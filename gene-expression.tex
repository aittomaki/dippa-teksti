%!TEX root = dippa.tex
%%% This file contains the Gene expression section of my master's thesis.
%%% This section covers the basics of gene expression.
%%% Author: Viljami Aittomäki

\section{Gene expression}\label{gene-expression}

Genetic information is encoded in molecules of deoxyribonucleic acid (DNA). A
gene is a section of DNA that serves as a template for a functional
ribonucleic acid (RNA) molecule. Gene expression refers to this process of
synthesizing a functional end-product from the information contained in gene.
DNA and gene expression serve as the basis of all currently known life. \citep{geneticskirja}.

Most of gene expression is dedicated to making proteins. The Central Dogma of
Molecular Biology, postulated by Francis Crick in 1970, describes the general
schema of how genetic information flows from genes to proteins; DNA is first
transcribed into messenger RNA (mRNA), which is then translated into a polypeptides,
which ultimately form proteins \citep{Crick1970} (illustrated in Figure \ref{fig:central-dogma}.
The flow is not strictly
one-directional, though, as there exist reverse transcriptases, enzymes that
synthesize DNA from an RNA template.

All genes are read to form RNA, but not all encode a protein product. The
human genome\footnote{The genome refers to the whole genetic material of an
organism or an individual.} has been suggested to contain approximately 20 500
protein-coding genes, which encompass only around 1.5\% of the whole genetic
sequence \citep{Clamp2007}. The vast majority of the human genome
was, thus, previously thought to be without function and referred to as "junk DNA".
More recently, however, it has become increasingly evident, that noncoding
portions of the genome are often functional and possibly have important
cellular functions \citep{ENCODE}. Noncoding genes give rise to noncoding RNA,
a class of RNA molecules that are mostly involved in aiding and regulating the
expression of other genes. Different types of noncoding RNA and their
functions are summarized in Table \ref{table:rnas}.




\subsection{Regulation of gene expression}\label{regulation-of-gene-expression}

Regulation of gene expression refers to controlling the abundance of
the gene end-product a cell produces. This control is paramount so that cells
can respond to external signals, changes in their environment, and 
move through different developmental stages. Gene expression can be regulated
at any stage of the process, and regulation can be roughly divided into transcriptional
and post-transcriptional regulation.

Transcriptional regulation with e.g. TFs, methylation, adenylation, splicing.

Post-transcriptional

MicroRNAs (miRNAs) and small interfering RNAs (siRNAs) are components of the
so called RNA interference (RNAi) pathway, which is a mechanism for regulating
gene expression post-transcriptionally. miRNAs and siRNAs are practically
interchangeable as substrates for RNAi, both act as target mRNA recognizing
templates, but have different biogenesis, where miRNAs are cut from endogenous hairpin
structures and siRNAs are processed from exogenous long double-stranded RNAs (dsRNAs).
\cite{Du2005} MicroRNAs are the focus of this study and are discussed in more
detail in the next section.

% The Argonaute (Ago) family of proteins is closely associated with small RNAs and
% Ago act as effectors in RNAi \cite{Ha2014}.

\begin{itemize}
  \item transcription factors
  \item post-transcriptional regulation
  \item translational factors
  \item protein degradation
\end{itemize}




\subsection{Measuring gene expression}\label{measurement-of-gene-expression}

The physical measurement of gene expression can be done on either the level of
messenger RNA molecules or protein molecules present in cells. Although
proteins are the eventual effectors molecules within cells -- at least for
protein-coding genes -- often gene expression is synonymous with mRNA
expression. This is because measuring mRNA abundances is significantly easier than
measuring protein abundances due to the chemistry of base-pair formation
and the relative ease of replicating DNA (or RNA) sequences by
exploiting cellular machinery evolved for this purpose.

Techniques shortly in one paragraph:
\begin{itemize}
  \item blotting, qPCR
  \item microarrays
  \item sequencing-based methods
  \item protein arrays (RPPA)
\end{itemize}




\subsubsection{Microarrays}

\begin{itemize}
  \item basics of operation
  \item shortly on microarray analysis
  \begin{itemize}
    \item preprocessing and normalization
    \item probe annotation problems
  \end{itemize}
\end{itemize}


The general assumption has been that mRNA expression is representative of gene
expression and that changes in mRNA abundances also reflect changes in protein
abundances and, therefore, cellular processes. This assumption has recently
been challenged by experiments indicating that the expression of mRNA and
corresponding protein often correlate poorly, with mRNA expression
explaining around 40\% of variation in protein abundances \citep{Vogel2012}.
There are also contrary findings of modest to good correlation
and one study suggested that mRNA-protein correlation is generally higher for
genes that have differing mRNA expression between studied conditions
(e.g. cancerous versus healthy tissue) \citep{Koussounadis2015}.

Nonetheless, Payne recently concluded that "proteome and transcriptome abundances are not
sufficiently correlated to act as proxies for each other" and that
most of this difference is likely caused by biological regulation and not
by measurement technology \cite{Payne2015}.
% This regulation can be post-transcriptional, translational
% or protein-degradation related, as discussed above.
Therefore, it is interesting, even necessary, to integrate
measurements from different stages of gene expression -- for example
mRNA, microRNA and protein abundances -- to gain new insights
into biological processes.
