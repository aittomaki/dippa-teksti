
%%% This file contains the Background section of my master's thesis.
%%% Author: Viljami Aittomaki


\section{Background}\label{background}










\subsection{Cancer}\label{cancer}

Cancer is generally viewed as a disease of genes. It is born of malfunction
and dysregulation of the genetic machinery in a cell.
\ldots{}

A classic idea of X was that there exist oncogenes, whose overactivity
promotes tumor growth, and tumor suppressors, whose underacticity and
suppression promotes tumorigenesis. MUTAATIOISTA? KAHDESTA HITISTA?

The idea of oncogenes and tumor suppressor genes was later expanded on by
Douglas Hanahan and Robert Weinberg in their seminal article The Hallmarks of
Cancer. The hallmarks are a set of six features which tumors often acquire to
become cancerous. Weinberg and X postulated that of these six features,
1.2.3.4.5.6., at least three are required for invasive cancer.

A few years later, Hanahan and Weinberg updated the hallmarks with two new
upcoming hallmarks, 1 and 2, and two enabling characteristics, genomic
instability and recruiting the immune system.










\subsection{MicroRNAs}\label{micrornas}

MicroRNAs (miRNAs) are a class of noncoding, small RNA molecules that function
as post-transcriptional regulators of gene expression. This section serves as
an introduction to microRNAs, their biogenes, function, and regulation.

The first microRNA, lin-4, was discovered in 1993 by two research groups
studying the larval development of the nematode \emph{Caenorhabditis elegans}.
They noted that lin-4 does not encode a protein, small RNAs encoded by lin-4
have several conserved antisense complimentarity sites in the lin-14 mRNA, and
that these sites are necessary and sufficient for the normal regulation of
lin-14 expression by lin-4 \cite{Lee1993,Wightman1993}. Since then,
\textbf{microRNAs have been found in more than XXXX species, including plants
\cite{CITE}. miRNAs have been found to regulate and control several key cellular
processes, including LISTAAA \cite{CITE}.} Many miRNAs are also highly
conserved in evolution \cite{Bartel2004}, highlighting the importance of their
regulatory function.

In their mature form these tiny regulatory RNAs are single stranded and
approximately 20 to 23 nucleotides long. MiRNAs are not translated into
protein
-- hence noncoding -- rather, they regulate gene expression by influencing the
   translation of messenger RNAs (mRNAs). They assert extensive control over
   the transcriptome; it has been estimated that more than 60 \% of human mRNA
   transcripts are regulated by miRNAs
\cite{CITE}.

MiRBase is the de facto database of known microRNAs, containing miRNA
sequences and genomic annotations \cite{CITE}. Its latest version, miRBase 21,
currently hosts \textbf{XX known miRNA precursors for X species. It contains
over 2500 mature human miRNAs \cite{CITE}. The number of precursor records has
gone up from XX in XX, illustrating ETTÄ LÖYDETTY PALJON LISÄÄ}.





\subsubsection{MicroRNA biogenesis}\label{microrna-biogenesis}

The biogenesis of microRNAs is illustrated in figure
\ref{fig:mirna-biogenesis} and is presented here as reviewed in
\cite{CITE}. MicroRNAs are transcribed by RNA polymerase II. miRNAs are
arranged in the genome as single miRNA genes, with their own promotor regions,
or in polycistronic miRNA clusters that share a common promotor and are
transcribed simultaneously, or in same cases within introns or even exons of
other genes, which can be protein coding or noncoding.

The transcribed miRNA molecule is called a pri-miRNA. The pri-miRNA is cut by
the endonuclease Drosha (also known as Rnasen) to form one or several pre-
miRNAs. In the case of miRNAs located within introns of other genes
(mirtrons), the pre-miRNA is formed in RNA splicing by the spliceosome. Pre-
miRNA contain two inverted repeats, which fold together to form a
characteristic hairpin RNA molecule. A typical pre-miRNA structure is shown in
figure \ref{fig:premirna-structure}. The hairpin is then exported from the
nucleus to the cytoplasm.

In the cytoplasm, the endonuclease Dicer cleaves the head of the hairpin to
form a double-stranded miRNA/miRNA* duplex. The duplex is then bound by a
protein complex containing an argonaut protein, in humans one of Ago1-4. The
argonaut separates the strands of the duplex and retains one of them. The
retained strand is known as the guide strand (or miRNA). The other strand,
called the passenger (miRNA*), is released and degraded.

There exist exceptions to this general scheme. Some miRNAs are not dependent
on Drosha, such as mirtrons, which are cut into pre-miRNA by the spliceosome.
Other miRNAs are independent of Dicer, including miR-451 WHICH JOTAIN.
Additionally, in some cases the miRNA* can also be bound to an miRISC and used
as a miRNA \cite{CITE}.





\subsubsection{MicroRNA function}\label{microrna-function}

The microRNA bound to an argonaut form the core part of a protein complex
known as the microRNA-induced silencing complex (miRISC). The miRISC seeks
messenger RNA molecules that match the bound template miRNA and binds them.

\begin{itemize}
\tightlist
\item
  RISC ja Argonaut
\item
  base pairing of seed sequence - sequence similarity families!
\item
  target degradation or inhibition of translation
\item
  combinatorial effect (many-to-one)
\end{itemize}





\subsubsection{MicroRNA regulation and dysregulation}\label{microrna-
regulation-and-dysregulation}

MicroRNAs themselves are regulated by several different mechanisms, similarly
to mRNA and gene expression. Possible control points include transcription of
genomic miRNAs, post-transcriptional modification of pri- and pre-miRNAs,
JATKA LISTAA. MiRNAs also form transcriptional feedback loops with proteins
that act as transcription factors (TFs) for miRNAs. For example, JOKU TÄHÄN JA
KUVA SIITÄ. Some of these regulation mechanisms have a known important
biological function, such as miR-XXX and miR-XXX which function as a cell fate
switch, deciding the differentiation of ASF.

Dysregulation of microRNAs can happen at any of these putative regulation
steps, and many, if not all, of them have been implicated in tumorgenesis.
Examples of this include LISTAA. There are also additional mechanisms that can
lead to miRNA dysregulation, such as mutations in the miRNA sequence or the
target mRNA sequence, genomic rearrangements,
\textbf{LISÄÄ?}. The role of miRNAs in cancer has recently been
thoroughly reviewed by FOO-ET-AL \cite{CITE}.

It has also been shown that inhibiting the miRNA biogenesis pathway leads to
death in several organisms \cite{CITE} and, further, that perturbing the
biogenesis often leads to tumorgenesis \cite{CITE}. This highlights the
importance of microRNAs and their regulatory function.










\subsection{MiRNA target prediction}\label{mirna-target-prediction}

This section presents computational methods that have been used to predict
putative target genes for microRNAs and regulatory networks between genes and
microRNAs.





\subsubsection{Sequence-based target prediction}\label{sequence-based-target-
prediction}

\textbf{SELITÄ TÄÄ HOMMA JA ERI ALGORITMIT.}

Sequence-based prediction suffers from two major drawbacks. First, there are
high numbers of false positives \textbf{SYY JA VIITE
\cite{Sethupathy2006?}}. Second, the predictions are static and do not
account for different tissues or disease states.





\subsubsection{Integration of expression data with sequence data}\label
{integration-of-expression-data-with-sequence-data}

Recently, a plethora of new methdos have been published that integrate
sequence-based target prediction with expression data. This helps combat the
high false-positive rate of sequence-only methods and enables tissue and
disease specific support for the predictions in real-world data. Recent
evidence indicates that miRNAs act predominantly through degradation
\cite{CITE}. Thus, it is feasible to use miRNA and mRNA expression data to
infer target relationships, since the regulatory effect of miRNAs should be
directly reflected in mRNA levels. All proposed methods use sequence-based
predictions as a starting point by considering only miRNA-mRNA pairs predicted
by at least one sequence-based prediction algorithm. This section provides a
review of algorithms for integrating miRNA and mRNA expression and
implementations in select target prediction methods.

\textbf{KOLME KATEGORIAA: KORRELAATIO, REGRESSIO, BAYESIAN?}



\paragraph{Correlation methods (incl MI)}\label{correlation-methods}

Mutual information (MI) is a simple measure of similarity between two
variables. \textbf{SELITÄ MI TARKEMMIN JA KAAVAN KANSSA JA LÄHDE} Thus, MI can
be used to measure the interdependence of miRNA-mRNA pairs from expression
data. However, MI does not distinguish the direction of the interaction, which
is highly relevant for miRNAs that are believed to mostly downregulate mRNA
expression. This constitutes a major drawback.

Correlation \textbf{SELITÄ KORRELAATIO JA LÄHDE}.

MAGIA \cite{Sales2010} is a webservice that implements both the MI and
correlation approaches. It also constructs a bipartite network of the top 250
predicted miRNA-mRNA pairs and provides links to several databases for further
examination and validation of results.

\textbf{VAI LAITTAISKO MAGIAN JO MI:N YHTEYTEEN JA TOISTAA VAAN
KORRELAATIOSSA?}



\paragraph{Regression methods}\label{regression-methods}

\textbf{SELITÄ REGRESSIO.}

Engelmann et al used least angle regression to show that gene expression can
be predicted from miRNA expression \cite{Engelmann}.

\textbf{SELITÄ REGULARISOIDUT/SHRINKAGE-TYYPPISET REGRESSIOMALLIT.}

While aiding with interpretability, shrinkage also has several drawbacks.
First, only a limited number of covariates may be included in the model, and
thus some relevant associations can be missed by number of included covariates
alone \cite{vanIterson2013}. Second, shrinkage may remove covariates highly
associated with and functionally regulating the response, instead retaining an
uninvolved covariate that correlates with actual regulators \cite{Engelmann}.
Relating to both limitations, van Iterson et al showed for one dataset that
lasso did not consistently select highly correlated miRNA-mRNA pairs
\cite{vanIterson2013}.

\textbf{SAISKO NÄIHIN VIITTEET REGRESSIOTEORIASTA EIKÄ TOSTA PAPERISTA?}


\subparagraph{Global test (regression)}\label{global-test-regression}

van Iterson et al recently proposed a miRNA target prediction method based on
the global test \cite{vanIterson2013}. \textbf{SELITÄ GLOBAL TEST}.

The method by van Iterson et al uses TargetScan, microCosm and PITA for
putative sequence-based targets and is available as an R package called
miRNAmRNA \cite{vanItersonWeb}.





\subsubsection{Bayesian methods}\label{bayesian-methods}

Advantages of Bayesian methods:
\begin{itemize}
\tightlist
\item quantification of uncertainty as probabilities, knowledge about anything
  unknown is described as a probability distribution, easy to comprehend
\item
  common-sense interpretation of credible intervals compared to frequentist
  confidence intervals
\item
  flexibility allows constructing complex models with relative ease (e.g.
  hierarchical models)
\item
  Adding more data sequentially is possible by using the previous posterior
  distribution as the new prior distribution. This can be especially useful in
  a clinical research context, where data are often collected within a long
  timespan.
\end{itemize}
The challenge in Bayesian analysis is setting up proper probability models for
the parameters and observations.

