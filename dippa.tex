%%%%%%%%%%%%%%%%%%%%%%%%%%%%%%%%%%%%%%%%%%%%%%%%%%%%%%%%%%%%%%%%%%%%
%%%%%%%%%%%%%%%%%%%%%%%%%%%%%%%%%%%%%%%%%%%%%%%%%%%%%%%%%%%%%%%%%%%%
%%                                                                %%
%% An example for writting your thesis using LaTeX                %%
%% Original version by Luis Costa,  changes by Perttu Puska       %%
%% Support for Swedish added 15092014                             %%
%%                                                                %%
%% Explanatory comments in this example begin with                %%
%% the characters %%, and changes that the user can make          %%
%% with the character %                                           %%
%%                                                                %%
%%%%%%%%%%%%%%%%%%%%%%%%%%%%%%%%%%%%%%%%%%%%%%%%%%%%%%%%%%%%%%%%%%%%
%%%%%%%%%%%%%%%%%%%%%%%%%%%%%%%%%%%%%%%%%%%%%%%%%%%%%%%%%%%%%%%%%%%%

%% Uncomment one of these:
%% the 1st when using pdflatex, which directly typesets your document in
%% pdf (use jpg or pdf figures), or
%% the 2nd when producing a ps file (use eps figures, don't use ps figures!).
\documentclass[english,12pt,a4paper,pdftex,elec,utf8]{aaltothesis}
%\documentclass[english,12pt,a4paper,dvips]{aaltothesis}

%% To the \documentclass above
%% specify your school: arts, biz, chem, elec, eng, sci
%% specify the character encoding scheme used by your editor: utf8, latin1

\usepackage{graphicx}

%% Use this if you write hard core mathematics, these are usually needed
\usepackage{amsfonts,amssymb,amsbsy}

%% Use the macros in this package to change how the hyperref package below 
%% typesets its hypertext -- hyperlink colour, font, etc. See the package
%% documentation. It also defines the \url macro, so use the package when 
%% not using the hyperref package.
%%
%\usepackage{url}

%% Use this if you want to get links and nice output. Works well with pdflatex.
\usepackage{hyperref}
\hypersetup{pdfpagemode=UseNone, pdfstartview=FitH,
  colorlinks=true,urlcolor=red,linkcolor=blue,citecolor=black,
  pdftitle={MicroRNA regulation in breast cancer},pdfauthor={Viljami Aittom\"aki},
  pdfkeywords={Bayesian analysis, breast cancer, microRNA}}


%% User-added packages
\usepackage[numbers,square]{natbib}
\usepackage[parfill]{parskip}



%% All that is printed on paper starts here
\begin{document}
\renewcommand{\thesissupervisorname}{Thesis supervisors}

%% Change the school field to specify your school if the automatically 
%% set name is wrong
% \university{aalto-yliopisto}
% \university{aalto University}
% \school{Sähkötekniikan korkeakoulu}
% \school{School of Electrical Engineering}

%% ONLY FOR M.Sc. AND LICENTIATE THESIS: Specify your department,
%% professorship and professorship code. 
%%
\department{Department of WHATTHEHELL}
\professorship{Computational WHATEVER}
%%

%% Valitse yksi näistä kolmesta
%%
%% Choose one of these:
%\univdegree{BSc}
\univdegree{MSc}
%\univdegree{Lic}

%% Your own name (should be self explanatory...)
\author{Viljami Aittom\"aki}

%% Your thesis title comes here and again before a possible abstract in
%% Finnish or Swedish . If the title is very long and latex does an
%% unsatisfactory job of breaking the lines, you will have to force a
%% linebreak with the \\ control character. 
%% Do not hyphenate titles.
%% 
\thesistitle{MicroRNA regulation in breast cancer -- a Bayesian analysis of expression data}

\place{Espoo}

%% For B.Sc. thesis use the date when you present your thesis.
%% BUT WHAT IS IT FOR MSc THESIS?
\date{3.8.2016}

%% B.Sc. or M.Sc. thesis supervisor 
%% Note the "\" after the comma. This forces the following space to be 
%% a normal interword space, not the space that starts a new sentence. 
%% This is done because the fullstop isn't the end of the sentence that
%% should be followed by a slightly longer space but is to be followed
%% by a regular space.
%%
\supervisor{Prof.\ Aki Vehtari}

%% B.Sc. or M.Sc. thesis advisors(s).
%%
\advisor{Ph.D.\ Rainer Lehtonen}

%% Aalto logo: syntax:
%% \uselogo{aaltoRed|aaltoBlue|aaltoYellow|aaltoGray|aaltoGrayScale}{?|!|''}
%% Logo language is set to be the same as the document language.
%%
\uselogo{aaltoRed}{''}


%% Create the coverpage
%%
\makecoverpage















%% English abstract.
%% All the information required in the abstract (your name, thesis title, etc.)
%% is used as specified above.
%% Specify keywords
%%
\keywords{breast cancer, microRNA, Bayes, microarray}
%% Abstract text
\begin{abstractpage}[english]
  'tis with magic that jolly interesting microRNA were conjured.
  The results were dazzling to say the least.
\end{abstractpage}

%% Force a new page so that the Finnish abstract starts on a new page
\newpage

%% Abstract in Finnish.  Delete if you don't need it. 
\thesistitle{MikroRNA-säätely rintasyövässä -- Bayeslainen analyysi ekspressiodatasta}
\advisor{Dosentti Rainer Lehtonen}
\degreeprogram{Electronics and electrical engineering}
\department{TÄÄ PITÄÄ KEKSIÄ JOSTAIN}
\professorship{JA TÄÄ}
%% Avainsanat
\keywords{rintasy\"op\"a, mikroRNA, Bayes, mikrosiru}
%% Tiivistelman tekstiosa
\begin{abstractpage}[finnish]
  Juhannustaikoja ja muuta huttua.
\end{abstractpage}




















%% Force new page so that the preface starts from a new page
\newpage

%% Preface
%%
\mysection{Preface}
Ja heillä kaikilla oli niin mukavaa.\\

\vspace{5cm}
Helsinki, 1.6.2016

\vspace{5mm}
{\hfill Viljami Aittom\"aki \hspace{1cm}}










%% Force new page after preface
\newpage


%% Table of contents. 
\thesistableofcontents














%% Symbols and abbreviations
\mysection{Symbols and abbreviations}

\subsection*{Symbols}

\begin{tabular}{ll}
$w$ & vector of weights for the covariates \\
$x$ & vector of covariates \\
$y$ & PREDICTED variable
% $\mathbf{B}$  & magnetic flux density  \\
% $c$              & speed of light in vacuum $\approx 3\times10^8$ [m/s]\\
% $\omega_{\mathrm{D}}$    & Debye frequency \\
% $\omega_{\mathrm{latt}}$ & average phonon frequency of lattice \\
% $\uparrow$       & electron spin direction up\\
% $\downarrow$     & electron spin direction down
\end{tabular}

% \subsection*{Operators}

% \begin{tabular}{ll}
% $\nabla \times \mathbf{A}$              & curl of vectorin $\mathbf{A}$\\
% $\displaystyle\frac{\mbox{d}}{\mbox{d} t}$ & derivative with respect to 
% variable $t$\\[3mm]
% $\displaystyle\frac{\partial}{\partial t}$  & partial derivative with respect 
% to variable $t$ \\[3mm]
% $\sum_i $                       & sum over index $i$\\
% $\mathbf{A} \cdot \mathbf{B}$    & dot product of vectors $\mathbf{A}$ and 
% $\mathbf{B}$
% \end{tabular}

\subsection*{Abbreviations}

\begin{tabular}{ll}
miRNA       & microRNA \\
mRNA        & messenger RNA \\
RNA         & ribonucleic acid \\
RPPA        & reverse phosphatase protein array
\end{tabular}


%% Tweaks the page numbering to meet the requirement of the thesis format:
%% Begin the pagenumbering in Arabian numerals (and leave the first page
%% of the text body empty, see \thispagestyle{empty} below).
%% Additionally, force the actual text to begin on a new page with the 
%% \clearpage command.
%% \clearpage is similar to \newpage, but it also flushes the floats (figures
%% and tables).
%% There is no need to change these
%%
\cleardoublepage
\storeinipagenumber
\pagenumbering{arabic}
\setcounter{page}{1}














%%=========================================================
%% Text body begins




%\section{Introduction}
%% Leave first page empty
%\thispagestyle{empty}

\section{Introduction}
\thispagestyle{empty}

Breast cancer is the most common of female cancers and causes remarkable
morbidity and mortality word-wide \citep{??}.  Annually more than 1.5 million
women develop breast cancer. Thus, breast cancer is a major global health
problem.

Cancer is a genetic disease caused by mutations in the genome of the cancer
cells. Some of these mutations can be inherited, while some are somatic, that
is, mutations that arise during the life-time of an individual in the tissue,
where cancer develops. To understand how cancer develops, the tumor evolution
of breast cancer, it is of paramount importance to identify genes in which
mutations initiate breast cancer development. It is already known that many of
these mutations perturb the expression of the gene in question. Previous
expression studies comparing the expression profiles of normal breast tissue
and breast cancer tissue have revealed ...

MicroRNA control of gene expression .... MicroRNAs have been indicated as
potential causative agents in numerous diseases, including several types of
cancer. Therefore, the study of microRNAs and their function in cancer can
offer insights into tumorigenesis and cancer progression as well as potential
new treatments.

Several different methods for computational identification of microRNA target
genes have been proposed. Early methods are mainly based on sequence similarity of the
microRNA and messenger RNA of putative target genes. More recent methods use
expression profiles to either augment sequence data or as sole predictors for
interactions. These are mostly based on correlation, or a derivative thereof,
of microRNA and gene expression. The combinatorial nature of miRNA action, however,
makes expression based strategies difficult, as one transcript may be regulated
by several different miRNAs simultaneously and the contribution of each
miRNA may be small. Additionally, most miRNAs have several target transcripts.

The aim of this thesis was to apply a recently proposed Bayesian variable
selection method to microRNA target discovery in breast cancer. The variable
selection was applied in the context of Bayesian regression of expression
profiles to elucidate which microRNAs are highly relevant in regulating
protein expression levels. The results were compared with a similar recent
study and also with known validated microRNA targets as well as microRNAs
previously indicated in breast cancer.






%\clearpage
%\section{Background}\label{background}

%%% This file contains the Background section of my master's thesis.
%%% Author: Viljami Aittomaki


\section{Background}\label{background}










\subsection{Cancer}\label{cancer}

Cancer is generally viewed as a disease of genes. It is born of malfunction
and dysregulation of the genetic machinery in a cell.
\ldots{}

A classic idea of X was that there exist oncogenes, whose overactivity
promotes tumor growth, and tumor suppressors, whose underacticity and
suppression promotes tumorigenesis. \textbf{MUTAATIOISTA? KAHDESTA HITISTA?}

The idea of oncogenes and tumor suppressor genes was later expanded on by
Douglas Hanahan and Robert Weinberg in their seminal article The Hallmarks of
Cancer \cite{Hanahan2000}. The hallmarks are a set of six features which
tumors often acquire to become cancerous. Weinberg and Hanahan postulated that
of these six features at least three are required for invasive cancer.

Recently, Hanahan and Weinberg updated the hallmarks with two new upcoming
hallmarks and two enabling characteristics \cite{Hanahan2011}. The new
hallmarks are deregulation of cellular energetics and avoiding immune
destruction. The enabling characteristics of cancerous tumors are genome
instability and mutation, and tumor-promoting inflammation through recrution
of the immune system.




\subsubsection{Breast cancer}\label{breast-cancer}

Cancer of the breast constitutes a significant health issue globally. It is
the most common cancer in women in 140 nations. Approximately 1.7 million
women get the disease annually in the world, and in 2012 there were 522 000
breast cancer related deaths globally. \textbf{IARCista viite.} In Finland the
incidence of breast cancer is approximately 4 800 new cases annually, and
there were 874 breast cancer related deaths in 2013 \cite{Syoparekisteri}.

Most breast cancers are sporadic; only 5-7\% of breast cancer cases are of
familial type \cite{Melchor2013}. However, 15-30\% of breast cancer patients
have a family member or relative with breast cancer. This VIITTAA that there
are unknown genetic factors and possible YMPÄRISTÖTEKIJÄ that have an impact
on breast cancer development.


\paragraph{Breast cancer classification}\label{breast-cancer-classification}

\begin{itemize}
\tightlist
\item
  histological types, (NMT classification?)
\item
  PAM50 molecular classification
\item
  new three-gene-model classification \textbf{tän edut? onko kukaan viitannu/käyttäny tätä?}
\end{itemize}









\subsection{Molecular biology}\label{molecular-biology}

Molecular biology is concerned with the study of different molecules and their
interactions at the cellular level and the effects these have on cellular
processes. Central species to molecular biology are deoxyribonucleic acid
(DNA), ribonucleic acid (RNA) and proteins; these are the main actors of
different cellular functions. The Central Dogma of Molecular Biology,
postulated by Crick in 1970 and illustrated in figure \ref{central-dogma},
describes the general schema in which genes are first transcribed into
messenger RNA (mRNA), which is then translated into protein
\cite{Crick1970}.

\subsubsection{Ribonucleic acids}\label{ribonucleic-acids}










\subsection{MicroRNAs}\label{micrornas}

MicroRNAs (miRNAs) are a class of noncoding, small RNA molecules that function
as post-transcriptional regulators of gene expression. This section serves as
an introduction to microRNAs, their biogenes, function, and regulation.

The first microRNA, lin-4, was discovered in 1993 by two research groups
studying the larval development of the nematode \emph{Caenorhabditis elegans}.
The groups noted that lin-4 does not encode a protein, but instead produces a
pair of small RNAs \cite{Lee1993}. These RNAs encoded by lin-4 have several
conserved antisense complimentarity sites in an untranslated region of the
lin-14 mRNA, and these sites are necessary and sufficient for the normal
downregulation of lin-14 expression by lin-4 \cite{Lee1993,Wightman1993}. It
should be noted, that one these groups also showed that lin-4 reduces the
amount of the LIN-14 protein -- the end-product of the lin-14 gene -- without
significantly affecting the cellular concentration of the lin-14 mRNA
\cite{??}. We will return to this issue of microRNA action in chapter
\ref{microrna-function}.

Since then,
\textbf{microRNAs have been found in more than XXXX species, including plants
\cite{CITE}. miRNAs have been found to regulate and control several key cellular
processes, including LISTAAA \cite{CITE}.} Many miRNAs are also highly
conserved in evolution \cite{Bartel2004}, highlighting the importance of their
regulatory function.

In their mature form these tiny regulatory RNAs are single stranded and
approximately 20 to 23 nucleotides long. MiRNAs are not translated into
protein -- hence noncoding -- rather, they regulate gene expression by
influencing the translation of messenger RNAs (mRNAs). They assert
extensive control over the transcriptome; it has been estimated that more
than 60 \% of human mRNA transcripts are regulated by miRNAs
\cite{CITE}.

MiRBase is the de facto database of known microRNAs, containing miRNA
sequences and genomic annotations \cite{CITE}. Its latest version, miRBase 21,
currently hosts \textbf{XX known miRNA precursors for X species. It contains
over 2500 mature human miRNAs \cite{CITE}. The number of precursor records has
gone up from XX in XX, illustrating ETTÄ LÖYDETTY PALJON LISÄÄ}.





\subsubsection{MicroRNA biogenesis}\label{microrna-biogenesis}

The biogenesis of microRNAs is illustrated in figure
\ref{fig:mirna-biogenesis} and is presented here as reviewed by Bartel 
\cite{Bartel2004} and Denzler et al \cite{Denzler2015}. MicroRNAs are
transcribed by RNA polymerase II. miRNAs are arranged in the genome as
single miRNA genes, with their own promotor regions,
or in polycistronic miRNA clusters that share a common promotor and are
transcribed simultaneously, or in same cases within introns or even exons of
other genes, which can be protein coding or noncoding.

The transcribed miRNA molecule is called a pri-miRNA. The pri-miRNA is cut by
the endonuclease Drosha (also known as Rnasen) to form one or several pre-
miRNAs. In the case of miRNAs located within introns of other genes
(mirtrons), the pre-miRNA is formed in RNA splicing by the spliceosome. Pre-
miRNA contain two inverted repeats, which fold together to form a
characteristic hairpin RNA molecule. A typical pre-miRNA structure is shown in
figure \ref{fig:premirna-structure}. The hairpin is then exported from the
nucleus to the cytoplasm.

In the cytoplasm, the endonuclease Dicer cleaves the head of the hairpin to
form a double-stranded miRNA/miRNA* duplex. The duplex is then bound by a
protein complex containing an argonaut protein, in humans one of Ago1-4. The
argonaut separates the strands of the duplex and retains one of them. The
retained strand is known as the guide strand (or miRNA). The other strand,
called the passenger (miRNA*), is released and degraded.

There exist exceptions to this general scheme. Some miRNAs are not dependent
on Drosha, such as mirtrons, which are cut into pre-miRNA by the spliceosome.
Other miRNAs are independent of Dicer, including miR-451 WHICH JOTAIN.
Additionally, in some cases the miRNA* can also be bound to an miRISC and used
as a miRNA \cite{CITE}.





\subsubsection{MicroRNA function}\label{microrna-function}

The microRNA bound to an argonaut form the core part of a protein complex
known as the microRNA-induced silencing complex (miRISC). The miRISC seeks
messenger RNA molecules that match the bound template miRNA and binds them.

\begin{itemize}
\tightlist
\item
  RISC ja Argonaut
\item
  base pairing of seed sequence - sequence similarity families!
\item
  target degradation or inhibition of translation
\item
  combinatorial effect (many-to-one)
\end{itemize}





\subsubsection{MicroRNA regulation and dysregulation}\label{microrna-
regulation-and-dysregulation}

MicroRNAs themselves are regulated by several different mechanisms, similarly
to mRNA and gene expression. Possible control points include transcription of
genomic miRNAs, post-transcriptional modification of pri- and pre-miRNAs,
JATKA LISTAA. MiRNAs also form transcriptional feedback loops with proteins
that act as transcription factors (TFs) for miRNAs. For example, JOKU TÄHÄN JA
KUVA SIITÄ. Some of these regulation mechanisms have a known important
biological function, such as miR-XXX and miR-XXX which function as a cell fate
switch, deciding the differentiation of ASF.

Dysregulation of microRNAs can happen at any of these putative regulation
steps, and many, if not all, of them have been implicated in tumorgenesis.
Examples of this include LISTAA. There are also additional mechanisms that can
lead to miRNA dysregulation, such as mutations in the miRNA sequence or the
target mRNA sequence, genomic rearrangements,
\textbf{LISÄÄ?}. The role of miRNAs in cancer has recently been
thoroughly reviewed by FOO-ET-AL \cite{CITE}.

It has also been shown that inhibiting the miRNA biogenesis pathway leads to
death in several organisms \cite{CITE} and, further, that perturbing the
biogenesis often leads to tumorgenesis \cite{CITE}. This highlights the
importance of microRNAs and their regulatory function.










\subsection{Expression microarrays}

\begin{itemize}
  \item two-color arrays
  \item oligonucleotide arrays
  \item miRNA arrays
  \item preprocessing and normalization
  \item probe annotation problems
  \item protein arrays (RPPA)
\end{itemize}









\subsection{MiRNA target prediction}\label{mirna-target-prediction}

This section presents computational methods that have been used to predict
putative target genes for microRNAs and regulatory networks between genes and
microRNAs.





\subsubsection{Sequence-based target prediction}\label{sequence-based-target-
prediction}

\textbf{SELITÄ TÄÄ HOMMA JA ERI ALGORITMIT.}

Sequence-based prediction suffers from two major drawbacks. First, there are
high numbers of false positives \textbf{SYY JA VIITE
\cite{Sethupathy2006?}}. Second, the predictions are static and do not
account for different tissues or disease states.


\paragraph{MicroRNA target databases}





\subsubsection{Integration of expression data with sequence data}\label
{integration-of-expression-data-with-sequence-data}

Recently, a plethora of new methdos have been published that integrate
sequence-based target prediction with expression data. This helps combat the
high false-positive rate of sequence-only methods and enables tissue and
disease specific support for the predictions in real-world data. Recent
evidence indicates that miRNAs act predominantly through degradation
\cite{CITE}. Thus, it is feasible to use miRNA and mRNA expression data to
infer target relationships, since the regulatory effect of miRNAs should be
directly reflected in mRNA levels. All proposed methods use sequence-based
predictions as a starting point by considering only miRNA-mRNA pairs predicted
by at least one sequence-based prediction algorithm. This section provides a
review of algorithms for integrating miRNA and mRNA expression and
implementations in select target prediction methods.

\textbf{KOLME KATEGORIAA: KORRELAATIO, REGRESSIO, BAYESIAN?}



\paragraph{Correlation methods (incl MI)}\label{correlation-methods}

Mutual information (MI) is a simple measure of similarity between two
variables. \textbf{SELITÄ MI TARKEMMIN JA KAAVAN KANSSA JA LÄHDE} Thus, MI can
be used to measure the interdependence of miRNA-mRNA pairs from expression
data. However, MI does not distinguish the direction of the interaction, which
is highly relevant for miRNAs that are believed to mostly downregulate mRNA
expression. This constitutes a major drawback.

Correlation \textbf{SELITÄ KORRELAATIO JA LÄHDE}.

MAGIA \cite{Sales2010} is a webservice that implements both the MI and
correlation approaches. It also constructs a bipartite network of the top 250
predicted miRNA-mRNA pairs and provides links to several databases for further
examination and validation of results.

\textbf{VAI LAITTAISKO MAGIAN JO MI:N YHTEYTEEN JA TOISTAA VAAN
KORRELAATIOSSA?}



\paragraph{Regression methods}\label{regression-methods}

\textbf{SELITÄ REGRESSIO.}

Engelmann et al used least angle regression to show that gene expression can
be predicted from miRNA expression \cite{Engelmann}.

\textbf{SELITÄ REGULARISOIDUT/SHRINKAGE-TYYPPISET REGRESSIOMALLIT.}

While aiding with interpretability, shrinkage also has several drawbacks.
First, only a limited number of covariates may be included in the model, and
thus some relevant associations can be missed by number of included covariates
alone \cite{vanIterson2013}. Second, shrinkage may remove covariates highly
associated with and functionally regulating the response, instead retaining an
uninvolved covariate that correlates with actual regulators \cite{Engelmann}.
Relating to both limitations, van Iterson et al showed for one dataset that
lasso did not consistently select highly correlated miRNA-mRNA pairs
\cite{vanIterson2013}.

\textbf{SAISKO NÄIHIN VIITTEET REGRESSIOTEORIASTA EIKÄ TOSTA PAPERISTA?}


\subparagraph{Global test (regression)}\label{global-test-regression}

van Iterson et al recently proposed a miRNA target prediction method based on
the global test \cite{vanIterson2013}. \textbf{SELITÄ GLOBAL TEST}.

The method by van Iterson et al uses TargetScan, microCosm and PITA for
putative sequence-based targets and is available as an R package called
miRNAmRNA \cite{vanItersonWeb}.





\subsubsection{Bayesian methods}\label{bayesian-methods}

First published Bayesian analysis by Thomas Bayes in 1763 \cite{Gelman2013}.

Advantages of Bayesian methods:
\begin{itemize}
\tightlist
  \item
  quantification of uncertainty as probabilities, knowledge about anything
  unknown is described as a probability distribution, easy to comprehend
  \item
  common-sense interpretation of credible intervals compared to frequentist
  confidence intervals
  \item
  flexibility allows constructing complex models with relative ease (e.g.
  hierarchical models)
  \item
  Adding more data sequentially is possible by using the previous posterior
  distribution as the new prior distribution. This can be especially useful in
  a clinical research context, where data are often collected within a long
  timespan.
\end{itemize}
The challenge in Bayesian analysis is setting up proper probability models for
the parameters and observations. \cite{Gelman2013}

The posterior distribution represents a compromise between the prior (and
hence, prior information) and the observed data with the data having a larger
effect as the sample size increases. \cite{Gelman2013}.


\paragraph{Simulation}\label{simulation}

Bayesian anaylis often involves simulation if the form of sampling from the
obtained posterior distribution. This is convenient -- and necessary -- when
the exact probability density function cannot be explicitly obtained through
integration. Additionally, simulation often has the advantage of pointing out
problems in the model specification when simulated values are extremely small
of large.

Simulation methods:
\begin{itemize}
\tightlist
  \item
  sampling from probability distributions (easy with modern pseudorandom
  number generators)
  \item
  \textbf{mikäseyksinkertasinonkaan}
  \item
  Gibbs
  \item
  Hamiltonian Monte Carlo (this is used by Stan)
\end{itemize}


\paragraph{Bayesian regression}\label{bayesian-regression}

\TEXTBF{SELITÄ MITEN BAYES-REGRESSIO EROAA TAVALLISESTA.}







%\clearpage
%\section{Research material and methods}
%!TEX root = dippa.tex
%%% This file contains the materials and methods section of my master's thesis.
%%% Author: Viljami Aittomäki


\section{Materials and methods}\label{materials-and-methods}

\subsection{Research material}

The data analysed in this thesis consists of 283 tumor samples collected from
280 breast cancer patients treated in two Norwegian hospitals. Protein, mRNA
and microRNA expression were measured from each sample. The data were
published by Aure et al. \citep{Aure2015} and are publicly available. Analyses
performed in this thesis used the publicly available preprocessed data.

The data are part of the Oslo2 cohort, which consists of breast cancer
patients with primarily operable disease -- that is stage cT1--cT2 -- treated
in several Norwegian hospitals. Collection of the cohort started in 2006 and
is still ongoing. Therefore, no survival data were available for analysis.
Clinical data for included patients were kindly provided by Aure and
associates and a compiled summary is presented in Table
\ref{clinical-data}. Notably, the vast majority of tumors in the data
were ductal carcinomas, which is in general the most common histological type
of breast cancer. Patient ages ranged from X to Y with a median of 45 years.
No matched control samples of healthy breast tissue were available.
It is worth noting that the data represent a very heterogeneous sampling
of different kinds of breast tumors.

% latex table generated in R 3.3.0 by xtable 1.8-2 package
% Thu Oct  6 19:58:33 2016
\begin{table}[ht]
\centering
\caption{
Clinical features of the 283 tumor samples included in the analysis.
Table compiled from original data kindly provided by Aure and associates.
See \citep{Tavassoli2003} for definitions of the TNM classification.
CIS: carcinoma in situ. ER: estrogen receptor. PR: progesterone receptor.
HER2: human epidermal growth factor receptor 2. Multifocality: whether there is a single primary tumor or several. \
positive: receptor present in cancer cells. negative: receptor not present.
missing: information not available
\label{clinical-data}
} 
\begingroup\footnotesize
\begin{tabular}{llrr}
% latex table generated in R 3.3.0 by xtable 1.8-2 package
% Thu Oct  6 19:58:33 2016
  \hline
Variable & Level & Count & Fraction \\ 
  \hline
Histology & Ductal & 228 & 81\% \\ 
   & Ductal CIS & 7 & 2\% \\ 
   & Lobular & 23 & 8\% \\ 
   & Medullary & 1 & 0\% \\ 
   & Metaplastic & 1 & 0\% \\ 
   & Mixed & 4 & 1\% \\ 
   & Mucinous & 5 & 2\% \\ 
   & Papillary CIS & 1 & 0\% \\ 
   & Tubular & 5 & 2\% \\ 
   & missing & 5 & 2\% \\ 
Tumor size (T) & pT1a & 1 & 0\% \\ 
   & pT1b & 20 & 7\% \\ 
   & pT1c & 122 & 44\% \\ 
   & pT2 & 109 & 39\% \\ 
   & pT3 & 13 & 5\% \\ 
   & pTis & 7 & 2\% \\ 
   & pTx & 3 & 1\% \\ 
   & missing & 5 & 2\% \\ 
Metastasis (M) & M0 & 36 & 13\% \\ 
   & M1 & 6 & 2\% \\ 
   & Mx & 236 & 84\% \\ 
   & missing & 2 & 1\% \\ 
Lymph node status (N) & pN0 & 166 & 59\% \\ 
   & pN1 & 80 & 29\% \\ 
   & pN1a & 2 & 1\% \\ 
   & pN2 & 19 & 7\% \\ 
   & pN3 & 8 & 3\% \\ 
   & missing & 5 & 2\% \\ 
Grade & I & 42 & 15\% \\ 
   & II & 108 & 39\% \\ 
   & III & 125 & 45\% \\ 
   & missing & 5 & 2\% \\ 
ER & negative & 52 & 19\% \\ 
   & positive & 216 & 77\% \\ 
   & missing & 12 & 4\% \\ 
PR & negative & 81 & 29\% \\ 
   & positive & 187 & 67\% \\ 
   & missing & 12 & 4\% \\ 
HER2 & negative & 240 & 86\% \\ 
   & positive & 28 & 10\% \\ 
   & missing & 12 & 4\% \\ 
Multifocality & Multifocal & 44 & 16\% \\ 
   & Single tumor & 225 & 80\% \\ 
   & missing & 11 & 4\% \\ 
   \hline
\end{tabular}
\endgroup
\end{table}


The mRNA and microRNA expression were measured using Agilent Technologies
SurePrint G3 Human GE 8x60K and Human miRNA Microarray Kit (V2) microarrays,
respectively. These microarrays measure 27958 genes and 887 miRNAs, according
to manufacturer annotation. Protein expression was measured using a reverse
phospatase protein array (RPPA) containing a set of 105 proteins. Most of the
proteins are found on the PI3K/AKT intracellular pathway, which is important
for cell-cycle regulation and, thus, cancer. \textbf{(ref to G.Mills?)}.

% The mRNA and microRNA expression data are publicly available in preprocessed
% format in the Gene Expression Omnibus (GEO) database \citep{GEO} under
% accession IDs xx and xx respectively. For the purpose of this thesis, the raw
% Agilent expression data were kindly provided and used for the analyses instead
% of the preprocessed data. The protein expression data is available in
% Additional file 4 of \citet{norjis} also in preprocessed format.

% Clinical data concerning each patient and cancer were also provided. A summary
% of the clinical parameters is presented in table \ref{clinical-data}. The
% predominant tumor type in the data was ductal carcinoma, which is in general
% the most common histological type of breast cancer.

% Use danish data for validation?





\subsection{Methods}

A Bayesian regression model of protein, mRNA and miRNA expression data was
constructed, and projection predictive variable selection was used to predict
microRNA targets in breast cancer data. Details of the methods used are
presented in this section. All computational analyses were performed in R
\citep{R} and workflow management was handled with Anduril \citep{Ovaska2010}.
Monte-Carlo simulations for the Bayesian regression models were performed with
\emph{RStan} \citep{RStan} using the No-U-Turn variant of a Hamiltonian Monte Carlo
algorithm for sampling posterior distributions. Simulations were run using
computer resources within the Aalto University School of Science "Science-IT"
project.


\subsubsection{Preprocessing and quality control}

MicroRNA and mRNA expression data were downloaded from the Gene Expression
Omnibus (accessions GSE8210 and GSE8212, respectively \citep{Edgar2002}) using
the GEOquery R package \citep{GEOquery}. Protein expression data were
downloaded as a Microsoft Excel sheet from the supplementary data of Aure et
al \citep{Aure2015}. All of the data are in log scale, as is usual for
expression data analyses. For regression analyses, all variables (miRNA, mRNA
and protein) were further scaled to have zero mean and unit variance,
a commonly used transformation in regression.

Notably, the protein data have only a single set of measurements for two sets
of related genes, namely AKT1, AKT2, AKT3 and GSK3A, GSK3B (labeled
respectively as "AKT1/2/3" and "GSK3A/GSK3B" in the original data). Presumably
the protein array used cannot differentiate the proteins of these genes, but
no mention of this is made by Aure et al. For the analyses presented here,
each of these genes was considered separately, and the same expression values
from the single measurement set were used for all of the related genes
respectively.

Out of the 421 microRNAs present in the public dataset, eleven miRNAs (hsa-
miR-1274a, hsa-miR-1274b, hsa-miR-1280, hsa-miR-1308, hsa- miR-1826, hsa-
miR-1974, hsa-miR-1975, hsa-miR-1977, hsa-miR-1979, hsa-miR-720, hsa-
miR-886-3p) were reported as missing from miRBase by miRBase Tracker. Reviewed
on miRBase, these miRNAs are reported as being fragments of other RNA species
(e.g. tRNA or rRNA) and, thus, removed from the database. The eleven miRNAs
were therefore removed from subsequent analyses, leaving 410 miRNAs.

The publicly available expression data are in preprocessed form, and as such,
no further preprocessing of the actual measurements was done. For details on
the preprocessing, the reader is referred to the supplementary data of Aure et
al \citep{Aure2015}. The mRNA expression data are available as probe-level
measurements, these were summarized to gene-level using manufacturer probe
annotations by taking the median of all probes targeting the same gene.
Only the genes present in the protein data were used in analyses.

For assessing the quality of the expression data, distributions of each
microarray and each variable (miRNA, mRNA and protein) were plotted. The data
have been collected from two different hospitals and to assess possible bias
introduced by the separate sites, a principal component analysis (PCA) and
hierarchical clustering of samples were performed separately for each data
type.



\subsubsection{Correlation analysis}


\subsubsection{Regression models}

For predicting protein expression from mRNA and miRNA expression, a similar
regression model to Aure et al was used:
\begin{equation}
	\label{eq:reg-model}
	y = \beta_0 + z \beta_g + X \beta + \epsilon,
\end{equation}
where $y$ denotes the protein expression vector for the protein, which
is produced from translation of the mRNA $z$, $w_g$
is the regression coefficient for the mRNA, $X$ is the matrix of miRNA
expression vectors, and $w_0$ is the intercept term
(for a justification of this equation, see Aure et al \citep{Aure2015}).
A separate model was fitted for each gene.
A model with only the mRNA expression covariate (called the \emph{gene-only model}),
defined as $y = \beta_0 + z \beta_g + \epsilon$, was used as a baseline.
A normally distributed error term with equal errors and no correlation
between observations was assumed for all models.

The likelihood for Bayesian regression was therefore defined as
\begin{equation}
	y | \mathbf{\beta}, \sigma, z, X \sim N(\beta_0 + z \beta_g + X \beta, \sigma^2I),
\end{equation}
where $\mathbf{\beta} = [\beta_0, \beta_g, \beta]$ for convenience.
The intercept and mRNA coefficient were given diffuse Gaussian priors
and $\sigma$ a uniform prior:
\begin{subequations}
  \begin{align}
    \beta_0 & \sim N(0, 5^2) \\
    \beta_g & \sim N(0, 5^2) \\
    \sigma  & \propto 1.
  \end{align}
\end{subequations}
These were also used in the gene-only model.
A hierarchical shrinkage prior was applied to the miRNA coefficients $\beta$,
as defined in Equation \eqref{eq:hs-prior}. The degrees of freedom for the
$\lambda_j$ priors was set at $\nu=3$ (similar to Piironen and Vehtari
\citep{Piironen2015}). The prior for $\tau$ was defined as:
\begin{equation}
    \tau ~ \textup{half-Cauchy}(0, \frac{p_n}/{n}*\sqrt{\textup{log}(n/p_n)}),
\end{equation}
combining the previous suggestions of half-Cauchy and fixed $tau$.
The assumed number of relevant miRNAs, $p_n$, was estimated as follows.
Ensembl gene ID's were downloaded for all protein-coding genes in the human
genome using biomaRt \citep{biomaRt}. From these, a sample of 1000 genes was
taken, and known validated microRNA interaction partners for each sampled gene
were downloaded from miRWalk \citep{Dweep2015}. Genes for which there were no
validated miRNA interactors were assumed to have zero. The mean number of
miRNA interactors per gene was used as an estimate, giving
$\hat{p_n} = 13.75$.



\subsubsection{Variable selection}

Projection predictive variable selection (as described in Section
\ref{sec:bayes-variable- selection}) was used to obtain the relevant set
of microRNAs for each gene. A full model was fitted by drawing 2000 samples
from the posterior using RStan (4 chains, 1000 samples each and the first half
discarded as burn-in). A random sample of $S=1000$ simulation samples from the
full posterior was used to increase projection speed. Then a series of
submodels was obtained by projection from the full model and using a forward
search strategy. That is, the search started from a model including only the
intercept, the mRNA expression $z$ was always added as the first covariate,
and then at each subsequent step, the miRNA covariate $x_j$ giving the largest
decrease in KL-divergence between the full and projected models was chosen.
The forward search was continued up to 200 variables.

For choosing the model size, 10-fold cross validation was used, as proposed by
Piironen and Vehtari \citep{Piironen2016}. That is, the above model selection
process was performed $K=10$ times, each leaving $n/K$ observations out for
evaluation. To judge the appropriote model size, estimation of submodel
predictive performance was used as explained below.

\paragraph{Predictive performance}
Given submodel $M_\perp$ with the posterior predictive distribution
$p(\tilde{y}| \tilde{z}, \tilde{X}_\perp, \mathbf{\theta}_\perp, D_\perp)$,
where $D_\perp$ is the observed data in the current submodel and
$\mathbf{\theta}_\perp$ the projected parameters, the predictive performance
of each submodel was evaluated with the logarithm of the predictive density
(LPD) at each of the left-out observations $(y_*, z_*, X_*)$ The LPD was
estimated by averaging over the simulated posterior samples:
\[
	textup{LPD}_*(M_\perp) \approx \textup{log}\frac{1}{S} \sum_s^S p(y_*| z_*, X_*, \mathbf{\beta}_\perp, \sigma_\perp).
\]
The LPD's from each fold were pooled and a mean over the full set of data
(MLPD) was used as a summary. To compare the predictive performance of a
sbumodel to the full model the difference in MLPD ($\Delta \textup{MLPD}$) was
used. Bayesian bootstrap (with 5000 samples) was used to estimate a
distribution for $\Delta \textup{MLPD}$ by:
\[
	\Delta \textup{MLPD}^{s} = \sum_i^n w_i^{(s)} \left [ LPD_i(M_\perp) - LPD_i(M_{\textup{full}}) \right ],
\]
where $w_i^{(s)}, i = 1, \dotsc,n$, are the bootstrap weights
for the $s$th bootstrap sample (subject to $\sum_i w_i^{(s)} = 1$).
The \emph{bayesboot} R package was used for computing the bootstrap.

\paragraph{Choosing model size}
The threshold for choosing the model size was defined as the smallest
model satisfying
\begin{equation}
	\label{eq:size-condition}
	\textup{Pr}(\Delta \textup{LPD} < U) \geq \alpha,
\end{equation}
where $U = \gamma \textup{E}(LPD_{\textup{full}}-LPD_{0})$, and
$LPD_0$ refers to the intercept-only model. This means that model
size was chosen such that the probability of the difference in utility
between projected and full model being smaller than a constant times
the difference between intercept-only and full model was at least $\alpha$.
Several different values for $\alpha$ and $\gamma$ were experimented with.

\paragraph{Final model selection}
The final projected model was obtained by redoing the projection search up to
the chosen number of variables, using all data for each gene. The miRNAs for
which the 95\% posterior interval did not include the origo (called
\emph{significant miRNAs}) were considered putative interactors with the gene
in question. For some genes the condition in \eqref{eq:size-condition} was not
met after including 200 covariates. In these cases it was concluded that the
miRNA covariates provided no additional information on the protein expression
and, thus, none of them were deemed as targeting the gene. In some cases the
condition was met already by the model with only the mRNA covariate, and the
same conclusion was made.

\paragraph{Lasso regression}
A lasso regression model was also fitted for each gene using the
\emph{glmnet} R package. In this case the mRNA variable was treated equal
to the miRNA variables and subjected to the lasso regularization.
For choosing the regularization parameter $\lambda$, a 10-fold cross
validation was done, and the largest $\lambda$ that gave
a mean square prediction error (MSE) at most one standard deviation
apart from the lowest MSE was used. The covariates included
in the model using the chosen $\lambda$ were considered putative
target interactors. For same genes, this criterion was met by
the intercept-only model, and again, in these cases
none of the miRNAs were deemed as targeting the gene.



\subsubsection{Measuring model fit}

The coefficient of determination $R^2$ of a predictive model is defined as
\begin{equation}
	R^2 = \frac{SS_{\textup{residuals}}}{SS_{\textup{total}}} = \frac{\sum_{i=1}^{n}(\hat{y}_i-y_i)^2}{\sum_{i=1}^{n}(y_i-\bar{y})^2},
\end{equation}
where $\hat{y}_i$ are the predictions made by the model and $\bar{y}$ is the mean of the outcome variable.
$R^2$ indicates the proportion of variance of the outcome variable that is explained
by a statistical model. It provides a measure of how well the model
replicates the observed values and can be used as a measure of model fit.
$R^2$ has the property of being invariant to variable scaling, which
makes it suitable for use with expression data, as expression data
do not have a well defined scale, $R^2$. A caveat of
$R^2$ is that in linear regression it often increases monotonically by adding more explanatory
variables. The adjusted $R^2$, defined as $\bar{R}^2 = 1-(1-R^2)*(n-1)/(n-p)$
(where $n$ is the number of observations and $p$ the number of explanatory
variables) adjusts for the number of regressors relative to the number of
observations, thus penalizing inclusion of additional variables. $\bar{R}^2$
was used for comparing the projected model with the gene only model.



\subsubsection{Verification of predicted targets}








\clearpage
\section{Results}






%\clearpage
%\section{Discussion} 
%!TEX root = dippa.tex
%%% This file contains the discussion section of my master's thesis.
%%% Author: Viljami Aittomäki

\section{Discussion}

This thesis presents a review of the basics of gene expression, microRNAs and
the computational prediction of microRNA target genes. A modern Bayesian
variable selection method was applied in the context of regression, to predict
protein expression from mRNA and miRNA expression in breast cancer tumor
samples, with the goal of identifying putative miRNA target. The Bayesian
method was compared to lasso regression, a popular method for target
prediction from expression data.

Quality control plots suggested the presence of noise in the data, especially
in the miRNA microarrays. More strict filtering of miRNA data could lead to
better overall results. The choice of data model can also alleviate the
effects of noise. The
student-$t$ distribution, which has heavier tails than the normal
distribution, would likely be a better choice for the regression model
likelihood. This, however, would require a different solution to the
projection Eq. \eqref{eq:projection} than the one derived by Piironen
and Vehtari \citep{Piironen2015} and, therefore, was not tried in this work.

Correlation between mRNA and protein measurements for the same gene were
mostly low, a finding supported by previous studies \citep{Payne2015}.
Interestingly, there was practically no difference between correlations of
validated miRNA-target pairs or randomly chosen ones. This supports the view
that modeling single interaction pairs individually is unlikely to be
sufficient for effective target prediction using expression data.

It is possible that low correlation of miRNA-target pairs was mostly due to the
dataset used and measurement noise it contained, however, 
%as discussed in Section \ref{expr-methods},
correlation has low power to detect weak relationships and
cannot capitalize on combinatorial effects. The low correlations also suggest
that simple (Pearson) correlation of expression data is possibly inadequate for
finding miRNA targets. Poor prediction results using correlation have been
reported before (see for example \citep{Muniategui2012}).

There could be several explanations for the increasingly poor performance of
larger models, evident in Figure \ref{fig:n-miRNAs-vs-R2}. The largest models
could be a result of the covariates not explaining the outcome variable well.
This results in a large model, as the inclusion of each covariate can provide
only a minute improvement to submodel performance and, therefore, a large
number of covariates is required to satisfy the criterion for choosing model
size. The assumption that the reference model represents the best current knowledge
could also be false, as seemed to be the case for some genes where the
submodels performed consistently better than the full model. This likely means
that the reference model has overfit the training data (in the cross-
validation).

From a biological perspective, poor performance means that the microRNAs did
not provide additional information for predicting protein expression after
accounting for gene expression. This could be due to relevant miRNAs missing
from the dataset, the number of observations being too small leading to low
statistical power (unable to capture the often small effect that miRNAs have),
or the biological heterogeneity of breast cancer.
% a gene truly not having
% miRNA regulators in breast tissue, insufficient data (the true regulators of a
% given gene might be missing from the data)
The proposed regression model could also be inadequate for capturing the actual	
biological effect of miRNAs, though previous research seems to suggest otherwise.


% Secondly, the marginal posterior of a single predictor can
% indicate non-significance, while the joint posterior of several predictors
% combined might still achieve significance. This would indicate that the effect
% of miRNAs is only significant when acting simultaneously, a hypothesis that is
% supported by experimental evidence, as discussed in section
% \ref{microrna-function}.

Compared to lasso regression PPVS achieved better model fit, yet from a target
prediction perspective, performance of the two methods was similar. There
was little overlap of predictions made by the two methods, a common issue in
microRNA target prediction \citep{vanIterson2013}. Only a small fraction of
predicted targets were validated according to TarBase and miRTarBase, however,
this is probably true of all miRNA targets in general; only a
limited number of validation studies have been published. The computational cost
of PPVS was significantly higher than that of lasso: MCMC simulations took hours
compared to less than a minute per model for lasso.

Approximately half of the regression coefficients for miRNAs were positive,
suggesting that those miRNAs increase gene expression. Some of these
could indicate indirect regulation. However, this proportion seems too
high, as the vast majority of known microRNA interactions are suppressive. In
fact, of all the experimentally validated human miRNA targets listed in
TarBase, only approximately 0.2\% show positive regulation by the miRNA.
Therefore, many of the predicted activating interactions are possibly false
findings. They could be caused by miRNA expression mirroring the involvement of other
regulatory factors not included in the data. To correct for this, the model could
easily be restricted to only negative interactions (using a non-positive prior
for $\beta$). This has previously been reported to increase prediction
performance \citep{Muniategui2013}.

Previous studies have shown that microRNA signatures correlate
with different breast cancer subtypes \citep{Blenkiron2007}. This suggests that
using pooled datasets of various tumors, such as the data used in this
study, is likely to miss subtype-specific miRNA effects, unless this is accounted
for in the model. This could be achieved in the proposed method by
constructing a hierarchical model that includes tumor-subtype data.

Another way to improve the proposed model would be to include sequence-based
target information, as most published methods do. This could be achieved with
indicator variables, a weighting scheme, or more elaborately by including
sequence-based data within the hierarchical-shrinkage prior to impose less
regularization on putative target pairs. However, as the authors of GenMir
noted, including sequence features did not result in a significant improvement
of their method \citep{Huang2008}.\footnote{It should be noted, that GenMir
uses sequence-based predictions as a preliminary filter step. Therefore,
it is perhaps not surprising that including the same type of data within the
model does not produce substantial improvement.}

% One disadvantage of the proposed method relates to scaling of the regression
% covariates, that is, the mRNA and miRNA expression variables. Sequencing studies have shown
% that a relatively small number of miRNAs accounts for over 80\% of tissue microRNA
% \citep{Landgraf2007}. Therefore, changes in the expression of these highly
% abundant miRNAs may have a relatively large impact on protein levels, where as
% similar changes in less expressed miRNAs may have little to no effect on target gene
% expression. This difference is lost by scaling all miRNA variables to a similar scale.

% However, many of the most abundant miRNAs are ubiquitously expressed across
% different tissues \citep{Landgraf2007} and, therefore, possibly less
% interesting with regards to disease pathogenesis. Additionally, inclusion of
% the gene expression measurement in the regression necessitates some form of
% scaling, as mRNA and miRNA expression profiles are measured with different platforms
% and processed using different algorithms and, therefore, are not directly
% comparable. One could easily envision a scaling procedure where the relative
% levels of different miRNAs are preserved and gene expression is scaled
% relative to mean miRNA expression, for example, but it is debatable whether
% this would be appropriate either.

The proposed model does not account for the fact that microRNAs have
several, even hundreds, of target transcripts \citep{Friedman2009}. Therefore, the
regulatory effect of a single miRNA is most likely spread across several
genes. In combination with transcripts having several regulating miRNAs, this
many-to-many nature of microRNA regulation ultimately calls for computational
methods that model the whole regulatory network at once, such as regression
models with multivariate targets. This, however, becomes a much more difficult
problem than multivariate linear regression.

Aure et al \citep{Aure2015} used lasso regression for a similar analysis of the same dataset.
%prediction of protein expression from mRNA and miRNA expression 
%to identify miRNAs significantly affecting protein expression in breast cancer
They used a multi-step process, where only miRNAs deemed
significant in a univariate regression model were used as input in multivariate
lasso regression. This approach is flawed in the sense, that it loses some of
the power of multivariate models to identify singly weak but combinatorially
strong effects, as univariate modeling is used as a filtering step. It also
effectively uses the same data twice, causing bias, and introduces a
multiple hypothesis testing problem. Therefore, a multivariate
approach (such as the one presented here or earlier ones with slight modification)
would likely be preferable.

In conclusion, the work in this thesis shows that the proposed method of
projection predictive variable selection is applicable to microRNA target
prediction. However, further refinements to the model are warranted to improve
performance. In the presented form, compared to a simpler alternative, the
method offered only a limited advantage from a modeling perspective, and no
apparent advantage from a biological perspective, but incurred a large
computational burden. The choice of parameters $\alpha$ and $\gamma$, which
define the threshold for model size, proved nontrivial. The values chosen
had a large impact on the sizes of resulting models and, therefore,
a data-driven approach for optimizing the parameter values would perhaps be useful.




\subsection*{Future prospects}

The recent development of CLIP-seq and similar methods has made high-throughput
experimental microRNA target discovery possible, partially
replacing the need for computational target prediction. Nonetheless,
experimental (particularly high-throughput) methods are not immune to error,
and gene regulation is vastly complex with many unconventional regulatory
mechanisms having been discovered. Integrative computational approaches
beyond correlation -- combining several types of data -- will, thus,
remain important in the future. Possibilities for integrating data include
incorporating copy number variation, other regulatory RNAs, transcription factors,
other protein-level regulatory factors such as phosphorylation, and
epigenetic mechanisms into models.

The elucidation of complex regulatory networks using network-level modeling is
becoming feasible with modern experimental and computational methods.
Employing this approach will be essential, as it has the ability to better
capture the true nature of gene regulation and cellular biology.

Many aspects of microRNA biology and function still remain unknown. Uncovering
miRNA function offers interesting possibilities in diagnostics and treatment
of disease, and will further our understanding of the complexities of
molecular cell biology. Therefore, microRNAs remain an exciting avenue
of research.
























\clearpage
%% References
%% The \phantomsection command is nessesary for hyperref to jump to the 
%% correct page, in other words it puts a hyper marker on the page.
%% MITES TAA TEHAAN BIBTEXILLA???

\phantomsection
\addcontentsline{toc}{section}{\refname}
%\addcontentsline{toc}{section}{References}

\bibliographystyle{plainnat}
\bibliography{dippa}
%\begin{thebibliography}{99}
%% Alla pilkun j\"alkeen on pakotettu oikea v\"ali \<v\"alily\"onti>-merkeill\"a.
%\bibitem{Kauranen} Kauranen,\ I., Mustakallio,\ M. ja Palmgren,\ V.
%  \textit{Tutkimusraportin kirjoittamisen opas opinn\"aytety\"on
%    tekij\"oille.}  Espoo, Teknillinen korkeakoulu, 2006.
%\end{thebibliography}











%% Appendices
%\clearpage
%\thesisappendix

%\section{Esimerkki liitteest\"a\label{LiiteA}}


\end{document}
