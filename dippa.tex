%% Master's thesis for Aalto University
%% Author: Viljami Aittomaki
%%
%% Written on official LaTeX template:
%% Original version by Luis Costa, changes by Perttu Puska


\documentclass[english,12pt,a4paper,pdftex,elec,utf8]{aaltothesis}


%%%% PAKCAGES %%%%

\usepackage{graphicx}
%% Use this if you write hard core mathematics, these are usually needed
\usepackage{amsmath,amsfonts,amssymb,amsbsy}

%% Use the macros in this package to change how the hyperref package below 
%% typesets its hypertext -- hyperlink colour, font, etc. See the package
%% documentation. It also defines the \url macro, so use the package when 
%% not using the hyperref package.
%\usepackage{url}

%% Use this if you want to get links and nice output. Works well with pdflatex.
\usepackage{hyperref}
\hypersetup{pdfpagemode=UseNone, pdfstartview=FitH,
  colorlinks=true,urlcolor=red,linkcolor=black,citecolor=black,
  pdftitle={MicroRNA regulation in breast cancer},pdfauthor={Viljami Aittom\"aki},
  pdfkeywords={Bayesian analysis, breast cancer, microRNA}}

%% User-added packages
\usepackage[numbers,square]{natbib}
\usepackage[parfill]{parskip}
\usepackage{longtable}
\usepackage[font={small,sf},labelfont={bf},format=plain,indention=.5cm]{caption}
\usepackage{subcaption}
\captionsetup[sub]{font={scriptsize,sf},labelfont={sf}}
\usepackage{pdflscape}
\setlength\LTcapwidth{\textwidth}
\setlength{\LTleft}{-20cm plus -1fill}
%\setlength{\LTleft}{-30cm}
\setlength{\LTright}{\LTleft}
\usepackage{colortbl}

%% Which sections to include
%\includeonly{introduction}







%%=========================================================
%% All that is printed on paper starts here
\begin{document}
\renewcommand{\thesissupervisorname}{Thesis supervisor}

%% Change the school field to specify your school if the automatically 
%% set name is wrong
% \university{aalto-yliopisto}
% \university{aalto University}
% \school{Sähkötekniikan korkeakoulu}
% \school{School of Electrical Engineering}

%% ONLY FOR M.Sc. AND LICENTIATE THESIS: Specify your department,
%% professorship and professorship code. 
%%
\department{Department of Computer Science}
\professorship{--}
%%

%% Valitse yksi näistä kolmesta
%%
%% Choose one of these:
%\univdegree{BSc}
\univdegree{MSc}
%\univdegree{Lic}

%% Your own name (should be self explanatory...)
\author{Viljami Aittom\"aki}

%% Your thesis title comes here and again before a possible abstract in
%% Finnish or Swedish . If the title is very long and latex does an
%% unsatisfactory job of breaking the lines, you will have to force a
%% linebreak with the \\ control character. 
%% Do not hyphenate titles.
%% 
\thesistitle{MicroRNA regulation in breast cancer -- \\a Bayesian analysis of expression data}

\place{Espoo,}
\date{7 October 2016}

%% Thesis supervisor 
\supervisor{Professor Aki Vehtari}

%% Thesis advisors(s).
\advisor{Senior Researcher Rainer Lehtonen}

%% Aalto logo: syntax:
%% \uselogo{aaltoRed|aaltoBlue|aaltoYellow|aaltoGray|aaltoGrayScale}{?|!|''}
%% Logo language is set to be the same as the document language.
%%
\uselogo{aaltoRed}{?}


%% Create the coverpage
\makecoverpage














%%=========================================================
%% English abstract.
%% All the information required in the abstract (your name, thesis title, etc.)
%% is used as specified above.
%% Specify keywords
%%
\keywords{Bayesian analysis, breast cancer, gene expression, microarray, microRNA, target prediction}
%% Abstract text
\begin{abstractpage}[english]

MicroRNAs are a class of small, non-coding RNAs, which regulate gene
expression post-transcriptionally. They downregulate genes by targeting
messenger RNA transcripts and causing their degradation and inhibition of
translation. Research has revealed microRNAs to participate in diverse
cellular functions, such as differentiation and apoptosis, and many
pathological processes, including cancer. \\

Identification of microRNA target genes is crucial in understanding their
function in cell biology and disease. A wide range of methods have been
proposed for computational prediction of microRNA targets. Early target
prediction methods used sequence information, while recent tools have
integrated expression measurements of target genes and microRNAs. A limited
number of studies have integrated protein, gene and microRNA
expression for target prediction. \\

Breast cancer is the most common cancer in women and a significant cause of
morbidity and mortality globally. Analyses of gene expression data have
provided insight into the pathogenesis of breast cancer, and intrinsic
subtypes correlating with prognosis have been identified. A range of microRNAs
have been indicated to contribute to breast cancer pathogenesis. \\

% provides a review of microRNA biology and target prediction.
In this thesis, a
recent Bayesian variable selection method was applied for uncovering putative
microRNA targets in breast cancer. The proposed model integrated
protein, gene and microRNA expression data. Results
were compared with another popular prediction method.
%and a recently published analysis of the same data.
Analyses showed that the proposed method is applicable to microRNA target
prediction. Limitations and refinements of the method and study are discussed,
and the importance of an integrative approach is highlighted.

\end{abstractpage}

%% Force a new page so that the Finnish abstract starts on a new page
\newpage

%% Abstract in Finnish.  Delete if you don't need it. 
\thesistitle{MikroRNA-säätely rintasyövässä -- ekspressiodatan bayesilainen analyysi}
\supervisor{Professori Aki Vehtari}
\advisor{Dosentti Rainer Lehtonen}
\degreeprogram{Electronics and electrical engineering}
\department{Tietotekniikan laitos}
\professorship{--}
%% Avainsanat
\keywords{bayesilainen analyysi, geeniekspressio, kohdegeenien ennustaminen, mikroRNA, mikrosiru, rintasy\"op\"a}
%% Tiivistelman tekstiosa
\begin{abstractpage}[finnish]

MikroRNA:t ovat lyhyitä RNA-molekyylejä, jotka säätelevät geeniekspressiota
sitoutumalla lähetti-RNA-molekyyleihin estäen siten niiden translaation
proteiiniksi. Aiemmat tutkimukset ovat osoittaneet, että mikroRNA:t
osallistuvat monipuolisesti solujen toiminnan säätelyyn, kuten erilaistumiseen ja
apoptoosiin, ja ovat osallisena monien tautien, kuten syövän synnyssä. \\

MikroRNA:n säätelemien kohdegeenien tunnistaminen on olennainen
askel mikroRNA:n toiminnan ymmärtämisessä. Kohdegeenien
ennustamiseen on kehitetty lukuisia laskennallisia menetelmiä. Varhaiset
menetelmät perustuivat RNA-sekvenssien vertailuun. Uudemmat työkalut
yhdistävät geeni- ja mikroRNA-ekspressiodataa kohdegeenien tunnistamiseksi.
Proteiini-, geeni- ja mikroRNA-ekspressiota yhdistäviä
kohdegeenien tunnistamiseen tähtääviä tutkimuksia on julkaistu toistaiseksi
suhteellisen vähän. \\

Rintasyöpä on naisten yleisin syöpä ja merkittävä sairastavuuden ja
kuolleisuuden aiheuttaja maailmanlaajuisesti. Geeniekspressiodatan analysointi
on lisännyt tietoa rintasyövän synnystä, ja geeniekspressioon perustuen on
kyetty tunnistamaan rintasyövän alatyyppejä, jotka korreloivat syövän
ennusteeseen. MikroRNA:n on todettu olevan osatekijä rintasyövän synnyssä. \\

Tässä diplomityössä sovellettiin äskettäin julkaistua bayesilaista
muuttujavalintamenetelmää mikroRNA-molekyylien kohdegeenien ennustamiseen
rintasyövässä. Tähän tarkoitukseen käytettiin proteiini-, geeni- ja mikroRNA-
ekspressiodataa. Tulokset osoittivat, että menetelmä soveltuu kohdegeenien
ennustamiseen. Työssä esitetään vaihtoehtoja mallin jatkokehittämiseksi.

\end{abstractpage}














%%=========================================================
%% Preface

%% Force new page so that the preface starts from a new page
\newpage

\mysection{Preface}

This Master's Thesis and the related research were done in the Systems Biology
of Drug Resistance in Cancer research group within the Genome-Scale Biology
research program of the University of Helsinki. I am grateful to Professor
Sampsa Hautaniemi for suggesting this work and providing facilities and also
for several years of interesting work in his group.

I want to thank Professor Aki Vehtari for supervision and suggesting the
method proposed in this work as well as for advice in implementing it. I would
like to thank my advisors Professor Sampsa Hautaniemi and Senior Researcher
Rainer Lehtonen for their guidance and trust and the time they have invested
into this work. I want to thank Juho Piironen for help with the modeling. I
would also like to thank Professor Antti Vaheri for discussions.

I owe a lifetime of gratitude to my parents for their endless love and
encouragement. Even in the darkest of moments, they have never stopped
believing in me. I also want to thank my mother for discussions relating
to genetics.

Finally, I wish to send a thank you and all of my love to all my family and
friends for their support during the writing of this thesis and in other
endeavors in my life. Special thanks to my friend Heidi for pushing me forward
and sharing cookies. \\

\vspace{5cm}
Espoo, 7 October 2016

\vspace{5mm}
{\hfill Viljami Aittom\"aki \hspace{1cm}}


%% Force new page after preface
\newpage


%% Table of contents. 
\thesistableofcontents













%%=========================================================
%% Symbols and abbreviations
\mysection{Symbols and abbreviations}

\subsection*{Symbols}

\begin{tabular}{ll}
$\beta$                    & Regression coefficients for explanatory variables \\
$\left\| \beta \right\|_1$ & The 1-norm of $\beta$ \\
$\textup{E}_{\theta}(x)$   & Expectation of random variable $x$ over parameters $\theta$ \\
$n$                        & Number of observations \\
$\textup{N}(\mu,\sigma^2)$ & Multivariate normal distribution with mean $\mu$ and variance $\sigma^2$ \\
$p_n$                      & Number of (assumed) true explanatory variables in variable seĺection \\
$p(y)$                     & Probability density of $x$ \\
$p(y,\theta)$              & Joint probability of $y$ and $\theta$ \\
$p(y|\theta)$              & Conditional probability of $y$ given $\theta$ \\
$\theta$                   & Parameters of a probability model \\
$x$                        & Explanatory variable (or microRNA expression vector) \\
$X$                        & Matrix of explanatory variables (or microRNA expression vectors) \\
$y$                        & Outcome variable (or protein expression vector) \\
$y \sim \textup{N}(.)$     & Random variable $y$ has probability distribution $N(.)$ \\
$y \propto x$              & $y$ is proportional to $x$, up to a constant factor \\
$z$                        & mRNA expression vector
\end{tabular}

% \subsection*{Expressions}
% \begin{tabular}{ll}
% $\nabla \times \mathbf{A}$                 & curl of vectorin $\mathbf{A}$\\
% $\displaystyle\frac{\mbox{d}}{\mbox{d} t}$ & derivative with respect to variable $t$\\[3mm]
% $\displaystyle\frac{\partial}{\partial t}$ & partial derivative with respect to variable $t$ \\[3mm]
% $\sum_i $                      & sum over index $i$\\
% $\mathbf{A} \cdot \mathbf{B}$  & dot product of vectors $\mathbf{A}$ and $\mathbf{B}$
% \end{tabular}

\subsection*{Abbreviations}

\begin{tabular}{lp{11cm}}
bp          & Base pairs (as a measure of double-stranded sequence length) \\
HMM         & Hidden Markov model \\
lasso       & Least absolute shrinkage and selection operator \\
LPD         & log predictive density \\
MCMC        & Markov chain Monte Carlo \\
miRNA       & MicroRNA \\
MLPD        & Mean log predictive density \\
MLR         & Multivariate linear regression \\
mRNA        & Messenger RNA \\
NGS         & Next-generation sequencing \\
nt          & Nucleotides (as a measure of sequence length) \\
PCR         & Polymerase chain reaction \\
PPVS        & Projection predictive variable selection \\
qPCR        & Quantitative PCR \\
RISC        & RNA-induced silencing complex \\
RNAi        & RNA interference \\
RPPA        & Reverse-phase protein array \\
SVM         & Support vector machine \\
UTR         & Untranslated region at the beginning (5') or end (3') of a messenger RNA \\
WHO         & World Health Organization \\
\end{tabular}


%% Tweaks the page numbering to meet the requirement of the thesis format:
%% Begin the pagenumbering in Arabian numerals (and leave the first page
%% of the text body empty, see \thispagestyle{empty} below).
%% Additionally, force the actual text to begin on a new page with the 
%% \clearpage command.
%% \clearpage is similar to \newpage, but it also flushes the floats (figures
%% and tables).
%% There is no need to change these
%%
\cleardoublepage
\storeinipagenumber
\pagenumbering{arabic}
\setcounter{page}{1}








%%=========================================================
%% Text body begins


\section{Introduction}
\thispagestyle{empty}

Breast cancer is the most common of female cancers and causes remarkable
morbidity and mortality word-wide \citep{??}.  Annually more than 1.5 million
women develop breast cancer. Thus, breast cancer is a major global health
problem.

Cancer is a genetic disease caused by mutations in the genome of the cancer
cells. Some of these mutations can be inherited, while some are somatic, that
is, mutations that arise during the life-time of an individual in the tissue,
where cancer develops. To understand how cancer develops, the tumor evolution
of breast cancer, it is of paramount importance to identify genes in which
mutations initiate breast cancer development. It is already known that many of
these mutations perturb the expression of the gene in question. Previous
expression studies comparing the expression profiles of normal breast tissue
and breast cancer tissue have revealed ...

MicroRNA control of gene expression .... MicroRNAs have been indicated as
potential causative agents in numerous diseases, including several types of
cancer. Therefore, the study of microRNAs and their function in cancer can
offer insights into tumorigenesis and cancer progression as well as potential
new treatments.

Several different methods for computational identification of microRNA target
genes have been proposed. Early methods are mainly based on sequence similarity of the
microRNA and messenger RNA of putative target genes. More recent methods use
expression profiles to either augment sequence data or as sole predictors for
interactions. These are mostly based on correlation, or a derivative thereof,
of microRNA and gene expression. The combinatorial nature of miRNA action, however,
makes expression based strategies difficult, as one transcript may be regulated
by several different miRNAs simultaneously and the contribution of each
miRNA may be small. Additionally, most miRNAs have several target transcripts.

The aim of this thesis was to apply a recently proposed Bayesian variable
selection method to microRNA target discovery in breast cancer. The variable
selection was applied in the context of Bayesian regression of expression
profiles to elucidate which microRNAs are highly relevant in regulating
protein expression levels. The results were compared with a similar recent
study and also with known validated microRNA targets as well as microRNAs
previously indicated in breast cancer.


%\clearpage (\include implicity does a \clearpage)
%!TEX root = dippa.tex
%%% This file contains the Gene expression section of my master's thesis.
%%% This section covers the basics of gene expression.
%%% Author: Viljami Aittomäki

\section{Gene expression}\label{gene-expression}

Genetic information is encoded in molecules of deoxyribonucleic acid (DNA). A
gene is a section of DNA that serves as a template for a functional
ribonucleic acid (RNA) molecule. Gene expression refers to this process of
synthesizing a functional end-product from the information contained in gene.
DNA and gene expression serve as the basis of all currently known life.

Most of gene expression is dedicated to making proteins. The Central Dogma of
Molecular Biology, postulated by Francis Crick in 1970, describes the general
schema of how genetic information flows from genes to proteins; DNA is first
transcribed into messenger RNA (mRNA), which is then translated into a polypeptides,
which ultimately form proteins \citep{Crick1970}. The flow is not strictly
one-directional though, as there exist reverse transcriptases, enzymes that
synthesize DNA from an RNA template.

All genes are read to form RNA, but not all encode a protein product. The
human genome\footnote{The genome refers to the whole genetic material of an
organism or an individual.} has been suggested to contain approximately 20 500
protein-coding genes, which encompass only around 1.5\% of the whole genetic
sequence \citep{Clamp2007}. Therefore, the vast majority of the human genome
was previously thought to be without function and referred to as "junk DNA".
More recently, however, it has become increasingly evident, that noncoding
portions of the genome are often functional and possibly have important
cellular functions \citep{ENCODE}. Noncoding genes give rise to noncoding RNA,
a class of RNA molecules that are mostly involved in aiding and regulating the
expression of other genes. Different types of noncoding RNA and their
functions are summarized in Table \ref{table:rnas}.




\subsection{Regulation of gene expression}\label{regulation-of-gene-expression}

Regulation of gene expression refers to controlling the abundance of
the gene end-product a cell produces. This control is paramount so that cells
can respond to external signals, changes in their environment, damage and also 
\textbf{pass through} different developmental stages. Gene expression can be regulated
at any stage of the process and regulation can be divided into transcriptional
and post-transcriptional regulation.

Transcriptional regulation with e.g. TFs, methylation, adenylation, splicing.

Post-transcriptional

MicroRNAs (miRNAs) and small interfering RNAs (siRNAs) are components of the
so called RNA interference (RNAi) pathway, which is a mechanism for regulating
gene expression post-transcriptionally. miRNAs and siRNAs are practically
interchangeable as substrates for RNAi, both act as target mRNA recognizing
templates, but have different biogenesis, where miRNAs are cut from endogenous hairpin
structures and siRNAs are processed from exogenous long double-stranded RNAs (dsRNAs).
\cite{Du2005} MicroRNAs are the focus of this study and are discussed in more
detail in the next section.

The Argonaute (Ago) family of proteins is closely associated with small RNAs and
Ago act as effectors in RNAi \cite{Ha2014}.

\begin{itemize}
  \item transcription factors
  \item post-transcriptional regulation
  \item translational factors
  \item protein degradation
\end{itemize}




\subsection{Measuring gene expression}\label{measurement-of-gene-expression}

The physical measurement of gene expression can be done on either the level of
messenger RNA molecules or protein molecules present in cells. Although
proteins are the eventual effectors molecules within cells -- at least for
protein-coding genes -- often gene expression is synonymous with mRNA
expression. This is because measuring mRNA abundances is significantly easier than
measuring protein abundances due to the chemistry of base-pair formation
and the relative ease of replicating DNA (or RNA) sequences by
exploiting cellular machinery evolved for this purpose.

Techniques shortly in one paragraph:
\begin{itemize}
  \item blotting, qPCR
  \item microarrays
  \item sequencing-based methods
  \item protein arrays (RPPA)
\end{itemize}




\subsubsection{Microarrays}

\begin{itemize}
  \item basics of operation
  \item shortly on microarray analysis
  \begin{itemize}
    \item preprocessing and normalization
    \item probe annotation problems
  \end{itemize}
\end{itemize}


The general assumption has been that mRNA expression is representative of gene
expression and that changes in mRNA abundances also reflect changes in protein
abundances and, therefore, cellular processes. This assumption has recently
been challenged by experiments indicating that the expression of mRNA and
corresponding protein often correlate poorly, with mRNA expression
explaining around 40\% of variation in protein abundances \citep{Vogel2012}.
There are also contrary findings of modest to good correlation
and one study suggested that mRNA-protein correlation is generally higher for
genes that have differing mRNA expression between studied conditions
(e.g. cancerous versus healthy tissue) \citep{Koussounadis2015}.

Nonetheless, Payne recently concluded that "proteome and transcriptome abundances are not
sufficiently correlated to act as proxies for each other" and that
most of this difference is likely caused by biological regulation and not
by measurement technology \cite{Payne2015}.
% This regulation can be post-transcriptional, translational
% or protein-degradation related, as discussed above.
Therefore, it is interesting, even necessary, to integrate
measurements from different stages of gene expression -- for example
mRNA, microRNA and protein abundances -- to gain new insights
into biological processes.


%!TEX root = dippa.tex
%%% This file contains the MicroRNAs section of my master's thesis.
%%% This section covers the biological background of miRNAs.
%%% Author: Viljami Aittomäki

\section{MicroRNAs}\label{micrornas}

MicroRNAs (miRNAs) are a family of endogenous (i.e. coming from within the
cell itself) noncoding small RNA molecules that function as post-transcriptional
regulators of gene expression \citep{Ambros2004}. In their
functional, mature form miRNAs are single stranded and approximately 22
nucleotides long. MicroRNAs are not translated into protein. Instead, they
have an important role in regulation of gene expression in a wide range of
physiological, developmental and pathological processes \citep{Bartel2009}.
MicroRNAs assert their regulatory function by destabilization and degradation
of target messenger RNA (mRNA) molecules and inhibition of mRNA translation
\citep{Fabian2010}.




\subsection{Discovery of microRNAs}

The first known microRNA, lin-4, was discovered in 1993 by two research groups
studying the larval development of the nematode \emph{Caenorhabditis elegans}.
The researchers noted that lin-4 does not encode a protein, but instead
produces a pair of small RNAs, the longer of which was proposed to be a
precursor to the shorter one \citep{Lee1993}. The RNAs encoded by lin-4 were
noted to have conserved antisense complementarity in several sites of an
untranslated region of the lin-14 mRNA, and these sites were found to be
necessary for the normal repression of lin-14 expression by lin-4
\citep{Lee1993,Wightman1993}.
%It
%should be noted, that one of these groups also showed that lin-4 reduces the
%amount of the LIN-14 protein -- the end-product of the lin-14 gene -- without
%significantly affecting the cellular concentration of the lin-14 mRNA
%\citep{??}. %oks tää Wightman1993:ssa?
%We will return to the issue of microRNA action in chapter \ref{microrna-function}.

Let-7, the second microRNA to be discovered, was also first found in \emph{C.
elegans}, however, homologues of let-7 were found in several other species
\citep{Pasquinelli2000}. Soon after, numerous microRNA genes were found across
a variety of species, and a registry was set up to serve as a comprehensive
knowledge base of published microRNAs and as an independent authority on
microRNA nomenclature \citep{GriffithsJones2004}. This registry later became
miRBase, the de facto reference database of known microRNAs, and now provides
sequence data, annotations as well as predicted and validated target genes for
miRNAs \citep{Kozomara2014}.

The number of known small RNAs has since vastly expanded
and microRNAs have been found in more than 200 organisms, including
all studied animals, plants \citep{JonesRhoades2006} and viruses \citep{Grundhoff2011}. 
The number of records in miRBase has risen exponentially
from %218 precursor and
218 mature miRNAs in the first release in 2002 to %28645 precursor and
35828 mature miRNAs in 223 species in the most recent version (v21, released June
2014 \citep{VanPeer2014,MiRBaseWeb}), illustrating the vast amount of novel
microRNA molecules discovered recently, mainly due to increasing efforts in
and availability of sequencing. miRBase lists 2588 known human miRNAs at the
time of writing this thesis. A web service called miRBaseTracker has been
developed by \citet{VanPeer2014} for updating miRNA nomenclature and
annotations across different versions of miRBase to allow correct comparison
of miRNA study results and reannotation of miRNA analysis platforms.




\subsection{MicroRNA genomics}\label{microrna-genomics}

MicroRNAs are highly conserved in evolution \citep{Bartel2004}, for example,
55\% of \emph{C. elegans} miRNAs have homologues in humans
\citep{IbanezVentoso2008}. Interestingly, the
appearance of multi-cellular organisms appears to correlate with the
appearance of the microRNA machinery, and organism complexity and speciation
are correlated with miRNA complexity, suggesting that microRNAs have had a
crucial role in the development of complex organisms \citep{Lee2007}.

MicroRNAs are found in varying genomic contexts. Approximately 50\% of
mammalian miRNAs are located in close proximity to other miRNAs and form
polycistronic miRNA clusters that are transcribed simultaneously. Some miRNAs
reside in the genome as dedicated miRNA genes, with their own promotor regions.
\citep{Kim2009} MicroRNAs and miRNA clusters can be situated in exons or
introns of nonconding genes and some are found in introns of protein-coding genes
\citep{Du2005}. MicroRNAs located in introns are sometimes referred to as
mirtrons \citep{Ruby2007}.

MicroRNAs are expressed in all tissues, however, different tissues
have differing miRNA expression profiles \citep{Krol2010}. Many microRNAs also
have differing expression in different developmental stages
of some organisms, e.g. let-7 functions to control the transition
from the second larval stage to \textbf{WHAT} in \emph{C. elegans} \citep{Bartel2004}.

% Tight evolutionary control, extensive transcriptome
% targeting, and the fact that miRNAs and their associated proteins are one of
% the most abundant molecules in the cytoplasm \citep{Bartel2004} highlight the
% importance of microRNAs.




\subsection{MicroRNA biogenesis}\label{microrna-biogenesis}

The canonical pathway of microRNA biogenesis is illustrated in Figure
\ref{fig:mirna-biogenesis} and is presented here as reviewed by \citet{Bartel2004},
\citet{Melo2011}, \citet{Ha2014}, and many others. 

Most microRNAs are transcribed from genomic DNA by RNA polymerase II to form a
long primary microRNA (pri-miRNA) molecule \citep{Lee2004}. The pri-miRNA molecule
contains a hairpin structure, with a 33-bp double-helix stem and a terminal
loop, and flanking single-strand sequences, which are several hundreds or
thousands of nucleotides long \citep{Kim2005}. Some miRNAs within Alu repeat elements
can be transcribed by RNA polymerase III \citep{Borchert2006}.

The pri-miRNA is cut by the ribonuclease Drosha to form %one or, in the case of polycistronic miRNAs, several
a pre-microRNA (pre-miRNA), which consists of the hairpin and is
approximately 70 nt long \citep{Lee2003}. A typical pre-miRNA structure is
shown in Figure \ref{fig:premirna-structure}. Drosha is aided by the essential
cofactor DGRC8 (the protein product of a gene deleted in DiGeorge syndrome \citep{Shiohama2003})
and they form a complex known as the Microprocessor \citep{Gregory2004}.
The hairpin is
then exported from the nucleus to the cytoplasm by exportin 5 (XPO5), which is
a member of the nuclear transport receptor family \citep{Lund2004}.

In the cytoplasm, the ribonuclease Dicer cleaves out the loop of the hairpin
to form a 22-nt-long double-stranded miRNA:miRNA* duplex corresponding to
the stem of the hairpin \citep{Bernstein2001}.
Dicer associates with a cofactor, in humans TRBP (Tar RNA-binding protein),
which is not required for effective dicing of the pre-miRNA,
but acts to physically bridge the Dicer to an Argonaute protein
% for further miRNA processing
\citep{Chendrimada2005}.

The duplex is then bound by the Argonaute protein, in mammals one of Ago1
through Ago4, forming what is called the RNA-induced silencing complex (RISC).
The RISC is a protein complex containing Dicer, TRBP and Ago \citep{Gregory2005}.
Ago, aided by Dicer and TRBP, unwinds the strands of the duplex and retains one of
them. The retained strand is known as the guide strand (or miRNA). The other
strand, called the passenger (or miRNA*), is released and typically degraded \citep{Du2005}.
In some instances, either one of the strands can become the guide
or both can be used \citep{Czech2009}.
%Notably, the Dicer cleaving, duplex unwinding and eventual
%mRNA regulation activity are, in fact, all coupled and performed by RISC
%\citep{Gregory2005}.

Not all miRNAs are generated through this canonical pathway of microRNA
biogenesis. Some miRNAs are not dependent on Drosha, such as mirtrons, which
are cut into pre-miRNA by the spliceosome, a molecular complex responsible for
removing introns (and sometimes exons) from precursor mRNA \citep{Ruby2007}. The
biogenesis of miR-451 is independent of Dicer; miR-451, which
has an important role in erythropoiesis, is cleaved by Ago2 \citep{Cheloufi2010}.




\subsection{MicroRNA mechanism of action}\label{microrna-mechanism}

The RISC is the effector of RNA interference. Ago functions as the catalytic
engine of the RISC and the miRNA bound to it guides the RISC to target
messenger RNAs \citep{Filipowicz2008}. The microRNA mechanism of action is
illustrated in Figure \ref{fig:mirna-action}.

Target recognition is based on sequence complementarity of the miRNA and mRNA.
In animal miRNAs this complementarity is almost always limited \citep{Ambros2004}.
Nucleotides at positions 2-8 of the 5' end of the microRNA have been found
crucial to target mRNA matching; these nucleotides are termed the miRNA "seed sequence".
miRNA target sequences are mostly located in the 3' UTR (untranslated region)
of the mRNA transcript, but in some instances target sites also reside in the
coding region or 5' UTR of the mRNA \citep{Bartel2009}.
\textbf{TÄÄ KAPPALE PAREMMIN, MUITA FIITSUJA MUKAAN, KOSKA NIITÄ KÄYTETÄÄN PREDICTIOSSA, KTS GRIMSON.)}

The action of microRNAs is \textbf{related} through inhibition of mRNA
translation or destabilization and subsequent degradation of mRNA. The exact
mechanisms by which the miRNA and Ago induce translational repression or
destabilization of mRNA are unclear \citep{Filipowicz2008}. Translational
inhibition was earlier believed to be the major form of miRNA action in animals,
but recent evidence suggests that mRNA destabilization dominates \citep{Guo2010}.
In some cases the mRNA is directly cleaved by Ago. Ago2 is the only mammalian
Argonaute capable of cleavage, and this is assumed to require
extensive base-pair matching between the miRNA and mRNA \citep{Du2005}.
However, mRNA cleavage appears to be rare in animals (while much more common
in plants).

mRNAs bound to RISC accumulate in so called processing bodies (P-bodies),
which are known sites of mRNA catabolism and translational repression in the
cytoplasm. The localization in P-bodies, however, appears to be a consequence
of RNA silencing, not the cause, and is reversible  \citep{Eulalio2007}.

Interestingly, several alternative mechanisms of action for microRNA have been
reported. For example, some miRNAs can increase the translation of target mRNA instead of
repressing it \citep{Vasudevan2007}, miR-373
was found to target DNA promoter areas and act to induce gene transcription
\citep{Place2008}, and miR-328 targets a protein to prevent inhibition of mRNA
translation \citep{Eiring2010}. This illustrates the complexity and diversity of
microRNA biology and gene regulation in general.




\subsection{MicroRNA function}\label{microrna-function}

SIIRRÄ TÄMÄ AIEMMAKSI?

MicroRNAs have been found to participate in regulation of almost all studied
cellular processes, including embryo development, cell proliferation,
apoptosis and metabolism, by regulating protein expression \cite{}.

miRNAs assert extensive control over the transcriptome. More than 60\% of
human mRNA transcripts are predicted to be regulated by miRNAs and a single
transcript has target sites for several different miRNA families
\citep{Friedman2009}. Furthermore, a single microRNA can have as many as hundreds or
thousands of target mRNAs. The effect of a
single miRNA on its target tends to be subtle, usually causing less than a
2-fold change in protein expression \citep{Baek2008}. However, a single
miRNA can have multiple binding sites in one mRNA and multiple miRNAs
acting on the same target have additive, and in some cases even synergistic
effects \citep{Bartel2009}.

% , and depending also on 
% of multiple miRNAs acting in tandem can, however, be much more pronounced and
% even multiplicative, achieving changes in excess of 10-fold \citep{}.

% A recent review concluded that mRNA degradation is the
% predominant form of miRNA action in mammals \citep{Guo2010}.

It should be noted, however, that the functional role and importance of many
miRNA-mRNA interactions are unknown, even for validated interaction pairs.
Uncovering these roles is challenging because of the subtle regulatory effects
that miRNAs often have and, additionally, because of the complexity and
robustness of most cellular regulatory networks \citep{Bartel2009}.
Furthermore, experimentally validated target mRNAs exist only for a subset of
all known microRNAs. Nonetheless, discovering miRNA targets is a critical
step in understanding their function.

The dysregulation of microRNAs is associated with many human diseases
\citep{Jiang2009,VAIHDATÄMÄREFE}. The first example of such an association was, in
fact, found in cancer, when miR-15 and miR-16 were found to be suppressed in
chronic lymphocytic leukemia \citep{Musilova2015}. A disease-promoting role
for miRNAs has since been implicated in many different cancers, including
breast cancer \citep{Melo2011}.




\subsection{Measuring MicroRNA expression}

The same methods that have been employed for measuring mRNA (i.e. gene)
expression are generally applicable for measuring microRNA expression as well,
ranging from northern blotting to microarrays and more recent studies using next
generation sequencing (NGS) \citep{Huang2011}. However, as Hunt and colleagues
in a recent and thorough review point out, there are several challenges in
detecting miRNAs in particular \citep{Hunt2015}.

MicroRNAs are very short and only comprise approximately 0.01\% of RNA
typically extracted from a sample \citep{Dong2013}. This implies that miRNA
detection must be highly sensitive. MicroRNAs from the same family can differ
by only one base, which in turn requires high specificity to be able to
distinguish between family members. On the other hand, variation in miRNA
processing can result in slight sequence variations, or isoforms, of a single
miRNA, also known as isomiRs \citep{StaregaRoslan2011,Lee2010}. This means
high specificity or an incorrect reference sequence (e.g. that of a
weakly-expressed isomiR) used for detection can cause inaccurate measurements.
IsomiRs may also have different functions resulting from altered target
specificity \citep{Chugh2012}. The existence of the pri-miRNA, pre-miRNA and
mature miRNA molecules provides an additional challenge for measurement
methods.
%, although differentiation between these maturation stages is not necessarily required.

Many of the challenges mentioned are solved by NGS, which is sensitive and
reliable in quantifying known microRNAs and identifying novel ones
\citep{Huang2011}. Sequencing can detect variations of even one nucleotide and
does not necessarily depend on previously identified sequences. However, not
all identified short RNAs are functional miRNAs, and NGS conveys its own set
of problems relating to high cost, significant computational complexity and
validation efforts to distinguish relevant data from noise
\citep{Chugh2012,Hunt2015}.

\textbf{TÄHÄN VIELÄ ERI PLATFORMIEN VERTAILUA JA PREPROSESSOINNISTA?}


%!TEX root = dippa.tex
%%% This file contains the Cancer section of my master's thesis.
%%% This section covers the basics of cancer and microRNA involvement.
%%% Author: Viljami Aittomäki

\section{Cancer}\label{cancer}

Cancer is a disease of uncontrolled overgrowth of a population of cells. It is
generally viewed as a genetic disease, albeit it is mostly not inherited, as it is
caused by  mutations in the genome of the tumor. These mutations cause malfunction
and dysregulation of the genetic machinery regulating cellular functions, such
as cell proliferation, differentiation and apoptosis, resulting in
unregulated growth and malignant tumor formation.

There are several classes of genes that influence tumor growth, the two main
categories being oncogenes and tumor suppressors. Oncogenes were first
identified in retroviruses and later shown to be proto-oncogenes, which by
mutation develop into oncogenes whose over activity promotes tumor growth
\citep{Varmus1988}. Tumor-suppressor genes are often regulators of cell
proliferation or other so called housekeeping genes that work to ensure the
proper functioning of cells and the apoptosis of misbehaving ones. The
inactivation of these genes can lead to tumor progression. The existence of
tumor suppressors was first hypothesized by Alfred Knudson, who formed the
“two-hit hypothesis” while studying the epidemiology of retinoblastoma
\citep{Knudson1971}. He suggested that, for cancer to develop, both copies of
a tumor suppressor gene should become inactivated and that in inherited
cancers one mutation is acquired in the germline and the other occurs in
somatic cells, whereas in sporadic cancers both mutations happen in somatic
cells.

The idea of oncogenes and tumor suppressor genes was later expanded on by
Douglas Hanahan and Robert Weinberg in their seminal article The Hallmarks of
Cancer \citep{Hanahan2000}. The hallmarks are a set of six features which
tumors often acquire to become malignant. The features are: sustaining
proliferative signaling, evading growth suppressors, resisting cell death,
enabling replicative immortality, inducing angiogenesis, and activating
invasion and metastasis. Weinberg and Hanahan postulated that at least three
of these six features are required for invasive cancer to develop.

Recently, Hanahan and Weinberg revised the hallmarks with two new upcoming
hallmarks and two enabling characteristics \citep{Hanahan2011}. The new
hallmarks are deregulation of cellular energetics and avoiding immune
destruction. The enabling characteristics of malignant tumors are genome
instability and mutation, and tumor-promoting inflammation through recruitment
of the immune system. Genome instability and mutation is of special importance
as much of cancer and tumor research focuses on identifying mutated or
dysregulated genes that promote tumor progression.




\subsection{Breast cancer}\label{breast-cancer}

Breast cancer constitutes a significant health issue globally. It is the most
common cancer in women and the second most common cancer overall;
approximately 1.7 million women develop breast cancer annually world-wide, and
in 2012 there were 522 000 breast cancer-related deaths
\citep{Ferlay2015}. In Finland there were 5008 new cases of breast cancer
 and 815 breast cancer-related deaths in 2014 \citep{Syoparekisteri}.

Most breast cancers are sporadic; only 5-7\% of breast cancer cases are of
familial type \citep{Melchor2013}. However, 15-30\% of breast cancer patients
have a family member or relative with breast cancer. This is
mostly due to the high frequency of breast cancer in many western populations,
but also suggests that there are unknown genetic factors and environmental
factors that have an impact on breast cancer development. Indeed, breast
cancer is a hormone-related disease and hormonal factors are known to have an
impact on breast cancer risk. The most important of these are
estrogen and progesterone.

The most important hereditary forms of breast cancer are those related to a
tumor predisposition syndrome caused by mutations in the breast cancer 1
(BRCA1) and BRCA2 genes, which explain about 25\% familial breast cancer in
many populations. These, however are rare on the population level and familial
clustering of beast cancer is multifactorial and caused by moderate risk and
low risk genetic variations, which are much more common \citep{Melchor2013}.




\subsubsection{Breast cancer classification}\label{breast-cancer-classification}

TÄTÄ PITÄÄ TIIVISTÄÄ!

The basic classification of cancer is based on the site -- that is, the organ
or tissue, such as breast or colon, where the primary tumor develops. Cancers
occurring within each organ are, however, heterogeneous in their nature and
have different behavior and prognosis. Thus, tumors are secondarily classified
by morphology, the microscopic structure of the cancer tissue.

The morphological classification of breast cancer is based on the WHO
classification from 2003 and includes altogether 19 histological subtypes of
invasive breast cancer \citep{Tavassoli2003,Weigelt2009}. Of these invasive
ductal carcinoma not otherwise specified (IDC NOS), accounts for the by far
largest histological group. Additionally, the WHO classification of
breast cancer includes the
TNM classification; characterization of the primary tumor (T), lymph node
status (N) and distant metastasis (M) and stage grouping based on the TNM
data. cTNM (or TNM) refers to a clinical TNM classification -- based on imaging studies,
surgical exploration and similar studies -- and pTNM to a pathological
classification -- based on postsurgical pathological analysis of tissue
biopsies. All invasive cancers are also graded into well (I), moderately (II)
or poorly (III) differentiated tumors based on microscopical examination
% based on three features; tubule formation as an expression of glandular
% differentiation, nuclear pleomorphism and mitotic counts
with less differentiated tumors having worse prognosis \citep{Tavassoli2003}.

In the clinic many other tumor characteristics are used. These
include the age of the patient, lymphovascular invasion as well as expression of
estrogen (ER) and progesterone receptors (PR), and Her2, which are routinely
studied for breast cancers. Together these data can be used to group patients
into risk categories for prognosis and choice of treatment using for example
St Gallen criteria \citep{Goldhirsch2007} or NIH criteria \citep{Eifel2001} or others.
The estrogen receptor also has a special role in treatment, as tumors that highly express ER
(termed ER-positive and covering 70\% of all cases) can be given antiestrogens
as endocrine therapy.

One of the problems with morphological classification of breast tumors is that
over 50\% do not show any particular features and are classified as IDC NOS,
in spite of the fact that tumors in this large group have very different
clinical outcomes. As mentioned above, tumors of the same tissue -- and even
of the same histological type -- are heterogeneous in nature and personalized
molecular diagnostics are required to exploit more targeted treatments. This
is even more critical in the case of de novo treatment resistance, where
tumors develop a molecular mechanism for resisting pharmacological therapies.
This is especially common with targeted therapies, for example ER-positive
tumors may acquire mutations to ER rendering antiestrogen therapy ineffective
\citep{Oesterreich2013}.

More recently, expression profiling has led to a suggestion
of new classification of breast cancers \citep{Perou2000,Sorlie2001}. By studying 65
samples from breast tumors with expression profiling Perou et al.
distinguished four subgroups based on gene expression, namely
ER+/luminal-like, basal-like, Erb-B2+ and normal breast \citep{Perou2000}.

Expression profiling has also led to development of new prognostic tests to
help to determine the need of adjuvant chemotherapy. These tests include
Oncotype DX (a 21-gene recurrence score), MammaPrint (a 70-gene test) and PAM50
(a 50-gene test), and have been evaluated by several groups and suggested to be
valid and promising, but their utility in clinical decision making remains
unclear \citep{Azim2013}. The PAM50 molecular classification ... 

\textbf{SUBTYPE KORRELAATIO PROGNOOSIIN JA HOITOON PITÄISI KERTOA PAREMMIN?}




\subsection{MicroRNAs and cancer}

Research has shown microRNAs to have important roles in tumor initiation,
progression and metastasis \citep{Lin2015}. MicroRNA expression signatures
correlate with numerous cancer features, such as tissue, staging, 
progression, prognosis and treatment response, and all studied cancers
have had miRNA expression profiles differing from healthy tissue, including breast
\citep{Calin2006}. In fact, microRNAs appear to be globally underexpressed in
cancers \citep{Lu2005}. Therefore, it is clear that microRNAs participate
in many of the pathways resulting in the hallmarks of cancer, and
examples of miRNAs influencing each of the hallmarks have been found.

Similarly to protein-coding genes, microRNAs can function as tumor
suppressors or oncogenes \citep{Lin2015}. In a meta-analysis of dysregulation
of miRNAs in breast cancer van Schoonveld found major oncogenic microRNAs in
breast cancer to include miR-10b, miR-21, miR- 155, miR-373, and miR-520c
\citep{vanSchooneveld2015}. They also list nine miRNAs as major tumor suppressors for breast cancer,
namely miR-125b, miR-205, miR-17-92, miR-206, miR-200, miR-146b, miR-126,
miR-335, and miR-31.

The genetic mechanisms for microRNA involvement in cancer are varied,
including mutations in miRNA or target mRNA sequence, chromosomal
rearrangements of the miRNA-encoding DNA regions and epigenetic changes in DNA
methylation or histones, leading to aberrant miRNA expression
\citep{Calin2006,Melo2011}. For example, a single-nucleotide polymorphism
in the microRNA miR-196a2 has been found to be associated with breast cancer
risk \citep{Gao2011}. A mutation in the sequence of estrogen receptor alpha,
in the target site of miR-453, has been suggested to be associated with a
lower breast cancer risk \citep{Tchatchou2009}, an example of a mutation in
a target transcript affecting miRNA function. MicroRNA function can also be
altered by abnormalities in the miRNA-processing machinery. For instance, a
mutation in the Dicer gene causes a tumor predisposition syndrome known as
DICER1 syndrome \citep{Slade2011}. Another example of this is apparent
dysregulation of Dicer and Drosha in breast cancer \citep{Yan2012}.

The different subtypes of breast cancer, explained above, reflect the genetic
background of the tumor and, accordingly, the subtypes differ in their gene
expression profiles. This also applies for miRNA expression, the different
intrinsic subtypes have different miRNA expression profiles, suggesting their
importance in breast cancer evolution \citep{Blenkiron2007}. de Rinaldis et al
identified a 46-miRNA signature that could be used in differentiating the
intrinsic subtypes from each other \citep{deRinaldis2013}. In addition to
tumor development, many miRNAs have been found to modulate the response to
breast cancer therapies. These include chemotherapy, antiendocrine therapy,
radiotherapy and targeted therapies.

Accordingly, miRNAs have been studied as biomarkers for diagnosing cancer and
cancer prognosis. Emmadi et al recently found let-7 expression to be negatively
correlated with the Oncotype DX recurrence score in breast cancer
\citep{Emmadi2015}. This corroborated with the earlier finding of let-7 being
downregulated in breast cancer stem cells (tumor cells possessing the ability
of self-renewal) \citep{Yu2007} and later research suggesting let-7 to act as
a tumor suppressor. Several miRNAs have also been associated with breast cancer
metastasis \citep{Chen2016}.

MicroRNAs also show promise as a novel therapeutic tool, several studies have
proposed miRNA-based cancer treatments in animal models, including for breast
cancer \citep{VanRooij2014}. However, more research this area is needed before
microRNA treatments are ready for the clinical setting.

% Cell-free biomarker:
% Lawrie C.H., Gal S., Dunlop H.M., Pushkaran B., Liggins A.P., Pulford K., Banham A.H., Pezzella F., Boultwood J., Wainscoat J.S., et al. Detection of elevated levels of tumour-associated microRNAs in serum of patients with diffuse large B-cell lymphoma. Br. J. Haematol. 2008;141:672-675.

% Treatments:
% Van Rooij E., Kauppinen S. Development of microRNA therapeutics is coming of age. EMBO Mol. Med. 2014;6:851-864.
% Garzon R., Marcucci G., Croce C.M. Targeting microRNAs in cancer: rationale, strategies and challenges. Nat. Rev. Drug Discov. 2010;9:775-789.


%!TEX root = dippa.tex
%%% This file contains the Computational identification of microRNA targets section of my master's thesis.
%%% This section cover the computational methods for miRNA target identification.
%%% Author: Viljami Aittomäki

\section{Computational identification of miRNA targets}

% This section presents computational methods that have been used to predict
% putative target genes for microRNAs and regulatory networks between genes and
% microRNAs.

% Considering the recent large body of research on microRNAs, and their
% potential utility as biomarkers and treatment targets, it is not surprising
% that a plethora of computational tools have been released to aid in microRNA
% research.

% A recent review covering many published methods for different tasks
% has been written by Akhtar et al \citep{Akhtar2016}.

Recognizing the targets of microRNAs is essential in understanding their
biological function and role in disease. Many target
interactions have been found in experimental laboratory studies. Common
methods for such studies include using cell lines and introducing exogenous
miRNAs by transfection or suppressing endogenous ones and measuring the
effect on mRNA or protein expression. For a detailed review of experimental
methods, see Thomson et al \citep{Thomson2011}.

Several public databases list currently known experimentally validated
microRNA targets. Examples include DIANA-TarBase \citep{Vlachos2015} and
mirTarBase \citep{Chou2016}, which are both manually curated from published
literature, and MiRWalk \citep{Dweep2015}, which combines data from several
other databases using text mining.

Although recent advances in high-throughput methodologies, such as CLIP-seq,
have significantly increased the scale of experimental studies, experimental
identification of microRNA targets remains laborious and costly, and many
methods still rely on computational processing of results \citep{Vlachos2015}. To
this end, a wide range of computational tools have been developed to aid in
miRNA target discovery.

Computational approaches to target prediction can be roughly classified into
solely sequence-based tools and tools based on analysis of expression data
(which often incorporate sequence-based predictions). This section presents an
overview of published methods developed for target prediction.
Examples of these methods are shown in Table \ref{table:prediction-methods}.
For more in-depth reviews, see references \citep{Yue2009,Muniategui2013}.


\begin{table}
  \caption{Examples of tools for computational prediction of miRNA targets. All listed methods,
  except MAGIA, allow only downregulation by miRNAs. \\
  Method: type of inference method used for predictions. \\
  SVM: support vector machine. HMM: hidden Markov model. MI: mutual information. \\
  Seq. used: sequence features considered (sequence-based methods) or use of
  previous sequence-based predictions (expression-based methods), see text for details \\
  prefilter: previous sequence-based predictions used as a prefilter
  }
  \label{table:prediction-methods}
  %\centering
  {\fontfamily{lmss}\fontsize{10pt}{13pt}\selectfont
  \begin{tabular}{ lp{3cm}lp{5cm} }
    %\\[-1ex] \hline\hline
    \hline
    \textbf{Name} & \textbf{Method} & \textbf{Seq. used} & \textbf{Additional notes} \\
    \hline \\
    \multicolumn{4}{l}{\textbf{Sequence-based methods}} \\
    \\[-.3cm]
    TargetScan \citep{Agarwal2015}  & rule based            & i,ii,iv,v     & First published target prediction tool (originally) \\
    miRanda \citep{Betel2008}       & rule based            & i,ii,v        & Aligns whole miRNA to mRNA 3' UTR \\
    mirTarget \citep{Wang2008}      & SVM                   & i,ii,iii,iv,v &  \\
    rna22 \citep{Miranda2006}       & Markov chain and rule & i,ii,iv       & Uses a Markov chain to identify potential regions in mRNA 3' UTR and then sequence-rule filtering \\
    PicTar \citep{Krek2005}         & rule and HMM          & i (iv,v)      & Semi-supervised-like approach, uses strict sequence rules$^{\textup{i,iv,v}}$ to obtain a training set for a HMM classifier \\
    TargetBoost \citep{Saetrom2005} & genetic \mbox{programming} & learned  & Learns sequence features and classifier from training data. \\
    \\
    \multicolumn{4}{l}{\textbf{Expression-based methods}} \\
    \\[-.3cm]
    MAGIA \citep{Sales2010}               & correlation, MI               & prefilter & Also produces a bipartite network of miRNA-mRNA interactions. \\
    TaLasso \citep{Muniategui2012}        & lasso \mbox{regression}       & prefilter &  \\
    Engelmann et al \citep{Engelmann2012} & least angle \mbox{regression} & none/prefilter & Least-angle regression is a specific implementation of lasso. \\
    miRNAmRNA \citep{vanIterson2013}      & global test                   & prefilter   & Uses mRNA expression profiles to predict miRNA expression. \\
    GenMir++/3 \citep{Huang2007,Huang2008}& Bayesian \mbox{regression}    & prefilter/i,iv,v & GenMir3 can incorporate sequence features into the model. \\
    Stingo et al \citep{Stingo2010}       & Bayesian \mbox{variable} \mbox{selection} & prior & Effectively a spike-and-slab variable selection approach, scores from any sequence-based method can be used as prior information. \\
    \hline
    \end{tabular}
    }
\end{table}




\subsection{Sequence-based target prediction}

Sequence-based prediction methods focus on finding miRNA-mRNA pairs that
have complementary sequences, as sequence complementarity
is the primary determinant of miRNA targeting.
% Hence, they can only determine pair-wise relationships.
From a machine learning perspective,
prediction of miRNA targets is essentially a \emph{classification}
problem, where the goal is to identify a set features (both of the miRNA and
mRNA) that allows classifying mRNAs as either a target or a non-target of any
given miRNA.

Most sequence-based approaches are essentially rule-based filters, where
features of both the miRNA and mRNA sequence are used to narrow down candidate
target lists \citep{Yeu2009}. These features are derived from earlier
experimental knowledge, and commonly used features include:
(i) sequence matches between the seed region of the miRNA and 3' UTR of the mRNA,
(ii) sequence matches outside the seed region (in the 3' UTR),
(iii) sequence matches in the 5' UTR or coding sequence (CDS) of the mRNA,
(iv) free energy of the bound miRNA-mRNA duplex, and
(v) evolutionary conservation of matches between species.
Rule-based prediction methods are unsupervised, i.e. no training data is
used to form the classifier. Instead, the relevance of the used features is
decided by the method's authors. An example of a rule-based algorithm
is depicted in Figure \ref{fig:miranda-flow}.

\begin{figure}[htb]
  \centering
  \includegraphics[width=0.3\linewidth]{figures/miRanda-flow.pdf}
  \caption{A schematic flowchart of the miRanda \citep{Betel2008} algorithm for
  miRNA target prediction. Modified with permission from \citep{Karhu2009}.}
  \label{fig:miranda-flow}
\end{figure}


% An example of a rule-based predictor is illustrated in Figure \ref{fig:rule-flow}.

Supervised machine-learning approaches has also been employed, where a
training data set consisting of experimentally validated targets and non-targets
(often obtained from expression data sets) is used to train a
classifier for classifying mRNAs as miRNA targets. Support vector machines (SVM)
are the most common choice for classifier. The features used for classification
are similar to rule-based tools, i.e. mostly derived from sequences, but supervised
learning allows the inclusion of much more features. For example mirTarget
uses a set of 113 features including seed region matches, conservation and a
range of different sequence features from different parts of the miRNA
sequence \citep{Wang2008}. More complex approaches have also been applied,
such as using genetic programming to learn sequence features, and using
Markov chains or hidden Markov models (HMM) as a sequence
generative model to estimate targeting probability.

% For a more thorough
% review of several different sequence features and algorithms using them,
% see the reviews by Yue et al \citep{Yue2009} and Bartel \citep{Bartel2009}.

The advantage of sequence-based methods is that they are based on
experimentally derived knowledge on molecular mechanisms and, thus, are likely
to represent causal relationships. As such, the predictions are easy to
interpret. Disadvantages of sequence-based methods include: considering only
pair-wise interactions cannot capture combinatorial effects; using sequence
conservation misses poorly conserved species-specific targets; requiring seed
region matches cannot identify miRNA targets without seed matches (while these
appear rare, they should not be discounted altogether \citep{Bartel2009}); a
sequence match does not always confer %\citep{Grimson2007}
repression and could be functionally inactive; and, finally, rule-based
methods are static and do not account for differing miRNA and mRNA expression
profiles in various tissues and disease states. There is also a general lack
of overlap between predictions from different sequence-based approaches,
suggesting that many results are spurious false positives
\citep{Muniategui2013}.




\subsection{Expression-data-based target prediction}\label{expr-methods}

Integrating expression data with sequence-based target prediction helps combat
the high false-positive rate of sequence-only methods and, importantly,
enables tissue and disease specific support for target predictions in real-world
data. Recent evidence indicates that miRNAs act predominantly through
mRNA degradation \citep{Guo2010}. Thus, it is feasible to use mRNA or protein
expression data to infer target relationships, since the regulatory effect of
miRNAs should be reflected in mRNA and protein abundances. Sequence-based
predictions are often incorporated as a preliminary filter step to limit the
potential interactions examined.

Various mathematical approaches, ranging from correlation to complex Bayesian
models, have been proposed for expression-based prediction. Most
methods limit the studied relationship to repression by the miRNA. This has
been suggested to improve performance \citep{Muniategui2012}, but has the
limitation of not being able to detect positive regulation, both direct and
indirect (mediated through regulation of other mRNAs) \citep{Engelmann2012}.
Notably, the majority of published efforts use either protein or mRNA 
expression together with miRNA expression, very few have combined all three.

% Mathematical models used in such expression
% analyses range from simple similarity measures to regression and complex
% Bayesian models, and examples are covered in this and the next section.

% Expression-based prediction methods use mathematical models that range from correlation
% to complex Bayesian regression models. The idea is to find (negatively)
% correlating miRNA-mRNA pairs or to predict mRNA or protein expression
% from miRNA expression patterns using regression. In the case of multiple regression
% target prediction becomes essentially a variable selection, where 
% the goal is to select microRNAs that best predict mRNA expression and then
% classify the mRNA in question as a target of chosen miRNAs.

Let us henceforth define $y_k = [y_{1k}, \dotsc y_{nK}]$ as a vector of expression
values of mRNA $k (k = 1, 2, \dotsc, K)$ and $X_{n \times p} = [x_j] =
[x_{ij}]$ as the matrix of expression values of miRNAs $j (j = 1, 2, \ldots,
p)$ for observations $i (i = 1, \ldots, n)$.

\paragraph{Correlation}
Several methods and publications use a straightforward approach to identifying
miRNA targets by finding miRNA-mRNA pairs whose expression patterns are
similar across observations. This is achieved with simple measures of variable
association. Pearson correlation is widely used because of its simplicity
and intuitive interpretation. Pearson correlation between mRNA $k$ and miRNA
$j$ is defined as:
\begin{equation}
	\rho_{kj} = (y_k)_{\mu_k=0|\sigma_k=1}^T \cdot (x_j)_{\mu_j=0|\sigma_j=1},
	\label{eq:pearson}
\end{equation}
where $\mu=0|\sigma=1$ indicates normalization to zero mean and unit
variance. Significantly correlated miRNA-mRNA pairs are classified as putative
target interactions. Other measures used include Spearman correlation and
mutual information (MI). A crucial limitation of correlation analysis is being
restricted to studying pair-wise associations. Single miRNAs often have a
small effect on mRNA expression, which leads to weak associations and,
therefore, low power to identify miRNA targets. This issue is worsened by a
large multiple-hypothesis testing problem when considering all possible miRNA-mRNA
pairs. Some approaches have used additional information, such as
sequence-based prediction or differential expression analysis, for limiting
examined miRNA-mRNA pairs to alleviate this to some extent
\citep{Muniategui2013}.
% The drawback of MI is that it only indicates similarity of the variables, but not
% the direction, or sign, of the relationship.

\paragraph{Multivariate linear regression}
Many proposed expression-based methods use some form of multivariate linear
regression (MLR) to examine the relationship between miRNAs and mRNAs.
Expression profiles of miRNAs are commonly used to predict the expression of a
single mRNA. Recently, Engelmann and Spang reported that miRNA expression can indeed be used to
predict mRNA expression \citep{Engelmann2012}. In the context of regression,
target prediction essentially becomes a \emph{variable selection} problem, where the
goal is to choose a set of miRNAs that best predict mRNA expression without
overfitting.

A linear regression model for the expression of mRNA $k$ is defined as
\begin{equation}
  \label{eq:linear-regression}
	y_k = \sum_{j=0}^{p} (\beta_{kj} \cdot x_j) + \epsilon_k =  X \beta_k + \epsilon_k,
\end{equation}
where $\beta_k = [\beta_{k0}, \dotsc, \beta_{kp}]$ is the vector of regression coefficients,
$\beta_{k0}$ is the intercept term, $\epsilon_k$ is the error term, and $X$ is the
matrix of covariates, i.e. the miRNA expression vectors (where a constant column
vector of $x_0=1$ has been added for the intercept). The parameter of
interest is $\beta_k$, which determines to the contribution of each miRNA to the
response variable $y_k$, i.e. mRNA expression. The regression error $\epsilon_k$
represents noise and fitting error caused by variation not captured by the
included covariates. $\epsilon_k$ is commonly assumed to be normally
distributed, with equal variance and no correlation between observations,
giving the \emph{normal linear model}. It is straightforward to incorporate previous
sequence-based predictions by adding an indicator variable $y_k = X c_k \beta_k + \epsilon_k$,
where $c_{kj} = 1$ if mRNA $k$ is a potential target of miRNA $j$, and $c_{kj} = 0$ otherwise.
% Commonly the solution is obtained by
% least squares, which involves minimizing the objective function,
% \begin{equation}
% 	min\{ || y_k - X \cdot \beta_k ||_2 \},
% \end{equation}
% which is equal to the sum of squares of the residuals.

The advantage of using regression for target prediction is the ability to
model the effect of several miRNAs on one gene simultaneously.
% This is desirable as the effect of a single
% miRNA on mRNA expression can be small, as discussed above.
Simple MLR is not applicable in cases, where the number covariates is larger
than the number of observations (here $p > n$)
\footnote{A characteristic that is very common
in analysis of high-throughput biological data, for example microarray
expression data.}, because the linear model is
undetermined and a single solution cannot be obtained.
Furthermore, simple MLR cannot solve the problem of
variable selection as the model fit improves asymptotically
by adding more covariates, leading to overfitting.

\paragraph{Regularized regression}
Both the dimensionality problem and overfitting can be overcome using regularized
regression. The most common approach is to apply regularized least squares,
where a penalty depending on the magnitude of the coefficients $\beta$
is applied to force them small. This entails minimizing the expression
\begin{equation}
	min\{ \left \| y_k - X \cdot \beta_k \right \|_2 + \lambda \cdot R(\beta_k) \} ,
\end{equation}
where the first term corresponds to fitting error (the sum of squared residuals),
$R(\beta_k)$ is the penalty function and $\lambda$ is a tuning
parameter that controls the amount of regularization. 
The 1-norm ($R(\beta_k) = \left \| \beta_k \right \|_1 = \sum_{j=0}^{j} \left | \beta_{gj} \right |$)
is frequently used for regularization; this is referred to as lasso regression
(shorthand for \emph{least absolute shrinkage and selection operator}).
% , where 
% regularizations include the 1-norm ($R(\beta) = ||\beta||_1$) in
% lasso regression, the 2-norm ($R(\beta) = ||\beta||_2$) in ridge
% regression and a combination of these ($R(w) =
% \lambda_1||\beta||_1 + \lambda_2||\beta||_2$) in what is called
% elastic-net regression.
Lasso regression in effect forces the number of non-zero coefficients in $\beta_k$
to be small, leading to a sparse solution that chooses seemingly important covariates.
% where as ridge regression results in a solution
% where coefficients are small but mostly non-zero \citep{Muniategui2013}.
While regularization solves the dimensionality problem and improves
interpretability, it has several important limitations. First, regularization
may remove covariates highly associated with and functionally regulating the
response, instead retaining an unimportant covariate that correlates with
actual regulators \citep{Engelmann2012}. Second, only a limited number of
covariates may be included in the model, and thus some relevant associations
can be missed by number of included covariates alone.% \citep{vanIterson2013}.
Relating to both limitations, van Iterson et al showed (for one dataset) that
lasso did not consistently select highly correlated miRNA-mRNA pairs
\citep{vanIterson2013}.

\paragraph{Other approaches}
Other suggested approaches used include the global test, which is a
generalization for testing the global null hypothesis ($H_0: \beta = 0$) of a
linear regression model when $p >> n$ \citep{vanIterson2013}, and approaches
similar to gene-set enrichment analysis, where the over-representation of
sequence-based target genes in differentially-expressed gene sets is considered
indicative of a target relationship in the studied condition. Le at al have proposed an ensemble
method, which combines predictions from several separate algorithms to build
on the advantages and compensate for the drawbacks of each
\citep{Le2015}.
Several Bayesian approaches have also been proposed; these are discussed in
the next section.


% A group of bioinformatics tools uses mRNA expression data to suggest
% potentially interesting miRNA-target interactions (MTIs).
% \begin{itemize}
%   \item
%   Input ist of interesting genes (e.g. DE vs normal)
%   \item
%   Look for miRNA regulation patterns in list (analogous to gene set enrichment REF REVIEW)
%   \item
%   Enriched miRNAs deemed interesting for this data
% \end{itemize}

% Ainakin nämä vaikka:
% \begin{itemize}
%   \item
%   DIANA-mirExTra (uusin versio NGS-datalle)
%   \item
%   GeneSet2miRNA
%   \item
%   Sylamer(?)
% \end{itemize}

% \paragraph{Correlation methods (incl MI)}\label{correlation-methods}

% Mutual information (MI) is a simple measure of similarity between two
% variables. \textbf{SELITÄ MI TARKEMMIN JA KAAVAN KANSSA JA LÄHDE} Thus, MI can
% be used to measure the interdependence of miRNA-mRNA pairs from expression
% data. However, MI does not distinguish the direction of the interaction, which
% is highly relevant for miRNAs that are believed to mostly downregulate mRNA
% expression. This constitutes a major drawback.


%!TEX root = dippa.tex
%%% This file contains the Bayesian analysis section of my master's thesis.
%%% Author: Viljami Aittomäki

\section{Bayesian analysis}\label{bayesian-analysis}

Bayesian data analysis is a modeling framework that is based on the principle
of quantifying uncertainty as probability. Current knowledge about unknown
model parameters, variables and future observations is described in terms of
probability statements \citep{Gelman2013}. This provides a framework which
is inherently suited to dealing with noisy real-world data, as measurement
noise is naturally incorporated into probability distributions.

\subsection{Basics of Bayesian analysis}

Bayesian analysis begins by defining a joint probability model $p(y,\theta)$
for observed data $y$ and unknown model parameters $\theta$.
The joint distribution can be written as a product of two probability distributions
\begin{equation}
  p(y,\theta) = p(\theta) p(y|\theta),
\end{equation}
which are referred to as the \emph{prior distribution} $p(\theta)$ and the
data distribution or \emph{likelihood} $p(y|\theta)$. The prior conveys
information on the presumed values a parameter may take and the likelihood
represents the likeliness of observed data for given parameter values (in the
context of the chosen data model). Applying the Bayes' theorem we obtain the
\emph{posterior distribution} for $\theta$ given the known values of the
observations $y$:
\begin{equation}
  \label{eq:bayes}
  p(\theta|y) = \frac{p(y,\theta)}{p(y)} = \frac{p(\theta) p(y|\theta)}{p(y)},
\end{equation}
where $p(y) = \int_{\theta} p(\theta) \cdot p(y|\theta) d\theta$.
Noting that $p(y)$ does not depend on $\theta$, we can write Equation
\ref{eq:bayes} as
\begin{equation}
  p(\theta|y) \propto p(\theta) p(y|\theta).
\end{equation}
The latter is referred to as the unnormalized posterior distribution. The
Bayes' theorem forms the heart of Bayesian inference and illustrates the core
concept of updating prior beliefs to account for observed evidence. The
posterior provides a probability assessment of the possible values of
a parameter, and represents a compromise between prior knowledge and information
contained in observed data. As the number of observations increases, the
data have increasing influence on the posterior \citep{Gelman2013}.

The virtue of modeling uncertainty as probabilities -- in addition to
naturally dealing with noise -- is that they are conceptually easy to grasp
and often allow common-sense interpretations of conclusions to be made.
\footnote{This is especially true compared to classical frequentist analysis,
which is defined within the context of repeated sampling (and inference) from
a fixed but unknown process generating the observations. For
example, frequentist confidence intervals strictly do not indicate that the
true value of the parameter is contained within with high probability -- a
common misconception -- where as Bayesian posterior intervals do (subject to
modeling assumptions, of course).}
Bayesian modeling is also flexible and can cope with complex problems with
relative ease. Prior and expert knowledge about the parameters of interest can be
embedded in the prior distribution and different priors may be assigned to
each parameter.
% additional data can be included sequentially using a
% previously obtained posterior as a new prior distribution.

Hierarchical models -- where parameters of the prior
distributions have their own priors called \emph{hyperpriors} -- provide
additional flexibility. In situations where model parameters are
related to each other, a common hyperprior may be used for several parameters.
Hierarchical models are particularly appropriate in settings where
data are sparse (as available information is be shared between parameters)
or the data are naturally structured into several levels, such as 
similar measurements from different hospitals or schools.

The challenge in Bayesian analysis is choosing proper probability models for
the parameters and observations, including prior distributions as well as the
likelihood \citep{Gelman2013}. In fact, Bayesian methodology has been
criticized for the subjectivity related to choosing suitable priors, as it is
based on the experience and reasoning of the statistician. One could, however,
argue that the choice of any model is always subjective to a certain degree,
irrespective of chosen methodology. Additionally, weakly informative or non-informative
priors can be used to decrease the effect of subjective information (or
in cases where no prior knowledge is available) and results from
inferences using non-informative priors often coincide with classical
analyses.


% In cases where no prior information is available, weakly informative
% or noninformative priors can be used. These convey little or no
% information on the presumed values a parameter may take, respectively.
% In many instances, using a non-informative prior results in similar or equal
% results as frequentist analysis, but the strength of Bayesian analysis comes
% from including prior knowledge in the prior distribution. \citep{Jaynes?} The
% posterior distribution represents a compromise between the prior (and, hence,
% prior information) and the observed data, with the data having an increasing
% effect as the sample size increases \citep{Gelman2013}. The posterior
% distribution also provides a more comprehensive view of
% one's knowledge on the parameter of interest than, say, a single confidence
% interval.

% The frequentist approach only considers the data to have a probability
% distribution, the likelihood. The process giving rise to the data, and the
% parameters that define it, are considered fixed. The observed data are
% assessed with respect to other data that might be generated by the same model.

% OMIN SANOIN: Confidence intervals work their best when you don't know much about a
% parameter beyond the information contained in a data set. And further,
% credibility intervals won't be able to improve much on confidence intervals
% unless there is prior information which the confidence interval can't take
% into account, or finding the sufficient and ancillary statistics is hard.


\subsection{Bayesian inference}\label{simulation}

The goal of Bayesian inference is often to make conclusions about uknown
parameters $\theta$ of uknown observations $\tilde{y}$ given observed data
$y$. These are formulated as probability statements.

Often it is not possible to obtain analytical solutions to integrals
. This is especially true for hierarchical models. It is therefore
oftentimes necessary to use numerical estimation or simulation to
derive estimates of the posterior.

% Bayesian analysis often involves simulation if the form of sampling from the
% obtained posterior distribution. This is convenient -- and necessary -- when
% the exact probability density function cannot be explicitly obtained through
% integration. Additionally, simulation often has the advantage of pointing out
% problems in the model specification when simulated values are extremely small
% of large.

% Often obtaining analytical solutions to integrals or explicit formulations of
% the posterior distribution is not possible. This is especially true for
% complex and hierarchical models. In these cases it is possible to use
% simulation to obtain samples from the posterior distribution of parameters and
% use these to compute estimates for parameters and other quantities of
% interest.

Simplest approaches to simulation include sampling directly from the posterior
distribution $p(\theta|y)$, where possible, or a distribution proportional
to $p(\theta|y)$. More commonly Markov chain methods are used. These entail
drawing iterative samples of $\theta$ from a distribution that approaches
the true posterior during iteration rounds, where each draw $\theta^t$ is
conditional (only) on the previous one $\theta^{t-1}$.
\footnote{This is essentially the definition of a Markov chain; a sequence of random
variables, where the probability density of each one is dependent on only the
previous one.} It is advisable to run several separate simulations with
different starting values $\theta^0$ to assure proper coverage of the whole
posterior. These are referred to as chains.

Markov chain simulations should always be assessed for proper converge, both
for each chain and between the chains. For this purpose, Gelman et al have
proposed the $\hat{R}$ measure\footnote{Not to be confused with $\bar{R}^2$,
the adjusted coefficient of determination of a regression model, defined in
the next section.}, which estimates the ratio of between-chain variation and
within-chain variation for each parameter \citep{Gelman2013}. Gelman et al
suggest $\hat{R}^2 < 1.1$ is a reasonable indicator of good convergence. It is
also good practice to discard the first simulation samples (commonly the first
half of each chain) to ensure that samples arise from a converged state.


\subsection{Bayesian regression}

Bayesian regression analysis aims to infer the posterior distributions
for the regression coefficients of covariates and other model parameters,
such as the variance (i.e. the error term) of the observation model.
Within the Bayesian framework, the normal linear regression defined in
Equation \ref{eq:linear-regression} can be expressed as
\begin{equation}
  y | \beta, \sigma, X \sim N(X \beta, \sigma^2I),
  \label{eq:bayesian-linear-regression}
\end{equation}
where $N$ is the multivariate normal distribution, and $I$ is the $n \times n$
identity matrix. The mean of $y$ is then the familiar linear sum of $x_k$
\begin{equation}
  \textrm{E}(y|\beta,X) = X \beta = \beta_0 + \beta_1 x_1 + \dotsb + \beta_p x_p.
\end{equation}
The posterior distribution for the regression coefficients (up to a
normalizing constant) is obtained from the marginal posterior
\begin{equation}
  p(\beta | \sigma, y, X) \propto \int p(y | \beta, \sigma, X) p(\beta | \sigma) p(\sigma) d\sigma,
\end{equation}
with the joint prior $p(\beta, \sigma) = p(\beta | \sigma) p(\sigma)$.
It is relatively straightforward to extend this simple model, for instance by
allowing unequal variances or correlation between observations, choosing a
different data distribution
% than the normal distribution
to represent the error term or including hyperparameters to construct a
hierarchical model.

As mentioned in Section \ref{expr-methods}, microRNA target prediction using
regression analysis of expression data is effectively a variable selection
problem. For Bayesian regression, several different priors that provide model shrinkage
have been proposed, including the Laplace prior (which is closely related to
lasso regression), the horseshoe prior, and the hierarchical shrinkage prior.
A hierarchical shrinkage prior for regression weights
$\beta = [\beta_1, \dotsc \beta_p]$ can be defined as
\begin{subequations}
  \label{eq:hs-prior}
  \begin{align}
    \beta_j | \lambda_j, \tau & \sim N(0, \lambda_j^2 \tau^2) \\
    \lambda_j                 & \sim t_\nu^+(0,1),
  \end{align}
\end{subequations}
where $t_\nu^+$ denotes the half-Student-$t$ prior with $\nu$ degrees of freedom
\citep{Piironen2015}. The $\lambda_j$ correspond to a local scale parameter and
$\tau$ to a global scale. As an example, in a very sparse model with many irrelevant
covariates, the model would ideally make $\tau$ small (so that $p(\beta)$ is shrunk
close to zero), but allow some $\lambda_j$ to be large to escape the shrinkage.
Additional priors need to be assigned to $\tau$, $\sigma$ and $\beta_0$, of course.

Bayesian shrinkage priors, however, do not lead to a sparse solution as there remains
uncertainty in the posterior distribution and no coefficient can be considered
exactly zero.


\subsection{Bayesian variable selection}

In order to find a small set of relevant predictive variables, a model
selection approach needs to be applied. To this end, a range of methods
applicable in Bayesian analysis have been proposed; examples include using
Bayesian cross validation, different information criteria, and projection
methods to determine the submodel giving the best compromise between
prediction accuracy and model size. A detailed review of these falls outside
the scope of this thesis, however, a comprehensive one has been recently
written by Vehtari and Ojanen
\citep{Vehtari2012}.

In the context of variable selection for regression,
Piironen and Vehtari recently suggested that, for problems where data
are scarce and the number of candidate variables high, using projection predictive
variable selection is effective \citep{Piironen2016}. The idea, proposed by
Dupuis and Robert \citep{Dupuis2003}, is to fit a full reference model
encompassing all candidate variables and uncertainties related to their
effect, and then use projection to find a submodel that gives similar
predictions compared to the reference model.

An issue with this method is deciding how many variables should be included
in the model, and what is considered similar-enough predictive performance
from the submodel. Piironen and Vehtari suggest using cross-validation
to guide the variable selection process and give a practical guideline
for stopping the selection, which we will return to in Section \ref{sec:PPVS}.

In this thesis, projection predictive variable selection is used for
inferring putative microRNA targets from breast cancer expression
data using Bayesian regression. Further details of the used method are
presented in section \ref{sec:PPVS}.


\subsection{Bayesian microRNA target-prediction methods}

% Bayesian analysis has previously been applied to miRNA target prediction.

One of the earliest tools to use expression data for target prediction was
GenMir++. It is based on A BAYESIAN SYSTEEMI, KIRJOITA AUKI.
The latest version of GenMir (GenMir3) now integrates sequence-based features
into the algorithm, although the authors note that this does not affect the
results to a significant extent.

Stingo et al have proposed a Bayesian method based on spike-and-slab variable
selection.

Previous Bayesian miRNA methods:
\begin{itemize}
  \item Stingo et al
  \item GenMir3/GenMir++
  \item Se joku graafinen (korkeintaan lauseen maininta)
\end{itemize}




%!TEX root = dippa.tex
%%% This file contains the materials and methods section of my master's thesis.
%%% Author: Viljami Aittomäki


\section{Materials and methods}\label{materials-and-methods}

\subsection{Research material}

The data analysed in this thesis consists of 283 tumor samples collected from
280 breast cancer patients treated in two Norwegian hospitals. Protein, mRNA
and microRNA expression were measured from each sample. The data were
published by Aure et al. \citep{Aure2015} and are publicly available. Analyses
performed in this thesis used the publicly available preprocessed data.

The data are part of the Oslo2 cohort, which consists of breast cancer
patients with primarily operable disease -- that is stage cT1--cT2 -- treated
in several Norwegian hospitals. Collection of the cohort started in 2006 and
is still ongoing. Therefore, no survival data were available for analysis.
Clinical data for included patients were kindly provided by Aure and
associates and a compiled summary is presented in Table
\ref{clinical-data}. Notably, the vast majority of tumors in the data
were ductal carcinomas, which is in general the most common histological type
of breast cancer. Patient ages ranged from X to Y with a median of 45 years.
No matched control samples of healthy breast tissue were available.
It is worth noting that the data represent a very heterogeneous sampling
of different kinds of breast tumors.

% latex table generated in R 3.3.0 by xtable 1.8-2 package
% Thu Oct  6 19:58:33 2016
\begin{table}[ht]
\centering
\caption{
Clinical features of the 283 tumor samples included in the analysis.
Table compiled from original data kindly provided by Aure and associates.
See \citep{Tavassoli2003} for definitions of the TNM classification.
CIS: carcinoma in situ. ER: estrogen receptor. PR: progesterone receptor.
HER2: human epidermal growth factor receptor 2. Multifocality: whether there is a single primary tumor or several. \
positive: receptor present in cancer cells. negative: receptor not present.
missing: information not available
\label{clinical-data}
} 
\begingroup\footnotesize
\begin{tabular}{llrr}
% latex table generated in R 3.3.0 by xtable 1.8-2 package
% Thu Oct  6 19:58:33 2016
  \hline
Variable & Level & Count & Fraction \\ 
  \hline
Histology & Ductal & 228 & 81\% \\ 
   & Ductal CIS & 7 & 2\% \\ 
   & Lobular & 23 & 8\% \\ 
   & Medullary & 1 & 0\% \\ 
   & Metaplastic & 1 & 0\% \\ 
   & Mixed & 4 & 1\% \\ 
   & Mucinous & 5 & 2\% \\ 
   & Papillary CIS & 1 & 0\% \\ 
   & Tubular & 5 & 2\% \\ 
   & missing & 5 & 2\% \\ 
Tumor size (T) & pT1a & 1 & 0\% \\ 
   & pT1b & 20 & 7\% \\ 
   & pT1c & 122 & 44\% \\ 
   & pT2 & 109 & 39\% \\ 
   & pT3 & 13 & 5\% \\ 
   & pTis & 7 & 2\% \\ 
   & pTx & 3 & 1\% \\ 
   & missing & 5 & 2\% \\ 
Metastasis (M) & M0 & 36 & 13\% \\ 
   & M1 & 6 & 2\% \\ 
   & Mx & 236 & 84\% \\ 
   & missing & 2 & 1\% \\ 
Lymph node status (N) & pN0 & 166 & 59\% \\ 
   & pN1 & 80 & 29\% \\ 
   & pN1a & 2 & 1\% \\ 
   & pN2 & 19 & 7\% \\ 
   & pN3 & 8 & 3\% \\ 
   & missing & 5 & 2\% \\ 
Grade & I & 42 & 15\% \\ 
   & II & 108 & 39\% \\ 
   & III & 125 & 45\% \\ 
   & missing & 5 & 2\% \\ 
ER & negative & 52 & 19\% \\ 
   & positive & 216 & 77\% \\ 
   & missing & 12 & 4\% \\ 
PR & negative & 81 & 29\% \\ 
   & positive & 187 & 67\% \\ 
   & missing & 12 & 4\% \\ 
HER2 & negative & 240 & 86\% \\ 
   & positive & 28 & 10\% \\ 
   & missing & 12 & 4\% \\ 
Multifocality & Multifocal & 44 & 16\% \\ 
   & Single tumor & 225 & 80\% \\ 
   & missing & 11 & 4\% \\ 
   \hline
\end{tabular}
\endgroup
\end{table}


The mRNA and microRNA expression were measured using Agilent Technologies
SurePrint G3 Human GE 8x60K and Human miRNA Microarray Kit (V2) microarrays,
respectively. These microarrays measure 27958 genes and 887 miRNAs, according
to manufacturer annotation. Protein expression was measured using a reverse
phospatase protein array (RPPA) containing a set of 105 proteins. Most of the
proteins are found on the PI3K/AKT intracellular pathway, which is important
for cell-cycle regulation and, thus, cancer. \textbf{(ref to G.Mills?)}.

% The mRNA and microRNA expression data are publicly available in preprocessed
% format in the Gene Expression Omnibus (GEO) database \citep{GEO} under
% accession IDs xx and xx respectively. For the purpose of this thesis, the raw
% Agilent expression data were kindly provided and used for the analyses instead
% of the preprocessed data. The protein expression data is available in
% Additional file 4 of \citet{norjis} also in preprocessed format.

% Clinical data concerning each patient and cancer were also provided. A summary
% of the clinical parameters is presented in table \ref{clinical-data}. The
% predominant tumor type in the data was ductal carcinoma, which is in general
% the most common histological type of breast cancer.

% Use danish data for validation?





\subsection{Methods}

A Bayesian regression model of protein, mRNA and miRNA expression data was
constructed, and projection predictive variable selection was used to predict
microRNA targets in breast cancer data. Details of the methods used are
presented in this section. All computational analyses were performed in R
\citep{R} and workflow management was handled with Anduril \citep{Ovaska2010}.
Monte-Carlo simulations for the Bayesian regression models were performed with
\emph{RStan} \citep{RStan} using the No-U-Turn variant of a Hamiltonian Monte Carlo
algorithm for sampling posterior distributions. Simulations were run using
computer resources within the Aalto University School of Science "Science-IT"
project.


\subsubsection{Preprocessing and quality control}

MicroRNA and mRNA expression data were downloaded from the Gene Expression
Omnibus (accessions GSE8210 and GSE8212, respectively \citep{Edgar2002}) using
the GEOquery R package \citep{GEOquery}. Protein expression data were
downloaded as a Microsoft Excel sheet from the supplementary data of Aure et
al \citep{Aure2015}. All of the data are in log scale, as is usual for
expression data analyses. For regression analyses, all variables (miRNA, mRNA
and protein) were further scaled to have zero mean and unit variance,
a commonly used transformation in regression.

Notably, the protein data have only a single set of measurements for two sets
of related genes, namely AKT1, AKT2, AKT3 and GSK3A, GSK3B (labeled
respectively as "AKT1/2/3" and "GSK3A/GSK3B" in the original data). Presumably
the protein array used cannot differentiate the proteins of these genes, but
no mention of this is made by Aure et al. For the analyses presented here,
each of these genes was considered separately, and the same expression values
from the single measurement set were used for all of the related genes
respectively.

Out of the 421 microRNAs present in the public dataset, eleven miRNAs (hsa-
miR-1274a, hsa-miR-1274b, hsa-miR-1280, hsa-miR-1308, hsa- miR-1826, hsa-
miR-1974, hsa-miR-1975, hsa-miR-1977, hsa-miR-1979, hsa-miR-720, hsa-
miR-886-3p) were reported as missing from miRBase by miRBase Tracker. Reviewed
on miRBase, these miRNAs are reported as being fragments of other RNA species
(e.g. tRNA or rRNA) and, thus, removed from the database. The eleven miRNAs
were therefore removed from subsequent analyses, leaving 410 miRNAs.

The publicly available expression data are in preprocessed form, and as such,
no further preprocessing of the actual measurements was done. For details on
the preprocessing, the reader is referred to the supplementary data of Aure et
al \citep{Aure2015}. The mRNA expression data are available as probe-level
measurements, these were summarized to gene-level using manufacturer probe
annotations by taking the median of all probes targeting the same gene.
Only the genes present in the protein data were used in analyses.

For assessing the quality of the expression data, distributions of each
microarray and each variable (miRNA, mRNA and protein) were plotted. The data
have been collected from two different hospitals and to assess possible bias
introduced by the separate sites, a principal component analysis (PCA) and
hierarchical clustering of samples were performed separately for each data
type.



\subsubsection{Correlation analysis}


\subsubsection{Regression models}

For predicting protein expression from mRNA and miRNA expression, a similar
regression model to Aure et al was used:
\begin{equation}
	\label{eq:reg-model}
	y = \beta_0 + z \beta_g + X \beta + \epsilon,
\end{equation}
where $y$ denotes the protein expression vector for the protein, which
is produced from translation of the mRNA $z$, $w_g$
is the regression coefficient for the mRNA, $X$ is the matrix of miRNA
expression vectors, and $w_0$ is the intercept term
(for a justification of this equation, see Aure et al \citep{Aure2015}).
A separate model was fitted for each gene.
A model with only the mRNA expression covariate (called the \emph{gene-only model}),
defined as $y = \beta_0 + z \beta_g + \epsilon$, was used as a baseline.
A normally distributed error term with equal errors and no correlation
between observations was assumed for all models.

The likelihood for Bayesian regression was therefore defined as
\begin{equation}
	y | \mathbf{\beta}, \sigma, z, X \sim N(\beta_0 + z \beta_g + X \beta, \sigma^2I),
\end{equation}
where $\mathbf{\beta} = [\beta_0, \beta_g, \beta]$ for convenience.
The intercept and mRNA coefficient were given diffuse Gaussian priors
and $\sigma$ a uniform prior:
\begin{subequations}
  \begin{align}
    \beta_0 & \sim N(0, 5^2) \\
    \beta_g & \sim N(0, 5^2) \\
    \sigma  & \propto 1.
  \end{align}
\end{subequations}
These were also used in the gene-only model.
A hierarchical shrinkage prior was applied to the miRNA coefficients $\beta$,
as defined in Equation \eqref{eq:hs-prior}. The degrees of freedom for the
$\lambda_j$ priors was set at $\nu=3$ (similar to Piironen and Vehtari
\citep{Piironen2015}). The prior for $\tau$ was defined as:
\begin{equation}
    \tau ~ \textup{half-Cauchy}(0, \frac{p_n}/{n}*\sqrt{\textup{log}(n/p_n)}),
\end{equation}
combining the previous suggestions of half-Cauchy and fixed $tau$.
The assumed number of relevant miRNAs, $p_n$, was estimated as follows.
Ensembl gene ID's were downloaded for all protein-coding genes in the human
genome using biomaRt \citep{biomaRt}. From these, a sample of 1000 genes was
taken, and known validated microRNA interaction partners for each sampled gene
were downloaded from miRWalk \citep{Dweep2015}. Genes for which there were no
validated miRNA interactors were assumed to have zero. The mean number of
miRNA interactors per gene was used as an estimate, giving
$\hat{p_n} = 13.75$.



\subsubsection{Variable selection}

Projection predictive variable selection (as described in Section
\ref{sec:bayes-variable- selection}) was used to obtain the relevant set
of microRNAs for each gene. A full model was fitted by drawing 2000 samples
from the posterior using RStan (4 chains, 1000 samples each and the first half
discarded as burn-in). A random sample of $S=1000$ simulation samples from the
full posterior was used to increase projection speed. Then a series of
submodels was obtained by projection from the full model and using a forward
search strategy. That is, the search started from a model including only the
intercept, the mRNA expression $z$ was always added as the first covariate,
and then at each subsequent step, the miRNA covariate $x_j$ giving the largest
decrease in KL-divergence between the full and projected models was chosen.
The forward search was continued up to 200 variables.

For choosing the model size, 10-fold cross validation was used, as proposed by
Piironen and Vehtari \citep{Piironen2016}. That is, the above model selection
process was performed $K=10$ times, each leaving $n/K$ observations out for
evaluation. To judge the appropriote model size, estimation of submodel
predictive performance was used as explained below.

\paragraph{Predictive performance}
Given submodel $M_\perp$ with the posterior predictive distribution
$p(\tilde{y}| \tilde{z}, \tilde{X}_\perp, \mathbf{\theta}_\perp, D_\perp)$,
where $D_\perp$ is the observed data in the current submodel and
$\mathbf{\theta}_\perp$ the projected parameters, the predictive performance
of each submodel was evaluated with the logarithm of the predictive density
(LPD) at each of the left-out observations $(y_*, z_*, X_*)$ The LPD was
estimated by averaging over the simulated posterior samples:
\[
	textup{LPD}_*(M_\perp) \approx \textup{log}\frac{1}{S} \sum_s^S p(y_*| z_*, X_*, \mathbf{\beta}_\perp, \sigma_\perp).
\]
The LPD's from each fold were pooled and a mean over the full set of data
(MLPD) was used as a summary. To compare the predictive performance of a
sbumodel to the full model the difference in MLPD ($\Delta \textup{MLPD}$) was
used. Bayesian bootstrap (with 5000 samples) was used to estimate a
distribution for $\Delta \textup{MLPD}$ by:
\[
	\Delta \textup{MLPD}^{s} = \sum_i^n w_i^{(s)} \left [ LPD_i(M_\perp) - LPD_i(M_{\textup{full}}) \right ],
\]
where $w_i^{(s)}, i = 1, \dotsc,n$, are the bootstrap weights
for the $s$th bootstrap sample (subject to $\sum_i w_i^{(s)} = 1$).
The \emph{bayesboot} R package was used for computing the bootstrap.

\paragraph{Choosing model size}
The threshold for choosing the model size was defined as the smallest
model satisfying
\begin{equation}
	\label{eq:size-condition}
	\textup{Pr}(\Delta \textup{LPD} < U) \geq \alpha,
\end{equation}
where $U = \gamma \textup{E}(LPD_{\textup{full}}-LPD_{0})$, and
$LPD_0$ refers to the intercept-only model. This means that model
size was chosen such that the probability of the difference in utility
between projected and full model being smaller than a constant times
the difference between intercept-only and full model was at least $\alpha$.
Several different values for $\alpha$ and $\gamma$ were experimented with.

\paragraph{Final model selection}
The final projected model was obtained by redoing the projection search up to
the chosen number of variables, using all data for each gene. The miRNAs for
which the 95\% posterior interval did not include the origo (called
\emph{significant miRNAs}) were considered putative interactors with the gene
in question. For some genes the condition in \eqref{eq:size-condition} was not
met after including 200 covariates. In these cases it was concluded that the
miRNA covariates provided no additional information on the protein expression
and, thus, none of them were deemed as targeting the gene. In some cases the
condition was met already by the model with only the mRNA covariate, and the
same conclusion was made.

\paragraph{Lasso regression}
A lasso regression model was also fitted for each gene using the
\emph{glmnet} R package. In this case the mRNA variable was treated equal
to the miRNA variables and subjected to the lasso regularization.
For choosing the regularization parameter $\lambda$, a 10-fold cross
validation was done, and the largest $\lambda$ that gave
a mean square prediction error (MSE) at most one standard deviation
apart from the lowest MSE was used. The covariates included
in the model using the chosen $\lambda$ were considered putative
target interactors. For same genes, this criterion was met by
the intercept-only model, and again, in these cases
none of the miRNAs were deemed as targeting the gene.



\subsubsection{Measuring model fit}

The coefficient of determination $R^2$ of a predictive model is defined as
\begin{equation}
	R^2 = \frac{SS_{\textup{residuals}}}{SS_{\textup{total}}} = \frac{\sum_{i=1}^{n}(\hat{y}_i-y_i)^2}{\sum_{i=1}^{n}(y_i-\bar{y})^2},
\end{equation}
where $\hat{y}_i$ are the predictions made by the model and $\bar{y}$ is the mean of the outcome variable.
$R^2$ indicates the proportion of variance of the outcome variable that is explained
by a statistical model. It provides a measure of how well the model
replicates the observed values and can be used as a measure of model fit.
$R^2$ has the property of being invariant to variable scaling, which
makes it suitable for use with expression data, as expression data
do not have a well defined scale, $R^2$. A caveat of
$R^2$ is that in linear regression it often increases monotonically by adding more explanatory
variables. The adjusted $R^2$, defined as $\bar{R}^2 = 1-(1-R^2)*(n-1)/(n-p)$
(where $n$ is the number of observations and $p$ the number of explanatory
variables) adjusts for the number of regressors relative to the number of
observations, thus penalizing inclusion of additional variables. $\bar{R}^2$
was used for comparing the projected model with the gene only model.



\subsubsection{Verification of predicted targets}




\section{Results}

\begin{center}
  /home/viljami/wrk/dippa-analyysi/execute/finalModelTable/report/finalModelTable-Table.tex
\end{center}


%!TEX root = dippa.tex
%%% This file contains the discussion section of my master's thesis.
%%% Author: Viljami Aittomäki

\section{Discussion}

This thesis presents a review of the basics of gene expression, microRNAs and
the computational prediction of microRNA target genes. A modern Bayesian
variable selection method was applied in the context of regression, to predict
protein expression from mRNA and miRNA expression in breast cancer tumor
samples, with the goal of identifying putative miRNA target. The Bayesian
method was compared to lasso regression, a popular method for target
prediction from expression data.

Quality control plots suggested the presence of noise in the data, especially
in the miRNA microarrays. More strict filtering of miRNA data could lead to
better overall results. The choice of data model can also alleviate the
effects of noise. The
student-$t$ distribution, which has heavier tails than the normal
distribution, would likely be a better choice for the regression model
likelihood. This, however, would require a different solution to the
projection Eq. \eqref{eq:projection} than the one derived by Piironen
and Vehtari \citep{Piironen2015} and, therefore, was not tried in this work.

Correlation between mRNA and protein measurements for the same gene were
mostly low, a finding supported by previous studies \citep{Payne2015}.
Interestingly, there was practically no difference between correlations of
validated miRNA-target pairs or randomly chosen ones. This supports the view
that modeling single interaction pairs individually is unlikely to be
sufficient for effective target prediction using expression data.

It is possible that low correlation of miRNA-target pairs was mostly due to the
dataset used and measurement noise it contained, however, 
%as discussed in Section \ref{expr-methods},
correlation has low power to detect weak relationships and
cannot capitalize on combinatorial effects. The low correlations also suggest
that simple (Pearson) correlation of expression data is possibly inadequate for
finding miRNA targets. Poor prediction results using correlation have been
reported before (see for example \citep{Muniategui2012}).

There could be several explanations for the increasingly poor performance of
larger models, evident in Figure \ref{fig:n-miRNAs-vs-R2}. The largest models
could be a result of the covariates not explaining the outcome variable well.
This results in a large model, as the inclusion of each covariate can provide
only a minute improvement to submodel performance and, therefore, a large
number of covariates is required to satisfy the criterion for choosing model
size. The assumption that the reference model represents the best current knowledge
could also be false, as seemed to be the case for some genes where the
submodels performed consistently better than the full model. This likely means
that the reference model has overfit the training data (in the cross-
validation).

From a biological perspective, poor performance means that the microRNAs did
not provide additional information for predicting protein expression after
accounting for gene expression. This could be due to relevant miRNAs missing
from the dataset, the number of observations being too small leading to low
statistical power (unable to capture the often small effect that miRNAs have),
or the biological heterogeneity of breast cancer.
% a gene truly not having
% miRNA regulators in breast tissue, insufficient data (the true regulators of a
% given gene might be missing from the data)
The proposed regression model could also be inadequate for capturing the actual	
biological effect of miRNAs, though previous research seems to suggest otherwise.


% Secondly, the marginal posterior of a single predictor can
% indicate non-significance, while the joint posterior of several predictors
% combined might still achieve significance. This would indicate that the effect
% of miRNAs is only significant when acting simultaneously, a hypothesis that is
% supported by experimental evidence, as discussed in section
% \ref{microrna-function}.

Compared to lasso regression PPVS achieved better model fit, yet from a target
prediction perspective, performance of the two methods was similar. There
was little overlap of predictions made by the two methods, a common issue in
microRNA target prediction \citep{vanIterson2013}. Only a small fraction of
predicted targets were validated according to TarBase and miRTarBase, however,
this is probably true of all miRNA targets in general; only a
limited number of validation studies have been published. The computational cost
of PPVS was significantly higher than that of lasso: MCMC simulations took hours
compared to less than a minute per model for lasso.

Approximately half of the regression coefficients for miRNAs were positive,
suggesting that those miRNAs increase gene expression. Some of these
could indicate indirect regulation. However, this proportion seems too
high, as the vast majority of known microRNA interactions are suppressive. In
fact, of all the experimentally validated human miRNA targets listed in
TarBase, only approximately 0.2\% show positive regulation by the miRNA.
Therefore, many of the predicted activating interactions are possibly false
findings. They could be caused by miRNA expression mirroring the involvement of other
regulatory factors not included in the data. To correct for this, the model could
easily be restricted to only negative interactions (using a non-positive prior
for $\beta$). This has previously been reported to increase prediction
performance \citep{Muniategui2013}.

Previous studies have shown that microRNA signatures correlate
with different breast cancer subtypes \citep{Blenkiron2007}. This suggests that
using pooled datasets of various tumors, such as the data used in this
study, is likely to miss subtype-specific miRNA effects, unless this is accounted
for in the model. This could be achieved in the proposed method by
constructing a hierarchical model that includes tumor-subtype data.

Another way to improve the proposed model would be to include sequence-based
target information, as most published methods do. This could be achieved with
indicator variables, a weighting scheme, or more elaborately by including
sequence-based data within the hierarchical-shrinkage prior to impose less
regularization on putative target pairs. However, as the authors of GenMir
noted, including sequence features did not result in a significant improvement
of their method \citep{Huang2008}.\footnote{It should be noted, that GenMir
uses sequence-based predictions as a preliminary filter step. Therefore,
it is perhaps not surprising that including the same type of data within the
model does not produce substantial improvement.}

% One disadvantage of the proposed method relates to scaling of the regression
% covariates, that is, the mRNA and miRNA expression variables. Sequencing studies have shown
% that a relatively small number of miRNAs accounts for over 80\% of tissue microRNA
% \citep{Landgraf2007}. Therefore, changes in the expression of these highly
% abundant miRNAs may have a relatively large impact on protein levels, where as
% similar changes in less expressed miRNAs may have little to no effect on target gene
% expression. This difference is lost by scaling all miRNA variables to a similar scale.

% However, many of the most abundant miRNAs are ubiquitously expressed across
% different tissues \citep{Landgraf2007} and, therefore, possibly less
% interesting with regards to disease pathogenesis. Additionally, inclusion of
% the gene expression measurement in the regression necessitates some form of
% scaling, as mRNA and miRNA expression profiles are measured with different platforms
% and processed using different algorithms and, therefore, are not directly
% comparable. One could easily envision a scaling procedure where the relative
% levels of different miRNAs are preserved and gene expression is scaled
% relative to mean miRNA expression, for example, but it is debatable whether
% this would be appropriate either.

The proposed model does not account for the fact that microRNAs have
several, even hundreds, of target transcripts \citep{Friedman2009}. Therefore, the
regulatory effect of a single miRNA is most likely spread across several
genes. In combination with transcripts having several regulating miRNAs, this
many-to-many nature of microRNA regulation ultimately calls for computational
methods that model the whole regulatory network at once, such as regression
models with multivariate targets. This, however, becomes a much more difficult
problem than multivariate linear regression.

Aure et al \citep{Aure2015} used lasso regression for a similar analysis of the same dataset.
%prediction of protein expression from mRNA and miRNA expression 
%to identify miRNAs significantly affecting protein expression in breast cancer
They used a multi-step process, where only miRNAs deemed
significant in a univariate regression model were used as input in multivariate
lasso regression. This approach is flawed in the sense, that it loses some of
the power of multivariate models to identify singly weak but combinatorially
strong effects, as univariate modeling is used as a filtering step. It also
effectively uses the same data twice, causing bias, and introduces a
multiple hypothesis testing problem. Therefore, a multivariate
approach (such as the one presented here or earlier ones with slight modification)
would likely be preferable.

In conclusion, the work in this thesis shows that the proposed method of
projection predictive variable selection is applicable to microRNA target
prediction. However, further refinements to the model are warranted to improve
performance. In the presented form, compared to a simpler alternative, the
method offered only a limited advantage from a modeling perspective, and no
apparent advantage from a biological perspective, but incurred a large
computational burden. The choice of parameters $\alpha$ and $\gamma$, which
define the threshold for model size, proved nontrivial. The values chosen
had a large impact on the sizes of resulting models and, therefore,
a data-driven approach for optimizing the parameter values would perhaps be useful.




\subsection*{Future prospects}

The recent development of CLIP-seq and similar methods has made high-throughput
experimental microRNA target discovery possible, partially
replacing the need for computational target prediction. Nonetheless,
experimental (particularly high-throughput) methods are not immune to error,
and gene regulation is vastly complex with many unconventional regulatory
mechanisms having been discovered. Integrative computational approaches
beyond correlation -- combining several types of data -- will, thus,
remain important in the future. Possibilities for integrating data include
incorporating copy number variation, other regulatory RNAs, transcription factors,
other protein-level regulatory factors such as phosphorylation, and
epigenetic mechanisms into models.

The elucidation of complex regulatory networks using network-level modeling is
becoming feasible with modern experimental and computational methods.
Employing this approach will be essential, as it has the ability to better
capture the true nature of gene regulation and cellular biology.

Many aspects of microRNA biology and function still remain unknown. Uncovering
miRNA function offers interesting possibilities in diagnostics and treatment
of disease, and will further our understanding of the complexities of
molecular cell biology. Therefore, microRNAs remain an exciting avenue
of research.





%%=========================================================
%% References

\clearpage
%% The \phantomsection command is nessesary for hyperref to jump to the 
%% correct page, in other words it puts a hyper marker on the page.
\phantomsection

\addcontentsline{toc}{section}{\refname}
%\addcontentsline{toc}{section}{References}

\bibliographystyle{apalike}
\bibliography{dippa}




%%=========================================================
%% Appendices

\clearpage
\thesisappendix

%!TEX root = dippa.tex
%%% This file contains the final model table appendix of my master's thesis
%%% Author: Viljami Aittomäki

\section{Table of model properties\label{app:model-table}}

%\begin{landscape}
/home/viljami/wrk/dippa-analyysi/execute/finalModelTable/report/finalModelTable-Table.tex
%\end{landscape}


%!TEX root = dippa.tex
%%% This file contains the final model size distribution appendix of my master's thesis
%%% Author: Viljami Aittomäki

\section{Model size distributions\label{app:model-sizes}}

\begin{figure}[htb]
  \centering
  \makebox[\textwidth][c]{\includegraphics[width=1.1\linewidth]{figures/varNumHistogram/ZZ_variable_number_hist.pdf}}
  \caption{Distributions of chosen model sizes for differrent values of model-size parameters $\alpha$ and $\gamma$.
  $\# 0$ refers to the number of models with no covariates (i.e. not even the mRNA covariate was chosen)
  and $\# \textup{NA}$ to the number of models where the model-size criterion was not met.
  The parameter values had a large impact on the final model sizes. Strict values
  ($\alpha$ close to one and small $\gamma$) generated very large models and had the
  effect of the size-criterion not being met for many genes (larger $\# \textup{NA}$).}
  \label{fig:model-size-distribution}
\end{figure}


%!TEX root = dippa.tex
%%% This file contains the QC plots appendix for my master's thesis
%%% Author: Viljami Aittomäki



\section{Quality control plots\label{app:qc-plots}}


\begin{figure}
	\centering
	\begin{subfigure}{.9\textwidth}
		\centering
	 	\includegraphics[width=1\linewidth]{figures/proteinQCPlots/proteinQCPlots-sample_boxplot.pdf}
	 	\subcaption{\label{fig:protein-sample-boxplot}}
	\end{subfigure}
	\begin{subfigure}{.9\textwidth}
		\centering
	 	\includegraphics[width=1\linewidth]{figures/geneQCPlots/geneQCPlots-sample_boxplot.pdf}
	 	\subcaption{\label{fig:gene-sample-boxplot}}
	\end{subfigure}
	\begin{subfigure}{.9\textwidth}
		\centering
		\includegraphics[width=1\linewidth]{figures/mirnaQCPlots/mirnaQCPlots-sample_boxplot.pdf}
		\subcaption{\label{fig:mirna-sample-boxplot}}
	\end{subfigure}%
	\caption{A figure with two subfigures}
	\label{fig:test}
\end{figure}






%%=========================================================
%% The End  ==DDDDDDD

\end{document}
