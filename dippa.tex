%%%%%%%%%%%%%%%%%%%%%%%%%%%%%%%%%%%%%%%%%%%%%%%%%%%%%%%%%%%%%%%%%%%%
%%%%%%%%%%%%%%%%%%%%%%%%%%%%%%%%%%%%%%%%%%%%%%%%%%%%%%%%%%%%%%%%%%%%
%%                                                                %%
%% An example for writting your thesis using LaTeX                %%
%% Original version by Luis Costa,  changes by Perttu Puska       %%
%% Support for Swedish added 15092014                             %%
%%                                                                %%
%% Explanatory comments in this example begin with                %%
%% the characters %%, and changes that the user can make          %%
%% with the character %                                           %%
%%                                                                %%
%%%%%%%%%%%%%%%%%%%%%%%%%%%%%%%%%%%%%%%%%%%%%%%%%%%%%%%%%%%%%%%%%%%%
%%%%%%%%%%%%%%%%%%%%%%%%%%%%%%%%%%%%%%%%%%%%%%%%%%%%%%%%%%%%%%%%%%%%

%% Uncomment one of these:
%% the 1st when using pdflatex, which directly typesets your document in
%% pdf (use jpg or pdf figures), or
%% the 2nd when producing a ps file (use eps figures, don't use ps figures!).
\documentclass[english,12pt,a4paper,pdftex,elec,utf8]{aaltothesis}
%\documentclass[english,12pt,a4paper,dvips]{aaltothesis}

%% To the \documentclass above
%% specify your school: arts, biz, chem, elec, eng, sci
%% specify the character encoding scheme used by your editor: utf8, latin1

\usepackage{graphicx}

%% Use this if you write hard core mathematics, these are usually needed
\usepackage{amsmath,amsfonts,amssymb,amsbsy}

%% Use the macros in this package to change how the hyperref package below 
%% typesets its hypertext -- hyperlink colour, font, etc. See the package
%% documentation. It also defines the \url macro, so use the package when 
%% not using the hyperref package.
%%
%\usepackage{url}

%% Use this if you want to get links and nice output. Works well with pdflatex.
\usepackage{hyperref}
\hypersetup{pdfpagemode=UseNone, pdfstartview=FitH,
  colorlinks=true,urlcolor=red,linkcolor=blue,citecolor=black,
  pdftitle={MicroRNA regulation in breast cancer},pdfauthor={Viljami Aittom\"aki},
  pdfkeywords={Bayesian analysis, breast cancer, microRNA}}


%% User-added packages
\usepackage[numbers,square]{natbib}
\usepackage[parfill]{parskip}
\usepackage{longtable}
\usepackage[font={small,it},labelfont={sf}]{caption}
\usepackage{subcaption}
\captionsetup[sub]{font={scriptsize},labelfont={sf}}
\usepackage{pdflscape}
\setlength\LTcapwidth{\textwidth}
\setlength{\LTleft}{-20cm plus -1fill}
%\setlength{\LTleft}{-30cm}
\setlength{\LTright}{\LTleft}

%% Define which sections to include
%\includeonly{appendix-QC-plots}







%%=========================================================
%% All that is printed on paper starts here
\begin{document}
\renewcommand{\thesissupervisorname}{Thesis supervisor}

%% Change the school field to specify your school if the automatically 
%% set name is wrong
% \university{aalto-yliopisto}
% \university{aalto University}
% \school{Sähkötekniikan korkeakoulu}
% \school{School of Electrical Engineering}

%% ONLY FOR M.Sc. AND LICENTIATE THESIS: Specify your department,
%% professorship and professorship code. 
%%
\department{Department of Computer Science}
%\professorship{Computational ROFESSUURI}
%%

%% Valitse yksi näistä kolmesta
%%
%% Choose one of these:
%\univdegree{BSc}
\univdegree{MSc}
%\univdegree{Lic}

%% Your own name (should be self explanatory...)
\author{Viljami Aittom\"aki}

%% Your thesis title comes here and again before a possible abstract in
%% Finnish or Swedish . If the title is very long and latex does an
%% unsatisfactory job of breaking the lines, you will have to force a
%% linebreak with the \\ control character. 
%% Do not hyphenate titles.
%% 
\thesistitle{MicroRNA regulation in breast cancer -- a Bayesian analysis of expression data}

\place{Espoo}

%% For B.Sc. thesis use the date when you present your thesis.
%% BUT WHAT IS IT FOR MSc THESIS?
\date{10.10.2016}

%% B.Sc. or M.Sc. thesis supervisor 
%% Note the "\" after the comma. This forces the following space to be 
%% a normal interword space, not the space that starts a new sentence. 
%% This is done because the fullstop isn't the end of the sentence that
%% should be followed by a slightly longer space but is to be followed
%% by a regular space.
%%
\supervisor{Aki Vehtari, professor}

%% B.Sc. or M.Sc. thesis advisors(s).
%%
\advisor{Rainer Lehtonen, associate professor}

%% Aalto logo: syntax:
%% \uselogo{aaltoRed|aaltoBlue|aaltoYellow|aaltoGray|aaltoGrayScale}{?|!|''}
%% Logo language is set to be the same as the document language.
%%
\uselogo{aaltoRed}{''}


%% Create the coverpage
%%
\makecoverpage














%%=========================================================
%% English abstract.
%% All the information required in the abstract (your name, thesis title, etc.)
%% is used as specified above.
%% Specify keywords
%%
\keywords{Bayesian analysis, breast cancer, gene expression, microRNA, microarray}
%% Abstract text
\begin{abstractpage}[english]

Breast cancer is the most common cancer in women and a significant cause of
morbidity and mortality globally. Analyses of gene expression data have
provided insight into the pathogenesis of breast cancer and  intrinsic subtypes
correlating with prognosis have been identified. \\

MicroRNAs are a class of small RNAs that regulate gene expression post-
transcriptionally by targeting messenger RNA transcripts. Recent research has
revealed microRNAs to participate in almost all cellular functions and many
pathological processes, including cancer. Several microRNAs have been
indicated to contribute to breast cancer pathogenesis. \\

Identification of target genes of microRNAs is crucial to understand their
function in normal cell biology and disease. A wide array of methods have been
proposed for computational prediction of microRNA targets. Early methods used
sequence information and recent tools have integrated mRNA and microRNA
expression measurements. Very few published analyses have included both
protein and mRNA expression. \\

This thesis provides a review of microRNA biology and target prediction. A
recently published Bayesian variable selection method is applied for
uncovering putative microRNA targets in breast cancer using microarray
expression data. Results are compared with a recently published analysis of
the same data and previous literature. The analysis shows that the choice of
modeling parameters is not trivial and has impact on results. The importance
of an integrating different data types is highlighted.

\end{abstractpage}

%% Force a new page so that the Finnish abstract starts on a new page
\newpage

%% Abstract in Finnish.  Delete if you don't need it. 
\thesistitle{MikroRNA-säätely rintasyövässä -- ekspressiodatan bayeslainen analyysi}
\supervisor{Professori Aki Vehtari}
\advisor{Dosentti Rainer Lehtonen}
\degreeprogram{Electronics and electrical engineering}
\department{Tietotekniikan laitos}
%\professorship{JA TÄÄ}
%% Avainsanat
\keywords{Bayeslainen analyysi, geeniekspressio, mikroRNA, mikrosiru, rintasy\"op\"a}
%% Tiivistelman tekstiosa
\begin{abstractpage}[finnish]

\textbf{Käännä suomeksi enkkuabstrakti.}

\end{abstractpage}














%%=========================================================
%% Preface

%% Force new page so that the preface starts from a new page
\newpage

\mysection{Preface}
Ja heillä kaikilla oli niin mukavaa.\\

\vspace{5cm}
Helsinki, 10.10.2016

\vspace{5mm}
{\hfill Viljami Aittom\"aki \hspace{1cm}}


%% Force new page after preface
\newpage


%% Table of contents. 
\thesistableofcontents













%%=========================================================
%% Symbols and abbreviations
\mysection{Symbols and abbreviations}

\subsection*{Symbols}

\begin{tabular}{ll}
$N(\mu,\sigma^2)$ & Normal distribution with mean $\mu$ and variance $\sigma^2$ \\
$p(y)$            & Probability density of $x$ \\
$p(y,\theta)$     & Joint probability of $y$ and $\theta$ \\
$p(y|\theta)$     & Conditional probability of $y$ given $\theta$ \\
$\theta$          & Parameter vector for a probability model \\
$w$               & Regression coefficients for explanatory variables \\
$x$               & Explanatory variable (or microRNA expression vector) \\
$X$               & Matrix of explanatory variables (or microRNA expression values) \\
$y$               & Outcome variable (or protein expression vector) \\
$y \sim N(.)$     & Random variable $y$ has probability distribution $N(.)$ \\
$y \propto x$     & $y$ is proportional to $x$, up to a constant factor \\
$z$               & mRNA expression vector
\end{tabular}

% \subsection*{Expressions}
% \begin{tabular}{ll}
% $\nabla \times \mathbf{A}$                 & curl of vectorin $\mathbf{A}$\\
% $\displaystyle\frac{\mbox{d}}{\mbox{d} t}$ & derivative with respect to variable $t$\\[3mm]
% $\displaystyle\frac{\partial}{\partial t}$ & partial derivative with respect to variable $t$ \\[3mm]
% $\sum_i $                      & sum over index $i$\\
% $\mathbf{A} \cdot \mathbf{B}$  & dot product of vectors $\mathbf{A}$ and $\mathbf{B}$
% \end{tabular}

\subsection*{Abbreviations}

\begin{tabular}{ll}
bp          & Base pairs (as a measure of sequence length) \\
nt          & Nucleotides (as a measure of sequence length) \\
miRNA       & MicroRNA \\
MLR         & Multivariate linear regression \\
mRNA        & Messenger RNA \\
NGS         & Next generation sequencing \\
PPVS        & Projection predictive variable selection \\
RISC        & RNA-induced silencing complex \\
RNA         & Ribonucleic acid \\
RPPA        & Reverse phosphatase protein array
\end{tabular}


%% Tweaks the page numbering to meet the requirement of the thesis format:
%% Begin the pagenumbering in Arabian numerals (and leave the first page
%% of the text body empty, see \thispagestyle{empty} below).
%% Additionally, force the actual text to begin on a new page with the 
%% \clearpage command.
%% \clearpage is similar to \newpage, but it also flushes the floats (figures
%% and tables).
%% There is no need to change these
%%
\cleardoublepage
\storeinipagenumber
\pagenumbering{arabic}
\setcounter{page}{1}









%%=========================================================
%% Text body begins


\section{Introduction}
\thispagestyle{empty}

\textbf{REFET PUUTTUU!}

Breast cancer is the most common of female cancers and causes remarkable
morbidity and mortality word-wide. Annually more than 1.5 million
women develop breast cancer. Thus, breast cancer is a major global health
problem.

Cancer is a genetic disease caused by mutations in the genome of the cancer
cells. Some of these mutations can be inherited, while some are somatic, that
is, mutations that arise during the life-time of an individual in the tissue,
where cancer develops. To understand how cancer develops, the tumor evolution
of breast cancer, it is of paramount importance to identify genes in which
mutations initiate breast cancer development. It is already known that many of
these mutations perturb the expression of the gene in question. Previous
expression studies comparing the expression profiles of normal breast tissue
and breast cancer tissue have revealed ...

MicroRNAs are short single-stranded RNA-molecules that act in
post-transcriptional regulation of gene expression. They have been found in a wide
array of species including mammals, nematodes, insects, plants and even
viruses. MicroRNAs are higly conserved in evolution and function in diverse
physiological, developmental and pathological processes. MicroRNAs have been
indicated as potential causative agents in numerous diseases, including
several types of cancer. Therefore, the study of microRNAs and their function
in cancer can offer insights into tumorigenesis and cancer progression as well
as potential new biomarkers and treatments.

A plethora of different methods for computational identification of microRNA
target genes have been proposed. Early methods are mainly based on sequence
similarity of the microRNA and messenger RNA of putative target genes. More
recent methods use expression profiles to either augment sequence data or as
sole predictors for interaction. Expression methods are mostly based on
correlation, or a derivative thereof such as regression, of microRNA and gene
expression. The combinatorial nature of miRNA action, however, makes
expression based strategies difficult, as one transcript may be regulated by
several different miRNAs simultaneously and the contribution of each miRNA may
be small. Additionally, most miRNAs have a large number of target transcripts.

The aim of this thesis was to apply a recently proposed Bayesian variable
selection method to microRNA target discovery in breast cancer. The variable
selection was applied in the context of Bayesian regression of expression
profiles to elucidate which microRNAs are highly relevant in regulating
protein expression levels. The results were compared with a similar recent
study and also with known validated microRNA targets as well as microRNAs
previously indicated in breast cancer.


%\clearpage (\include implicity does a \clearpage)
%!TEX root = dippa.tex
%%% This file contains the Background section of my master's thesis.
%%% Author: Viljami Aittomäki


\section{Background}\label{background}











\subsection{Gene expression}\label{gene-expression}

Genetic information is encoded in molecules of deoxyribonucleic acid (DNA). A
gene is a section of DNA that serves as a template for a functional
ribonucleic acid (RNA) molecule. Gene expression refers to this process of
synthesizing a functional end-product from the information contained in gene.
DNA and gene expression serve as the basis of all currently known life.

Most of gene expression is dedicated to making proteins. The Central Dogma of
Molecular Biology, postulated by Francis Crick in 1970, describes the general
schema of how genetic information flows from genes to proteins; DNA is first
transcribed into messenger RNA (mRNA), which is then translated into a polypeptides,
which ultimately form proteins \citep{Crick1970}. The flow is not strictly
one-directional though, as there exist reverse transcriptases, enzymes that
synthesize DNA from an RNA template.

All genes are used to form RNA, but not all are protein-coding. These
noncoding genes give rise to noncoding RNA, a class of RNA molecules that
are mostly involved in aiding and regulating the expression of other genes.
Different types of noncoding RNA and their functions are summarized in
table \ref{table:rnas}.

The genome refers to the whole genetic material of an organism or an individual.
The human genome has been suggested to contain approximately 20 500 protein-coding
genes, which encompass only around 1.5\% of the whole genetic sequence
\citep{Clamp2007}. Therefore, the vast majority of the human genome was previously thought
to be without function and referred to as "junk DNA". More recently, however,
it has become increasingly evident, that noncoding portions of the genome
are often functional and possibly have important cellular functions \citep{ENCODE}.




\subsubsection{Regulation of gene expression}\label{regulation-of-gene-expression}

Regulation of gene expression refers to controlling the abundance of
the gene end-product a cell produces. This control is paramount so that cells
can respond to external signals, changes in their environment, damage and also 
\textbf{pass through} different developmental stages. Gene expression can be regulated
at any stage of the process and regulation can be divided into transcriptional
and post-transcriptional regulation.

Transcriptional regulation with e.g. TFs, methylation, adenylation, splicing.

Post-ranscriptional

\begin{itemize}
  \item transcription factors
  \item post-transcriptional regulation
  \item translational factors
  \item protein degradation
\end{itemize}



\subsubsection{Ribonucleic acids}\label{ribonucleic-acids}

\textbf{SIIRRÄ TÄÄ TOHON EKAAN YLÄKAPPALEESEEN?}

Messenger RNA.

Long non-coding RNA

Short non-coding RNA.

MicroRNAs (miRNAs) and small interfering RNAs (siRNAs) are components of the
so called RNA interference (RNAi) pathway, which is a mechanism for regulating
gene expression post-transcriptionally. miRNAs and siRNAs are practically
interchangeable as substrates for RNAi, both act as target mRNA recognizing
templates, but have different biogenesis, where miRNAs are cut from endogenous hairpin
strucutres and siRNAs are processed from exogenous long double-stranded RNAs (dsRNAs).
\cite{Du2005} MicroRNAs are the focus of this study and are discussed in more
detail in the next section, as is also the origin of RNAi.

The Argonaute (Ago) family of proteins is closely associated with small RNAs and
Ago act as effectors in RNAi \cite{Ha2014}.







\subsection{MicroRNAs}\label{micrornas}

MicroRNAs (miRNAs) are a family of endogenous (i.e. coming from within the
cell itself) noncoding small RNA molecules that function as post-transcriptional
regulators of gene expression \citep{Ambros2004}. In their
functional, mature form miRNAs are single stranded and approximately 22
nucleotides long. MicroRNAs are not translated into protein. Instead, they
have an important role in regulation of gene expression in a wide range of
physiological, developmental and pathological processes \citep{Bartel2009}.
MicroRNAs assert their regulatory function by destabilization and degradation
of target messenger RNA (mRNA) molecules and inhibition of mRNA translation
\citep{Fabian2010}.



\subsubsection{Discovery of microRNAs}

The first known microRNA, lin-4, was discovered in 1993 by two research groups
studying the larval development of the nematode \emph{Caenorhabditis elegans}.
The researchers noted that lin-4 does not encode a protein, but instead
produces a pair of small RNAs, the longer of which was proposed to be a
precursor to the shorter one \citep{Lee1993}. The RNAs encoded by lin-4 were
noted to have conserved antisense complementarity in several sites of an
untranslated region of the lin-14 mRNA, and these sites were found to be
necessary for the normal repression of lin-14 expression by lin-4
\citep{Lee1993,Wightman1993}.
%It
%should be noted, that one of these groups also showed that lin-4 reduces the
%amount of the LIN-14 protein -- the end-product of the lin-14 gene -- without
%significantly affecting the cellular concentration of the lin-14 mRNA
%\citep{??}. %oks tää Wightman1993:ssa?
%We will return to the issue of microRNA action in chapter \ref{microrna-function}.

Let-7, the second microRNA to be discovered, was also first found in \emph{C.
elegans}, however, homologues of let-7 were found in several other species
\citep{Pasquinelli2000}. Soon after, numerous microRNA genes were found across
a variety of species, and a registry was set up to serve as a comprehensive
knowledgebase of published microRNAs and as an independent authority on
microRNA nomenclature \citep{GriffithsJones2004}. This registry later became
miRBase, the de facto reference database of known microRNAs, and now provides
sequence data, annotations as well as predicted and validated target genes for
miRNAs \citep{Kozomara2014}.

The number of known small RNAs has since vastly expanded
and microRNAs have been found in more than 200 organisms, including
all studied animals, plants \citep{JonesRhoades2006} and viruses \citep{Grundhoff2011}. 
The number of records in miRBase has risen exponentially
from %218 precursor and
218 mature miRNAs in the first release in 2002 to %28645 precursor and
35828 mature miRNAs in 223 species in the most recent version (v21, released June
2014 \citep{VanPeer2014,MiRBaseWeb}), illustrating the vast amount of novel
microRNA molecules discovered recently, mainly due to increasing efforts in
and availability of sequencing. miRBase lists 2588 known human miRNAs at the
time of writing this thesis. A web service called miRBaseTracker has been
developed by \citet{VanPeer2014} for updating miRNA nomenclature and
annotations across different versions of miRBase to allow correct comparison
of miRNA study results and reannotation of miRNA analysis platforms.



\subsubsection{MicroRNA genomics}\label{microrna-genomics}

MicroRNAs are highly conserved in evolution \citep{Bartel2004}, for example,
55\% of \emph{C. elegans} miRNAs have homologues in humans
\citep{IbanezVentoso2008}. Interestingly, the
appearance of multi-cellular organisms appears to correlate with the
appearance of the microRNA machinery, and organism complexity and speciation
are correlated with miRNA complexity, suggesting that microRNAs have had a
crucial role in the development of complex organisms \citep{Lee2007}.

MicroRNAs are found in varying genomic contexts. Approximately 50\% of
mammalian miRNAs are located in close proximity to other miRNAs and form
polycistronic miRNA clusters that are transcribed simultaneously. Some miRNAs
reside in the genome as dedicated miRNA genes, with their own promotor regions.
\citep{Kim2009} MicroRNAs and miRNA clusters can be situated in exons or
introns of nonconding genes and some are found in introns of protein-coding genes
\citep{Du2005}. MicroRNAs located in introns are sometimes referred to as
mirtrons \citep{Ruby2007}.

MicroRNAs are expressed in all tissues, however, different tissues
have differing miRNA expression profiles \citep{Krol2010}. Many microRNAs also
have differing expression in different developmental stages
of some organisms, e.g. let-7 functions to control the transition
from the second larval stage to \textbf{WHAT} in \emph{C. elegans} \citep{Bartel2004}.

% Tight evolutionary control, extensive transcriptome
% targeting, and the fact that miRNAs and their associated proteins are one of
% the most abundant molecules in the cytoplasm \citep{Bartel2004} highlight the
% importance of microRNAs.






\subsubsection{MicroRNA biogenesis}\label{microrna-biogenesis}

The canonical pathway of microRNA biogenesis is illustrated in figure
\ref{fig:mirna-biogenesis} and is presented here as reviewed by \citet{Bartel2004},
\citet{Melo2011}, \citet{Ha2014}, and many others. 

Most microRNAs are transcribed from genomic DNA by RNA polymerase II to form a
long primary microRNA (pri-miRNA) molecule \citep{Lee2004}. The pri-miRNA molecule
contains a hairpin structure, with a 33-bp double-helix stem and a terminal
loop, and flanking single-strand sequences, which are several hundreds or
thousands of nucleotides long \citep{Kim2005}. Some miRNAs within Alu repeat elements
can be transcribed by RNA polymerase III \citep{Borchert2006}.

The pri-miRNA is cut by the ribonuclease Drosha to form %one or, in the case of polycistronic miRNAs, several
a pre-microRNA (pre-miRNA), which consists of the hairpin and is
approximately 70 nt long \citep{Lee2003}. A typical pre-miRNA structure is
shown in figure \ref{fig:premirna-structure}. Drosha is aided by the essential
cofactor DGRC8 (the protein product of a gene deleted in DiGeorge syndrome \citep{Shiohama2003})
and they form a complex known as the Microprocessor \citep{Gregory2004}.
The hairpin is
then exported from the nucleus to the cytoplasm by exportin 5 (XPO5), which is
a member of the nuclear transport receptor family \citep{Lund2004}.

In the cytoplasm, the ribonuclease Dicer cleaves out the loop of the hairpin
to form a 22-nt-long double-stranded miRNA:miRNA* duplex corresponding to
the stem of the hairpin \citep{Bernstein2001}.
Dicer associates with a cofactor, in humans TRBP (Tar RNA-binding protein),
which is not required for effective dicing of the pre-miRNA,
but acts to physically bridge the Dicer to an Argonaute protein
% for further miRNA processing
\citep{Chendrimada2005}.

The duplex is then bound by the Argonaute protein, in mammals one of Ago1
through Ago4, forming what is called the RNA-induced silencing complex (RISC).
The RISC is a protein complex containing Dicer, TRBP and Ago \citep{Gregory2005}.
Ago, aided by Dicer and TRBP, unwinds the strands of the duplex and retains one of
them. The retained strand is known as the guide strand (or miRNA). The other
strand, called the passenger (or miRNA*), is released and typically degraded \citep{Du2005}.
In some instances, either one of the strands can become the guide
or both can be used \citep{Czech2009}.
%Notably, the Dicer cleaving, duplex unwinding and eventual
%mRNA regulation activity are, in fact, all coupled and performed by RISC
%\citep{Gregory2005}.

Not all miRNAs are generated through this canonical pathway of microRNA
biogenesis. Some miRNAs are not dependent on Drosha, such as mirtrons, which
are cut into pre-miRNA by the spliceosome, a molecular complex responsible for
removing introns (and sometimes exons) from precursor mRNA \citep{Ruby2007}. The
biogenesis of miR-451 is independent of Dicer; miR-451, which
has an important role in erythropoiesis, is cleaved by Ago2 \citep{Cheloufi2010}.






\subsubsection{MicroRNA mechanism}\label{microrna-mechanism}

The RISC is the effector of RNA interference. Ago functions as the catalytic
engine of the RISC and the miRNA bound to it guides the RISC to target
messenger RNAs \citep{Filipowicz2008}. The microRNA mechanism of action is
illustrated in figure \ref{fig:mirna-action}.

Target recognition is based on sequence complimentarity of the miRNA and mRNA.
In animal miRNAs this complimentarity is almost always limited \citep{Ambros2004}.
Nucleotides at positions 2-8 of the 5' end of the microRNA have been found
crucial to target mRNA matching; these nucleotides are termed the miRNA "seed sequence".
miRNA target sequences are mostly located in the 3' UTR (untranslated region)
of the mRNA transcript, but in some instances target sites also reside in the
coding region or 5' UTR of the mRNA \citep{Bartel2009}.
\textbf{TÄÄ KAPPALE PAREMMIN, MUITA FIITSUJA MUKAAN, KOSKA NIITÄ KÄYTETÄÄN PREDICTIOSSA, KTS GRIMSON.)}

The action of microRNAs is \textbf{related} through inhibition of mRNA
translation or destabilization and subsequent degradation of mRNA. The exact
mechanisms by which the miRNA and Ago induce translational repression or
destabilization of mRNA are unclear \citep{Filipowicz2008}. Translational
inhibition was earlier believed to be the major form of miRNA action in animals,
but recent evidence suggests that mRNA destabilization dominates \citep{Guo2010}.
In some cases the mRNA is directly cleaved by Ago. Ago2 is the only mammalian
Argonaute capable of cleavage, and this is assumed to require
extensive base-pair matching between the miRNA and mRNA \citep{Du2005}.
However, mRNA cleavage appears to be rare in animals (while much more common
in plants).

mRNAs bound to RISC accumulate in so called processing bodies (P-bodies),
which are known sites of mRNA catabolism and translational repression in the
cytoplasm. The localization in P-bodies, however, appears to be a consequence
of RNA silencing, not the cause, and is reversible  \citep{Eulalio2007}.

Interestingly, several alternative mechanisms of action for microRNA have been
reported. For example, some miRNAs can increase the translation of target mRNA instead of
repressing it \citep{Vasudevan2007}, miR-373
was found to target DNA promoter areas and act to induce gene transcription
\citep{Place2008}, and miR-328 targets a protein to prevent inhibition of mRNA
translation \citep{Eiring2010}. This illustrates the complexity and diversity of
microRNA biology and gene regulation in general.






\subsubsection{MicroRNA function}\label{microrna-function}

MicroRNAs have been found to participate in regulation of almost all studied
cellular processes, including embryo development, cell proliferation,
apoptosis and metabolism, by regulating protein expression \cite{}.

miRNAs assert extensive control over the transcriptome. More than 60\% of
human mRNA transcripts are predicted to be regulated by miRNAs and a single
transcript has target sites for several different miRNA families
\citep{Friedman2009}. Furthermore, a single microRNA can have as many as hundreds or
thousands of target mRNAs. The effect of a
single miRNA on its target tends to be subtle, usually causing less than a
2-fold change in protein expression \citep{Baek2008}. However, a single
miRNA can have multiple binding sites in one mRNA and multiple miRNAs
acting on the same target have additive, and in some cases even synergistic
effects \citep{Bartel2009}.

% , and depending also on 
% of multiple miRNAs acting in tandem can, however, be much more pronounced and
% even multiplicative, achieving changes in excess of 10-fold \citep{}.

% A recent review concluded that mRNA degradation is the
% predominant form of miRNA action in mammals \citep{Guo2010}.

It should be noted, however, that the functional role and importance of many
miRNA-mRNA interactions are unknown, even for validated interaction pairs.
Uncovering these roles is challenging because of the subtle regulatory effects
that miRNAs often have and, additionally, because of the complexity and
robustness of most cellular regulatory networks \citep{Bartel2009}.
Furthermore, experimentally validated target mRNAs exist only for a subset of
all known microRNAs. Nonetheless, discovering miRNA targets is a critical
step in understanding their function.

The dysregulation of microRNAs is associated with many human diseases
\citep{Jiang2009,VAIHDATÄMÄREFE}. The first example of such an association was, in
fact, found in cancer, when miR-15 and miR-16 were found to be suppressed in
chronic lymphocytic leukemia \citep{Musilova2015}. A disease-promoting role
for miRNAs has since been implicated in many different cancers, including
breast cancer \citep{Melo2011}.









\subsection{Cancer}\label{cancer}

Cancer is a disease of uncontrolled overgrowth of a population of cells. It is
generally viewed as a genetic disease, albeit it is mostly not inherited, as it is
caused by  mutations in the genome of the tumor. These mutations cause malfunction
and dysregulation of the genetic machinery regulating cellular functions, such
as cell proliferation, differentiation and apoptosis, resulting in
unregulated growth and malignant tumor formation.

There are several classes of genes that influence tumor growth, the two main
categories being oncogenes and tumor suppressors. Oncogenes were first
identified in retroviruses and later shown to be proto-oncogenes, which by
mutation develop into oncogenes whose over activity promotes tumor growth
\citep{Varmus1988}. Tumor suppressor genes are often regulators of cell
proliferation or other so called housekeeping genes that work to ensure the
proper functioning of cells and the apoptosis of misbehaving ones. The
inactivation of these genes can lead to tumor progression. The existence of
tumor suppressors was first hypothesized by Alfred Knudson, who formed the “two-
hit hypothesis” while studying the epidemiology of retinoblastoma
\citep{Knudson1971}. He suggested that, for cancer to develop, both copies of
a tumor suppressor gene should become inactivated and that in inherited
cancers one mutation is acquired in the germline and the other occurs in
somatic cells, whereas in sporadic cancers both mutations happen in somatic
cells.

The idea of oncogenes and tumor suppressor genes was later expanded on by
Douglas Hanahan and Robert Weinberg in their seminal article The Hallmarks of
Cancer \citep{Hanahan2000}. The hallmarks are a set of six features which
tumors often acquire to become malignant. The features are: sustaining
proliferative signaling, evading growth suppressors, resisting cell death,
enabling replicative immortality, inducing angiogenesis, and activating
invasion and metastasis. Weinberg and Hanahan postulated that at least three
of these six features are required for invasive cancer to develop.

Recently, Hanahan and Weinberg revised the hallmarks with two new upcoming
hallmarks and two enabling characteristics \citep{Hanahan2011}. The new
hallmarks are deregulation of cellular energetics and avoiding immune
destruction. The enabling characteristics of malignant tumors are genome
instability and mutation, and tumor-promoting inflammation through recruition
of the immune system. Genome instability and mutation is of special importance
as much of cancer and tumor research focuses on identifying mutated or
dysregulated genes that promote tumor progression.



\subsubsection{Breast cancer}\label{breast-cancer}

Breast cancer constitutes a significant health issue globally. It is the most
common cancer in women and the second most common cancer overall;
approximately 1.7 million women develop breast cancer annually world-wide, and
in 2012 there were 522 000 breast cancer-related deaths
\citep{Ferlay2015}. In Finland there were 5008 new cases of breast cancer
 and 815 breast cancer-related deaths in 2014 \citep{Syoparekisteri}.

Most breast cancers are sporadic; only 5-7\% of breast cancer cases are of
familial type \citep{Melchor2013}. However, 15-30\% of breast cancer patients
have a family member or relative with breast cancer. This is
mostly due to the high frequency of breast cancer in many western populations,
but also suggests that there are unknown genetic factors and environmental
factors that have an impact on breast cancer development. Indeed, breast
cancer is a hormone-related disease and hormonal factors are known to have an
impact on breast cancer risk. The most important of these are
estrogen and progesterone.

The most important heriditary forms of breast cancer are those related to a
tumor predisposition syndrome caused by mutations in the breast cancer 1
(BRCA1) and BRCA2 genes, which explain about 25\% familial breast cancer in
many populations. These, however are rare on the population level and familial
clusering of beast cancer is also caused by moderate risk and low risk genetic
variations, which are much more common \citep{Melchor2013}.


\subsubsection{Breast cancer classification}\label{breast-cancer-classification}

The basic classification of cancer is based on the site -- that is, the organ
or tissue, such as breast or colon, where the primary tumor develops. Cancers
occurring within each organ are, however, heterogeneous in their nature and
have different behavior and prognosis. Thus, tumors are secondarily classified
by morphology, the microscopic structure of the cancer tissue.

The morphological classification of breast cancer is based on the WHO
classification from 2003 and includes altogether 19 histological subtypes of
invasive breast cancer \citep{Tavassoli2003,Weigelt2009}. Of these invasive
ductal carcinoma not otherwise specified (IDC NOS), accounts for the by far
largest histological group. Additionally, the WHO classification of
breast cancer includes the
TNM classification; characterization of the primary tumor (T), lymph node
status (N) and distant metastasis (M) and stage grouping based on the TNM
data. cTNM (or TNM) refers to a clinical TNM classification -- based on imaging studies,
surgical exploration and similar studies -- and pTNM to a pathological
classification -- based on postsurgical pathological analysis of tissue
biopsies. All invasive cancers are also graded into well (I), moderately (II)
or poorly (III) differentiated tumors based on microscopical examination
% based on three features; tubule formation as an expression of glandular
% differentiation, nuclear pleomorphism and mitotic counts
with less differentiated tumors having worse prognosis \citep{Tavassoli2003}.

In the clinic many other tumor characteristics are used. These
include the age of patient, lympho-vascular invasion as well as expression of
estrogen (ER) and progesterone receptors (PR), and Her2, which are routinely
studied for breast cancers. Together these data can be used to group patients
into risk categories for prognosis and choice of treatment using for example
StGallen criteria \citep{Goldhirsch2007} or NIH criteria \citep{Eifel2001} or others.
The estrogen receptor also has a special role in treatment, as tumors that higly express ER
(termed ER-positive and covering 70\% of all cases) can be given antiestrogens
as endocrine therapy.

One of the problems with morphological classification of breast tumors is that
over 50\% do not show any particular features and are classified as IDC NOS,
in spite of the fact that tumors in this large group have very different
clinical outcomes. As mentioned above, tumors of the same tissue -- and even
of the same histological type -- are heterogeneous in nature and personalized
molecular diagnostics are required to exploit more targeted treatments. This
is even more critical in the case of de novo treatment resistance, where
tumors develop a molecular mechanism for resisting pharmacological therapies.
This is especially common with targeted therapies, for example ER-positive
tumors may acquire mutations to ER rendering antiestrogen therapy ineffective
\citep{Oesterreich2013}.

More recently, expression profiling has led to a suggestion
of new classification of breast cancers \citep{Perou2000,Sorlie2001}. By studying 65
samples from breast tumors with expression profiling Perou et al.
distinguished four subgroups based on gene expression, namely
ER+/luminal-like, basal-like, Erb-B2+ and normal breast \citep{Perou2000}.

Expression profiling has also led to development of new prognostic tests to
help to determine the need of adjuvant chemotherapy. These tests include
Oncotype DX (a 21-gene recurrence score), MammaPrint (a 70-gene test) and PAM50
(a 50-gene test), and have been evaluated by several groups and suggested to be
valid and promising, but their utility in clinical decision making remains
unclear \citep{Azim2013}. The PAM50 molecular classification ... 

\textbf{TÄSTÄ LISÄÄ? AINAKIN SUBTYPE KORRELAATIO PROGNOOSIIN JA HOITOON PITÄISI
KERTOA?}






\subsubsection{MicroRNAs and cancer}

Research has shown microRNAs to have important roles in tumor initiation,
progression and metastasis \citep{Lin2015}. MicroRNA expression signatures
correlate with numerous cancer features, such as tissue, staging, 
progression, prognosis and treatment response, and all studied cancers
have had miRNA expression profiles differing from healthy tissue, including breast
\citep{Calin2006}. In fact, microRNAs appear to be globally underexpressed in
cancers \citep{Lu2005}. Therefore, it seems clear that microRNAs participate
in many of the pathways resulting in the hallmarks of cancer.

Similarily to protein-coding genes, microRNAs can function as tumor
suppressors or oncogenes \citep{Lin2015}. In a meta-analysis of dysregulation
of miRNAs in breast cancer van Schoonveld found major oncogenic microRNAs in
breast cancer to include miR-10b, miR-21, miR- 155, miR-373, and miR-520c
\citep{vanSchooneveld2015}. They also list nine miRNAs as major tumor suppressors for breast cancer,
namely miR-125b, miR-205, miR-17-92, miR-206, miR-200, miR-146b, miR-126,
miR-335, and miR-31.

The genetic mechanisms for microRNA involvement in cancer are varied,
including mutations in miRNA or target mRNA sequence, chromosomal
rearrangements of the miRNA-encoding DNA regions and epigenetic changes in DNA
methylation or histones, leading to aberrant miRNA expression
\citep{Calin2006,Melo2011}. For example, a single-nucleotide polymorphism
in the microRNA miR-196a2 has been found to be associated with breast cancer
risk \citep{Gao2011}. A mutation in the sequence of estrogen receptor alpha,
in the target site of miR-453, has been suggested to be associated with a
lower breast cancer risk \citep{Tchatchou2009}, as an example of a mutation in
a target transcript affecting miRNA function. MicroRNA function can also be
altered by abnormalities in the miRNA-processing machinery, for example, a
mutation in the Dicer gene causes a tumor predisposition syndrome known as
DICER1 syndrome \citep{Slade2011}. Another example of this is apparent
dysregulation of Dicer and Drosha in breast cancer
\citep{Yan2012}.

The different subtypes of breast cancer, explained above, reflect the genetic
background of the tumor and, accordingly, the subtypes differ in their gene
expression profiles. This also applies for miRNA expression, the different
intrinsic subtypes have different miRNA expression profiles, suggesting their
importance in breast cancer evolution \citep{Blenkiron2007}. de Rinaldis et al
identified a 46-miRNA signature that could be used in differentiating the
intrinsic subtypes from each other \citep{deRinaldis2013}. In addition to
tumor development, many miRNAs have been found to modulate the response to
breast cancer therapies. These include chemotherapy, antiendocrine therapy,
radiotherapy and targeted therapies.

Accordingly, miRNAs have been studied as biomarkers for diagnosing cancer and
cancer prognosis \citep{}. Emmadi et al recently found let-7 expression to be negatively
correlated with the Oncotype DX recurrence score in breast cancer
\citep{Emmadi2015}. This corroborated with the earlier finding of let-7 being
downregulated in breast cancer stem cells (tumor cells possessing the ability
of self-renewal) \citep{Yu2007} and later research suggesting let-7 to act as
a tumor suppressor. Several miRNAs have also been associated with breast cancer
metastasis \citep{Chen2016}.

\textbf{JOTAIN CELL-FREE BIOMARKEREISTA?}

MicroRNAs also show promise as a novel therapeutic tool, several studies have
proposed miRNA-based cancer treatments in animal models, including for breast
cancer \citep{VanRooij2014}. However, more research this area is needed before
microRNA treatments are ready for the clinical setting.

% Cell-free biomarker:
% Lawrie C.H., Gal S., Dunlop H.M., Pushkaran B., Liggins A.P., Pulford K., Banham A.H., Pezzella F., Boultwood J., Wainscoat J.S., et al. Detection of elevated levels of tumour-associated microRNAs in serum of patients with diffuse large B-cell lymphoma. Br. J. Haematol. 2008;141:672-675.

% Van Rooij E., Kauppinen S. Development of microRNA therapeutics is coming of age. EMBO Mol. Med. 2014;6:851-864.
%  Garzon R., Marcucci G., Croce C.M. Targeting microRNAs in cancer: rationale, strategies and challenges. Nat. Rev. Drug Discov. 2010;9:775-789.







\subsection{Measuring gene expression}\label{measurement-of-gene-expression}

The physical measurement of gene expression can be done on either the level of
messenger RNA molecules or protein molecules present in cells. Although
proteins are the eventual effectors molecules within cells -- atleast for
protein-coding genes -- often gene expression is synonymous with mRNA
expression, since measuring mRNA abundances is significantly easier than
measuring protein abundances. This is due to the chemistry of base-pair
hybridization and the relative ease of replicating DNA (or RNA) sequences by
exploiting cellular machinery evolved for this purpose.

The general assumption has been that mRNA expression is representative of gene
expression and that changes in mRNA abundances also reflect changes in protein
abundances and, therefore, cellular processes. This assumption has recently
been challenged by experiments indicating that the expression of mRNA and
correspoding protein correlate poorly in general \citep{tahanOliNiitaRefeja}.
There are also contrary findings of modest to good correlation
and one study suggested that mRNA-protein correlation is generally higher for
genes that have differing mRNA expression between studied conditions
(e.g. cancerous versus healthy tissue) \citep{seYksPaperi}.

Nonetheless, Payne recently concluded that "proteome and transcriptome abundances are not
sufficiently correlated to act as proxies for each other" and that
most of this difference is likely caused by biological regulation and not
by measurement technology \cite{Payne2015}.
% This regulation can be post-transcriptional, translational
% or protein-degradation related, as discussed above.
Therefore, it is interesting, even necessary, to integrate
measurements from different parts of this process -- for example
mRNA, microRNA and protein abundances -- to gain new insights
into biological processes.

Techniques shortly in one paragraph:
\begin{itemize}
  \item blotting, qPCR
  \item microarrays
  \item sequencing-based methods
  \item protein arrays (RPPA)
\end{itemize}





\subsubsection{Microarrays}

\begin{itemize}
  \item two-color arrays
  \item oligonucleotide arrays
  \item microarray analysis
  \begin{itemize}
    \item preprocessing and normalization
    \item probe annotation problems
  \end{itemize}
\end{itemize}



\subsubsection{MicroRNA detection}

The same methods that have been employed for measuring mRNA (i.e. gene)
expression are generally applicable for measuring microRNA expression as well,
ranging from northern blotting to microarrays and more recent studies using next
generation sequencing (NGS) \citep{Huang2011}. However, as Hunt and colleagues
in a recent and thorough review point out, there are several challenges in
detecting miRNAs in particular \citep{Hunt2015}.

MicroRNAs are very short and only comprise approximately 0.01\% of RNA
typically extracted from a sample \citep{Dong2013}. This implies that miRNA
detection must be highly sensitive. MicroRNAs from the same family can differ
by only one base, which in turn requires high specifity to be able to
distinguish between family members. On the other hand, variation in miRNA
processing can result in slight sequence variations, or isoforms, of a single
miRNA, also known as isomiRs \citep{StaregaRoslan2011,Lee2010}. This means
high specifity or an incorrect reference sequence (e.g. that of a 
weakly-expressed isomiR) used for detection can cause inaccurate measurements.
IsomiRs may also have different functions resulting from altered target
specifity \citep{Chugh2012}. The existence of the pri-miRNA, pre-miRNA and
mature miRNA molecules provides an additional challenge for measurement
methods.
%, although differentiation between these maturation stages is not necessarily required.

Many of the challenges mentioned are solved by NGS, which is sensitive and
reliable in quantifying known microRNAs and identifying novel ones
\citep{Huang2011}. Sequencing can detect variations of even one nucleotide and
does not necessarily depend on previously identified sequences. However, not
all identified short RNAs are functional miRNAs, and NGS conveys its own set
of problems relating to high cost, significant computational complexity and
validation efforts to distinguish relevant data from noise
\citep{Chugh2012,Hunt2015}.

\textbf{TÄHÄN VIELÄ ERI PLATFORMIEN VERTAILUA JA PREPROSESSOINNISTA?}





\subsection{Identification of microRNA targets}

% This section presents computational methods that have been used to predict
% putative target genes for microRNAs and regulatory networks between genes and
% microRNAs.

% Considering the recent large body of research on microRNAs, and their
% potential utility as biomarkers and treatment targets, it is not surprising
% that a plethora of computational tools have been released to aid in microRNA
% research.

% A recent review covering many published methods for different tasks
% has been written by Akhtar et al \citep{Akhtar2016}.

Recognizing the targets of microRNAs is essential to understand their
biological function and role in disease such as cancer. Many target
interactions have been found in experimental laboratory studies. Common
methods for such studies include using cell lines and introducing exogenous
miRNAs or suppressing endogenous ones and measuring the effects on mRNA or
protein expression. For a dateiled review of experimental methods, see for
example Thomson et al \citep{Thomson2011}.

Several databases exist that list currently known experimentally validated
microRNA targets. Examples include DIANA-TarBase \citep{Vlachos2015} and
MirTarBase \citep{Chou2016}, which are both manually curated from published
literature, and MiRWalk \citep{Dweep2015}, which combines data from several
other databases using text mining.

Although recent advances in high-throuput methodologies, such as CLIP-seq,
have significantly increased the scale of experimental studies, experimental
identification of microRNA targets remains laborous and costly and many
methods rely on computational processing of results \citep{Vlachos2015}. To
this end, a wide range of computational tools have been developed to aid in
miRNA taget discovery. This section presents an overview of published target
prediction methods. For more in-depth reviews, see for example
\citep{Yue2009,Muniategui2012}.




\subsubsection{Sequence-based target prediction}

From a machine learning perspective, pair-wise prediction of miRNA-mRNA
targets is essentially a classification problem. The goal is to identify a set
features (both of the miRNA and mRNA) that allows classifying mRNAs as either
a target or a non-target of any given miRNA. Early (and many recent) target
prediction methods use sequence information to derive these features, as
target recognition of RISC is guided by sequence complimentarity of miRNA and
mRNA. Table \ref{table:sequence-methods} lists examples of sequence-based methods.

\begin{table}
  \caption{Examples of sequence-based approaches to miRNA target prediction.
  See text for details on features. Ref. $=$ Reference}
  \label{table:sequence-methods}
  \centering
  \begingroup\small
  \begin{tabular}{ llllp{4cm} }
    \\[-1ex] \hline\hline
    Name & Ref. & Type & Features Used & Additional Notes \\
    \hline
    TargetScan & \citep{Agarwal2015} & Rule based & i, ii, iii & The first published prediction method. \\
    mirTarget & & SVM & i, ii, iii, iv, v &  \\
    rna22 & & HMM & ??? &  \\
    Kallen juttu? & & ? &  &  \\
    \hline\hline
    \end{tabular}
    \endgroup
\end{table}

Most sequence-based approaches are essentially rule-based filters, where
features of both the miRNA and mRNA sequence are used to narrow down candidate
target lists \citep{Yeu2009}. These features are derived from earlier
experimental knowledge. Commonly used features include: (i) sequence
complimentarity between the seed region of the miRNA and 3' UTR of the mRNA
(see section \ref{mirna-mechanism}), (ii) seed matches in the coding region or
5' UTR of the mRNA, (iii) evolutionary conservation of seed matches between
species, (iv) target site accessibility and (v) free energy of the bound
miRNA-mRNA duplex.

An example of a rule-based predictor is illustrated in figure \ref{fig:rule-flow}.
These rule-based prediction methods are unsupervised, i.e. no training data
is used to form the classifier. Instead, the relevance of the used features
is decided by the method's authors. 

Several methods use a supervised approach, where a training data set
consisting of experimentally validated targets and non-targets -- obtained
from literature or expression data sets -- is used to train
a classifier. Most commonly a support vector machine (SVM) is used
as the classifier. The features used for classification are
similar to rule-based tools, namely sequence features, but
supervised learning allows the inclusion of many more features.
For example mirTarget uses a set of 113 features including seed matches,
conservation and a range of different sequence features from different
parts of the miRNA sequence \citep{mirTarget}.


More complex tools have used hidden markov models (HMMs) to 
train...

One tool, SEJOKU, uses a semisupervised method, where a very strict
rule-based filtering is used to form a training set, which then
is used to train a hidden markov model (HMM). The HMM is used
to model the sequence of target sites for groups of coexpressed miRNAs
and the output is a likelihood that a given mRNA is a target of said
group of miRNAs. The advantage of this approach is that it models
the effect of several miRNAs together.

Another tool using a HMM, is NONONONOO. JOKA TEKEE JOTAIN.


For a more thorough
review of several different sequence features and algorithms using them,
see the reviews by Yue et al \citep{Yeu2009} and Bartel \citep{Bartel2009}.

There are some drawbacks in using target-site prediction to identify microRNA
targets. Firstly, using conservation results in nonconserved and novel sites
being disregarded. Nonetheless, it is perhaps the most used feature.
Additionally, methods requiring some degree of seed region matching
cannot identify miRNA targets without seed matches; while these appear
rare, they should not be discounted altogether \citep{Bartel2009}. Finally,
rule-based methods are static do not account for differing miRNA
and mRNA expression profiles in various tissues and disease states (although
some methods, such as ToppMir, do include this type of data). Sequence-based
predictions also need to be regularly updated to reflect the latest knowledge
of miRNA and mRNA sequences.


% - A seed match does not always confer repression by the matching miRNA.\citep{Grimson2007} %MicroRNA targeting specifity in mammals






\subsubsection{Target prediction based on expression data}

Integrating expression data with sequence-based target prediction helps combat the
high false-positive rate of sequence-only methods and, importantly, enables tissue and
disease specific support for target predictions in real-world data. Recent
evidence indicates that miRNAs act predominantly through mRNA degradation
\citep{Guo2010}. Thus, it is feasible to use mRNA or protein expression data to
infer target relationships, since the regulatory effect of miRNAs should be
reflected in mRNA and protein abundances. Mathematical models used such expression
analyses range from simple similarity measures to regression and complex
Bayesian models, and examples are covered in this and the next section. 
Sequence-based predictions are often incorporated as a preliminary filter step to
limit the potential interactions examined. Examples of published methods
using expression data are listed in table \ref{table:expression-methods}.

% Expression-based prediction methods use mathematical models that range from correlation
% to complex Bayesian regression models. The idea is to find (negatively)
% correlating miRNA-mRNA pairs or to predict mRNA or protein expression
% from miRNA expression patterns using regression. In the case of multiple regression
% target prediction becomes essentially a variable selection, where 
% the goal is to select microRNAs that best predict mRNA expression and then
% classify the mRNA in question as a target of chosen miRNAs.

\begin{table}
  \caption{Examples of expression-data based approaches to miRNA target prediction.
  See text for details. Ref. $=$ Reference, Method = Analysis method used.}
  \label{table:expression-methods}
  \centering
  \begingroup\small
  \begin{tabular}{ lllp{6cm} }
    \\[-1ex] \hline\hline
    Name & Ref. & Method & Additional Notes \\
    \hline
    MAGIA & \citep{Sales2010} & MI or correlation & Also produces a bipartite network of miRNA-mRNA interactions. \\
    TaLasso & \citep{Muniategui2015} & lasso regression &  \\
    Engelmann & \citep{Engelmann} & least-angle regression &  \\
    miRNAmRNA & \citep{vanIterson2013} & global test &  \\
    GenMir++ & \citep{} & Bayesian regression &  \\
    \hline\hline
    \end{tabular}
    \endgroup
\end{table}

Let us henceforth define $\mathbf{X_{NxJ}} = [\mathbf{x_j}] = [x_{nj}]$ as the matrix
of expression values of mRNAs $j (j = 1, 2, ..., J)$ and
$\mathbf{Z_{NxK}} = [\mathbf{x_k}] = [x_{nk}]$ as the matrix
of expression values of miRNAs $k (k = 1, 2, ..., J)$ for observations
$n (n = 1, 2, ..., N)$.
In the case of regression analysis (discussed below), for 
simplicity of notation, $\mathbf{Z}$ denotes
a matrix $\mathbf{Z_{NxK+1}}$,
where a constant vector $\mathbf{z_0} = [1, ..., 1]^T$ has been added.

\paragraph{Pairwise similarity measures}
The relationship between miRNAs and mRNAs can be easily analyzed using
measures of variable similarity (or, inversely, independence), such as
correlation or mutual information (MI). The goal is to find miRNA-mRNA pairs
whose expression pattern is similar or correlated across observations. The
drawback of MI is that it only indicates similarity of the variables, but not
the direction, or sign, of the relationship. Pearson correlation is widely
used because of its simplicity and intuitive interpretation. Pearson
correlation between mRNA $j$ and miRNA $k$ is defined as:
\begin{equation}
	\rho_{jk} = (\mathbf{x_j})_{\mu_k=0|\sigma_k=1}^T \cdot (\mathbf{z_k})_{\mu_k=0|\sigma_k=1},
	\label{eq:correlation}
\end{equation}
where $\mu_i=0|\sigma_i=1$ indicates normalization to zero mean and standard
deviation of one and $\cdot$ a vector dot product. miRNA-mRNA pairs
significantly correlated are classified as putative target interactions.
Simple correlation analysis, however, often results in very high numbers of
putative targets and many tools use additional information, such as
differential expression analysis and sequence-based predictions, to limit
results \citep{Muniategui2010}. Another limitation of similarity measures is
that they can only be applied for pairwise miRNA-mRNA comparisons.

\paragraph{Multiple linear regression}
Multiple linear regression (MLR) can be used to examine the relationship
between miRNA and mRNA. Most tools use expression of miRNAs to predict mRNA
(or protein) expression. In the case of MLR, target prediction essentially
becomes a variable selection problem, where the goal is to choose a set miRNAs
predicts mRNA expression as closely as possible, but without overfitting. A
linear model for mRNA $j$ can be defined as:
\begin{equation}
	\mathbf{x_j} = \sum_{k=0}^{K} (\mathbf{z_k} \cdot w_{jk} + \epsilon_{jk}) =  \mathbf{Z} \cdot \mathbf{w_j} + \mathbf{\epsilon_j},
\end{equation}
where $\mathbf{w_j} = [w_j0, ..., w_jK]$ is the vector of regression
coefficients, $w_j0$ is the intercept term, $\epsilon_j$ is the error term
(often assumed to be normally distributed), and $\mathbf{Z}$ is the matrix of
covariates, i.e. the miRNA expression vectors. The parameter of interest is
$\mathbf{w_j}$, which determines to the contribution of each miRNA to the
response variable, namely mRNA expression. Commonly the solution is obtained by
least squares, which involves minimizes the objective function,
\begin{equation}
	min\{ || \mathbf{x_j} - \mathbf{w_j} \cdot \mathbf{Z} ||_2 \} ,
\end{equation}
which is equal to the sum of squares of the residuals. The advantage of using
regression for target prediction is the possibility of modeling the effect of
several miRNAs simultaneously. This is desirable as the effect of a single
miRNA on mRNA expression can be small, as discussed above. MLR is not
applicable in cases, where the number covariates is larger than the number of
samples (here $K > N$), because the simple linear model is undetermined and a
single solution cannot be obtained. This is quite common with high-throughput
biological data, for example in analysis of microarray expresssion data.

\paragraph{Regularized regression}
There are several solutions to the data dimensionality problem, one of which
is using regularized regression. In regularized least squares a penalty term is
applied to the coefficients $\mathbf{w_j}$ to force them small. This approach
involves minimizing,
\begin{equation}
	min\{ || \mathbf{x_j} - \mathbf{w_j} \cdot \mathbf{Z} ||^2_2 + \lambda \cdot R(\mathbf{w_j}) \} ,
\end{equation}
where $R(\mathbf{w_j})$ is the penalty function and $\lambda$ is a tuning
parameter that controls the amount of regularization. Commonly used
regularizations include the 1-norm ($R(\mathbf{w_j}) = ||\mathbf{w_j}||_1$) in
lasso regression, the 2-norm ($R(\mathbf{w_j}) = ||\mathbf{w_j}||_2$) in ridge
regression and a combination of these ($R(\mathbf{w_j}) =
\lambda_1||\mathbf{w_j}||_1 + \lambda_2||\mathbf{w_j}||_2$) in what is called
elastic-net regression. Lasso regression in effect forces the number of non-
zero coefficients to be small, where as ridge regression results in a solution
where coefficients are small but mostly non-zero \citep{Muniategui2010}. While regularization solves
the dimensionality problem and improves interpretability, it has several
drawbacks. First, only a limited number of covariates may be included in the
model, and thus some relevant associations can be missed by number of included
covariates alone \citep{vanIterson2013}. Second, regularization may remove
covariates highly associated with and functionally regulating the response,
instead retaining an uninvolved covariate that correlates with actual
regulators \citep{Engelmann}. Relating to both limitations, van Iterson et al
showed for one dataset that lasso did not consistently select highly
correlated miRNA-mRNA pairs \citep{vanIterson2013}.

\paragraph{Other approaches}
Another solution to the dimensionality problem is least angle regression 
(LARS, not covered here), and Engelmann et al used a LARS approach to
show that mRNA expression can, in fact, be predicted from miRNA expression
\citep{Engelmann}. Other suggested approaches used include the global test
and several Bayesian methods, which are covered in the next section.


A group of bioinformatics tools uses mRNA expression data to suggest
potentially interesting miRNA-target interactions (MTIs).
\begin{itemize}
  \item
  Input ist of interesting genes (e.g. DE vs normal)
  \item
  Look for miRNA regulation patterns in list (analogous to gene set enrichment REF REVIEW)
  \item
  Enriched miRNAs deemed interesting for this data
\end{itemize}


% All proposed methods use sequence-based
% predictions as a starting point by considering only miRNA-mRNA pairs predicted
% by at least one sequence-based prediction algorithm. This section provides a
% review of algorithms for integrating miRNA and mRNA expression and
% implementations in select target prediction methods.

Ainakin nämä vaikka:
\begin{itemize}
  \item
  DIANA-mirExTra (uusin versio NGS-datalle)
  \item
  GeneSet2miRNA
  \item
  Sylamer(?)
\end{itemize}

These methods, however, rely on previous target site predictions based on
miRNA and mRNA sequences, some of which may be outdated, and 
cannot suggest novel target interactions. They also do not account for
the differences in miRNA expression inherent to different tissues and
possibly aberrant expression in diseases.


\subsubsection{Integration of mRNA and miRNA expression}

Since it is evident that miRNA regulation affects mRNA expression levels, as
mentioned above, it seems feasible to predict MTIs by integrating miRNA and
mRNA expression data. Many tools have been developed that do this using
correlation, mutual information or various forms of regression.

\textbf{KOLME KATEGORIAA: KORRELAATIO, REGRESSIO, BAYESIAN?}

\paragraph{Correlation methods (incl MI)}\label{correlation-methods}

Mutual information (MI) is a simple measure of similarity between two
variables. \textbf{SELITÄ MI TARKEMMIN JA KAAVAN KANSSA JA LÄHDE} Thus, MI can
be used to measure the interdependence of miRNA-mRNA pairs from expression
data. However, MI does not distinguish the direction of the interaction, which
is highly relevant for miRNAs that are believed to mostly downregulate mRNA
expression. This constitutes a major drawback.

Correlation \textbf{SELITÄ KORRELAATIO JA LÄHDE}.

MAGIA \citep{Sales2010} is a webservice that implements both the MI and
correlation approaches. It also constructs a bipartite network of the top 250
predicted miRNA-mRNA pairs and provides links to several databases for further
examination and validation of results.



\paragraph{Regression methods}\label{regression-methods}

\subparagraph{Global test (regression)}\label{global-test-regression}

van Iterson et al recently proposed a miRNA target prediction method based on
the global test \citep{vanIterson2013}. \textbf{SELITÄ GLOBAL TEST}.

The method by van Iterson et al uses TargetScan, microCosm and PITA for
putative sequence-based targets and is available as an R package called
miRNAmRNA \citep{vanItersonWeb}.








\subsubsection{Bayesian methods}\label{bayesian-methods}

\textbf{TÄSTÄ PITÄNEE VÄHÄN KARSIA.}

\begin{itemize}
  \item The first published Bayesian analysis by Thomas Bayes in 1763 \citep{Gelman2013}.
  \item Bayes' rule
  \item posteriors and priors
  \item hierarchical models and hyperpriors
  \item Advantages of Bayesian methods:
  \begin{itemize}
    \item
    quantification of uncertainty as probabilities, knowledge about anything
    unknown is described as a probability distribution, easy to comprehend
    \item
    common-sense interpretation of credible intervals compared to frequentist
    confidence intervals
    \item
    flexibility allows constructing complex models with relative ease (e.g.
    hierarchical models)
    \item
    Adding more data sequentially is possible by using the previous posterior
    distribution as the new prior distribution. This can be especially useful in
    a clinical research context, where data are often collected within a long
    timespan.
  \end{itemize}
\end{itemize}
The challenge in Bayesian analysis is setting up proper probability models for
the parameters and observations \citep{Gelman2013}. This includes the prior
distributions of the parameters as well as the likelihood of the observed data.
The choice of prior is also a source of much controversy, as it is based
on experience and reasoning of the statistician.

One could also argue that the choice of model is always subjective to some
degree, irrespective of chosen methodology, and \textbf{MALLIN TARKISTELU
LOPUKSI} is always necessary.

In many instances, using a non-informative prior results in similar or equal
results as frequentist analysis, but the strength of Bayesian analysis comes
from including prior knowledge in the prior distribution. \citep{Jaynes?} The
posterior distribution represents a compromise between the prior (and, hence,
prior information) and the observed data, with the data having an increasing
effect as the sample size increases \citep{Gelman2013}. The posterior
distribution also provides a more comprehensive view of
one's knowledge on the parameter of interest than, say, a single confidence
interval.

The frequentist approach only considers the data to have a probability
distribution, the likelihood. The process giving rise to the data, and the
parameters that define it, are considered fixed. The observed data are
assessed with respect to other data that might be generated by the same model.

% OMIN SANOIN: Confidence intervals work their best when you don't know much about a
% parameter beyond the information contained in a data set. And further,
% credibility intervals won't be able to improve much on confidence intervals
% unless there is prior information which the confidence interval can't take
% into account, or finding the sufficient and ancillary statistics is hard.





Previous Bayesian miRNA methods:
\begin{itemize}
  \item
  ElMMo
  \item
  TaLasso-artikkelissa joku
\end{itemize}




\paragraph{Simulation}\label{simulation}

Bayesian analysis often involves simulation if the form of sampling from the
obtained posterior distribution. This is convenient -- and necessary -- when
the exact probability density function cannot be explicitly obtained through
integration. Additionally, simulation often has the advantage of pointing out
problems in the model specification when simulated values are extremely small
of large.

Simulation methods:
\begin{itemize}
  \item
  sampling from probability distributions (easy with modern pseudorandom
  number generators)
  \item
  \textbf{mikäseyksinkertasinonkaan}
  \item
  Gibbs
  \item
  Hamiltonian Monte Carlo (this is used by Stan)
\end{itemize}


\paragraph{Bayesian regression}\label{bayesian-regression}

Bayesian regression analysis aims to infer the posterior distributions
for the regression coefficients of covariates and other model parameters,
such as the variance (i.e. noise) of the observation model.


%!TEX root = dippa.tex
%%% This file contains the Bayesian analysis section of my master's thesis.
%%% Author: Viljami Aittomäki

\section{Bayesian analysis}\label{bayesian-analysis}

Bayesian data analysis is a modeling framework that is based on the principle
of quantifying uncertainty as probability. Current knowledge about unknown
model parameters, variables and future observations is described in terms of
probability statements \citep{Gelman2013}. This provides a framework which is
inherently suited to dealing with noisy real-world data, as measurement noise
is naturally incorporated into probability distributions. This section
provides a brief introduction into Bayesian analysis and discusses Bayesian
regression and variable selection as applicable to the problem of microRNA
target prediction.




\subsection{Basics of Bayesian analysis}

Bayesian analysis begins by defining a joint probability model $p(y,\theta)$
for observed data $y$ and unknown model parameters $\theta$.
The joint distribution can be written as a product of two probability distributions
\begin{equation}
  p(y,\theta) = p(\theta) p(y\mid\theta),
\end{equation}
which are referred to as the \emph{prior distribution} $p(\theta)$ and the
data distribution or \emph{likelihood} $p(y\mid\theta)$. The prior conveys
information on the presumed values a parameter may take and the likelihood
represents the likeliness of observed data for given parameter values (in the
context of the chosen data model). Applying the Bayes' theorem we obtain the
\emph{posterior distribution} for $\theta$ given the known values of the
observations $y$:
\begin{equation}
  \label{eq:bayes}
  p(\theta \mid y) = \frac{p(y,\theta)}{p(y)} = \frac{p(\theta) p(y\mid\theta)}{p(y)},
\end{equation}
where $p(y) = \int_{\theta} p(\theta) \cdot p(y\mid\theta) d\theta$.
Noting that $p(y)$ does not depend on $\theta$, we can write Equation
\eqref{eq:bayes} as
\begin{equation}
  p(\theta \mid y) \propto p(\theta) p(y\mid\theta).
\end{equation}
The latter is referred to as the unnormalized posterior distribution. The
Bayes' theorem forms the heart of Bayesian inference and illustrates the core
concept of updating prior beliefs to account for observed evidence. The
posterior provides a probability assessment of the possible values of
a parameter, and represents a compromise between prior knowledge and information
obtained from observations. As the number of observations increases, the
data have increasing influence on the posterior \citep{Gelman2013}.

Hierarchical models -- where parameters of prior distributions have their own
priors, called \emph{hyperpriors} -- allow significant flexibility in Bayesian
modeling. In situations where model parameters are related to each other, a
common hyperprior may be used for several parameters. Hierarchical models are
particularly appropriate in settings where data are sparse (as available
information is shared between parameters) or the data are naturally structured
into several levels, such as similar measurements from different hospitals or
schools.

The virtue of modeling uncertainty as probabilities -- in addition to
naturally dealing with noise -- is that they are conceptually easy to grasp
and often allow common-sense interpretations of conclusions to be
made.\footnote{This is especially true compared to classical frequentist inference,
which is defined within the context of repeated sampling (and inference) from
a fixed but unknown process generating the observations. For example,
frequentist confidence intervals strictly do not indicate that the true value
of the parameter is contained within with high probability -- a common
misconception -- where as Bayesian posterior intervals do (subject to modeling
assumptions).} Perhaps the foremost advantage of Bayesian modeling,
however, is its flexibility and extensibility; it can cope with complex
problems and data with relative ease. Prior knowledge about the parameters of
interest can be embedded in prior distributions, priors can be assigned freely
to each parameter, hierarchical models can be used to model layered data, and
new observations can be added sequentially to update previous conclusions.

The challenge in Bayesian analysis is choosing proper probability models for
the parameters and observations, including prior distributions as well as the
likelihood \citep{Gelman2013}. In fact, Bayesian methodology has been
criticized for the subjectivity related to choosing suitable priors. One could, however,
argue that the choice of any model is always subjective to a certain degree,
irrespective of chosen methodology. Additionally, weakly informative or non-informative
priors can be used to decrease the effect of subjective information (or
in cases where no prior knowledge is available) and conclusions from
inferences using non-informative priors often coincide with classical
analyses.

% In cases where no prior information is available, weakly informative
% or noninformative priors can be used. These convey little or no
% information on the presumed values a parameter may take, respectively.
% In many instances, using a non-informative prior results in similar or equal
% results as frequentist analysis, but the strength of Bayesian analysis comes
% from including prior knowledge in the prior distribution. \citep{Jaynes?} The
% posterior distribution represents a compromise between the prior (and, hence,
% prior information) and the observed data, with the data having an increasing
% effect as the sample size increases \citep{Gelman2013}. The posterior
% distribution also provides a more comprehensive view of
% one's knowledge on the parameter of interest than, say, a single confidence
% interval.

% The frequentist approach only considers the data to have a probability
% distribution, the likelihood. The process giving rise to the data, and the
% parameters that define it, are considered fixed. The observed data are
% assessed with respect to other data that might be generated by the same model.

% OMIN SANOIN: Confidence intervals work their best when you don't know much about a
% parameter beyond the information contained in a data set. And further,
% credibility intervals won't be able to improve much on confidence intervals
% unless there is prior information which the confidence interval can't take
% into account, or finding the sufficient and ancillary statistics is hard.




\subsection{Bayesian inference}\label{sec:bayesian-inference}

The goal of Bayesian inference is to make conclusions about unknown parameters
$\theta$ or unknown observations $\tilde{y}$, given the observed data $y$.
These are formulated as posterior distributions or features describing them,
such as point or interval estimates. In simple cases, the posterior $p(\theta \mid y)$
can be derived in analytical form. In practice, however, it is often not possible to obtain
explicit forms of posteriors or analytical solutions to integrals involved in
inference, especially with complex and hierarchical models. Therefore,
numerical estimation or simulation methods, in the form of sampling from
probability distributions, are frequently used to approximate the posterior.

% In practice, one is often interested
% in a subset of the model parameters. \emph{Marginalization} over the
% uninteresting \emph{nuisance} parameters can be employed to obtain the marginal posterior
% \[
%   p(\theta_1 \mid \theta_2, y) = \int p(\theta_1, \theta_2 \mid y) d\theta_2,
% \]
% where $\theta_1$ are the parameters of interest and $\theta_2$ the
% nuisance parameters.

% Bayesian analysis often involves simulation if the form of sampling from the
% obtained posterior distribution. This is convenient -- and necessary -- when
% the exact probability density function cannot be explicitly obtained through
% integration. Additionally, simulation often has the advantage of pointing out
% problems in the model specification when simulated values are extremely small
% of large.

% Often obtaining analytical solutions to integrals or explicit formulations of
% the posterior distribution is not possible. This is especially true for
% complex and hierarchical models. In these cases it is possible to use
% simulation to obtain samples from the posterior distribution of parameters and
% use these to compute estimates for parameters and other quantities of
% interest.

The simplest approaches to simulation include sampling directly from the
posterior distribution $p(\theta \mid y)$, when possible, or from a simpler distribution
proportional to $p(\theta \mid y)$ using e.g. rejection sampling \citep{Gelman2013}.
More sophisticated methods are often needed when dealing with complex models.
Markov chain Monte Carlo (MCMC) is a general approach to simulation
that is based on drawing samples of $\theta$ from an approximating
distribution. The draws are corrected at each iteration so that the
approximating distribution becomes closer to $p(\theta \mid y)$.
Each draw $\theta^{(t)}$ is conditional (only) on $\theta^{(t-1)}$, the previous draw
.\footnote{This is essentially the definition of a Markov chain;
a sequence of random variables, where the probability density of each one is
dependent on only the previous one.}
MCMC methods are applicable to arbitrary posterior distributions and a range
of programs for running simulations of full Bayesian inferences are available.

A key issue with iterative methods, such as MCMC, is running the simulation
long enough, so that the distribution for drawing samples has converged 
close enough to the target distribution. Basic solutions to this include discarding a
burn-in period of samples from the beginning of each simulation (to assure
that the samples arise from a converged state), and running several separate
simulations (chains) with different starting points to improve coverage of the
posterior. Various approaches to measuring convergence have been proposed.




\subsection{Bayesian regression}

Bayesian regression analysis aims to infer the posterior distributions
for the regression coefficients of covariates and other model parameters,
such as the variance (i.e. the error term) of the observation model.
Within the Bayesian framework, the \emph{normal linear regression} defined in
Eq. \eqref{eq:linear-regression} can be expressed as
\begin{equation}
  y \mid \beta, \sigma, X \sim N(X \beta, \sigma^2I),
  \label{eq:bayesian-linear-regression}
\end{equation}
where $N$ is the multivariate normal distribution, and $I$ is the $n \times n$
identity matrix (the gene index $k$ has been suppressed for clarity).
The mean of $y$ is then the familiar linear sum of $x_k$
\begin{equation}
  \textrm{E}(y\mid\beta,X) = X \beta = \beta_0 + \beta_1 x_1 + \dotsb + \beta_p x_p.
\end{equation}
The posterior distribution for the regression coefficients (up to a
normalizing constant) is obtained from the marginal posterior
\begin{equation}
  p(\beta \mid \sigma, y, X) \propto \int p(y \mid \beta, \sigma, X) p(\beta \mid \sigma) p(\sigma) d\sigma,
\end{equation}
with the joint prior $p(\beta, \sigma) = p(\beta \mid \sigma) p(\sigma)$.
It is relatively straightforward to extend this simple model, for instance by
allowing unequal variances or correlation between observations, choosing a
different data distribution
% than the normal distribution
to represent the error term, or including hyperparameters to construct a
hierarchical model.

As mentioned in Section \ref{expr-methods}, microRNA target prediction using
regression analysis of expression data is effectively a variable selection
problem. For Bayesian regression, several different priors that provide model shrinkage
have been proposed, including the Laplace prior (which is closely related to
lasso regression), the horseshoe prior, and the hierarchical shrinkage prior
(a generalization of the horseshoe).
A hierarchical shrinkage prior for regression weight
$\beta_j$ can be formulates as
\begin{equation}
  \label{eq:hs-prior}
  \begin{aligned}
    \beta_j \mid \lambda_j, \tau & \sim N(0, \lambda_j^2 \tau^2) \\
    \lambda_j                 & \sim t_\nu^+(0,1),
  \end{aligned}
\end{equation}
where $t_\nu^+$ denotes the half-Student-$t$ prior with $\nu$ degrees of
freedom \citep{Piironen2015}. The $\lambda_j$ correspond to a local scale
parameter and $\tau$ controls the amount of global shrinkage. As an example,
in a very sparse model with many irrelevant covariates, the model would
ideally have small $\tau$ (so that $p(\beta)$ is mostly shrunk close to zero), but
allow some $\lambda_j$ to be large to escape the shrinkage.

A weakly
informative half-Cauchy distribution is often the suggested choice of prior
for $\tau$, but van der Pas et al have also proposed (for the horseshoe prior)
using a fixed value of $\tau = \frac{p_n}{n}\sqrt{\textup{log}(n/p_n)}$,
where $p_n$ is the assumed number of relevant covariates and $n$ is the number
of observations \citep{vanderpas2014}. Bayesian shrinkage priors, however, do
not lead to a sparse solution as there remains uncertainty in the posterior
distribution and no coefficient can be considered exactly zero.




\subsection{Bayesian variable selection}\label{sec:bayes-variable-selection}

In order to find a small set of relevant predictive variables, a model
selection approach needs to be applied. To this end, a range of methods
applicable in Bayesian analysis have been proposed; examples include using
cross validation, different information criteria, and projection
methods to determine the submodel giving the best compromise between
prediction accuracy and model size. A detailed review of these falls outside
the scope of this thesis, however, a comprehensive one has been recently
written by Vehtari and Ojanen \citep{Vehtari2012}.

In the context of variable selection for regression,
Piironen and Vehtari \citep{Piironen2016} recently suggested that, for problems where data
are scarce and the number of candidate variables high, using projection predictive
variable selection is effective. The idea, proposed by
Dupuis and Robert \citep{Dupuis2003}, is to fit a full reference model $M_{\textup{ref}}$
encompassing all candidate variables,
%and uncertainties related to their effect,
and then project the information in the reference posterior onto a submodel $M_\perp$
so that the predictions are as similar as possible.

Given the reference model parameters $\theta_{\textup{r}}$, the projection $\theta_\perp$
in the parameter space of $M_\perp$ is obtained by solving
\begin{equation}
  \label{eq:projection}
  \theta_\perp = \textup{arg}\underset{\theta}{\textup{min}} \frac{1}{n}\sum_i^n \textup{KL} \left ( p(\tilde{y} \mid x_i, \theta_{\textup{r}}, M_{\textup{ref}}) \parallel p(\tilde{y} \mid x_i, \theta, M_\perp) \right ),
\end{equation}
where $\textup{KL} \left ( P \parallel Q \right )$ is the Kullback-Leibler divergence
between probability distributions $P$ and $Q$. The discrepancy between the reference
model  $M_{\textup{ref}}$ and submodel $M_\perp$ is then defined as the expectation
of the divergence over the posterior of the reference model:
\begin{equation}
  \label{eq:model-discrepancy}
  \delta(M_{\textup{ref}} \parallel M_\perp) = \frac{1}{n} \sum_i^n \textup{E}_{\theta_{\textup{r}} \mid D,M_\textup{ref}} \left [ \textup{KL} \left ( p(\tilde{y} \mid x_i, \theta_{\textup{r}}, M_{\textup{ref}}) \parallel p(\tilde{y} \mid x_i, \theta_\perp, M_\perp) \right ) \right ].
\end{equation}
The posterior expectation in \eqref{eq:model-discrepancy} can, in practice, be
estimated by drawing samples from the reference posterior (using e.g. MCMC).
It can be shown, that in the case of normal linear regression, the
minimization in \eqref{eq:projection} can be solved analytically and the
discrepancy \eqref{eq:model-discrepancy} has a simple form depending only on
the reference and projected model variances $\sigma^2$ \citep{Piironen2015}.

A practical issue with this method is deciding how many variables should be
included in the submodel. This depends on what is considered acceptable
loss of prediction performance compared to the submodel. Piironen and Vehtari suggest using cross-validation to
guide the variable selection process and give a practical guideline for
stopping the selection. This is discussed Section \ref{sec:methods-variable-selection}.
In this thesis, projection predictive variable selection was used
for inferring putative microRNA targets from breast cancer expression data
using Bayesian regression. Further details of the used method are presented in
Section \ref{sec:methods-variable-selection}.




\subsection{Bayesian microRNA target-prediction methods}

One of the earliest tools to use expression data for target prediction was
GenMir++ \citep{Huang2007}. It takes as input a candidate set of miRNA targets (the authors
have used TargetScanS) and uses mRNA and miRNA expression data across multiple tissues
to predict whether a given candidate miRNA-target interaction is real.
GenMir is based on a Bayesian regression model, where
mRNAs are assumed to share a tissue-specific common background expression,
which is downregulated by regulating miRNAs. Let $\mu_t$ represent the
background mRNA expression in tissue $t$, the probability model for
mRNA expression $y_{kt}$ is defined in GenMir as:
\begin{equation}
  y_{kt} \mid X, S_k, \Gamma, \Lambda, \mu_t, \Sigma \sim \textup{N} \left( \mu_t - \gamma_t \sum_j^p \lambda_j s_{kj} x_k, \Sigma \right),
\end{equation}
where $\gamma_t$ is a tissue-specific scaling factor (modeling differences in
mRNA and miRNA measurements and normalization) and $\lambda_j$ are regulatory
weights of the miRNAs (irrespective of candidate target $k$), and $s_{kj}$ an
indicator variable of target interaction. Compared to Eq.
\eqref{eq:bayesian-linear-regression}, the degree of regulation of a miRNA becomes $\beta_kj =
\gamma_t \lambda_j s_kj$. The goal is to infer the posterior
$p(s_{kj} \mid c_{kj} = 1, D, M)$, that is, the probability of a candidate interaction being true,
where $c_{kj}$ is an indicator variable of the input putative interactions.
A log-odds score is given for each miRNA-mRNA interaction.

The latest version of GenMir (GenMir3) defines Gamma priors for $\gamma_t$ and
$\lambda_t$ and Bernoulli priors for $s_{kj}$ (leading to negative
interactions only) and allows including sequence features in a logit
hyperprior for the prior $p(s_{kj})$. The model is solved simultaneously for
all genes and tissues, using a variational Bayes method, to obtain an
approximate posterior for $S$. The authors note, that the method can be easily
extended to add protein expression.

Another Bayesian approach to target prediction is a Bayesian network method published by Stingo et al
\citep{Stingo2010}. The approach is essentially an implementation of the spike-and-slab variable
selection method \citep{Vehtari2012} applied to a Bayesian regression model
equivalent to Eq. \eqref{eq:bayesian-linear-regression}. Sequence features are
included in the (Bernoulli) prior of the covariate inclusion variable $S$. The
posterior for $S$ is obtained with MCMC methods. A time-dependent coefficients version
is also presented for data measured for several time points.
Additionally, many published analyses of different expression data for target prediction
have utilized various probabilistic or Bayesian approaches.



%%% This file contains the materials and methods section of my master's thesis.
%%% Author: Viljami Aittomäki


\section{Research material and methods}\label{material-and-methods}

\subsection{Research material}

The data analysed in this thesis consisted of tumor samples from 283 breast
cancer patients treated in two Norwegian hospitals. Protein, mRNA and microRNA
expression were measured from each sample. The data were published by
\citet{Aure2015} and are publicly available.

The patients in the data are part of the OSLO2 cohort, the collection of which
started in 2006 and is still ongoing. Therefore, no survival data were
available for analysis. Clinical data for included patients is summarised in
Table \ref{clinical-data}. All included patients had primarily operable
disease, that is stage cT1-cT2. No control samples were available.

The mRNA and microRNA expression were measured using Agilent Technologies
SurePrint G3 Human GE 8x60K and Human miRNA Microarray Kit (V2), respectively,
which contain 27958 genes and 887 miRNAs according to manufacturer annotation.
Protein expression was measured using a reverse phospatase protein array
(RPPA) containing a set of 105 proteins, most of which are found on the PI3K-
pathway \textbf{ref to G.Mills?}. Only the gene expression values
corresponding to each measured protein were used in the analysis. Analyses
performed in this thesis used the publicly available preprocessed data
\textbf{paitsi jos ehtii tehä vielä preprocin ite}.

MicroRNAs detected in less than 10\% of samples had been removed from the
publicly available OSLO2 data, leaving 421 miRNAs. Out of these, eleven miRNAs
(hsa-miR-1274a, hsa-miR-1274b, hsa-miR-1280, hsa-miR-1308, hsa- miR-1826, hsa-
miR-1974, hsa-miR-1975, hsa-miR-1977, hsa-miR-1979, hsa-miR-720, hsa-
miR-886-3p) were reported as missing by miRBase Tracker. Reviewed on miRBase,
these miRNAs are reported as being fragments of other RNA species (e.g. tRNA
or rRNA) and, thus, not actual miRNAs and removed from the database.
Therefore, these eleven miRNAs were removed from subsequent analyses.




% The mRNA and microRNA expression data are publicly available in preprocessed
% format in the Gene Expression Omnibus (GEO) database \citep{GEO} under
% accession IDs xx and xx respectively. For the purpose of this thesis, the raw
% Agilent expression data were kindly provided and used for the analyses instead
% of the preprocessed data. The protein expression data is available in
% Additional file 4 of \citet{norjis} also in preprocessed format.

% Clinical data concerning each patient and cancer were also provided. A summary
% of the clinical parameters is presented in table \ref{clinical-data}. The
% predominant tumor type in the data was ductal carcinoma, which is in general
% the most common histological type of breast cancer.

% Use danish data for validation?


\subsection{Methods}

\subsubsection{Preprocessing data}

\subsubsection{Variable selection with projection prediction}




\section{Results}

\textbf{Jos ehtis, niin analyysit voisi ajaa uudestaan niin, että
mRNA coef rajoitettu >0 ja miRNA coef rajoitettu <0.}

\begin{itemize}
  \item QC plots
  \begin{itemize}
    \item distributions of arrays
    \item PCA and hierarch. clust of samples with hospital colors
  \end{itemize}
  \item correlation plots
  \begin{itemize}
    \item mRNA-protein
    \item miRNA-mRNA (target vs non-target)
    \item miRNA-protein (target vs non-target)
    \item all three combined (additionally with grouping by intrinsic subtype?)
  \end{itemize}
  \item cross-validation results
  \begin{itemize}
    \item example plots of search path ("good" and "bad")
  \end{itemize}
  \item miRNA coefficients from final models
  \begin{itemize}
  	\item plot of different threshold values
    \item number of miRNAs included
    \item percentage of significant coefs
    \item number of negative miRNA coefs vs positive
    \item number of negative mRNA coefs
    \item percentage predicted by seq algos (num miRNAs vs num algos, included vs significant)
    \item percentage validated in experiments (included vs significant, mirTarBase)
    \item number of validated targets in PPVS vs Lasso vs GenMiR equiv (jos ehtii, tää olis kyllä mielenkiintonen), also num implicated in BRCA
    \item median of miRNA coef vs variance/prob.weight (sum of dens<0 or >0)
    \item magnitude of chosen miRNA coefs vs gene and constant
    \item posterior mean vs sd scatter (järkevä?)
  \end{itemize}
  \item simple network analysis?
  \begin{itemize}
    \item connections between included miRNAs and genes
    \item some pathway analysis or geneset enrichment? (moot since proteins from PIK3-path?)
  \end{itemize}
  \item DE analysis of included/significant miRNAs?
  \begin{itemize}
    \item triple negative cancers vs others
    \item alternatively correlations between significant miRNAs-proteins grouped by intrinsic subtypes?
  \end{itemize}

\end{itemize}

%\begin{landscape}
    { % This table is generated by CSV2Latex.
\footnotesize{\begin{longtable}{llllrp{7cm}}
\caption{
Properties of fitted models for all 105 genes. A missing value for N miRNAs indicates a projected model was not found
(i.e. the stopping criterion was not satisfied before reaching 200 covariates),
a zero indicates no miRNA variables were chosen.
$R^2$ was not computed for models with no miRNA variables.
Only the significant miRNAs chosen are listed for compactness of display. \\
$R^2_{\textup{gene}}$: $R^2$ for gene-only model \\
$R^2_\perp$: $R^2$ for projected model obtained with PPVS \\
$\Delta\bar{R}^2$: Difference of adjusted $R^2$ of projected model versus gene-only model \\
$^{\ast}$: gene expression variable is significant (95\% credible interval) \\
$\textup{N}_{\textup{miRNA}}$: number of miRNA variables in projected model (number of significant miRNA variables, 95\% credible interval) 
\label{table:final-models}
}
\label{table:finalModelTable}\\\hline
\textbf{Gene} & \textbf{$R^2_{\textup{gene}}$} & \textbf{$R^2_\perp$} & \textbf{$\Delta\bar{R}^2$} & $\mathbf{\textup{N}_{miRNA}}$ & \textbf{Significant miRNAs}\\\hline
\endfirsthead{}%
\textbf{Gene} & \textbf{$R^2_{\textup{gene}}$} & \textbf{$R^2_\perp$} & \textbf{$\Delta\bar{R}^2$} & $\mathbf{\textup{N}_{miRNA}}$ & \textbf{Significant miRNAs}\\\hline
\endhead\hline\multicolumn{6}{r}{\textit{Continued on next page\ldots\/}}
\endfoot\hline\endlastfoot{}%
ACACA&0.462$^{\ast}$&0.524$^{\ast}$&0.062&2 (2)&\raggedright{miR-30a, miR-370} \tabularnewline\rowcolor[rgb]{0.96,0.96,0.96}{}%
AKT1&0.111$^{\ast}$&0.174$^{\ast}$&0.064&2 (2)&\raggedright{miR-449a, miR-342-5p} \tabularnewline{}%
AKT2&0.001&0.230&0.195&14 (4)&\raggedright{miR-342-5p, miR-449a, miR-96, miR-146b-5p} \tabularnewline\rowcolor[rgb]{0.96,0.96,0.96}{}%
AKT3&0.000&0.251&0.216&15 (4)&\raggedright{miR-342-5p, miR-449a, miR-96, miR-146b-5p} \tabularnewline{}%
ANXA1&0.380$^{\ast}$&0.425$^{\ast}$&0.047&1 (1)&\raggedright{miR-765} \tabularnewline\rowcolor[rgb]{0.96,0.96,0.96}{}%
AR&0.713$^{\ast}$&&&0&\raggedright{} \tabularnewline{}%
BAK1&0.119$^{\ast}$&0.228$^{\ast}$&0.110&2 (2)&\raggedright{miR-29c, miR-505*} \tabularnewline\rowcolor[rgb]{0.96,0.96,0.96}{}%
BAX&0.130$^{\ast}$&0.405$^{\ast}$&0.228&23 (4)&\raggedright{miR-557, miR-659, miR-142-3p, miR-199a-3p} \tabularnewline{}%
BCL2&0.728$^{\ast}$&&&0&\raggedright{} \tabularnewline\rowcolor[rgb]{0.96,0.96,0.96}{}%
BCL2L1&0.088$^{\ast}$&0.354$^{\ast}$&0.123&53 (1)&\raggedright{miR-622} \tabularnewline{}%
BCL2L11&0.176$^{\ast}$&0.318$^{\ast}$&0.143&2 (2)&\raggedright{miR-29c, miR-34a} \tabularnewline\rowcolor[rgb]{0.96,0.96,0.96}{}%
BECN1&0.005&&&&\raggedright{} \tabularnewline{}%
BID&0.015$^{\ast}$&0.134&0.122&1 (1)&\raggedright{miR-1246} \tabularnewline\rowcolor[rgb]{0.96,0.96,0.96}{}%
BIRC2&0.018$^{\ast}$&0.243$^{\ast}$&0.226&2 (2)&\raggedright{miR-1246, miR-425} \tabularnewline{}%
BRAF&0.094$^{\ast}$&0.456$^{\ast}$&0.352&8 (8)&\raggedright{miR-638, miR-125b, miR-1321, miR-107, miR-765, miR-148a, miR-135b, miR-505*} \tabularnewline\rowcolor[rgb]{0.96,0.96,0.96}{}%
CASP8&0.014&0.111$^{\ast}$&0.091&4 (4)&\raggedright{miR-99a, miR-148a, miR-631, miR-126*} \tabularnewline{}%
CAV1&0.284$^{\ast}$&0.432$^{\ast}$&0.147&3 (3)&\raggedright{miR-551b, miR-24} \tabularnewline\rowcolor[rgb]{0.96,0.96,0.96}{}%
CCNB1&0.638$^{\ast}$&0.714$^{\ast}$&0.076&2 (2)&\raggedright{miR-199a-5p, miR-30a} \tabularnewline{}%
CCND1&0.334$^{\ast}$&0.497$^{\ast}$&0.153&8 (7)&\raggedright{miR-936, miR-181c, miR-622, miR-493*, miR-19a, miR-126, miR-9*} \tabularnewline\rowcolor[rgb]{0.96,0.96,0.96}{}%
CCNE1&0.544$^{\ast}$&&&0&\raggedright{} \tabularnewline{}%
CDH1&0.449$^{\ast}$&&&0&\raggedright{} \tabularnewline\rowcolor[rgb]{0.96,0.96,0.96}{}%
CDH2&0.006&&&0&\raggedright{} \tabularnewline{}%
CDH3&0.165$^{\ast}$&0.584$^{\ast}$&0.398&17 (13)&\raggedright{miR-155, miR-10b*, miR-502-5p, miR-224, miR-489, miR-148a, miR-195, miR-197, miR-361-5p, miR-650, miR-150, miR-501-5p, miR-582-5p} \tabularnewline\rowcolor[rgb]{0.96,0.96,0.96}{}%
CDK1&0.037$^{\ast}$&&&&\raggedright{} \tabularnewline{}%
CDKN1B&0.418$^{\ast}$&0.491$^{\ast}$&0.075&1 (1)&\raggedright{miR-195} \tabularnewline\rowcolor[rgb]{0.96,0.96,0.96}{}%
CHEK1&0.019$^{\ast}$&0.168&0.122&11 (0)&\raggedright{} \tabularnewline{}%
CHEK2&0.512$^{\ast}$&&&0&\raggedright{} \tabularnewline\rowcolor[rgb]{0.96,0.96,0.96}{}%
CLDN7&0.234$^{\ast}$&0.363$^{\ast}$&0.125&4 (4)&\raggedright{miR-29c, miR-200c, miR-30b, miR-150} \tabularnewline{}%
COL6A1&0.000&0.325&0.321&4 (4)&\raggedright{miR-125b, miR-638, miR-210, miR-24} \tabularnewline\rowcolor[rgb]{0.96,0.96,0.96}{}%
CTNNA1&0.081$^{\ast}$&0.113$^{\ast}$&0.033&2 (1)&\raggedright{miR-125a-3p} \tabularnewline{}%
CTNNB1&0.002&0.339&0.287&22 (5)&\raggedright{miR-711, miR-10a, miR-31*, miR-16, miR-28-5p} \tabularnewline\rowcolor[rgb]{0.96,0.96,0.96}{}%
DIABLO&0.089$^{\ast}$&0.483$^{\ast}$&0.336&31 (6)&\raggedright{miR-378, miR-339-3p, miR-762, miR-15b, miR-144*, miR-582-5p} \tabularnewline{}%
DVL3&0.068$^{\ast}$&0.377$^{\ast}$&0.283&14 (9)&\raggedright{miR-24, miR-498, miR-140-3p, miR-223, miR-29c, miR-21, miR-432, miR-662, miR-204} \tabularnewline\rowcolor[rgb]{0.96,0.96,0.96}{}%
EEF2&0.009&0.330&0.292&14 (3)&\raggedright{miR-106b, miR-196b, miR-29c*} \tabularnewline{}%
EEF2K&0.310$^{\ast}$&&&0&\raggedright{} \tabularnewline\rowcolor[rgb]{0.96,0.96,0.96}{}%
EGFR&0.148$^{\ast}$&0.356$^{\ast}$&0.189&10 (7)&\raggedright{miR-181d, miR-181b, miR-1182, miR-495, miR-30c, miR-126, miR-183*} \tabularnewline{}%
EIF4E&0.100$^{\ast}$&0.252$^{\ast}$&0.050&36 (0)&\raggedright{} \tabularnewline\rowcolor[rgb]{0.96,0.96,0.96}{}%
EIF4EBP1&0.495$^{\ast}$&&&0&\raggedright{} \tabularnewline{}%
ERBB2&0.729$^{\ast}$&&&0&\raggedright{} \tabularnewline\rowcolor[rgb]{0.96,0.96,0.96}{}%
ERBB3&0.204$^{\ast}$&0.334$^{\ast}$&0.128&3 (3)&\raggedright{miR-199a-5p, miR-451, miR-484} \tabularnewline{}%
ERCC1&0.004&&&0&\raggedright{} \tabularnewline\rowcolor[rgb]{0.96,0.96,0.96}{}%
ERRFI1&0.012&&&&\raggedright{} \tabularnewline{}%
ESR1&0.825$^{\ast}$&&&0&\raggedright{} \tabularnewline\rowcolor[rgb]{0.96,0.96,0.96}{}%
FN1&0.495$^{\ast}$&&&0&\raggedright{} \tabularnewline{}%
FOXO3&0.094$^{\ast}$&0.225$^{\ast}$&0.128&3 (3)&\raggedright{miR-140-3p, miR-631, miR-197} \tabularnewline\rowcolor[rgb]{0.96,0.96,0.96}{}%
GAB2&0.597$^{\ast}$&&&0&\raggedright{} \tabularnewline{}%
GATA3&0.708$^{\ast}$&&&0&\raggedright{} \tabularnewline\rowcolor[rgb]{0.96,0.96,0.96}{}%
GSK3A&0.106$^{\ast}$&0.473$^{\ast}$&0.319&26 (5)&\raggedright{miR-21, miR-29a, miR-20a, miR-100, miR-92a} \tabularnewline{}%
GSK3B&0.064$^{\ast}$&0.524$^{\ast}$&0.275&81 (0)&\raggedright{} \tabularnewline\rowcolor[rgb]{0.96,0.96,0.96}{}%
IGF1R&0.637$^{\ast}$&&&0&\raggedright{} \tabularnewline{}%
IGFBP2&0.553$^{\ast}$&&&0&\raggedright{} \tabularnewline\rowcolor[rgb]{0.96,0.96,0.96}{}%
INPP4B&0.754$^{\ast}$&&&0&\raggedright{} \tabularnewline{}%
IRS1&0.397$^{\ast}$&0.441$^{\ast}$&0.046&1 (1)&\raggedright{miR-93} \tabularnewline\rowcolor[rgb]{0.96,0.96,0.96}{}%
KDR&0.000&0.281&0.272&6 (6)&\raggedright{miR-150, miR-663, miR-495, miR-24-1*, miR-363, miR-140-3p} \tabularnewline{}%
KIT&0.581$^{\ast}$&&&0&\raggedright{} \tabularnewline\rowcolor[rgb]{0.96,0.96,0.96}{}%
KRAS&0.037$^{\ast}$&0.335&0.264&16 (2)&\raggedright{miR-96, miR-21} \tabularnewline{}%
MAP2K1&0.006&0.141&0.138&1 (1)&\raggedright{miR-21} \tabularnewline\rowcolor[rgb]{0.96,0.96,0.96}{}%
MAPK14&0.017$^{\ast}$&0.346$^{\ast}$&0.301&14 (9)&\raggedright{miR-145, miR-92a, miR-181c, miR-142-3p, miR-425, miR-339-3p, miR-342-5p, miR-18b, miR-1226*} \tabularnewline{}%
MAPK9&0.193$^{\ast}$&0.251$^{\ast}$&0.061&1 (1)&\raggedright{miR-342-5p} \tabularnewline\rowcolor[rgb]{0.96,0.96,0.96}{}%
MAPT&0.020$^{\ast}$&0.335&0.270&20 (3)&\raggedright{miR-30c, miR-132, miR-17*} \tabularnewline{}%
MET&0.031$^{\ast}$&0.085$^{\ast}$&0.054&2 (2)&\raggedright{miR-125b, miR-139-5p} \tabularnewline\rowcolor[rgb]{0.96,0.96,0.96}{}%
MRE11A&0.012&0.290&0.052&70 (0)&\raggedright{} \tabularnewline{}%
MSH2&0.333$^{\ast}$&&&0&\raggedright{} \tabularnewline\rowcolor[rgb]{0.96,0.96,0.96}{}%
MSH6&0.340$^{\ast}$&0.628$^{\ast}$&0.265&19 (10)&\raggedright{miR-125b, miR-324-5p, miR-25, miR-195*, miR-451, miR-154, miR-551b, miR-26b, miR-513a-5p} \tabularnewline{}%
MYC&0.025$^{\ast}$&0.101$^{\ast}$&0.076&2 (2)&\raggedright{miR-150*, miR-24} \tabularnewline\rowcolor[rgb]{0.96,0.96,0.96}{}%
NCOA3&0.132$^{\ast}$&&&0&\raggedright{} \tabularnewline{}%
NF2&0.132$^{\ast}$&0.321$^{\ast}$&0.187&3 (3)&\raggedright{miR-638, miR-125b, miR-22} \tabularnewline\rowcolor[rgb]{0.96,0.96,0.96}{}%
NOTCH1&0.216$^{\ast}$&0.279$^{\ast}$&0.064&2 (2)&\raggedright{miR-199b-5p, miR-502-5p} \tabularnewline{}%
NOTCH3&0.150$^{\ast}$&0.401$^{\ast}$&0.245&5 (5)&\raggedright{miR-125b, miR-27b, miR-193a-5p, miR-32, miR-139-5p} \tabularnewline\rowcolor[rgb]{0.96,0.96,0.96}{}%
PARK7&0.288$^{\ast}$&0.428$^{\ast}$&0.140&2 (2)&\raggedright{miR-93, miR-29c} \tabularnewline{}%
PCNA&0.229$^{\ast}$&0.283$^{\ast}$&0.057&1 (1)&\raggedright{miR-199a-5p} \tabularnewline\rowcolor[rgb]{0.96,0.96,0.96}{}%
PECAM1&0.054$^{\ast}$&0.169$^{\ast}$&-0.028&43 (0)&\raggedright{} \tabularnewline{}%
PGR&0.819$^{\ast}$&&&0&\raggedright{} \tabularnewline\rowcolor[rgb]{0.96,0.96,0.96}{}%
PIK3CA&0.051$^{\ast}$&&&0&\raggedright{} \tabularnewline{}%
PIK3R1&0.077$^{\ast}$&0.377$^{\ast}$&0.295&5 (5)&\raggedright{miR-142-3p, miR-342-3p, miR-501-5p, miR-145*, miR-92a} \tabularnewline\rowcolor[rgb]{0.96,0.96,0.96}{}%
PRKAA1&0.119$^{\ast}$&0.240$^{\ast}$&0.121&2 (2)&\raggedright{miR-199a-3p, miR-342-3p} \tabularnewline{}%
PRKCA&0.368$^{\ast}$&&&0&\raggedright{} \tabularnewline\rowcolor[rgb]{0.96,0.96,0.96}{}%
PTCH1&0.029$^{\ast}$&0.167$^{\ast}$&0.127&6 (2)&\raggedright{miR-145, miR-934} \tabularnewline{}%
PTEN&0.029$^{\ast}$&0.468$^{\ast}$&0.370&35 (5)&\raggedright{miR-204, miR-498, miR-324-3p, miR-30d, miR-22} \tabularnewline\rowcolor[rgb]{0.96,0.96,0.96}{}%
PTGS2&0.065$^{\ast}$&0.114$^{\ast}$&0.052&1 (1)&\raggedright{miR-1246} \tabularnewline{}%
PTK2&0.021$^{\ast}$&0.273$^{\ast}$&0.242&6 (5)&\raggedright{miR-1246, miR-150, miR-21, miR-200c, miR-32} \tabularnewline\rowcolor[rgb]{0.96,0.96,0.96}{}%
PXN&0.133$^{\ast}$&0.431$^{\ast}$&0.140&63 (0)&\raggedright{} \tabularnewline{}%
RAB25&0.031$^{\ast}$&&&&\raggedright{} \tabularnewline\rowcolor[rgb]{0.96,0.96,0.96}{}%
RAD50&0.076$^{\ast}$&0.106$^{\ast}$&0.033&1 (1)&\raggedright{miR-139-5p} \tabularnewline{}%
RAD51&0.011&0.205&0.195&2 (2)&\raggedright{miR-1246, miR-181c} \tabularnewline\rowcolor[rgb]{0.96,0.96,0.96}{}%
RAF1&0.121$^{\ast}$&0.446$^{\ast}$&0.306&12 (9)&\raggedright{miR-125b, miR-1260, miR-498, miR-449a, miR-130a, miR-30b, miR-28-5p, miR-374a, miR-195*} \tabularnewline{}%
RB1&0.006&0.233$^{\ast}$&0.222&4 (4)&\raggedright{miR-145, miR-1260, miR-451, miR-195*} \tabularnewline\rowcolor[rgb]{0.96,0.96,0.96}{}%
RPS6KB1&0.516$^{\ast}$&0.617$^{\ast}$&0.102&2 (2)&\raggedright{miR-497, miR-106b} \tabularnewline{}%
SMAD1&0.359$^{\ast}$&0.417$^{\ast}$&0.060&1 (1)&\raggedright{miR-106b} \tabularnewline\rowcolor[rgb]{0.96,0.96,0.96}{}%
SMAD3&0.354$^{\ast}$&0.475$^{\ast}$&0.112&7 (7)&\raggedright{miR-7, miR-451, miR-181d, miR-29b, miR-365, miR-1225-5p, miR-1299} \tabularnewline{}%
SMAD4&0.000&&&0&\raggedright{} \tabularnewline\rowcolor[rgb]{0.96,0.96,0.96}{}%
SNAI1&0.001&&&0&\raggedright{} \tabularnewline{}%
SRC&0.308$^{\ast}$&&&0&\raggedright{} \tabularnewline\rowcolor[rgb]{0.96,0.96,0.96}{}%
STAT5A&0.292$^{\ast}$&0.438$^{\ast}$&0.147&2 (2)&\raggedright{miR-155, miR-324-5p} \tabularnewline{}%
STMN1&0.000&0.364&-0.028&109 (0)&\raggedright{} \tabularnewline\rowcolor[rgb]{0.96,0.96,0.96}{}%
SYK&0.227$^{\ast}$&0.714$^{\ast}$&0.462&26 (20)&\raggedright{miR-125b, miR-155, miR-324-5p, miR-195, miR-449a, miR-711, miR-762, miR-940, miR-22, miR-615-3p, miR-663, miR-377*, miR-598, miR-126, miR-29b, miR-1228*, miR-144, miR-204, miR-601, miR-30d} \tabularnewline{}%
TP53&0.017$^{\ast}$&0.186$^{\ast}$&0.161&5 (5)&\raggedright{miR-301b, miR-29c*, miR-551b, miR-19a, miR-874} \tabularnewline\rowcolor[rgb]{0.96,0.96,0.96}{}%
TP53BP1&0.260$^{\ast}$&0.357$^{\ast}$&0.099&1 (1)&\raggedright{miR-96} \tabularnewline{}%
TSC2&0.065$^{\ast}$&0.318$^{\ast}$&0.225&13 (3)&\raggedright{miR-1915, miR-30b, miR-92a} \tabularnewline\rowcolor[rgb]{0.96,0.96,0.96}{}%
VASP&0.151$^{\ast}$&0.193$^{\ast}$&0.045&1 (1)&\raggedright{miR-29c*} \tabularnewline{}%
XIAP&0.089$^{\ast}$&0.224$^{\ast}$&0.136&2 (2)&\raggedright{miR-638, miR-99a} \tabularnewline\rowcolor[rgb]{0.96,0.96,0.96}{}%
XRCC1&0.130$^{\ast}$&&&0&\raggedright{} \tabularnewline{}%
YAP1&0.164$^{\ast}$&0.351$^{\ast}$&0.183&4 (4)&\raggedright{miR-106b, miR-486-5p, miR-28-5p, miR-595} \tabularnewline\rowcolor[rgb]{0.96,0.96,0.96}{}%
YBX1&0.016$^{\ast}$&0.329$^{\ast}$&0.314&2 (2)&\raggedright{miR-125b, miR-96} \tabularnewline{}%
YWHAE&0.001&0.145&0.138&4 (4)&\raggedright{miR-21, miR-1260, miR-365*, miR-204}%
\end{longtable}}
} % End of table

%\end{landscape}

Figure \ref{n-miRNAs-vs-R2} shows the number of miRNA variables included in
each full model plotted against the number of significant miRNA variables, the
$R^2$ of the full model, and the difference in $\bar{R}^2$ between the full
model and the model with only the gene variable. Each point in each plot
corresponds to one full model, and smoothed curves were fitted using locally
weighted scatter plot smoothing (LOESS). A trend emerges, where including more
miRNA variables tends to achieve better performance up to a turning point --
around 28 miRNA variables -- after which larger models seem to perform
increasingly poorly. Similar plots were created with more strict cross-
validation thresholds ($\gamma$ and $\alpha$); the models were larger on
average, but an equal trend and turning point were seen (data not shown).

\begin{figure}[htb]
\centering \includegraphics[height=11cm]{n_miRNAs_vs_R2.pdf}
\caption{Number of miRNA variables included in full model (N miRNAs) versus number of significant miRNA variables, R$^2$ of full model, and difference of $\bar{R}^2$ between full and gene only model ($\Delta\bar{R}^2$). Each point represents one model fitted for one gene. Note that some points are overlaid on top of each other. Curves were fitted with locally
weighted scatter plot smoothing (LOESS) and shaded areas represent 95\% confidence interval. \label{n-miRNAs-vs-R2}}
\end{figure}

There could be several explanations for this trend. Firstly, the largest
models could simply be a result of the predictors not fitting the observed
variable well. This often results in a large model, because including more
predictors always makes the model fit asymptotically better -- both in the
sense of $R^2$ and generally also the predictive density, which was used as
the measure of fit for the variable-selection process. From a biological
perspective, this means that the mRNA and miRNA expression are not sufficient
to explain the protein expression; other factors, that the model does not
account for (e.g. protein degradation), possibly dominate the protein
abundance. Secondly, the marginal posterior of a single predictor can indicate
non-significance, while the joint posterior of several predictors combined
might still achieve significance. This would indicate that the effect of
miRNAs is only significant when acting simultaneosly, a hypothesis that is
supported by experimental evidence, as discussed in section
\ref{microrna-function}.

To examine the latter point, \textbf{SOMETHING COULD BE DONE? AKI?}.


%!TEX root = dippa.tex
%%% This file contains the discussion section of my master's thesis.
%%% Author: Viljami Aittomäki

\section{Discussion}

This thesis presented a review of the basics of gene expression, microRNAs and
the computational prediction of microRNA target genes. A modern Bayesian
variable selection method was applied in the context of regression, to predict
protein expression from mRNA and miRNA expression in breast cancer tumor
samples, with the goal of identifying putative miRNA target genes in breast
cancer. The Bayesian method was compared to lasso regression, a popular method
for target prediction from expression data.


\subsection*{Overview of results}

Correlation between mRNA and protein expression was mostly low, a finding
supported by many previous studies \citep{Payne2015}. Interestingly, there was
no virtually difference between correlations of validated miRNA-target pairs
or randomly chosen ones. This illustrates the complexity of expression
regulation and supports the view that modeling single interaction pairs
individually is unlikely to be sufficient in general.

Compared to lasso regression PPVS achieved better model fit, yet from a target
prediction point of view, performance of the two methods was similar.
There was little overlap of predictions made by the two methods, a common
issue in microRNA target prediction. Only a small fraction of predicted
targets were validated according to TarBase, but it should be noted, that is
probably true in general of all miRNA targets; only a limited number of
validation studies have been done.

Aure et al used lasso regression for a similar analysis of the same dataset
%prediction of protein expression from mRNA and miRNA expression 
to identify miRNAs significantly affecting protein expression in breast cancer
\citep{Aure2015}. They used a multi-step process, where miRNAs deemed
significant in a simple univariate regression model were used as input in
lasso regression. This approach is flawed in the sense, that it loses some of
the power of multivariate models to identify singly weak but combinatorially
strong effects, since univariate modeling is used as a filtering step. It also
uses the same data twice, causing bias.


\subsection*{Possible improvements}

\textbf{KERTOIMIEN RAJOITTAMINEN NEGATIIVISIKSI!}

An obvious way to improve the model would be to include sequence-based target
information. Bayesian modeling provides a formal and relatively easy way to
achieve this with hierarchical models. MicroRNAs that that are putative
regulators of a gene based on sequence could have priors (or hyperpriors) with
less regularization than miRNAs without sequence matches. However, as
the authors of GenMir noted, including sequence features did not result
in a significant improvement of their method \citep{Huang2007}.

One disadvantage of the proposed method relates to scaling of the regression
covariates, that is, the mRNA and miRNA expression variables. Sequencing studies have shown
that a relatively small number of miRNAs accounts for over 80\% of tissue microRNA
\citep{Landgraf2007}. Therefore, changes in the expression of these highly
abundant miRNAs may have a relatively large impact on protein levels, where as
similar changes in less expressed miRNAs may have little to no effect on target gene
expression. This difference is lost by scaling all miRNA variables to a similar scale.

However, many of the most abundant miRNAs are ubiquitously expressed across
different tissues \citep{Landgraf2007} and, therefore, possibly less
interesting with regards to disease pathogenesis. Additionally, inclusion of
the gene expression measurement in the regression necessitates some form of
scaling, as mRNA and miRNA expression profiles are measured with different platforms
and processed using different algorithms and, therefore, are not directly
comparable. One could easily envision a scaling procedure where the relative
levels of different miRNAs are preserved and gene expression is scaled
relative to mean miRNA expression, for example, but it is debatable whether
this would be appropriate either.

Approximately half of the predicted interactions were positive. Some of these
could indicate indirect regulation. However, this proportion seems much too
high, as most known microRNA interactions are suppressive. In fact, of all the
experimentally validated human miRNA targets listed in DIANA-TarBase, only
around 0.2\% report positive regulation by the miRNA. Therefore, many of the
predicted positive interactions are possibly false positives. To amend this,
the model could easily be restricted to only negative interactions (using a
non-positive prior for $\beta$), and this has previously been reported to
increase prediction accuracy \citep{Muniategui2013}.

The proposed model also does not account for the fact that microRNAs have
several, even hundreds, of target transcripts \citep{}. Therefore, the regulatory
effect of a single miRNA is most likely spread across several genes. In
combination with transcripts having several regulating miRNAs, this many-to-many
nature of microRNA regulation ultimately calls for computational methods
that model the whole regulatory network at once. This, however, becomes a much
more difficult problem than multivariate linear regression.

In conclusion, the work in this thesis shows that the proposed method of
projection predictive variable selection is applicable to microRNA target
predction. However, further refinements to the model are warranted to improve
preformance. In the presented form, compared to a simpler alternative, the
method offered only a small advantage from a modeling perspective, and
apparent advantage from a biological perspective , but incurred a large
computational burden. The choice of model-size parameters $\alpha$ and
$\gamma$, and therefore the choice of final model size and ultimately the
model itself, proved untrivial. A data-driven approach for optimizing the
parameter values would perhaps be of value.


\subsection*{Future prospects}

The recent development of CLIPseq and similar methods has made high-throughput
experimental microRNA target discovery possible, partially replacing the need
for computational target prediction. Nonetheless, experimental methods
(particularly high-throughput ones) are not immune to error, and gene
regulation is vastly complex with many unconventional regulatory mechanisms
having been discovered. Integrative computational approaches --
combining several levels and types of data -- beyond correlation will, thus, remain
important in the future.

The elucidation of complex regulatory networks using
network-level modeling is becoming feasible with modern experimental and
computational methods. Employing this approach will be essential, as it has the
ability to better capture the true nature of gene regulation and cellular
biology.

Many aspects of microRNA biology and function are still unknown. As well as
helping us understand the complexities of molecular cell biology, uncovering
microRNA function also offers interesting possibilities in diagnostics and even
treatment of cancer, as has already been shown for instance for breast
cancer. MicroRNAs will, therefore, remain an exciting avenue of research in the future.
















%%=========================================================
%% References

\clearpage
%% The \phantomsection command is nessesary for hyperref to jump to the 
%% correct page, in other words it puts a hyper marker on the page.

\phantomsection
\addcontentsline{toc}{section}{\refname}
%\addcontentsline{toc}{section}{References}

\bibliographystyle{apalike}
\bibliography{dippa}
%\begin{thebibliography}{99}
%% Alla pilkun j\"alkeen on pakotettu oikea v\"ali \<v\"alily\"onti>-merkeill\"a.
%\bibitem{Kauranen} Kauranen,\ I., Mustakallio,\ M. ja Palmgren,\ V.
%  \textit{Tutkimusraportin kirjoittamisen opas opinn\"aytety\"on
%    tekij\"oille.}  Espoo, Teknillinen korkeakoulu, 2006.
%\end{thebibliography}










%%=========================================================
%% Appendices

\clearpage
\thesisappendix

\section{Table of model properties\label{app:full-model-table}}
%\begin{landscape}
{ % This table is generated by CSV2Latex.
\footnotesize{\begin{longtable}{llllrp{7cm}}
\caption{
Properties of fitted models for all 105 genes. A missing value for N miRNAs indicates a projected model was not found
(i.e. the stopping criterion was not satisfied before reaching 200 covariates),
a zero indicates no miRNA variables were chosen.
$R^2$ was not computed for models with no miRNA variables.
Only the significant miRNAs chosen are listed for compactness of display. \\
$R^2_{\textup{gene}}$: $R^2$ for gene-only model \\
$R^2_\perp$: $R^2$ for projected model obtained with PPVS \\
$\Delta\bar{R}^2$: Difference of adjusted $R^2$ of projected model versus gene-only model \\
$^{\ast}$: gene expression variable is significant (95\% credible interval) \\
$\textup{N}_{\textup{miRNA}}$: number of miRNA variables in projected model (number of significant miRNA variables, 95\% credible interval) 
\label{table:final-models}
}
\label{table:finalModelTable}\\\hline
\textbf{Gene} & \textbf{$R^2_{\textup{gene}}$} & \textbf{$R^2_\perp$} & \textbf{$\Delta\bar{R}^2$} & $\mathbf{\textup{N}_{miRNA}}$ & \textbf{Significant miRNAs}\\\hline
\endfirsthead{}%
\textbf{Gene} & \textbf{$R^2_{\textup{gene}}$} & \textbf{$R^2_\perp$} & \textbf{$\Delta\bar{R}^2$} & $\mathbf{\textup{N}_{miRNA}}$ & \textbf{Significant miRNAs}\\\hline
\endhead\hline\multicolumn{6}{r}{\textit{Continued on next page\ldots\/}}
\endfoot\hline\endlastfoot{}%
ACACA&0.462$^{\ast}$&0.524$^{\ast}$&0.062&2 (2)&\raggedright{miR-30a, miR-370} \tabularnewline\rowcolor[rgb]{0.96,0.96,0.96}{}%
AKT1&0.111$^{\ast}$&0.174$^{\ast}$&0.064&2 (2)&\raggedright{miR-449a, miR-342-5p} \tabularnewline{}%
AKT2&0.001&0.230&0.195&14 (4)&\raggedright{miR-342-5p, miR-449a, miR-96, miR-146b-5p} \tabularnewline\rowcolor[rgb]{0.96,0.96,0.96}{}%
AKT3&0.000&0.251&0.216&15 (4)&\raggedright{miR-342-5p, miR-449a, miR-96, miR-146b-5p} \tabularnewline{}%
ANXA1&0.380$^{\ast}$&0.425$^{\ast}$&0.047&1 (1)&\raggedright{miR-765} \tabularnewline\rowcolor[rgb]{0.96,0.96,0.96}{}%
AR&0.713$^{\ast}$&&&0&\raggedright{} \tabularnewline{}%
BAK1&0.119$^{\ast}$&0.228$^{\ast}$&0.110&2 (2)&\raggedright{miR-29c, miR-505*} \tabularnewline\rowcolor[rgb]{0.96,0.96,0.96}{}%
BAX&0.130$^{\ast}$&0.405$^{\ast}$&0.228&23 (4)&\raggedright{miR-557, miR-659, miR-142-3p, miR-199a-3p} \tabularnewline{}%
BCL2&0.728$^{\ast}$&&&0&\raggedright{} \tabularnewline\rowcolor[rgb]{0.96,0.96,0.96}{}%
BCL2L1&0.088$^{\ast}$&0.354$^{\ast}$&0.123&53 (1)&\raggedright{miR-622} \tabularnewline{}%
BCL2L11&0.176$^{\ast}$&0.318$^{\ast}$&0.143&2 (2)&\raggedright{miR-29c, miR-34a} \tabularnewline\rowcolor[rgb]{0.96,0.96,0.96}{}%
BECN1&0.005&&&&\raggedright{} \tabularnewline{}%
BID&0.015$^{\ast}$&0.134&0.122&1 (1)&\raggedright{miR-1246} \tabularnewline\rowcolor[rgb]{0.96,0.96,0.96}{}%
BIRC2&0.018$^{\ast}$&0.243$^{\ast}$&0.226&2 (2)&\raggedright{miR-1246, miR-425} \tabularnewline{}%
BRAF&0.094$^{\ast}$&0.456$^{\ast}$&0.352&8 (8)&\raggedright{miR-638, miR-125b, miR-1321, miR-107, miR-765, miR-148a, miR-135b, miR-505*} \tabularnewline\rowcolor[rgb]{0.96,0.96,0.96}{}%
CASP8&0.014&0.111$^{\ast}$&0.091&4 (4)&\raggedright{miR-99a, miR-148a, miR-631, miR-126*} \tabularnewline{}%
CAV1&0.284$^{\ast}$&0.432$^{\ast}$&0.147&3 (3)&\raggedright{miR-551b, miR-24} \tabularnewline\rowcolor[rgb]{0.96,0.96,0.96}{}%
CCNB1&0.638$^{\ast}$&0.714$^{\ast}$&0.076&2 (2)&\raggedright{miR-199a-5p, miR-30a} \tabularnewline{}%
CCND1&0.334$^{\ast}$&0.497$^{\ast}$&0.153&8 (7)&\raggedright{miR-936, miR-181c, miR-622, miR-493*, miR-19a, miR-126, miR-9*} \tabularnewline\rowcolor[rgb]{0.96,0.96,0.96}{}%
CCNE1&0.544$^{\ast}$&&&0&\raggedright{} \tabularnewline{}%
CDH1&0.449$^{\ast}$&&&0&\raggedright{} \tabularnewline\rowcolor[rgb]{0.96,0.96,0.96}{}%
CDH2&0.006&&&0&\raggedright{} \tabularnewline{}%
CDH3&0.165$^{\ast}$&0.584$^{\ast}$&0.398&17 (13)&\raggedright{miR-155, miR-10b*, miR-502-5p, miR-224, miR-489, miR-148a, miR-195, miR-197, miR-361-5p, miR-650, miR-150, miR-501-5p, miR-582-5p} \tabularnewline\rowcolor[rgb]{0.96,0.96,0.96}{}%
CDK1&0.037$^{\ast}$&&&&\raggedright{} \tabularnewline{}%
CDKN1B&0.418$^{\ast}$&0.491$^{\ast}$&0.075&1 (1)&\raggedright{miR-195} \tabularnewline\rowcolor[rgb]{0.96,0.96,0.96}{}%
CHEK1&0.019$^{\ast}$&0.168&0.122&11 (0)&\raggedright{} \tabularnewline{}%
CHEK2&0.512$^{\ast}$&&&0&\raggedright{} \tabularnewline\rowcolor[rgb]{0.96,0.96,0.96}{}%
CLDN7&0.234$^{\ast}$&0.363$^{\ast}$&0.125&4 (4)&\raggedright{miR-29c, miR-200c, miR-30b, miR-150} \tabularnewline{}%
COL6A1&0.000&0.325&0.321&4 (4)&\raggedright{miR-125b, miR-638, miR-210, miR-24} \tabularnewline\rowcolor[rgb]{0.96,0.96,0.96}{}%
CTNNA1&0.081$^{\ast}$&0.113$^{\ast}$&0.033&2 (1)&\raggedright{miR-125a-3p} \tabularnewline{}%
CTNNB1&0.002&0.339&0.287&22 (5)&\raggedright{miR-711, miR-10a, miR-31*, miR-16, miR-28-5p} \tabularnewline\rowcolor[rgb]{0.96,0.96,0.96}{}%
DIABLO&0.089$^{\ast}$&0.483$^{\ast}$&0.336&31 (6)&\raggedright{miR-378, miR-339-3p, miR-762, miR-15b, miR-144*, miR-582-5p} \tabularnewline{}%
DVL3&0.068$^{\ast}$&0.377$^{\ast}$&0.283&14 (9)&\raggedright{miR-24, miR-498, miR-140-3p, miR-223, miR-29c, miR-21, miR-432, miR-662, miR-204} \tabularnewline\rowcolor[rgb]{0.96,0.96,0.96}{}%
EEF2&0.009&0.330&0.292&14 (3)&\raggedright{miR-106b, miR-196b, miR-29c*} \tabularnewline{}%
EEF2K&0.310$^{\ast}$&&&0&\raggedright{} \tabularnewline\rowcolor[rgb]{0.96,0.96,0.96}{}%
EGFR&0.148$^{\ast}$&0.356$^{\ast}$&0.189&10 (7)&\raggedright{miR-181d, miR-181b, miR-1182, miR-495, miR-30c, miR-126, miR-183*} \tabularnewline{}%
EIF4E&0.100$^{\ast}$&0.252$^{\ast}$&0.050&36 (0)&\raggedright{} \tabularnewline\rowcolor[rgb]{0.96,0.96,0.96}{}%
EIF4EBP1&0.495$^{\ast}$&&&0&\raggedright{} \tabularnewline{}%
ERBB2&0.729$^{\ast}$&&&0&\raggedright{} \tabularnewline\rowcolor[rgb]{0.96,0.96,0.96}{}%
ERBB3&0.204$^{\ast}$&0.334$^{\ast}$&0.128&3 (3)&\raggedright{miR-199a-5p, miR-451, miR-484} \tabularnewline{}%
ERCC1&0.004&&&0&\raggedright{} \tabularnewline\rowcolor[rgb]{0.96,0.96,0.96}{}%
ERRFI1&0.012&&&&\raggedright{} \tabularnewline{}%
ESR1&0.825$^{\ast}$&&&0&\raggedright{} \tabularnewline\rowcolor[rgb]{0.96,0.96,0.96}{}%
FN1&0.495$^{\ast}$&&&0&\raggedright{} \tabularnewline{}%
FOXO3&0.094$^{\ast}$&0.225$^{\ast}$&0.128&3 (3)&\raggedright{miR-140-3p, miR-631, miR-197} \tabularnewline\rowcolor[rgb]{0.96,0.96,0.96}{}%
GAB2&0.597$^{\ast}$&&&0&\raggedright{} \tabularnewline{}%
GATA3&0.708$^{\ast}$&&&0&\raggedright{} \tabularnewline\rowcolor[rgb]{0.96,0.96,0.96}{}%
GSK3A&0.106$^{\ast}$&0.473$^{\ast}$&0.319&26 (5)&\raggedright{miR-21, miR-29a, miR-20a, miR-100, miR-92a} \tabularnewline{}%
GSK3B&0.064$^{\ast}$&0.524$^{\ast}$&0.275&81 (0)&\raggedright{} \tabularnewline\rowcolor[rgb]{0.96,0.96,0.96}{}%
IGF1R&0.637$^{\ast}$&&&0&\raggedright{} \tabularnewline{}%
IGFBP2&0.553$^{\ast}$&&&0&\raggedright{} \tabularnewline\rowcolor[rgb]{0.96,0.96,0.96}{}%
INPP4B&0.754$^{\ast}$&&&0&\raggedright{} \tabularnewline{}%
IRS1&0.397$^{\ast}$&0.441$^{\ast}$&0.046&1 (1)&\raggedright{miR-93} \tabularnewline\rowcolor[rgb]{0.96,0.96,0.96}{}%
KDR&0.000&0.281&0.272&6 (6)&\raggedright{miR-150, miR-663, miR-495, miR-24-1*, miR-363, miR-140-3p} \tabularnewline{}%
KIT&0.581$^{\ast}$&&&0&\raggedright{} \tabularnewline\rowcolor[rgb]{0.96,0.96,0.96}{}%
KRAS&0.037$^{\ast}$&0.335&0.264&16 (2)&\raggedright{miR-96, miR-21} \tabularnewline{}%
MAP2K1&0.006&0.141&0.138&1 (1)&\raggedright{miR-21} \tabularnewline\rowcolor[rgb]{0.96,0.96,0.96}{}%
MAPK14&0.017$^{\ast}$&0.346$^{\ast}$&0.301&14 (9)&\raggedright{miR-145, miR-92a, miR-181c, miR-142-3p, miR-425, miR-339-3p, miR-342-5p, miR-18b, miR-1226*} \tabularnewline{}%
MAPK9&0.193$^{\ast}$&0.251$^{\ast}$&0.061&1 (1)&\raggedright{miR-342-5p} \tabularnewline\rowcolor[rgb]{0.96,0.96,0.96}{}%
MAPT&0.020$^{\ast}$&0.335&0.270&20 (3)&\raggedright{miR-30c, miR-132, miR-17*} \tabularnewline{}%
MET&0.031$^{\ast}$&0.085$^{\ast}$&0.054&2 (2)&\raggedright{miR-125b, miR-139-5p} \tabularnewline\rowcolor[rgb]{0.96,0.96,0.96}{}%
MRE11A&0.012&0.290&0.052&70 (0)&\raggedright{} \tabularnewline{}%
MSH2&0.333$^{\ast}$&&&0&\raggedright{} \tabularnewline\rowcolor[rgb]{0.96,0.96,0.96}{}%
MSH6&0.340$^{\ast}$&0.628$^{\ast}$&0.265&19 (10)&\raggedright{miR-125b, miR-324-5p, miR-25, miR-195*, miR-451, miR-154, miR-551b, miR-26b, miR-513a-5p} \tabularnewline{}%
MYC&0.025$^{\ast}$&0.101$^{\ast}$&0.076&2 (2)&\raggedright{miR-150*, miR-24} \tabularnewline\rowcolor[rgb]{0.96,0.96,0.96}{}%
NCOA3&0.132$^{\ast}$&&&0&\raggedright{} \tabularnewline{}%
NF2&0.132$^{\ast}$&0.321$^{\ast}$&0.187&3 (3)&\raggedright{miR-638, miR-125b, miR-22} \tabularnewline\rowcolor[rgb]{0.96,0.96,0.96}{}%
NOTCH1&0.216$^{\ast}$&0.279$^{\ast}$&0.064&2 (2)&\raggedright{miR-199b-5p, miR-502-5p} \tabularnewline{}%
NOTCH3&0.150$^{\ast}$&0.401$^{\ast}$&0.245&5 (5)&\raggedright{miR-125b, miR-27b, miR-193a-5p, miR-32, miR-139-5p} \tabularnewline\rowcolor[rgb]{0.96,0.96,0.96}{}%
PARK7&0.288$^{\ast}$&0.428$^{\ast}$&0.140&2 (2)&\raggedright{miR-93, miR-29c} \tabularnewline{}%
PCNA&0.229$^{\ast}$&0.283$^{\ast}$&0.057&1 (1)&\raggedright{miR-199a-5p} \tabularnewline\rowcolor[rgb]{0.96,0.96,0.96}{}%
PECAM1&0.054$^{\ast}$&0.169$^{\ast}$&-0.028&43 (0)&\raggedright{} \tabularnewline{}%
PGR&0.819$^{\ast}$&&&0&\raggedright{} \tabularnewline\rowcolor[rgb]{0.96,0.96,0.96}{}%
PIK3CA&0.051$^{\ast}$&&&0&\raggedright{} \tabularnewline{}%
PIK3R1&0.077$^{\ast}$&0.377$^{\ast}$&0.295&5 (5)&\raggedright{miR-142-3p, miR-342-3p, miR-501-5p, miR-145*, miR-92a} \tabularnewline\rowcolor[rgb]{0.96,0.96,0.96}{}%
PRKAA1&0.119$^{\ast}$&0.240$^{\ast}$&0.121&2 (2)&\raggedright{miR-199a-3p, miR-342-3p} \tabularnewline{}%
PRKCA&0.368$^{\ast}$&&&0&\raggedright{} \tabularnewline\rowcolor[rgb]{0.96,0.96,0.96}{}%
PTCH1&0.029$^{\ast}$&0.167$^{\ast}$&0.127&6 (2)&\raggedright{miR-145, miR-934} \tabularnewline{}%
PTEN&0.029$^{\ast}$&0.468$^{\ast}$&0.370&35 (5)&\raggedright{miR-204, miR-498, miR-324-3p, miR-30d, miR-22} \tabularnewline\rowcolor[rgb]{0.96,0.96,0.96}{}%
PTGS2&0.065$^{\ast}$&0.114$^{\ast}$&0.052&1 (1)&\raggedright{miR-1246} \tabularnewline{}%
PTK2&0.021$^{\ast}$&0.273$^{\ast}$&0.242&6 (5)&\raggedright{miR-1246, miR-150, miR-21, miR-200c, miR-32} \tabularnewline\rowcolor[rgb]{0.96,0.96,0.96}{}%
PXN&0.133$^{\ast}$&0.431$^{\ast}$&0.140&63 (0)&\raggedright{} \tabularnewline{}%
RAB25&0.031$^{\ast}$&&&&\raggedright{} \tabularnewline\rowcolor[rgb]{0.96,0.96,0.96}{}%
RAD50&0.076$^{\ast}$&0.106$^{\ast}$&0.033&1 (1)&\raggedright{miR-139-5p} \tabularnewline{}%
RAD51&0.011&0.205&0.195&2 (2)&\raggedright{miR-1246, miR-181c} \tabularnewline\rowcolor[rgb]{0.96,0.96,0.96}{}%
RAF1&0.121$^{\ast}$&0.446$^{\ast}$&0.306&12 (9)&\raggedright{miR-125b, miR-1260, miR-498, miR-449a, miR-130a, miR-30b, miR-28-5p, miR-374a, miR-195*} \tabularnewline{}%
RB1&0.006&0.233$^{\ast}$&0.222&4 (4)&\raggedright{miR-145, miR-1260, miR-451, miR-195*} \tabularnewline\rowcolor[rgb]{0.96,0.96,0.96}{}%
RPS6KB1&0.516$^{\ast}$&0.617$^{\ast}$&0.102&2 (2)&\raggedright{miR-497, miR-106b} \tabularnewline{}%
SMAD1&0.359$^{\ast}$&0.417$^{\ast}$&0.060&1 (1)&\raggedright{miR-106b} \tabularnewline\rowcolor[rgb]{0.96,0.96,0.96}{}%
SMAD3&0.354$^{\ast}$&0.475$^{\ast}$&0.112&7 (7)&\raggedright{miR-7, miR-451, miR-181d, miR-29b, miR-365, miR-1225-5p, miR-1299} \tabularnewline{}%
SMAD4&0.000&&&0&\raggedright{} \tabularnewline\rowcolor[rgb]{0.96,0.96,0.96}{}%
SNAI1&0.001&&&0&\raggedright{} \tabularnewline{}%
SRC&0.308$^{\ast}$&&&0&\raggedright{} \tabularnewline\rowcolor[rgb]{0.96,0.96,0.96}{}%
STAT5A&0.292$^{\ast}$&0.438$^{\ast}$&0.147&2 (2)&\raggedright{miR-155, miR-324-5p} \tabularnewline{}%
STMN1&0.000&0.364&-0.028&109 (0)&\raggedright{} \tabularnewline\rowcolor[rgb]{0.96,0.96,0.96}{}%
SYK&0.227$^{\ast}$&0.714$^{\ast}$&0.462&26 (20)&\raggedright{miR-125b, miR-155, miR-324-5p, miR-195, miR-449a, miR-711, miR-762, miR-940, miR-22, miR-615-3p, miR-663, miR-377*, miR-598, miR-126, miR-29b, miR-1228*, miR-144, miR-204, miR-601, miR-30d} \tabularnewline{}%
TP53&0.017$^{\ast}$&0.186$^{\ast}$&0.161&5 (5)&\raggedright{miR-301b, miR-29c*, miR-551b, miR-19a, miR-874} \tabularnewline\rowcolor[rgb]{0.96,0.96,0.96}{}%
TP53BP1&0.260$^{\ast}$&0.357$^{\ast}$&0.099&1 (1)&\raggedright{miR-96} \tabularnewline{}%
TSC2&0.065$^{\ast}$&0.318$^{\ast}$&0.225&13 (3)&\raggedright{miR-1915, miR-30b, miR-92a} \tabularnewline\rowcolor[rgb]{0.96,0.96,0.96}{}%
VASP&0.151$^{\ast}$&0.193$^{\ast}$&0.045&1 (1)&\raggedright{miR-29c*} \tabularnewline{}%
XIAP&0.089$^{\ast}$&0.224$^{\ast}$&0.136&2 (2)&\raggedright{miR-638, miR-99a} \tabularnewline\rowcolor[rgb]{0.96,0.96,0.96}{}%
XRCC1&0.130$^{\ast}$&&&0&\raggedright{} \tabularnewline{}%
YAP1&0.164$^{\ast}$&0.351$^{\ast}$&0.183&4 (4)&\raggedright{miR-106b, miR-486-5p, miR-28-5p, miR-595} \tabularnewline\rowcolor[rgb]{0.96,0.96,0.96}{}%
YBX1&0.016$^{\ast}$&0.329$^{\ast}$&0.314&2 (2)&\raggedright{miR-125b, miR-96} \tabularnewline{}%
YWHAE&0.001&0.145&0.138&4 (4)&\raggedright{miR-21, miR-1260, miR-365*, miR-204}%
\end{longtable}}
} % End of table

%\end{landscape}


\section{Model size distributions\label{app:model-sizes}}

\begin{figure}[htb]
  \centering
  \includegraphics[width=1\linewidth]{figures/varNumHistogram/ZZ_variable_number_hist.pdf}
  \caption{Distributions of model sizes for differrent values of parameters $\alpha$ and $\gamma$.
  $\# 0$ refers to the number of models with no covariates (i.e. not even the mRNA covariate was chosen)
  and $\# \textup{NA}$ to the number of models where the model-size criterion was not met.}
  \label{fig:model-size-distribution}
\end{figure}


%!TEX root = dippa.tex
%%% This file contains the QC plots appendix for my master's thesis
%%% Author: Viljami Aittomäki



\section{Quality control plots\label{app:qc-plots}}

\captionsetup{font={up,footnotesize},labelfont={sf}}

\begin{figure}[!h]
	\centering
	\begin{subfigure}{.7\textwidth}
		\centering
		\includegraphics[width=1\linewidth]{figures/proteinQCPlots/proteinQCPlots-pcaplot.pdf}
		\subcaption{protein\label{fig:protein-sample-pca}}
	\end{subfigure}
	\begin{subfigure}{.48\textwidth}
		\centering
		\includegraphics[width=1\linewidth]{figures/geneQCPlots/geneQCPlots-pcaplot.pdf}
		\subcaption{mRNA\label{fig:mrna-sample-pca}}
	\end{subfigure}
	\begin{subfigure}{.48\textwidth}
		\centering
		\includegraphics[width=1\linewidth]{figures/mirnaQCPlots/mirnaQCPlots-pcaplot.pdf}
		\subcaption{miRNA\label{fig:mirna-sample-pca}}
	\end{subfigure}

	\caption{Scatter plots of first two principal components for each tumor sample (or microarray)
	computed from (a) protein expression, (b) mRNA expression, and (c) miRNA expression data.
	The point color corresponds to the hospital where each sample was handled. The samples
	from different hospitals were not distinguishable using the principal components, which
	suggests that no significant hospital batch effect was present.}
	\label{fig:qc-pca}
\end{figure}


\begin{figure}
	\centering
	\begin{subfigure}{1\textwidth}
		\centering
		\makebox[\textwidth][c]{\includegraphics[width=1.1\linewidth]{figures/proteinQCPlots/proteinQCPlots-sample_boxplot.pdf}}
		\subcaption{Protein expression by tumor samples \label{fig:protein-sample-boxplot}}
	\end{subfigure}
	\begin{subfigure}{1\textwidth}
		\centering
		\makebox[\textwidth][c]{\includegraphics[width=1.1\linewidth]{figures/proteinQCPlots/proteinQCPlots-variable_boxplot.pdf}}
		\subcaption{Protein variables \label{fig:protein-variable-boxplot}}
	\end{subfigure}
	\begin{subfigure}{1\textwidth}
		\centering
		\makebox[\textwidth][c]{\includegraphics[width=1.1\linewidth]{figures/proteinQCPlots/proteinQCPlots-variable_boxplot_scaled.pdf}}
		\subcaption{Scaled protein variables ($\mu_y = 0, \sigma_y = 1$) \label{fig:protein-scaled-variable-boxplot}}
	\end{subfigure}

	\caption{Distribution of protein expression, grouped by
	(a) tumor samples,
	(b) protein variables (i.e. microarrays), 
	and (c) protein variables (and scaled to zero mean and unit variance).
	The fill color in (a) corresponds to the hospital where the sample was collected.
	It is evident in (c), that some of the protein variables have significant outliers
	(especially the two values beyond $10$ and $-10$).
	}
	\label{fig:qc-protein-boxplot}
\end{figure}


\begin{figure}[!h]
	\centering
	\begin{subfigure}{.49\textwidth}
		\subcaption{Protein expression by tumor samples \label{fig:protein-sample-density}}
		\includegraphics[width=1\linewidth]{figures/proteinQCPlots/proteinQCPlots-sample_density.pdf}
	\end{subfigure}
	\begin{subfigure}{.49\textwidth}
		\subcaption{Protein variables \label{fig:protein-variable-density}}
		\includegraphics[width=1\linewidth]{figures/proteinQCPlots/proteinQCPlots-variable_density.pdf}
	\end{subfigure}

	\caption{Density estimates of protein expression data for each tumor sample (a)
	and each protein variable (b). The color of the curves has no significance.
	Distributions were generally uniform, but
	long tails for several protein variables are visible in (b), suggesting the presence of significant
	outliers and that these variables were not approximated well by a normal ditribution.}
	\label{fig:qc-protein-density}
\end{figure}


\begin{figure}
	\centering
	\begin{subfigure}{1\textwidth}
		\centering
		\makebox[\textwidth][c]{\includegraphics[width=1.1\linewidth]{figures/geneQCPlots/geneQCPlots-sample_boxplot.pdf}}
		\subcaption{mRNA microarrays \label{fig:mrna-sample-boxplot}}
	\end{subfigure}
	\begin{subfigure}{1\textwidth}
		\centering
		\makebox[\textwidth][c]{\includegraphics[width=1.1\linewidth]{figures/geneQCPlots/geneQCPlots-variable_boxplot.pdf}}
		\subcaption{mRNA variables \label{fig:mrna-variable-boxplot}}
	\end{subfigure}
	\begin{subfigure}{1\textwidth}
		\centering
		\makebox[\textwidth][c]{\includegraphics[width=1.1\linewidth]{figures/geneQCPlots/geneQCPlots-variable_boxplot_scaled.pdf}}
		\subcaption{Scaled mRNA variables ($\mu_z = 0, \sigma_z = 1$). \label{fig:mrna-scaled-variable-boxplot}}
	\end{subfigure}

	\caption{Distribution of mRNA expression, grouped by
	(a) tumor samples (i.e. microarrays),
	(b) mRNA variables, 
	and (c) mRNA variables (and scaled to zero mean and unit variance).
	The fill color in (a) corresponds to the hospital where the sample was collected.
	The distributions were fairly uniform and only a few significant outliers (further than
	5 standard devations from the mean) were present in the mRNA variables.}
	\label{fig:qc-mrna-boxplot}
\end{figure}


\begin{figure}[!h]
	\centering
	\begin{subfigure}{.49\textwidth}
		\subcaption{mRNA microarrays \label{fig:mrna-sample-density}}
		\includegraphics[width=1\linewidth]{figures/geneQCPlots/geneQCPlots-sample_density.pdf}
	\end{subfigure}
	\begin{subfigure}{.49\textwidth}
		\subcaption{mRNA variables \label{fig:mrna-variable-density}}
		\includegraphics[width=1\linewidth]{figures/geneQCPlots/geneQCPlots-variable_density.pdf}
	\end{subfigure}

	\caption{Density estimates of mRNA expression data for each microarray (a)
	and each mRNA variable (b). The color of the curves has no significance.
	The array distributions were uniform and quite close to normal. Some of the variable distributions
	had long tails as seen in (b).}
	\label{fig:qc-mrna-density}
\end{figure}


\begin{figure}
	\centering
	\begin{subfigure}{1\textwidth}
		\centering
		\makebox[\textwidth][c]{\includegraphics[width=1.1\linewidth]{figures/mirnaQCPlots/mirnaQCPlots-sample_boxplot.pdf}}
		\subcaption{miRNA microarrays \label{fig:mirna-sample-boxplot}}
	\end{subfigure}
	\begin{subfigure}{1\textwidth}
		\centering
		\makebox[\textwidth][c]{\includegraphics[width=1.1\linewidth]{figures/mirnaQCPlots/mirnaQCPlots-variable_boxplot_sorted.pdf}}
		\subcaption{miRNA variables \label{fig:mirna-variable-boxplot}}
	\end{subfigure}
	\begin{subfigure}{1\textwidth}
		\centering
		\makebox[\textwidth][c]{\includegraphics[width=1.1\linewidth]{figures/mirnaQCPlots/mirnaQCPlots-variable_boxplot_scaled_sorted.pdf}}
		\subcaption{Scaled miRNA variables ($\mu_x = 0, \sigma_x = 1$) \label{fig:mirna-scaled-variable-boxplot}}
	\end{subfigure}

	\caption{Distribution of miRNA expression, grouped by
	(a) tumor samples (i.e. microarrays),
	(b) miRNA variables, 
	and (c) miRNA variables (and scaled to $\mu_x = 0, \sigma_x = 1$).
	Fill in (a) corresponds to hospital.
	The miRNA variables in (b) have been sorted by median expression to highlight that
	some miRNAs had very low expression. There appeared to be a gap at around $-8$
	(confirmed in Fig. \ref{fig:qc-mirna-density}). Measurements below the gap possibly corresponded
	to miRNAs not expressed in the samples, and thus, likely consisted
	mainly of background noise, Real miRNA abundances should follow a more continuous distribution.
	There also seemed to be some technical artifact around the gap, where some miRNA variables had
	exactly the same expression value. No biological phenomenon would explain this.
	The same order is retained in (c). The scaling did not
	correct for the highly skewed distributions of the most lowly expressed miRNAs.}
	\label{fig:qc-mirna-boxplot}
\end{figure}


\begin{figure}[!h]
	\centering
	\begin{subfigure}{.49\textwidth}
		\subcaption{miRNA microarrays \label{fig:mirna-sample-density}}
		\includegraphics[width=1\linewidth]{figures/mirnaQCPlots/mirnaQCPlots-sample_density.pdf}
	\end{subfigure}%
	\begin{subfigure}{.49\textwidth}
		\subcaption{miRNA variables \label{fig:mirna-variable-density}}
		\includegraphics[width=1\linewidth]{figures/mirnaQCPlots/mirnaQCPlots-variable_density.pdf}
	\end{subfigure}

	\caption{Density estimates of miRNA expression data for each microarray (a)
	and each miRNA variable (b). Color of the curves has no significance.
	Note the bimodal distributions in (a) and the corresponding break at around $-8$
	and large spread in location in (b). This suggests that the miRNAs below the gap were present in
	such low quantities, possibly not expressed at all, that the measured expression values likely
	consisted mostly of background noise.}
	\label{fig:qc-mirna-density}
\end{figure}


\end{document}
