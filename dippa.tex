%%%%%%%%%%%%%%%%%%%%%%%%%%%%%%%%%%%%%%%%%%%%%%%%%%%%%%%%%%%%%%%%%%%%
%%%%%%%%%%%%%%%%%%%%%%%%%%%%%%%%%%%%%%%%%%%%%%%%%%%%%%%%%%%%%%%%%%%%
%%                                                                %%
%% An example for writting your thesis using LaTeX                %%
%% Original version by Luis Costa,  changes by Perttu Puska       %%
%% Support for Swedish added 15092014                             %%
%%                                                                %%
%% Explanatory comments in this example begin with                %%
%% the characters %%, and changes that the user can make          %%
%% with the character %                                           %%
%%                                                                %%
%%%%%%%%%%%%%%%%%%%%%%%%%%%%%%%%%%%%%%%%%%%%%%%%%%%%%%%%%%%%%%%%%%%%
%%%%%%%%%%%%%%%%%%%%%%%%%%%%%%%%%%%%%%%%%%%%%%%%%%%%%%%%%%%%%%%%%%%%

%% Uncomment one of these:
%% the 1st when using pdflatex, which directly typesets your document in
%% pdf (use jpg or pdf figures), or
%% the 2nd when producing a ps file (use eps figures, don't use ps figures!).
\documentclass[english,12pt,a4paper,pdftex,elec,utf8]{aaltothesis}
%\documentclass[english,12pt,a4paper,dvips]{aaltothesis}

%% To the \documentclass above
%% specify your school: arts, biz, chem, elec, eng, sci
%% specify the character encoding scheme used by your editor: utf8, latin1

\usepackage{graphicx}

%% Use this if you write hard core mathematics, these are usually needed
\usepackage{amsfonts,amssymb,amsbsy}

%% Use the macros in this package to change how the hyperref package below 
%% typesets its hypertext -- hyperlink colour, font, etc. See the package
%% documentation. It also defines the \url macro, so use the package when 
%% not using the hyperref package.
%%
%\usepackage{url}

%% Use this if you want to get links and nice output. Works well with pdflatex.
\usepackage{hyperref}
\hypersetup{pdfpagemode=UseNone, pdfstartview=FitH,
  colorlinks=true,urlcolor=red,linkcolor=blue,citecolor=black,
  pdftitle={Default Title, Modify},pdfauthor={Your Name},
  pdfkeywords={Modify keywords}}


%% All that is printed on paper starts here
\begin{document}
\renewcommand{\thesissupervisorname}{Thesis supervisors}

%% Change the school field to specify your school if the automatically 
%% set name is wrong
% \university{aalto-yliopisto}
% \university{aalto University}
% \school{Sähkötekniikan korkeakoulu}
% \school{School of Electrical Engineering}

%% ONLY FOR M.Sc. AND LICENTIATE THESIS: Specify your department,
%% professorship and professorship code. 
%%
\department{Department of WHATTHEHELL}
\professorship{Computational WHATEVER}
%%

%% Valitse yksi näistä kolmesta
%%
%% Choose one of these:
%\univdegree{BSc}
\univdegree{MSc}
%\univdegree{Lic}

%% Your own name (should be self explanatory...)
\author{Viljami Aittom\"aki}

%% Your thesis title comes here and again before a possible abstract in
%% Finnish or Swedish . If the title is very long and latex does an
%% unsatisfactory job of breaking the lines, you will have to force a
%% linebreak with the \\ control character. 
%% Do not hyphenate titles.
%% 
\thesistitle{Thesis template}

\place{Espoo}

%% For B.Sc. thesis use the date when you present your thesis.
%% BUT WHAT IS IT FOR MSc THESIS?
\date{1.6.2016}

%% B.Sc. or M.Sc. thesis supervisor 
%% Note the "\" after the comma. This forces the following space to be 
%% a normal interword space, not the space that starts a new sentence. 
%% This is done because the fullstop isn't the end of the sentence that
%% should be followed by a slightly longer space but is to be followed
%% by a regular space.
%%
\supervisor{Prof.\ Aki Vehtari}

%% B.Sc. or M.Sc. thesis advisors(s).
%%
%\advisor{Prof.\ Pirjo Professori}
%\advisor{D.Sc.\ (Tech.) Olli Ohjaaja}
%\advisor{M.Sc.\ Polli Pohjaaja}
\advisor{Docent Rainer Lehtonen}

%% Aalto logo: syntax:
%% \uselogo{aaltoRed|aaltoBlue|aaltoYellow|aaltoGray|aaltoGrayScale}{?|!|''}
%% Logo language is set to be the same as the document language.
%%
\uselogo{aaltoRed}{''}


%% Create the coverpage
%%
\makecoverpage















%% English abstract.
%% All the information required in the abstract (your name, thesis title, etc.)
%% is used as specified above.
%% Specify keywords
%%
\keywords{breast cancer, microRNA, Bayes, microarray}
%% Abstract text
\begin{abstractpage}[english]
  Your abstract in English. Try to keep the abstract short; approximately 
  100 words should be enough. The abstract explains your research topic, 
  the methods you have used, and the results you obtained.  
\end{abstractpage}

%% Force a new page so that the Finnish abstract starts on a new page
\newpage

%% Abstract in Finnish.  Delete if you don't need it. 
\thesistitle{Opinnäyteohje}
\advisor{TkT Olli Ohjaaja}
\degreeprogram{Electronics and electrical engineering}
\department{Radiotieteen ja -tekniikan laitos}
\professorship{Piiriteoria}
%% Avainsanat
\keywords{rintasy\"op\"a, mikroRNA, Bayes, mikrosiru}
%% Tiivistelman tekstiosa
\begin{abstractpage}[finnish]
  Tiivistelmässä on lyhyt selvitys (noin 100 sanaa)
  kirjoituksen tärkeimmästä sisällöstä: mitä ja miten on tutkittu,
  sekä mitä tuloksia on saatu. 
\end{abstractpage}

%% Force new page so that the preface starts from a new page
\newpage

%% Preface
%%
\mysection{Preface}
Plaadi plaa plaa.\\

\vspace{5cm}
Helsinki, 1.6.2016

\vspace{5mm}
{\hfill Viljami Aittom\"aki \hspace{1cm}}

%% Force new page after preface
\newpage


%% Table of contents. 
\thesistableofcontents











%% Symbols and abbreviations
\mysection{Symbols and abbreviations}

\subsection*{Symbols}

\begin{tabular}{ll}
$w$ & vector of weights for the covariates \\
$x$ & vector of covariates \\
$y$ & PREDICTED variable
% $\mathbf{B}$  & magnetic flux density  \\
% $c$              & speed of light in vacuum $\approx 3\times10^8$ [m/s]\\
% $\omega_{\mathrm{D}}$    & Debye frequency \\
% $\omega_{\mathrm{latt}}$ & average phonon frequency of lattice \\
% $\uparrow$       & electron spin direction up\\
% $\downarrow$     & electron spin direction down
\end{tabular}

% \subsection*{Operators}

% \begin{tabular}{ll}
% $\nabla \times \mathbf{A}$              & curl of vectorin $\mathbf{A}$\\
% $\displaystyle\frac{\mbox{d}}{\mbox{d} t}$ & derivative with respect to 
% variable $t$\\[3mm]
% $\displaystyle\frac{\partial}{\partial t}$  & partial derivative with respect 
% to variable $t$ \\[3mm]
% $\sum_i $                       & sum over index $i$\\
% $\mathbf{A} \cdot \mathbf{B}$    & dot product of vectors $\mathbf{A}$ and 
% $\mathbf{B}$
% \end{tabular}

\subsection*{Abbreviations}

\begin{tabular}{ll}
miRNA       & microRNA \\
mRNA        & messenger RNA \\
RNA         & ribonucleic acid
\end{tabular}


%% Tweaks the page numbering to meet the requirement of the thesis format:
%% Begin the pagenumbering in Arabian numerals (and leave the first page
%% of the text body empty, see \thispagestyle{empty} below).
%% Additionally, force the actual text to begin on a new page with the 
%% \clearpage command.
%% \clearpage is similar to \newpage, but it also flushes the floats (figures
%% and tables).
%% There is no need to change these
%%
\cleardoublepage
\storeinipagenumber
\pagenumbering{arabic}
\setcounter{page}{1}














%%=========================================================







%% Text body begins. Note that since the text body
%% is mostly in Finnish the majority of comments are
%% also in Finnish after this point. There is no point in explaining
%% Finnish-language specific thesis conventions in English. Someday 
%% this text will possibly be translated to English.
%%
\section{Introduction}

%% Leave first page empty
\thispagestyle{empty}

Here be dragons!

And beware the harpies!











%% In a thesis, every section starts a new page, hence \clearpage
\clearpage
\section{Background}

























\clearpage
\section{Research material and methods}
























\clearpage
\section{Results}


















\clearpage
\section{Summary} 























\clearpage
%% References
%% The \phantomsection command is nessesary for hyperref to jump to the 
%% correct page, in other words it puts a hyper marker on the page.
%% MITES TAA TEHAAN BIBTEXILLA???

\phantomsection
\addcontentsline{toc}{section}{\refname}
%\addcontentsline{toc}{section}{References}

%\begin{thebibliography}{99}
%% Alla pilkun j\"alkeen on pakotettu oikea v\"ali \<v\"alily\"onti>-merkeill\"a.
%\bibitem{Kauranen} Kauranen,\ I., Mustakallio,\ M. ja Palmgren,\ V.
%  \textit{Tutkimusraportin kirjoittamisen opas opinn\"aytety\"on
%    tekij\"oille.}  Espoo, Teknillinen korkeakoulu, 2006.
%\end{thebibliography}





















%% Appendices
%\clearpage
%\thesisappendix

%\section{Esimerkki liitteest\"a\label{LiiteA}}

\end{document}
