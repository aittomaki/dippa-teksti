%!TEX root = dippa.tex
%%% This file contains the results section of my master's thesis.
%%% Author: Viljami Aittomäki

\section{Results}

This section presents the results of performed analyses, and is structured as
follows. First, a short summary of quality control. The second subsection
present results from correlation analyses between variables of different data
types. Then results from the model simulations for projection predictive
variable selection are presented. Last, the performance of projection
prediction and lasso regression is assessed from a target prediction point of
view.




\subsection{Quality control}

Quality-control plots are presented in Appendix \ref{app:qc-plots}. Two
observations are worth noting. First, several genes have significant outliers
in the protein data (most notably the genes CDK1, ERRFI1, PIK3CA). This is
most apparent in the scaled data. Second, the miRNA microarrays have bimodal
distributions, and many miRNA variables are highly skewed towards very small
values. This is indicative of generally low abundances of some miRNA molecules
and, although partly due to different preprocessing compared to the mRNA and
protein arrays, it raises suspicion of significant noise in the miRNA data.
More strict filtering of data from the miRNA microarrays could lead to better
overall results.

The first two principal components for each data type were plotted and are
shown in Figure \ref{fig:qc-pca}. No significant batch effect is visible. The
hierarchical clustering of samples (not shown) had a similar result.
Raw array data produced from Agilent array readers was not assessed.




\subsection{Correlation analysis}

Pearson correlations between variables from the different data types are shown in Figure
\ref{fig:correlations}. Protein-mRNA correlation was clearly higher for gene-matched pairs
(that is protein and mRNA expression corresponding to the same gene) than unmatched pairs,
and correlations for unmatched pairs were concentrated around zero, both expected
results. Notably, even the gene-matched correlations were quite low, with a
mean $\bar{\rho} \approx 0.37$. Using the squared correlation coefficient $\rho^2$ as a
measure of explained variance (although this has been criticized), the amount
of protein variance explained by the matched mRNA ranged from 0\% to 82\% with a mean of
21\%. This is in alignment with earlier results.

\begin{figure}[!h]
  \centering
  \begin{subfigure}{.45\textwidth}
    \centering
    \includegraphics[width=1\linewidth]{figures/correlationPlots/protein-gene-correlation.pdf}
    \subcaption{ \label{fig:protein-gene-cor}}
  \end{subfigure}
  \begin{subfigure}{.45\textwidth}
    \centering
    \includegraphics[width=1\linewidth]{figures/correlationPlots/protein-mirna-correlation.pdf}
    \subcaption{ \label{fig:protein-mirna-cor}}
  \end{subfigure}
  \begin{subfigure}{.45\textwidth}
    %\centering
    \includegraphics[width=1\linewidth]{figures/correlationPlots/gene-mirna-correlation.pdf}
    \subcaption{ \label{fig:gene-mirna-cor}}
  \end{subfigure}

  \caption{Distributions of Pearson correlations between variables of different expression data types.
  The distributions are trimmed at the smallest and largest values.
  (a) Correlation between protein and mRNA pairs, where "matched" refers to correlating
  protein and mRNA from the same gene.
  (b) Correlation between protein and miRNA pairs, where "validated" refers to the gene
  being a validated target of the miRNA (according to TarBase),
  and "random" to a randomly picked group of protein-miRNA pairs.
  (c) Correlation between mRNA and miRNA pairs,
  where grouping is the same as in (b).}
  \label{fig:correlations}
\end{figure}

Correlations between miRNAs and their validated target genes (fetched from
TarBase) are mostly low with no preference towards negative (or positive)
values. There is virtually no difference in correlations between validated
targets and randomly picked gene-miRNA pairs, contrary to what might be
expected. This suggests that (Pearson) correlation of expression data is not a
suitable method for finding miRNA targets. It is possible that this is due to
the dataset used (and the noise in said data), however, as discussed in
Section \ref{expr-methods}, correlation has low power to detect weak
relationships and cannot capitalize on combinatorial effects. Therefore,
correlation was not used for target prediction in this work; poor
results with using correlation have been reported before
(see e.g. \citep{Muniategui2012}).




\subsection{Model simulations}

Simulations for the cross validation (to determine good model size) took
between 4 to 6 hours (on a computing-cluster node with 2 12-core Xeon E5 2680
2.50GHz processors). Simulations for the final projected models took between 2
to 45 minutes, depending on the chosen model size. All parameters for all
simulations had low potential scale reduction $\hat{R}^2 < 1.1$, indicating
good convergence of simulation chains \citep{Gelman2013}.

The parameters for the model-size criterion, $\alpha$ and $\gamma$ had a
significant effect on the resulting model sizes (a plot of model-size
distributions is shown in appendix Figure \ref{fig:model-size-distribution}).
Values $\alpha = 0.50$ and $\gamma = 0.2$ were chosen to keep models
relatively sparse. This choice was, however, largely subjective. Out of the
105 genes in the data, a projected model (where at least one miRNA variable
was included, and the model-size criterion was met in under 200 variables) was
found for 74 genes. A table of properties for the projected models is
presented in Appendix \ref{app:model-table}.

Figure \ref{n-miRNAs-vs-R2} shows the number of miRNA variables included in
each projected model plotted against the number of significant miRNA
variables, the $R^2$ of the projected model, and the difference in $\bar{R}^2$
between the projected model and the gene-only model. Each point in each plot
corresponds to one gene. A trend emerges, where including more
miRNA variables tends to achieve better performance up to a turning point --
around 28 miRNA variables -- after which larger models seem to perform
increasingly poorly. Similar plots were created with more strict model-size
thresholds ($\gamma$ and $\alpha$); the models were larger on
average, but an equal trend and turning point were observed (data not shown).

\begin{figure}[!h]
  \centering
  \includegraphics[height=11cm]{n_miRNAs_vs_R2.pdf}
  \caption{Size of final model compared to model goodness-of-fit. Plotted is
  the total number of chosen miRNA variables in each full model (N miRNAs)
  versus number of significant miRNA variables, R$^2$ of full model, and
  difference of $\bar{R}^2$ between full and gene only model
  ($\Delta\bar{R}^2$). Each point represents one model fitted for one gene. A
  small jitter has been added to the points to help visualize all of them.
  Curves were fitted with locally weighted scatter plot smoothing (LOESS) and
  shaded areas represent 95\% confidence interval.}
  \label{n-miRNAs-vs-R2}
\end{figure}

There could be several explanations for this trend. Firstly, the largest
models could simply be a result of the predictors not fitting the observed
variable well. This can result in a large model, because including more
predictors often makes the model fit asymptotically better -- both in the
sense of $R^2$ and generally also the predictive density, which was used as
the measure of fit for the variable-selection process. From a biological
perspective, this means that the mRNA and miRNA expression are not sufficient
to explain the protein expression; other factors, that the model does not
account for (e.g. protein degradation), possibly dominate.
Secondly, the marginal posterior of a single predictor can indicate
non-significance, while the joint posterior of several predictors combined
might still achieve significance. This would indicate that the effect of
miRNAs is only significant when acting simultaneously, a hypothesis that is
supported by experimental evidence, as discussed in section
\ref{microrna-function}.

Lasso regression produced models with at least one miRNA for 74 genes, out of
which only 57 were common to the ones found by projection prediction (PPVS).
For 18 of the lasso models, the mRNA expression variable was not included.
Figure \ref{fig:model-size} shows a comparison of model size distribution
and $\Delta\bar{R}^2$ between PPVS and lasso.

\begin{figure}[!h]
  \centering
  \begin{subfigure}{.45\textwidth}
    \centering
    \includegraphics[width=1\linewidth]{figures/R2comparison/n_miRNA_comparison.pdf}
  \end{subfigure}
  \begin{subfigure}{.45\textwidth}
    \centering
    \includegraphics[width=1\linewidth]{figures/R2comparison/R2_comparison.pdf}
  \end{subfigure}

  \caption{Comparison of model sizes and $\Delta\bar{R}^2$ for
      projection prediction (PPVS) and lasso regression.}
  \label{fig:model-size}
\end{figure}




\subsubsection{Target prediction performance}

Considering each selected miRNA variable as a putative miRNA-mRNA target
interaction, PPVS generated a total 945 target predictions, out of which
253 were significant (in the sense described above). Table \ref{table:final-models}
lists the significant predictions.
Lasso regression generated all together 650 target predictions. Figure
\ref{fig:venn} shows the overlap of predictions by PPVS and lasso, and also
the overlap of significant predictions from PPVS and the same number of top
predictions from lasso, where lasso predictions were ranked by the absolute value of
the regression coefficient.\footnote{Lasso regression does not provide any tests of
significance or other ranking measures}

\begin{figure}[!h]
  \centering
  \begin{subfigure}{.45\textwidth}
    \centering
    \includegraphics[width=1\linewidth]{figures/compareModels/Venn-PPVS-lasso.pdf}
  \end{subfigure}
  \begin{subfigure}{.45\textwidth}
    \centering
    \includegraphics[width=1\linewidth]{figures/compareModels/Venn-PPVS_sig-lasso_top.pdf}
  \end{subfigure}

  \caption{Venn diagrams showing the overlap of target predictions from PPVS
  and lasso (shown left). Shown right is the overlap of significant predictions
  from PPVS and the same number of top predictions from lasso (ranked by
  absolute value of regression coefficient).}
  \label{fig:venn}
\end{figure}

Figure \ref{fig:scatter-ppvs-lasso} shows a comparison of the regression
coefficients from PPVS and lasso for targets predicted by both methods. There
were only three predicted targets for which the methods did not agree on the
sign of the coefficient. PPVS produced larger coefficients in general, and
correlation for the coefficients was fairly high ($0.86$).

\begin{figure}[htb]
  \centering
  \includegraphics[width=.6\linewidth]{figures/compareModels/Scatter-PPVS-lasso.pdf}
  \caption{Scatter plot of miRNA regression coefficients from projection
  prediction (PPVS) and lasso for the 145 common targets predicted by both methods.
  Note that PPVS coefficients generally have greater magnitude. Triangles
  mark significant coefficients in PPVS. The gray
  line marks unity ($x=y$).}
  \label{fig:scatter-ppvs-lasso}
\end{figure}

PPVS and lasso had similar performance in regards of discovering validated
targets, with approximately 12\% of PPVS and 14\% of lasso predictions
appearing in the union of TarBase and miRTarBase. For the significant PPVS and top lasso
predictions, 13\% and 14\% were in the validated reference set, respectively. Therefore,
limiting predictions to the ones with most confidence did not significantly increase
the proportion of validated targets for either method.

For both methods, approximately half of the coefficients for both all
predicted and validated targets were positive, indicating upregulation by the
miRNA. In contrast, in TarBase, all of the discovered validated interactions
were classified as downregulative. miRTarBase does not provide data on the
nature of the regulation.
