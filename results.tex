
\section{Results}

\begin{itemize}
  \item QC plots
  \begin{itemize}
    \item distributions of arrays
    \item PCA and hierarch. clust of samples with hospital colors
  \end{itemize}
  \item correlation plots
  \begin{itemize}
    \item mRNA-protein
    \item miRNA-mRNA (target vs non-target)
    \item miRNA-protein (target vs non-target)
    \item all three combined (additionally with grouping by intrinsic subtype?)
  \end{itemize}
  \item cross-validation results
  \begin{itemize}
    \item example plots of search path ("good" and "bad")
  \end{itemize}
  \item miRNA coefficients from final models
  \begin{itemize}
  	\item plot of different threshold values
    \item number of miRNAs included
    \item percentage of significant coefs
    \item percentage predicted by seq algos (num miRNAs vs num algos, included vs significant)
    \item percentage validated in experiments (included vs significant, mirTarBase)
    \item magnitude of chosen miRNA coefs vs gene and constant
    \item posterior mean vs sd scatter (järkevä?)
  \end{itemize}
  \item simple network analysis?
  \begin{itemize}
    \item connections between included miRNAs and genes
    \item some pathway analysis or geneset enrichment? (moot since proteins from PIK3-path?)
  \end{itemize}
  \item DE analysis of included/significant miRNAs?
  \begin{itemize}
    \item triple negative cancers vs others
    \item alternatively correlations between significant miRNAs-proteins grouped by intrinsic subtypes?
  \end{itemize}

\end{itemize}

%\begin{landscape}
    { % This table is generated by CSV2Latex.
\footnotesize{\begin{longtable}{llllrp{7cm}}
\caption{
Properties of fitted models for all 105 genes. A missing value for N miRNAs indicates a projected model was not found
(i.e. the stopping criterion was not satisfied before reaching 200 covariates),
a zero indicates no miRNA variables were chosen.
$R^2$ was not computed for models with no miRNA variables.
Only the significant miRNAs chosen are listed for compactness of display. \\
$R^2_{\textup{gene}}$: $R^2$ for gene-only model \\
$R^2_\perp$: $R^2$ for projected model obtained with PPVS \\
$\Delta\bar{R}^2$: Difference of adjusted $R^2$ of projected model versus gene-only model \\
$^{\ast}$: gene expression variable is significant (95\% credible interval) \\
$\textup{N}_{\textup{miRNA}}$: number of miRNA variables in projected model (number of significant miRNA variables, 95\% credible interval) 
\label{table:final-models}
}
\label{table:finalModelTable}\\\hline
\textbf{Gene} & \textbf{$R^2_{\textup{gene}}$} & \textbf{$R^2_\perp$} & \textbf{$\Delta\bar{R}^2$} & $\mathbf{\textup{N}_{miRNA}}$ & \textbf{Significant miRNAs}\\\hline
\endfirsthead{}%
\textbf{Gene} & \textbf{$R^2_{\textup{gene}}$} & \textbf{$R^2_\perp$} & \textbf{$\Delta\bar{R}^2$} & $\mathbf{\textup{N}_{miRNA}}$ & \textbf{Significant miRNAs}\\\hline
\endhead\hline\multicolumn{6}{r}{\textit{Continued on next page\ldots\/}}
\endfoot\hline\endlastfoot{}%
ACACA&0.462$^{\ast}$&0.524$^{\ast}$&0.062&2 (2)&\raggedright{miR-30a, miR-370} \tabularnewline\rowcolor[rgb]{0.96,0.96,0.96}{}%
AKT1&0.111$^{\ast}$&0.174$^{\ast}$&0.064&2 (2)&\raggedright{miR-449a, miR-342-5p} \tabularnewline{}%
AKT2&0.001&0.230&0.195&14 (4)&\raggedright{miR-342-5p, miR-449a, miR-96, miR-146b-5p} \tabularnewline\rowcolor[rgb]{0.96,0.96,0.96}{}%
AKT3&0.000&0.251&0.216&15 (4)&\raggedright{miR-342-5p, miR-449a, miR-96, miR-146b-5p} \tabularnewline{}%
ANXA1&0.380$^{\ast}$&0.425$^{\ast}$&0.047&1 (1)&\raggedright{miR-765} \tabularnewline\rowcolor[rgb]{0.96,0.96,0.96}{}%
AR&0.713$^{\ast}$&&&0&\raggedright{} \tabularnewline{}%
BAK1&0.119$^{\ast}$&0.228$^{\ast}$&0.110&2 (2)&\raggedright{miR-29c, miR-505*} \tabularnewline\rowcolor[rgb]{0.96,0.96,0.96}{}%
BAX&0.130$^{\ast}$&0.405$^{\ast}$&0.228&23 (4)&\raggedright{miR-557, miR-659, miR-142-3p, miR-199a-3p} \tabularnewline{}%
BCL2&0.728$^{\ast}$&&&0&\raggedright{} \tabularnewline\rowcolor[rgb]{0.96,0.96,0.96}{}%
BCL2L1&0.088$^{\ast}$&0.354$^{\ast}$&0.123&53 (1)&\raggedright{miR-622} \tabularnewline{}%
BCL2L11&0.176$^{\ast}$&0.318$^{\ast}$&0.143&2 (2)&\raggedright{miR-29c, miR-34a} \tabularnewline\rowcolor[rgb]{0.96,0.96,0.96}{}%
BECN1&0.005&&&&\raggedright{} \tabularnewline{}%
BID&0.015$^{\ast}$&0.134&0.122&1 (1)&\raggedright{miR-1246} \tabularnewline\rowcolor[rgb]{0.96,0.96,0.96}{}%
BIRC2&0.018$^{\ast}$&0.243$^{\ast}$&0.226&2 (2)&\raggedright{miR-1246, miR-425} \tabularnewline{}%
BRAF&0.094$^{\ast}$&0.456$^{\ast}$&0.352&8 (8)&\raggedright{miR-638, miR-125b, miR-1321, miR-107, miR-765, miR-148a, miR-135b, miR-505*} \tabularnewline\rowcolor[rgb]{0.96,0.96,0.96}{}%
CASP8&0.014&0.111$^{\ast}$&0.091&4 (4)&\raggedright{miR-99a, miR-148a, miR-631, miR-126*} \tabularnewline{}%
CAV1&0.284$^{\ast}$&0.432$^{\ast}$&0.147&3 (3)&\raggedright{miR-551b, miR-24} \tabularnewline\rowcolor[rgb]{0.96,0.96,0.96}{}%
CCNB1&0.638$^{\ast}$&0.714$^{\ast}$&0.076&2 (2)&\raggedright{miR-199a-5p, miR-30a} \tabularnewline{}%
CCND1&0.334$^{\ast}$&0.497$^{\ast}$&0.153&8 (7)&\raggedright{miR-936, miR-181c, miR-622, miR-493*, miR-19a, miR-126, miR-9*} \tabularnewline\rowcolor[rgb]{0.96,0.96,0.96}{}%
CCNE1&0.544$^{\ast}$&&&0&\raggedright{} \tabularnewline{}%
CDH1&0.449$^{\ast}$&&&0&\raggedright{} \tabularnewline\rowcolor[rgb]{0.96,0.96,0.96}{}%
CDH2&0.006&&&0&\raggedright{} \tabularnewline{}%
CDH3&0.165$^{\ast}$&0.584$^{\ast}$&0.398&17 (13)&\raggedright{miR-155, miR-10b*, miR-502-5p, miR-224, miR-489, miR-148a, miR-195, miR-197, miR-361-5p, miR-650, miR-150, miR-501-5p, miR-582-5p} \tabularnewline\rowcolor[rgb]{0.96,0.96,0.96}{}%
CDK1&0.037$^{\ast}$&&&&\raggedright{} \tabularnewline{}%
CDKN1B&0.418$^{\ast}$&0.491$^{\ast}$&0.075&1 (1)&\raggedright{miR-195} \tabularnewline\rowcolor[rgb]{0.96,0.96,0.96}{}%
CHEK1&0.019$^{\ast}$&0.168&0.122&11 (0)&\raggedright{} \tabularnewline{}%
CHEK2&0.512$^{\ast}$&&&0&\raggedright{} \tabularnewline\rowcolor[rgb]{0.96,0.96,0.96}{}%
CLDN7&0.234$^{\ast}$&0.363$^{\ast}$&0.125&4 (4)&\raggedright{miR-29c, miR-200c, miR-30b, miR-150} \tabularnewline{}%
COL6A1&0.000&0.325&0.321&4 (4)&\raggedright{miR-125b, miR-638, miR-210, miR-24} \tabularnewline\rowcolor[rgb]{0.96,0.96,0.96}{}%
CTNNA1&0.081$^{\ast}$&0.113$^{\ast}$&0.033&2 (1)&\raggedright{miR-125a-3p} \tabularnewline{}%
CTNNB1&0.002&0.339&0.287&22 (5)&\raggedright{miR-711, miR-10a, miR-31*, miR-16, miR-28-5p} \tabularnewline\rowcolor[rgb]{0.96,0.96,0.96}{}%
DIABLO&0.089$^{\ast}$&0.483$^{\ast}$&0.336&31 (6)&\raggedright{miR-378, miR-339-3p, miR-762, miR-15b, miR-144*, miR-582-5p} \tabularnewline{}%
DVL3&0.068$^{\ast}$&0.377$^{\ast}$&0.283&14 (9)&\raggedright{miR-24, miR-498, miR-140-3p, miR-223, miR-29c, miR-21, miR-432, miR-662, miR-204} \tabularnewline\rowcolor[rgb]{0.96,0.96,0.96}{}%
EEF2&0.009&0.330&0.292&14 (3)&\raggedright{miR-106b, miR-196b, miR-29c*} \tabularnewline{}%
EEF2K&0.310$^{\ast}$&&&0&\raggedright{} \tabularnewline\rowcolor[rgb]{0.96,0.96,0.96}{}%
EGFR&0.148$^{\ast}$&0.356$^{\ast}$&0.189&10 (7)&\raggedright{miR-181d, miR-181b, miR-1182, miR-495, miR-30c, miR-126, miR-183*} \tabularnewline{}%
EIF4E&0.100$^{\ast}$&0.252$^{\ast}$&0.050&36 (0)&\raggedright{} \tabularnewline\rowcolor[rgb]{0.96,0.96,0.96}{}%
EIF4EBP1&0.495$^{\ast}$&&&0&\raggedright{} \tabularnewline{}%
ERBB2&0.729$^{\ast}$&&&0&\raggedright{} \tabularnewline\rowcolor[rgb]{0.96,0.96,0.96}{}%
ERBB3&0.204$^{\ast}$&0.334$^{\ast}$&0.128&3 (3)&\raggedright{miR-199a-5p, miR-451, miR-484} \tabularnewline{}%
ERCC1&0.004&&&0&\raggedright{} \tabularnewline\rowcolor[rgb]{0.96,0.96,0.96}{}%
ERRFI1&0.012&&&&\raggedright{} \tabularnewline{}%
ESR1&0.825$^{\ast}$&&&0&\raggedright{} \tabularnewline\rowcolor[rgb]{0.96,0.96,0.96}{}%
FN1&0.495$^{\ast}$&&&0&\raggedright{} \tabularnewline{}%
FOXO3&0.094$^{\ast}$&0.225$^{\ast}$&0.128&3 (3)&\raggedright{miR-140-3p, miR-631, miR-197} \tabularnewline\rowcolor[rgb]{0.96,0.96,0.96}{}%
GAB2&0.597$^{\ast}$&&&0&\raggedright{} \tabularnewline{}%
GATA3&0.708$^{\ast}$&&&0&\raggedright{} \tabularnewline\rowcolor[rgb]{0.96,0.96,0.96}{}%
GSK3A&0.106$^{\ast}$&0.473$^{\ast}$&0.319&26 (5)&\raggedright{miR-21, miR-29a, miR-20a, miR-100, miR-92a} \tabularnewline{}%
GSK3B&0.064$^{\ast}$&0.524$^{\ast}$&0.275&81 (0)&\raggedright{} \tabularnewline\rowcolor[rgb]{0.96,0.96,0.96}{}%
IGF1R&0.637$^{\ast}$&&&0&\raggedright{} \tabularnewline{}%
IGFBP2&0.553$^{\ast}$&&&0&\raggedright{} \tabularnewline\rowcolor[rgb]{0.96,0.96,0.96}{}%
INPP4B&0.754$^{\ast}$&&&0&\raggedright{} \tabularnewline{}%
IRS1&0.397$^{\ast}$&0.441$^{\ast}$&0.046&1 (1)&\raggedright{miR-93} \tabularnewline\rowcolor[rgb]{0.96,0.96,0.96}{}%
KDR&0.000&0.281&0.272&6 (6)&\raggedright{miR-150, miR-663, miR-495, miR-24-1*, miR-363, miR-140-3p} \tabularnewline{}%
KIT&0.581$^{\ast}$&&&0&\raggedright{} \tabularnewline\rowcolor[rgb]{0.96,0.96,0.96}{}%
KRAS&0.037$^{\ast}$&0.335&0.264&16 (2)&\raggedright{miR-96, miR-21} \tabularnewline{}%
MAP2K1&0.006&0.141&0.138&1 (1)&\raggedright{miR-21} \tabularnewline\rowcolor[rgb]{0.96,0.96,0.96}{}%
MAPK14&0.017$^{\ast}$&0.346$^{\ast}$&0.301&14 (9)&\raggedright{miR-145, miR-92a, miR-181c, miR-142-3p, miR-425, miR-339-3p, miR-342-5p, miR-18b, miR-1226*} \tabularnewline{}%
MAPK9&0.193$^{\ast}$&0.251$^{\ast}$&0.061&1 (1)&\raggedright{miR-342-5p} \tabularnewline\rowcolor[rgb]{0.96,0.96,0.96}{}%
MAPT&0.020$^{\ast}$&0.335&0.270&20 (3)&\raggedright{miR-30c, miR-132, miR-17*} \tabularnewline{}%
MET&0.031$^{\ast}$&0.085$^{\ast}$&0.054&2 (2)&\raggedright{miR-125b, miR-139-5p} \tabularnewline\rowcolor[rgb]{0.96,0.96,0.96}{}%
MRE11A&0.012&0.290&0.052&70 (0)&\raggedright{} \tabularnewline{}%
MSH2&0.333$^{\ast}$&&&0&\raggedright{} \tabularnewline\rowcolor[rgb]{0.96,0.96,0.96}{}%
MSH6&0.340$^{\ast}$&0.628$^{\ast}$&0.265&19 (10)&\raggedright{miR-125b, miR-324-5p, miR-25, miR-195*, miR-451, miR-154, miR-551b, miR-26b, miR-513a-5p} \tabularnewline{}%
MYC&0.025$^{\ast}$&0.101$^{\ast}$&0.076&2 (2)&\raggedright{miR-150*, miR-24} \tabularnewline\rowcolor[rgb]{0.96,0.96,0.96}{}%
NCOA3&0.132$^{\ast}$&&&0&\raggedright{} \tabularnewline{}%
NF2&0.132$^{\ast}$&0.321$^{\ast}$&0.187&3 (3)&\raggedright{miR-638, miR-125b, miR-22} \tabularnewline\rowcolor[rgb]{0.96,0.96,0.96}{}%
NOTCH1&0.216$^{\ast}$&0.279$^{\ast}$&0.064&2 (2)&\raggedright{miR-199b-5p, miR-502-5p} \tabularnewline{}%
NOTCH3&0.150$^{\ast}$&0.401$^{\ast}$&0.245&5 (5)&\raggedright{miR-125b, miR-27b, miR-193a-5p, miR-32, miR-139-5p} \tabularnewline\rowcolor[rgb]{0.96,0.96,0.96}{}%
PARK7&0.288$^{\ast}$&0.428$^{\ast}$&0.140&2 (2)&\raggedright{miR-93, miR-29c} \tabularnewline{}%
PCNA&0.229$^{\ast}$&0.283$^{\ast}$&0.057&1 (1)&\raggedright{miR-199a-5p} \tabularnewline\rowcolor[rgb]{0.96,0.96,0.96}{}%
PECAM1&0.054$^{\ast}$&0.169$^{\ast}$&-0.028&43 (0)&\raggedright{} \tabularnewline{}%
PGR&0.819$^{\ast}$&&&0&\raggedright{} \tabularnewline\rowcolor[rgb]{0.96,0.96,0.96}{}%
PIK3CA&0.051$^{\ast}$&&&0&\raggedright{} \tabularnewline{}%
PIK3R1&0.077$^{\ast}$&0.377$^{\ast}$&0.295&5 (5)&\raggedright{miR-142-3p, miR-342-3p, miR-501-5p, miR-145*, miR-92a} \tabularnewline\rowcolor[rgb]{0.96,0.96,0.96}{}%
PRKAA1&0.119$^{\ast}$&0.240$^{\ast}$&0.121&2 (2)&\raggedright{miR-199a-3p, miR-342-3p} \tabularnewline{}%
PRKCA&0.368$^{\ast}$&&&0&\raggedright{} \tabularnewline\rowcolor[rgb]{0.96,0.96,0.96}{}%
PTCH1&0.029$^{\ast}$&0.167$^{\ast}$&0.127&6 (2)&\raggedright{miR-145, miR-934} \tabularnewline{}%
PTEN&0.029$^{\ast}$&0.468$^{\ast}$&0.370&35 (5)&\raggedright{miR-204, miR-498, miR-324-3p, miR-30d, miR-22} \tabularnewline\rowcolor[rgb]{0.96,0.96,0.96}{}%
PTGS2&0.065$^{\ast}$&0.114$^{\ast}$&0.052&1 (1)&\raggedright{miR-1246} \tabularnewline{}%
PTK2&0.021$^{\ast}$&0.273$^{\ast}$&0.242&6 (5)&\raggedright{miR-1246, miR-150, miR-21, miR-200c, miR-32} \tabularnewline\rowcolor[rgb]{0.96,0.96,0.96}{}%
PXN&0.133$^{\ast}$&0.431$^{\ast}$&0.140&63 (0)&\raggedright{} \tabularnewline{}%
RAB25&0.031$^{\ast}$&&&&\raggedright{} \tabularnewline\rowcolor[rgb]{0.96,0.96,0.96}{}%
RAD50&0.076$^{\ast}$&0.106$^{\ast}$&0.033&1 (1)&\raggedright{miR-139-5p} \tabularnewline{}%
RAD51&0.011&0.205&0.195&2 (2)&\raggedright{miR-1246, miR-181c} \tabularnewline\rowcolor[rgb]{0.96,0.96,0.96}{}%
RAF1&0.121$^{\ast}$&0.446$^{\ast}$&0.306&12 (9)&\raggedright{miR-125b, miR-1260, miR-498, miR-449a, miR-130a, miR-30b, miR-28-5p, miR-374a, miR-195*} \tabularnewline{}%
RB1&0.006&0.233$^{\ast}$&0.222&4 (4)&\raggedright{miR-145, miR-1260, miR-451, miR-195*} \tabularnewline\rowcolor[rgb]{0.96,0.96,0.96}{}%
RPS6KB1&0.516$^{\ast}$&0.617$^{\ast}$&0.102&2 (2)&\raggedright{miR-497, miR-106b} \tabularnewline{}%
SMAD1&0.359$^{\ast}$&0.417$^{\ast}$&0.060&1 (1)&\raggedright{miR-106b} \tabularnewline\rowcolor[rgb]{0.96,0.96,0.96}{}%
SMAD3&0.354$^{\ast}$&0.475$^{\ast}$&0.112&7 (7)&\raggedright{miR-7, miR-451, miR-181d, miR-29b, miR-365, miR-1225-5p, miR-1299} \tabularnewline{}%
SMAD4&0.000&&&0&\raggedright{} \tabularnewline\rowcolor[rgb]{0.96,0.96,0.96}{}%
SNAI1&0.001&&&0&\raggedright{} \tabularnewline{}%
SRC&0.308$^{\ast}$&&&0&\raggedright{} \tabularnewline\rowcolor[rgb]{0.96,0.96,0.96}{}%
STAT5A&0.292$^{\ast}$&0.438$^{\ast}$&0.147&2 (2)&\raggedright{miR-155, miR-324-5p} \tabularnewline{}%
STMN1&0.000&0.364&-0.028&109 (0)&\raggedright{} \tabularnewline\rowcolor[rgb]{0.96,0.96,0.96}{}%
SYK&0.227$^{\ast}$&0.714$^{\ast}$&0.462&26 (20)&\raggedright{miR-125b, miR-155, miR-324-5p, miR-195, miR-449a, miR-711, miR-762, miR-940, miR-22, miR-615-3p, miR-663, miR-377*, miR-598, miR-126, miR-29b, miR-1228*, miR-144, miR-204, miR-601, miR-30d} \tabularnewline{}%
TP53&0.017$^{\ast}$&0.186$^{\ast}$&0.161&5 (5)&\raggedright{miR-301b, miR-29c*, miR-551b, miR-19a, miR-874} \tabularnewline\rowcolor[rgb]{0.96,0.96,0.96}{}%
TP53BP1&0.260$^{\ast}$&0.357$^{\ast}$&0.099&1 (1)&\raggedright{miR-96} \tabularnewline{}%
TSC2&0.065$^{\ast}$&0.318$^{\ast}$&0.225&13 (3)&\raggedright{miR-1915, miR-30b, miR-92a} \tabularnewline\rowcolor[rgb]{0.96,0.96,0.96}{}%
VASP&0.151$^{\ast}$&0.193$^{\ast}$&0.045&1 (1)&\raggedright{miR-29c*} \tabularnewline{}%
XIAP&0.089$^{\ast}$&0.224$^{\ast}$&0.136&2 (2)&\raggedright{miR-638, miR-99a} \tabularnewline\rowcolor[rgb]{0.96,0.96,0.96}{}%
XRCC1&0.130$^{\ast}$&&&0&\raggedright{} \tabularnewline{}%
YAP1&0.164$^{\ast}$&0.351$^{\ast}$&0.183&4 (4)&\raggedright{miR-106b, miR-486-5p, miR-28-5p, miR-595} \tabularnewline\rowcolor[rgb]{0.96,0.96,0.96}{}%
YBX1&0.016$^{\ast}$&0.329$^{\ast}$&0.314&2 (2)&\raggedright{miR-125b, miR-96} \tabularnewline{}%
YWHAE&0.001&0.145&0.138&4 (4)&\raggedright{miR-21, miR-1260, miR-365*, miR-204}%
\end{longtable}}
} % End of table

%\end{landscape}

Figure \ref{n-miRNAs-vs-R2} shows the number of miRNA variables included in
each full model plotted against the number of significant miRNA variables, the
$R^2$ of the full model, and the difference in $\bar{R}^2$ between the full
model and the model with only the gene variable. Each point in each plot
corresponds to one full model, and smoothed curves were fitted using locally
weighted scatter plot smoothing (LOESS). A trend emerges, where including more
miRNA variables tends to achieve better performance up to a turning point --
around 28 miRNA variables -- after which larger models seem to perform
increasingly poorly. Similar plots were created with more strict cross-
validation thresholds ($\gamma$ and $\alpha$); the models were larger on
average, but an equal trend and turning point were seen (data not shown).

\begin{figure}[htb]
\centering \includegraphics[height=11cm]{n_miRNAs_vs_R2.pdf}
\caption{Number of miRNA variables included in full model (N miRNAs) versus number of significant miRNA variables, R$^2$ of full model, and difference of $\bar{R}^2$ between full and gene only model ($\Delta\bar{R}^2$). Each point represents one model fitted for one gene. Note that some points are overlaid on top of each other. Curves were fitted with locally
weighted scatter plot smoothing (LOESS) and shaded areas represent 95\% confidence interval. \label{n-miRNAs-vs-R2}}
\end{figure}

There could be several explanations for this trend. Firstly, the largest
models could simply be a result of the predictors not fitting the observed
variable well. This often results in a large model, because including more
predictors always makes the model fit asymptotically better -- both in the
sense of $R^2$ and generally also the predictive density, which was used as
the measure of fit for the variable-selection process. From a biological
perspective, this means that the mRNA and miRNA expression are not sufficient
to explain the protein expression; other factors, that the model does not
account for (e.g. protein degradation), possibly dominate the protein
abundance. Secondly, the marginal posterior of a single predictor can indicate
non-significance, while the joint posterior of several predictors combined
might still achieve significance. This would indicate that the effect of
miRNAs is only significant when acting simultaneosly, a hypothesis that is
supported by experimental evidence, as discussed in section
\ref{microrna-function}.

To examine the latter point, \textbf{SOMETHING COULD BE DONE? AKI?}.
