
\section{Introduction}
\thispagestyle{empty}

Breast cancer is the most common of female cancers and causes remarkable
morbidity and mortality word-wide \citep{??}.  Annually more than 1.5 million
women develop breast cancer. Thus, breast cancer is a major global health
problem.

Cancer is a genetic disease caused by mutations in the genome of the cancer
cells. Some of these mutations can be inherited, while some are somatic, that
is, mutations that arise during the life-time of an individual in the tissue,
where cancer develops. To understand how cancer develops, the tumor evolution
of breast cancer, it is of paramount importance to identify genes in which
mutations initiate breast cancer development. It is already known that many of
these mutations perturb the expression of the gene in question. Previous
expression studies comparing the expression profiles of normal breast tissue
and breast cancer tissue have revealed ...

MicroRNA control of gene expression .... MicroRNAs have been indicated as
potential causative agents in numerous diseases, including several types of
cancer. Therefore, the study of microRNAs and their function in cancer can
offer insights into tumorigenesis and cancer progression as well as potential
new treatments.

Several different methods for computational identification of microRNA target
genes have been proposed. Most are based on sequence similarity of the
microRNA and messenger RNA of putative target genes. More recent methods use
expression profiles to either augment sequence data or as sole predictors for
interactions. These are mostly based on correlation, or a derivative thereof,
of microRNA and gene expression.

The aim of this thesis was to apply a recently proposed Bayesian variable
selection method to microRNA target discovery in breast cancer. The variable
selection was applied in the context of Bayesian regression of expression
profiles to elucidate which microRNAs are highly relevant in regulating
protein expression levels. The results were compared with a similar recent
study and also with known validated microRNA targets as well as microRNAs
indicated in breast cancer.
