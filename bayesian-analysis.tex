%!TEX root = dippa.tex
%%% This file contains the Bayesian analysis section of my master's thesis.
%%% Author: Viljami Aittomäki

\section{Bayesian analysis}\label{bayesian-analysis}

Bayesian data analysis is a modeling framework that is based on the principle
of quantifying uncertainty as probability. Current knowledge about unknown
model parameters, variables and future observations is described in terms of
probability statements \citep{Gelman2013}. This provides a framework which is
inherently suited to dealing with noisy real-world data, as measurement noise
is naturally incorporated into probability distributions. This section
provides a brief introduction into Bayesian analysis and discusses Bayesian
regerssion and variable selection as applicable to the problem of microRNA
target prediction.

\subsection{Basics of Bayesian analysis}

Bayesian analysis begins by defining a joint probability model $p(y,\theta)$
for observed data $y$ and unknown model parameters $\theta$.
The joint distribution can be written as a product of two probability distributions
\begin{equation}
  p(y,\theta) = p(\theta) p(y\mid\theta),
\end{equation}
which are referred to as the \emph{prior distribution} $p(\theta)$ and the
data distribution or \emph{likelihood} $p(y\mid\theta)$. The prior conveys
information on the presumed values a parameter may take and the likelihood
represents the likeliness of observed data for given parameter values (in the
context of the chosen data model). Applying the Bayes' theorem we obtain the
\emph{posterior distribution} for $\theta$ given the known values of the
observations $y$:
\begin{equation}
  \label{eq:bayes}
  p(\theta \mid y) = \frac{p(y,\theta)}{p(y)} = \frac{p(\theta) p(y\mid\theta)}{p(y)},
\end{equation}
where $p(y) = \int_{\theta} p(\theta) \cdot p(y\mid\theta) d\theta$.
Noting that $p(y)$ does not depend on $\theta$, we can write Equation
\eqref{eq:bayes} as
\begin{equation}
  p(\theta \mid y) \propto p(\theta) p(y\mid\theta).
\end{equation}
The latter is referred to as the unnormalized posterior distribution. The
Bayes' theorem forms the heart of Bayesian inference and illustrates the core
concept of updating prior beliefs to account for observed evidence. The
posterior provides a probability assessment of the possible values of
a parameter, and represents a compromise between prior knowledge and information
obtained from observations. As the number of observations increases, the
data have increasing influence on the posterior \citep{Gelman2013}.

Hierarchical models -- where parameters of prior
distributions have their own priors, called \emph{hyperpriors} -- allow significant
flexibility in Bayesian modeling. In situations where model parameters are
related to each other, a common hyperprior may be used for several parameters.
Hierarchical models are particularly appropriate in settings where
data are sparse (as available information is be shared between parameters)
or the data are naturally structured into several levels, such as 
similar measurements from different hospitals or schools.

The virtue of modeling uncertainty as probabilities -- in addition to
naturally dealing with noise -- is that they are conceptually easy to grasp
and often allow common-sense interpretations of conclusions to be made.
\footnote{This is especially true compared to classical frequentist analysis,
which is defined within the context of repeated sampling (and inference) from
a fixed but unknown process generating the observations. For example,
frequentist confidence intervals strictly do not indicate that the true value
of the parameter is contained within with high probability -- a common
misconception -- where as Bayesian posterior intervals do (subject to modeling
assumptions, of course).} Perhaps the foremost advantage of Bayesian modeling,
however, is its flexibility and extensibility; it can cope with complex
problems and data with relative ease. Prior knowledge about the parameters of
interest can be embedded in prior distributions, priors can be assigned freely
to each parameter, hierachical models can be used to model layered data, and
new observations can be added sequentially to update previous conclusions.

The challenge in Bayesian analysis is choosing proper probability models for
the parameters and observations, including prior distributions as well as the
likelihood \citep{Gelman2013}. In fact, Bayesian methodology has been
criticized for the subjectivity related to choosing suitable priors. One could, however,
argue that the choice of any model is always subjective to a certain degree,
irrespective of chosen methodology. Additionally, weakly informative or non-informative
priors can be used to decrease the effect of subjective information (or
in cases where no prior knowledge is available) and conclusions from
inferences using non-informative priors often coincide with classical
analyses.

% In cases where no prior information is available, weakly informative
% or noninformative priors can be used. These convey little or no
% information on the presumed values a parameter may take, respectively.
% In many instances, using a non-informative prior results in similar or equal
% results as frequentist analysis, but the strength of Bayesian analysis comes
% from including prior knowledge in the prior distribution. \citep{Jaynes?} The
% posterior distribution represents a compromise between the prior (and, hence,
% prior information) and the observed data, with the data having an increasing
% effect as the sample size increases \citep{Gelman2013}. The posterior
% distribution also provides a more comprehensive view of
% one's knowledge on the parameter of interest than, say, a single confidence
% interval.

% The frequentist approach only considers the data to have a probability
% distribution, the likelihood. The process giving rise to the data, and the
% parameters that define it, are considered fixed. The observed data are
% assessed with respect to other data that might be generated by the same model.

% OMIN SANOIN: Confidence intervals work their best when you don't know much about a
% parameter beyond the information contained in a data set. And further,
% credibility intervals won't be able to improve much on confidence intervals
% unless there is prior information which the confidence interval can't take
% into account, or finding the sufficient and ancillary statistics is hard.


\subsection{Bayesian inference}\label{sec:bayesian-inference}

The goal of Bayesian inference is to make conclusions about unkown parameters
$\theta$ or unkown observations $\tilde{y}$, given the observed data $y$.
These are formulated as posterior distributions or features describing them,
such as point or interval estimates. In simple cases, the posterior $p(\theta \mid y)$
can be derived in analytical form. In practice, however, it is often not possible to obtain
explicit forms of posteriors or analytical solutions to integrals involved in
inference, epecially with complex and hierarchical models. Therefore,
numerical estimation or simulation methods, in the form of sampling from
probability distributions, are frequently used to approximate the posterior.

% In practice, one is often interested
% in a subset of the model parameters. \emph{Marginalization} over the
% uninteresting \emph{nuisance} parameters can be employed to obtain the marginal posterior
% \[
%   p(\theta_1 \mid \theta_2, y) = \int p(\theta_1, \theta_2 \mid y) d\theta_2,
% \]
% where $\theta_1$ are the parameters of interest and $\theta_2$ the
% nuisance parameters.

% Bayesian analysis often involves simulation if the form of sampling from the
% obtained posterior distribution. This is convenient -- and necessary -- when
% the exact probability density function cannot be explicitly obtained through
% integration. Additionally, simulation often has the advantage of pointing out
% problems in the model specification when simulated values are extremely small
% of large.

% Often obtaining analytical solutions to integrals or explicit formulations of
% the posterior distribution is not possible. This is especially true for
% complex and hierarchical models. In these cases it is possible to use
% simulation to obtain samples from the posterior distribution of parameters and
% use these to compute estimates for parameters and other quantities of
% interest.

The simplest approaches to simulation include sampling directly from the
posterior distribution $p(\theta \mid y)$, when possible, or from a simpler distribution
proportional to $p(\theta \mid y)$ using e.g. rejection sampling \citep{Gelman2013}.
More sophisticated methods are often needed when dealing with complex models.
Markov chain Monte Carlo (MCMC) is a general approach to simulation
that is based on drawing samples of $\theta$ sequentially from approximate
distributions. The draws are corrected at each iteration so that their
distribution becomes closer to $p(\theta \mid y)$.
Each draw $\theta^t$ is conditional (only) on the previous one
$\theta^{t-1}$.\footnote{This is essentially the definition of a Markov chain;
a sequence of random variables, where the probability density of each one is
dependent on only the previous one.}
MCMC methods are applicable to arbitrary posterior distributions and a range
of programs for running simulations of full Bayesian inferences are available.

A key issue with iterative methods, such as MCMC, is running the simulation
long enough, so that the distribution for drawing samples has converged 
close enough to the target distribution. Basic solutions to this include discarding a
burn-in period of samples from the beginning of each simulation (to assure
that the samples arise from a converged state), and running several separate
simulations (chains) with different starting points to improve coverage of the
posterior. Various approaches to measuring convergence have been proposed.




\subsection{Bayesian regression}

Bayesian regression analysis aims to infer the posterior distributions
for the regression coefficients of covariates and other model parameters,
such as the variance (i.e. the error term) of the observation model.
Within the Bayesian framework, the \emph{normal linear regression} defined in
Equation \eqref{eq:linear-regression} can be expressed as
\begin{equation}
  y \mid \beta, \sigma, X \sim N(X \beta, \sigma^2I),
  \label{eq:bayesian-linear-regression}
\end{equation}
where $N$ is the multivariate normal distribution, and $I$ is the $n \times n$
identity matrix. The mean of $y$ is then the familiar linear sum of $x_k$
\begin{equation}
  \textrm{E}(y\mid\beta,X) = X \beta = \beta_0 + \beta_1 x_1 + \dotsb + \beta_p x_p.
\end{equation}
The posterior distribution for the regression coefficients (up to a
normalizing constant) is obtained from the marginal posterior
\begin{equation}
  p(\beta \mid \sigma, y, X) \propto \int p(y \mid \beta, \sigma, X) p(\beta \mid \sigma) p(\sigma) d\sigma,
\end{equation}
with the joint prior $p(\beta, \sigma) = p(\beta \mid \sigma) p(\sigma)$.
It is relatively straightforward to extend this simple model, for instance by
allowing unequal variances or correlation between observations, choosing a
different data distribution
% than the normal distribution
to represent the error term, or including hyperparameters to construct a
hierarchical model.

As mentioned in Section \ref{expr-methods}, microRNA target prediction using
regression analysis of expression data is effectively a variable selection
problem. For Bayesian regression, several different priors that provide model shrinkage
have been proposed, including the Laplace prior (which is closely related to
lasso regression), the horseshoe prior, and the hierarchical shrinkage prior
(a generalization of the horseshoe).
A hierarchical shrinkage prior for regression weight
$\beta_j$ can be formulates as
\begin{equation}
  \label{eq:hs-prior}
  \begin{aligned}
    \beta_j \mid \lambda_j, \tau & \sim N(0, \lambda_j^2 \tau^2) \\
    \lambda_j                 & \sim t_\nu^+(0,1),
  \end{aligned}
\end{equation}
where $t_\nu^+$ denotes the half-Student-$t$ prior with $\nu$ degrees of
freedom \citep{Piironen2015}. The $\lambda_j$ correspond to a local scale
parameter and $\tau$ controls the amount of global shrinkage. As an example,
in a very sparse model with many irrelevant covariates, the model would
ideally have small $\tau$ (so that $p(\beta)$ is mostly shrunk close to zero), but
allow some $\lambda_j$ to be large to escape the shrinkage.

A weakly
informative half-Cauchy distribution is often the suggested choice of prior
for $\tau$, but van der Pas et al have also proposed (for the horseshoe prior)
using a fixed value of $\tau = \frac{p_n}{n}\sqrt{\textup{log}(n/p_n)}$,
where $p_n$ is the assumed number of relevant covariates and $n$ is the number
of observations \citep{vanderpas2014}. Bayesian shrinkage priors, however, do
not lead to a sparse solution as there remains uncertainty in the posterior
distribution and no coefficient can be considered exactly zero.


\subsection{Bayesian variable selection}\label{sec:bayes-variable-selection}

In order to find a small set of relevant predictive variables, a model
selection approach needs to be applied. To this end, a range of methods
applicable in Bayesian analysis have been proposed; examples include using
cross validation, different information criteria, and projection
methods to determine the submodel giving the best compromise between
prediction accuracy and model size. A detailed review of these falls outside
the scope of this thesis, however, a comprehensive one has been recently
written by Vehtari and Ojanen \citep{Vehtari2012}.

In the context of variable selection for regression,
Piironen and Vehtari recently suggested that, for problems where data
are scarce and the number of candidate variables high, using projection predictive
variable selection is effective \citep{Piironen2016}. The idea, proposed by
Dupuis and Robert \citep{Dupuis2003}, is to fit a full reference model $M_{\textup{r}}$
encompassing all candidate variables,
%and uncertainties related to their effect,
and then project the information in the reference posterior onto a submodel $M_\perp$
so that the predictions are as similar as possible.

Given the reference model parameters $\theta_{\textup{r}}$, the projection $\theta_\perp$
in the parameter space of $M_\perp$ is obtained by solving
\begin{equation}
  \label{eq:projection}
  \theta_\perp = \textup{arg}\underset{\theta}{\textup{min}} \frac{1}{n}\sum_i^n \textup{KL} \left ( p(\tilde{y} \mid x_i, \theta_{\textup{r}}, M_{\textup{r}}) \parallel p(\tilde{y} \mid x_i, \theta, M_\perp) \right ),
\end{equation}
where $\textup{KL} \left ( P \parallel Q \right )$ is the Kullback-Leibler divergence
between probability distributions $P$ and $Q$. The discrepancy between the reference
model  $M_{\textup{r}}$ and submodel $M_\perp$ is then defined as the expectation
of the divergence over the posterior of the reference model:
\begin{equation}
  \label{eq:model-discrepancy}
  \delta(M_{\textup{r}} \parallel M_\perp) = \frac{1}{n} \sum_i^n \textup{E}_{\theta_{\textup{r}} \mid D,M_\textup{r}} \left [ \textup{KL} \left ( p(\tilde{y} \mid x_i, \theta_{\textup{r}}, M_{\textup{r}}) \parallel p(\tilde{y} \mid x_i, \theta_\perp, M_\perp) \right ) \right ].
\end{equation}
The posterior expectation in \eqref{eq:model-discrepancy} can, in practice, be
estimated by drawing samples from the reference posterior (using e.g. MCMC).
It can be shown, that in the case of normal linear regression, the
minimization in \eqref{eq:projection} can be solved analytically and the
discrepancy \eqref{eq:model-divergence} has a simple form depending only on
the reference and projected model variances $\sigma^2$ \citep{Piironen2015}.

A practical issue with this method is deciding how many variables should be included
in the model; what is considered similar-enough predictive performance
for the submodel. Piironen and Vehtari suggest using cross-validation
to guide the variable selection process and give a practical guideline
for stopping the selection, which we will return to in Section \ref{sec:methods-variable-selection}.
In this thesis, projection predictive variable selection was used for
inferring putative microRNA targets from breast cancer expression
data using Bayesian regression. Further details of the used method are
presented in Section \ref{sec:methods-variable-selection}.


\subsection{Bayesian microRNA target-prediction methods}

% Bayesian analysis has previously been applied to miRNA target prediction.

One of the earliest tools to use expression data for target prediction was
GenMir++. It is based on A BAYESIAN SYSTEEMI, KIRJOITA AUKI.
The latest version of GenMir (GenMir3) now integrates sequence-based features
into the algorithm, although the authors note that this does not affect the
results to a significant extent.

Stingo et al have proposed a Bayesian method based on spike-and-slab variable
selection.

Previous Bayesian miRNA methods:
\begin{itemize}
  \item Stingo et al
  \item GenMir3/GenMir++
  \item Se joku graafinen (korkeintaan lauseen maininta)
  \item GroupMir
\end{itemize}
