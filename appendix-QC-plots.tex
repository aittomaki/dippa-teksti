%!TEX root = dippa.tex
%%% This file contains the QC plots appendix for my master's thesis
%%% Author: Viljami Aittomäki



\section{Quality control plots\label{app:qc-plots}}

\captionsetup{font={up},labelfont={sf}}

\begin{figure}[!h]
	\centering
	\begin{subfigure}{.7\textwidth}
		\centering
		\includegraphics[width=1\linewidth]{figures/proteinQCPlots/proteinQCPlots-pcaplot.pdf}
		\subcaption{protein\label{fig:protein-sample-pca}}
	\end{subfigure}
	\begin{subfigure}{.48\textwidth}
		\centering
		\includegraphics[width=1\linewidth]{figures/geneQCPlots/geneQCPlots-pcaplot.pdf}
		\subcaption{mRNA\label{fig:mrna-sample-pca}}
	\end{subfigure}
	\begin{subfigure}{.48\textwidth}
		\centering
		\includegraphics[width=1\linewidth]{figures/mirnaQCPlots/mirnaQCPlots-pcaplot.pdf}
		\subcaption{miRNA\label{fig:mirna-sample-pca}}
	\end{subfigure}

	\caption{Scatter plots of first two principal components for each tumor sample (or microarray)
	computed from (a) protein expression, (b) mRNA expression, and (c) miRNA expression data.
	The point color corresponds to the hospital where each sample was handled.}
	\label{fig:qc-pca}
\end{figure}


\begin{figure}
	\centering
	\begin{subfigure}{1\textwidth}
		\centering
		\makebox[\textwidth][c]{\includegraphics[width=1.1\linewidth]{figures/proteinQCPlots/proteinQCPlots-sample_boxplot.png}}
		\subcaption{Protein microarrays. \label{fig:protein-sample-boxplot}}
	\end{subfigure}
	\begin{subfigure}{1\textwidth}
		\centering
		\makebox[\textwidth][c]{\includegraphics[width=1.1\linewidth]{figures/proteinQCPlots/proteinQCPlots-variable_boxplot.png}}
		\subcaption{Protein variables. \label{fig:protein-variable-boxplot}}
	\end{subfigure}
	\begin{subfigure}{1\textwidth}
		\centering
		\makebox[\textwidth][c]{\includegraphics[width=1.1\linewidth]{figures/proteinQCPlots/proteinQCPlots-variable_boxplot_scaled.png}}
		\subcaption{Scaled protein variables ($\mu_y = 0, \sigma_y = 1$). \label{fig:protein-scaled-variable-boxplot}}
	\end{subfigure}

	\caption{Quality control boxplots for protein expression, plotted by
	(a) each tumor sample,
	(b) each protein expression variable (i.e. microarray), 
	and (c) scaled protein variables.
	The fill color in (a) corresponds to the hospital where the sample was handled. Note that
	the protein data differ from mRNA and miRNA data in that each array measures one protein instead of one sample.}
	\label{fig:qc-protein-boxplot}
\end{figure}


\begin{figure}[!h]
	\centering
	\begin{subfigure}{.45\textwidth}
		\centering
		\includegraphics[width=.9\linewidth]{figures/proteinQCPlots/proteinQCPlots-sample_density.pdf}
		\subcaption{Protein microarrays \label{fig:protein-sample-density}}
	\end{subfigure}
	\begin{subfigure}{.45\textwidth}
		\centering
		\includegraphics[width=.9\linewidth]{figures/proteinQCPlots/proteinQCPlots-variable_density.pdf}
		\subcaption{Protein variables \label{fig:protein-variable-density}}
	\end{subfigure}

	\caption{Density estimates for protein expression data. In (a) each curve represents one tumor sample,
	and in (b) each curve represents one protein variable. The color of the curves is purely for visualization,
	it bears no meaning.}
	\label{fig:qc-protein-density}
\end{figure}


\begin{figure}
	\centering
	\begin{subfigure}{1\textwidth}
		\centering
		\makebox[\textwidth][c]{\includegraphics[width=1.1\linewidth]{figures/geneQCPlots/geneQCPlots-sample_boxplot.png}}
		\subcaption{mRNA microarrays. \label{fig:mrna-sample-boxplot}}
	\end{subfigure}
	\begin{subfigure}{1\textwidth}
		\centering
		\makebox[\textwidth][c]{\includegraphics[width=1.1\linewidth]{figures/geneQCPlots/geneQCPlots-variable_boxplot.png}}
		\subcaption{mRNA variables. \label{fig:mrna-variable-boxplot}}
	\end{subfigure}
	\begin{subfigure}{1\textwidth}
		\centering
		\makebox[\textwidth][c]{\includegraphics[width=1.1\linewidth]{figures/geneQCPlots/geneQCPlots-variable_boxplot_scaled.png}}
		\subcaption{Scaled mRNA variables ($\mu_z = 0, \sigma_z = 1$). \label{fig:mrna-scaled-variable-boxplot}}
	\end{subfigure}

	\caption{Quality control boxplots for protein expression, plotted by
	(a) each tumor sample (i.e. microarray),
	(b) each protein expression variable, 
	and (c) scaled protein variables.
	The fill color in (a) corresponds to the hospital where the sample was handled.}
	\label{fig:qc-mrna-boxplot}
\end{figure}


\begin{figure}[!h]
	\centering
	\begin{subfigure}{.45\textwidth}
		\centering
		\includegraphics[width=.9\linewidth]{figures/geneQCPlots/geneQCPlots-sample_density.pdf}
		\subcaption{mRNA microarrays \label{fig:mrna-sample-density}}
	\end{subfigure}
	\begin{subfigure}{.45\textwidth}
		\centering
		\includegraphics[width=.9\linewidth]{figures/geneQCPlots/geneQCPlots-variable_density.pdf}
		\subcaption{mRNA variables \label{fig:mrna-variable-density}}
	\end{subfigure}

	\caption{Density estimates for mRNA expression data. In (a) each curve represents one microarray,
	and in (b) each curve represents one mRNA variable. The color of the curves is purely for visualization,
	it bears no meaning.}
	\label{fig:qc-mrna-density}
\end{figure}


\begin{figure}
	\centering
	\begin{subfigure}{1\textwidth}
		\centering
		\makebox[\textwidth][c]{\includegraphics[width=1.1\linewidth]{figures/mirnaQCPlots/mirnaQCPlots-sample_boxplot.png}}
		\subcaption{miRNA microarrays. \label{fig:mirna-sample-boxplot}}
	\end{subfigure}
	\begin{subfigure}{1\textwidth}
		\centering
		\makebox[\textwidth][c]{\includegraphics[width=1.1\linewidth]{figures/mirnaQCPlots/mirnaQCPlots-variable_boxplot_sorted.png}}
		\subcaption{miRNA variables. \label{fig:mirna-variable-boxplot}}
	\end{subfigure}
	\begin{subfigure}{1\textwidth}
		\centering
		\makebox[\textwidth][c]{\includegraphics[width=1.1\linewidth]{figures/mirnaQCPlots/mirnaQCPlots-variable_boxplot_scaled_sorted.png}}
		\subcaption{Scaled miRNA variables ($\mu_x = 0, \sigma_x = 1$). \label{fig:mirna-scaled-variable-boxplot}}
	\end{subfigure}

	\caption{Quality control boxplots for miRNA expression, plotted by
	(a) each tumor sample (i.e. microarray),
	(b) each miRNA expression variable, 
	and (c) scaled miRNA variables.
	The fill color in (a) corresponds to the hospital where the sample was handled
	The miRNA variables in (b) have been sorted by median expression to highlight that
	some miRNAs have very low expression. These measurements are likely to
	be heavily influenced by background noise, possibly even consisting of
	mainly noise. This order is retained in (c). The scaling does not
	correct for the highly skewed distributions of expression values.}
	\label{fig:qc-mirna-boxplot}
\end{figure}


\begin{figure}[!h]
	\centering
	\begin{subfigure}{.45\textwidth}
		\centering
		\includegraphics[width=.9\linewidth]{figures/mirnaQCPlots/mirnaQCPlots-sample_density.pdf}
		\subcaption{miRNA microarrays \label{fig:mirna-sample-density}}
	\end{subfigure}%
	\begin{subfigure}{.45\textwidth}
		\centering
		\includegraphics[width=.9\linewidth]{figures/mirnaQCPlots/mirnaQCPlots-variable_density.pdf}
		\subcaption{miRNA variables \label{fig:mirna-variable-density}}
	\end{subfigure}

	\caption{Density estimates for miRNA expression data. In (a) each curve represents one microarray,
	and in (b) each curve represents one miRNA variable. The color of the curves is purely for visualization,
	it bears no meaning. Note the bimodal distributions in (a) and the corresponding break
	and large spread in centroid location in (b).}
	\label{fig:qc-mirna-density}
\end{figure}
