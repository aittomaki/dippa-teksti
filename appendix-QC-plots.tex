%!TEX root = dippa.tex
%%% This file contains the QC plots appendix for my master's thesis
%%% Author: Viljami Aittomäki



\section{Quality control plots\label{app:qc-plots}}

\captionsetup{font={up,footnotesize},labelfont={sf}}

\begin{figure}[!h]
	\centering
	\begin{subfigure}{.7\textwidth}
		\centering
		\includegraphics[width=1\linewidth]{figures/proteinQCPlots/proteinQCPlots-pcaplot.pdf}
		\subcaption{protein\label{fig:protein-sample-pca}}
	\end{subfigure}
	\begin{subfigure}{.48\textwidth}
		\centering
		\includegraphics[width=1\linewidth]{figures/geneQCPlots/geneQCPlots-pcaplot.pdf}
		\subcaption{mRNA\label{fig:mrna-sample-pca}}
	\end{subfigure}
	\begin{subfigure}{.48\textwidth}
		\centering
		\includegraphics[width=1\linewidth]{figures/mirnaQCPlots/mirnaQCPlots-pcaplot.pdf}
		\subcaption{miRNA\label{fig:mirna-sample-pca}}
	\end{subfigure}

	\caption{Scatter plots of first two principal components for each tumor sample (or microarray)
	computed from (a) protein expression, (b) mRNA expression, and (c) miRNA expression data.
	The point color corresponds to the hospital where each sample was handled. The samples
	from different hospitals were not distinguishable using the principal components, which
	suggests that no significant hospital batch effect was present.}
	\label{fig:qc-pca}
\end{figure}


\begin{figure}
	\centering
	\begin{subfigure}{1\textwidth}
		\centering
		\makebox[\textwidth][c]{\includegraphics[width=1.1\linewidth]{figures/proteinQCPlots/proteinQCPlots-sample_boxplot.pdf}}
		\subcaption{Protein expression by tumor samples \label{fig:protein-sample-boxplot}}
	\end{subfigure}
	\begin{subfigure}{1\textwidth}
		\centering
		\makebox[\textwidth][c]{\includegraphics[width=1.1\linewidth]{figures/proteinQCPlots/proteinQCPlots-variable_boxplot.pdf}}
		\subcaption{Protein variables \label{fig:protein-variable-boxplot}}
	\end{subfigure}
	\begin{subfigure}{1\textwidth}
		\centering
		\makebox[\textwidth][c]{\includegraphics[width=1.1\linewidth]{figures/proteinQCPlots/proteinQCPlots-variable_boxplot_scaled.pdf}}
		\subcaption{Scaled protein variables ($\mu_y = 0, \sigma_y = 1$) \label{fig:protein-scaled-variable-boxplot}}
	\end{subfigure}

	\caption{Distribution of protein expression, grouped by
	(a) tumor samples,
	(b) protein variables (i.e. microarrays), 
	and (c) protein variables (and scaled to zero mean and unit variance).
	The fill color in (a) corresponds to the hospital where the sample was collected.
	It is evident in (c), that some of the protein variables have significant outliers
	(especially the two values beyond $10$ and $-10$).
	}
	\label{fig:qc-protein-boxplot}
\end{figure}


\begin{figure}[!h]
	\centering
	\begin{subfigure}{.49\textwidth}
		\subcaption{Protein expression by tumor samples \label{fig:protein-sample-density}}
		\includegraphics[width=1\linewidth]{figures/proteinQCPlots/proteinQCPlots-sample_density.pdf}
	\end{subfigure}
	\begin{subfigure}{.49\textwidth}
		\subcaption{Protein variables \label{fig:protein-variable-density}}
		\includegraphics[width=1\linewidth]{figures/proteinQCPlots/proteinQCPlots-variable_density.pdf}
	\end{subfigure}

	\caption{Density estimates of protein expression data for each tumor sample (a)
	and each protein variable (b). The color of the curves has no significance.
	Distributions were generally uniform, but
	long tails for several protein variables are visible in (b), suggesting the presence of significant
	outliers and that these variables were not approximated well by a normal ditribution.}
	\label{fig:qc-protein-density}
\end{figure}


\begin{figure}
	\centering
	\begin{subfigure}{1\textwidth}
		\centering
		\makebox[\textwidth][c]{\includegraphics[width=1.1\linewidth]{figures/geneQCPlots/geneQCPlots-sample_boxplot.pdf}}
		\subcaption{mRNA microarrays \label{fig:mrna-sample-boxplot}}
	\end{subfigure}
	\begin{subfigure}{1\textwidth}
		\centering
		\makebox[\textwidth][c]{\includegraphics[width=1.1\linewidth]{figures/geneQCPlots/geneQCPlots-variable_boxplot.pdf}}
		\subcaption{mRNA variables \label{fig:mrna-variable-boxplot}}
	\end{subfigure}
	\begin{subfigure}{1\textwidth}
		\centering
		\makebox[\textwidth][c]{\includegraphics[width=1.1\linewidth]{figures/geneQCPlots/geneQCPlots-variable_boxplot_scaled.pdf}}
		\subcaption{Scaled mRNA variables ($\mu_z = 0, \sigma_z = 1$). \label{fig:mrna-scaled-variable-boxplot}}
	\end{subfigure}

	\caption{Distribution of mRNA expression, grouped by
	(a) tumor samples (i.e. microarrays),
	(b) mRNA variables, 
	and (c) mRNA variables (and scaled to zero mean and unit variance).
	The fill color in (a) corresponds to the hospital where the sample was collected.
	The distributions were fairly uniform and only a few significant outliers (further than
	5 standard devations from the mean) were present in the mRNA variables.}
	\label{fig:qc-mrna-boxplot}
\end{figure}


\begin{figure}[!h]
	\centering
	\begin{subfigure}{.49\textwidth}
		\subcaption{mRNA microarrays \label{fig:mrna-sample-density}}
		\includegraphics[width=1\linewidth]{figures/geneQCPlots/geneQCPlots-sample_density.pdf}
	\end{subfigure}
	\begin{subfigure}{.49\textwidth}
		\subcaption{mRNA variables \label{fig:mrna-variable-density}}
		\includegraphics[width=1\linewidth]{figures/geneQCPlots/geneQCPlots-variable_density.pdf}
	\end{subfigure}

	\caption{Density estimates of mRNA expression data for each microarray (a)
	and each mRNA variable (b). The color of the curves has no significance.
	The array distributions were uniform and quite close to normal. Some of the variable distributions
	had long tails as seen in (b).}
	\label{fig:qc-mrna-density}
\end{figure}


\begin{figure}
	\centering
	\begin{subfigure}{1\textwidth}
		\centering
		\makebox[\textwidth][c]{\includegraphics[width=1.1\linewidth]{figures/mirnaQCPlots/mirnaQCPlots-sample_boxplot.pdf}}
		\subcaption{miRNA microarrays \label{fig:mirna-sample-boxplot}}
	\end{subfigure}
	\begin{subfigure}{1\textwidth}
		\centering
		\makebox[\textwidth][c]{\includegraphics[width=1.1\linewidth]{figures/mirnaQCPlots/mirnaQCPlots-variable_boxplot_sorted.pdf}}
		\subcaption{miRNA variables \label{fig:mirna-variable-boxplot}}
	\end{subfigure}
	\begin{subfigure}{1\textwidth}
		\centering
		\makebox[\textwidth][c]{\includegraphics[width=1.1\linewidth]{figures/mirnaQCPlots/mirnaQCPlots-variable_boxplot_scaled_sorted.pdf}}
		\subcaption{Scaled miRNA variables ($\mu_x = 0, \sigma_x = 1$) \label{fig:mirna-scaled-variable-boxplot}}
	\end{subfigure}

	\caption{Distribution of miRNA expression, grouped by
	(a) tumor samples (i.e. microarrays),
	(b) miRNA variables, 
	and (c) miRNA variables (and scaled to $\mu_x = 0, \sigma_x = 1$).
	Fill in (a) corresponds to hospital.
	The miRNA variables in (b) have been sorted by median expression to highlight that
	some miRNAs had very low expression. There appeared to be a gap at around $-8$
	(confirmed in Fig. \ref{fig:qc-mirna-density}). Measurements below the gap possibly corresponded
	to miRNAs not expressed in the samples, and thus, likely consisted
	mainly of background noise, Real miRNA abundances should follow a more continuous distribution.
	There also seemed to be some technical artifact around the gap, where some miRNA variables had
	exactly the same expression value. No biological phenomenon would explain this.
	The same order is retained in (c). The scaling did not
	correct for the highly skewed distributions of the most lowly expressed miRNAs.}
	\label{fig:qc-mirna-boxplot}
\end{figure}


\begin{figure}[!h]
	\centering
	\begin{subfigure}{.49\textwidth}
		\subcaption{miRNA microarrays \label{fig:mirna-sample-density}}
		\includegraphics[width=1\linewidth]{figures/mirnaQCPlots/mirnaQCPlots-sample_density.pdf}
	\end{subfigure}%
	\begin{subfigure}{.49\textwidth}
		\subcaption{miRNA variables \label{fig:mirna-variable-density}}
		\includegraphics[width=1\linewidth]{figures/mirnaQCPlots/mirnaQCPlots-variable_density.pdf}
	\end{subfigure}

	\caption{Density estimates of miRNA expression data for each microarray (a)
	and each miRNA variable (b). Color of the curves has no significance.
	Note the bimodal distributions in (a) and the corresponding break at around $-8$
	and large spread in location in (b). This suggests that the miRNAs below the gap were present in
	such low quantities, possibly not expressed at all, that the measured expression values likely
	consisted mostly of background noise.}
	\label{fig:qc-mirna-density}
\end{figure}
