%!TEX root = dippa.tex
%%% This file contains the Cancer section of my master's thesis.
%%% This section covers the basics of cancer and microRNA involvement.
%%% Author: Viljami Aittomäki

\section{Cancer}\label{cancer}

Cancer is a disease of uncontrolled overgrowth of a population of cells. It is
generally viewed as a genetic disease, albeit it is mostly not inherited, as it is
caused by  mutations in the genome of the tumor. These mutations cause malfunction
and dysregulation of the genetic machinery regulating cellular functions, such
as cell proliferation, differentiation and apoptosis, resulting in
unregulated growth and malignant tumor formation.

There are several classes of genes that influence tumor growth, the two main
categories being oncogenes and tumor suppressors. Oncogenes were first
identified in retroviruses and later shown to be proto-oncogenes, which by
mutation develop into oncogenes whose over activity promotes tumor growth
\citep{Varmus1988}. Tumor-suppressor genes are often regulators of cell
proliferation or other so called housekeeping genes that work to ensure the
proper functioning of cells and the apoptosis of misbehaving ones. The
inactivation of these genes can lead to tumor progression. The existence of
tumor suppressors was first hypothesized by Alfred Knudson, who formed the
“two-hit hypothesis” while studying the epidemiology of retinoblastoma
\citep{Knudson1971}. He suggested that, for cancer to develop, both copies of
a tumor suppressor gene should become inactivated and that in inherited
cancers one mutation is acquired in the germline and the other occurs in
somatic cells, whereas in sporadic cancers both mutations happen in somatic
cells.

The idea of oncogenes and tumor suppressor genes was later expanded on by
Douglas Hanahan and Robert Weinberg in their seminal article The Hallmarks of
Cancer \citep{Hanahan2000}. The hallmarks are a set of six features which
tumors often acquire to become malignant. The features are: sustaining
proliferative signaling, evading growth suppressors, resisting cell death,
enabling replicative immortality, inducing angiogenesis, and activating
invasion and metastasis. Weinberg and Hanahan postulated that at least three
of these six features are required for invasive cancer to develop.

Recently, Hanahan and Weinberg revised the hallmarks with two new upcoming
hallmarks and two enabling characteristics \citep{Hanahan2011}. The new
hallmarks are deregulation of cellular energetics and avoiding immune
destruction. The enabling characteristics of malignant tumors are genome
instability and mutation, and tumor-promoting inflammation through recruitment
of the immune system. Genome instability and mutation is of special importance
as much of cancer and tumor research focuses on identifying mutated or
dysregulated genes that promote tumor progression.




\subsection{Breast cancer}\label{breast-cancer}

Breast cancer constitutes a significant health issue globally. It is the most
common cancer in women and the second most common cancer overall;
approximately 1.7 million women develop breast cancer annually world-wide, and
in 2012 there were 522 000 breast cancer-related deaths
\citep{Ferlay2015}. In Finland there were 5008 new cases of breast cancer
 and 815 breast cancer-related deaths in 2014 \citep{Syoparekisteri}.

Most breast cancers are sporadic; only 5-7\% of breast cancer cases are of
familial type \citep{Melchor2013}. However, 15-30\% of breast cancer patients
have a family member or relative with breast cancer. This is
mostly due to the high frequency of breast cancer in many western populations,
but also suggests that there are unknown genetic factors and environmental
factors that have an impact on breast cancer development. Indeed, breast
cancer is a hormone-related disease and hormonal factors are known to have an
impact on breast cancer risk. The most important of these are
estrogen and progesterone.

The most important hereditary forms of breast cancer are those related to a
tumor predisposition syndrome caused by mutations in the breast cancer 1
(BRCA1) and BRCA2 genes, which explain about 25\% familial breast cancer in
many populations. These, however are rare on the population level and familial
clustering of beast cancer is multifactorial and caused by moderate risk and
low risk genetic variations, which are much more common \citep{Melchor2013}.




\subsubsection{Breast cancer classification}\label{breast-cancer-classification}

TÄTÄ PITÄÄ TIIVISTÄÄ!

The basic classification of cancer is based on the site -- that is, the organ
or tissue, such as breast or colon, where the primary tumor develops. Cancers
occurring within each organ are, however, heterogeneous in their nature and
have different behavior and prognosis. Thus, tumors are secondarily classified
by morphology, the microscopic structure of the cancer tissue.

The morphological classification of breast cancer is based on the WHO
classification from 2003 and includes altogether 19 histological subtypes of
invasive breast cancer \citep{Tavassoli2003,Weigelt2009}. Of these invasive
ductal carcinoma not otherwise specified (IDC NOS), accounts for the by far
largest histological group. Additionally, the WHO classification of
breast cancer includes the
TNM classification; characterization of the primary tumor (T), lymph node
status (N) and distant metastasis (M) and stage grouping based on the TNM
data. cTNM (or TNM) refers to a clinical TNM classification -- based on imaging studies,
surgical exploration and similar studies -- and pTNM to a pathological
classification -- based on postsurgical pathological analysis of tissue
biopsies. All invasive cancers are also graded into well (I), moderately (II)
or poorly (III) differentiated tumors based on microscopical examination
% based on three features; tubule formation as an expression of glandular
% differentiation, nuclear pleomorphism and mitotic counts
with less differentiated tumors having worse prognosis \citep{Tavassoli2003}.

In the clinic many other tumor characteristics are used. These
include the age of the patient, lymphovascular invasion as well as expression of
estrogen (ER) and progesterone receptors (PR), and Her2, which are routinely
studied for breast cancers. Together these data can be used to group patients
into risk categories for prognosis and choice of treatment using for example
St Gallen criteria \citep{Goldhirsch2007} or NIH criteria \citep{Eifel2001} or others.
The estrogen receptor also has a special role in treatment, as tumors that highly express ER
(termed ER-positive and covering 70\% of all cases) can be given antiestrogens
as endocrine therapy.

One of the problems with morphological classification of breast tumors is that
over 50\% do not show any particular features and are classified as IDC NOS,
in spite of the fact that tumors in this large group have very different
clinical outcomes. As mentioned above, tumors of the same tissue -- and even
of the same histological type -- are heterogeneous in nature and personalized
molecular diagnostics are required to exploit more targeted treatments. This
is even more critical in the case of de novo treatment resistance, where
tumors develop a molecular mechanism for resisting pharmacological therapies.
This is especially common with targeted therapies, for example ER-positive
tumors may acquire mutations to ER rendering antiestrogen therapy ineffective
\citep{Oesterreich2013}.

More recently, expression profiling has led to a suggestion
of new classification of breast cancers \citep{Perou2000,Sorlie2001}. By studying 65
samples from breast tumors with expression profiling Perou et al.
distinguished four subgroups based on gene expression, namely
ER+/luminal-like, basal-like, Erb-B2+ and normal breast \citep{Perou2000}.

Expression profiling has also led to development of new prognostic tests to
help to determine the need of adjuvant chemotherapy. These tests include
Oncotype DX (a 21-gene recurrence score), MammaPrint (a 70-gene test) and PAM50
(a 50-gene test), and have been evaluated by several groups and suggested to be
valid and promising, but their utility in clinical decision making remains
unclear \citep{Azim2013}. The PAM50 molecular classification ... 

\textbf{SUBTYPE KORRELAATIO PROGNOOSIIN JA HOITOON PITÄISI KERTOA PAREMMIN?}




\subsection{MicroRNAs and cancer}

Research has shown microRNAs to have important roles in tumor initiation,
progression and metastasis \citep{Lin2015}. MicroRNA expression signatures
correlate with numerous cancer features, such as tissue, staging, 
progression, prognosis and treatment response, and all studied cancers
have had miRNA expression profiles differing from healthy tissue, including breast
\citep{Calin2006}. In fact, microRNAs appear to be globally underexpressed in
cancers \citep{Lu2005}. Therefore, it is clear that microRNAs participate
in many of the pathways resulting in the hallmarks of cancer, and
examples of miRNAs influencing each of the hallmarks have been found.

Similarly to protein-coding genes, microRNAs can function as tumor
suppressors or oncogenes \citep{Lin2015}. In a meta-analysis of dysregulation
of miRNAs in breast cancer van Schoonveld found major oncogenic microRNAs in
breast cancer to include miR-10b, miR-21, miR- 155, miR-373, and miR-520c
\citep{vanSchooneveld2015}. They also list nine miRNAs as major tumor suppressors for breast cancer,
namely miR-125b, miR-205, miR-17-92, miR-206, miR-200, miR-146b, miR-126,
miR-335, and miR-31.

The genetic mechanisms for microRNA involvement in cancer are varied,
including mutations in miRNA or target mRNA sequence, chromosomal
rearrangements of the miRNA-encoding DNA regions and epigenetic changes in DNA
methylation or histones, leading to aberrant miRNA expression
\citep{Calin2006,Melo2011}. For example, a single-nucleotide polymorphism
in the microRNA miR-196a2 has been found to be associated with breast cancer
risk \citep{Gao2011}. A mutation in the sequence of estrogen receptor alpha,
in the target site of miR-453, has been suggested to be associated with a
lower breast cancer risk \citep{Tchatchou2009}, an example of a mutation in
a target transcript affecting miRNA function. MicroRNA function can also be
altered by abnormalities in the miRNA-processing machinery. For instance, a
mutation in the Dicer gene causes a tumor predisposition syndrome known as
DICER1 syndrome \citep{Slade2011}. Another example of this is apparent
dysregulation of Dicer and Drosha in breast cancer \citep{Yan2012}.

The different subtypes of breast cancer, explained above, reflect the genetic
background of the tumor and, accordingly, the subtypes differ in their gene
expression profiles. This also applies for miRNA expression, the different
intrinsic subtypes have different miRNA expression profiles, suggesting their
importance in breast cancer evolution \citep{Blenkiron2007}. de Rinaldis et al
identified a 46-miRNA signature that could be used in differentiating the
intrinsic subtypes from each other \citep{deRinaldis2013}. In addition to
tumor development, many miRNAs have been found to modulate the response to
breast cancer therapies. These include chemotherapy, antiendocrine therapy,
radiotherapy and targeted therapies.

Accordingly, miRNAs have been studied as biomarkers for diagnosing cancer and
cancer prognosis. Emmadi et al recently found let-7 expression to be negatively
correlated with the Oncotype DX recurrence score in breast cancer
\citep{Emmadi2015}. This corroborated with the earlier finding of let-7 being
downregulated in breast cancer stem cells (tumor cells possessing the ability
of self-renewal) \citep{Yu2007} and later research suggesting let-7 to act as
a tumor suppressor. Several miRNAs have also been associated with breast cancer
metastasis \citep{Chen2016}.

MicroRNAs also show promise as a novel therapeutic tool, several studies have
proposed miRNA-based cancer treatments in animal models, including for breast
cancer \citep{VanRooij2014}. However, more research this area is needed before
microRNA treatments are ready for the clinical setting.

% Cell-free biomarker:
% Lawrie C.H., Gal S., Dunlop H.M., Pushkaran B., Liggins A.P., Pulford K., Banham A.H., Pezzella F., Boultwood J., Wainscoat J.S., et al. Detection of elevated levels of tumour-associated microRNAs in serum of patients with diffuse large B-cell lymphoma. Br. J. Haematol. 2008;141:672-675.

% Treatments:
% Van Rooij E., Kauppinen S. Development of microRNA therapeutics is coming of age. EMBO Mol. Med. 2014;6:851-864.
% Garzon R., Marcucci G., Croce C.M. Targeting microRNAs in cancer: rationale, strategies and challenges. Nat. Rev. Drug Discov. 2010;9:775-789.
