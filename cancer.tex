%!TEX root = dippa.tex
%%% This file contains the Cancer section of my master's thesis.
%%% This section covers the basics of cancer and microRNA involvement.
%%% Author: Viljami Aittomäki

\section{Cancer}\label{cancer}

Cancer is a disease of uncontrolled overgrowth of a population of cells. It is
generally viewed as a genetic disease, albeit it is mostly not inherited, as it is
caused by  mutations in the genome of the tumor. These mutations cause malfunction
and dysregulation of the genetic machinery regulating cellular functions, such
as cell proliferation, differentiation and apoptosis, resulting in
unregulated growth and malignant tumor formation.

There are several classes of genes that influence tumor growth, the two main
categories being oncogenes and tumor suppressors. Oncogenes were first
identified in retroviruses and later shown to be proto-oncogenes, which by
mutation develop into oncogenes whose over activity promotes tumor growth
\citep{Varmus1988}. Tumor-suppressor genes are often regulators of cell
proliferation or other so called housekeeping genes that work to ensure the
proper functioning of cells and the apoptosis of misbehaving ones. The
inactivation of these genes can lead to tumor progression. The existence of
tumor suppressors was first hypothesized by Alfred Knudson, who formed the
“two-hit hypothesis” while studying the epidemiology of retinoblastoma
\citep{Knudson1971}. He suggested that, for cancer to develop, both copies of
a tumor suppressor gene should become inactivated and that in inherited
cancers one mutation is acquired in the germline and the other occurs in
somatic cells, whereas in sporadic cancers both mutations happen in somatic
cells.

The Hallmarks of Cancer were suggested Douglas Hanahan and Robert Weinberg \citep{Hanahan2000} in
their seminal article in 2000. The hallmarks are a set of
six features which tumors often acquire to become malignant. The features are:
sustaining proliferative signaling, evading growth suppressors, resisting cell
death, enabling replicative immortality, inducing angiogenesis, and activating
invasion and metastasis. Weinberg and Hanahan postulated that at least three
of these six features are required for invasive cancer to develop.

Recently, Hanahan and Weinberg \citep{Hanahan2011} revised the hallmarks with two new
hallmarks and two enabling characteristics. The new
hallmarks are deregulation of cellular energetics and avoiding immune
destruction. The enabling characteristics of malignant tumors are genome
instability and mutation, and tumor-promoting inflammation through recruitment
of the immune system. Genome instability and mutation is of special importance
as much of cancer and tumor research focuses on identifying mutated or
dysregulated genes that promote tumor progression.




\subsection{Breast cancer}\label{breast-cancer}

Breast cancer constitutes a significant health issue globally. It is the most
common cancer in women and the second most common cancer overall;
approximately 1.7 million women develop breast cancer annually world-wide, and
in 2012 there were 522,000 breast cancer-related deaths
\citep{Ferlay2015}. In Finland there were 5,008 new breast cancer cases
and 815 breast cancer-related deaths in 2014 \citep{Syoparekisteri}.

Most breast cancers are sporadic; only 5-7\% of breast cancer cases are of
familial type \citep{Melchor2013}. However, 15-30\% of breast cancer patients
have a family member or relative with breast cancer. This is
mostly due to the high frequency of breast cancer in many western populations,
but also suggests that there are unknown genetic factors and environmental
factors that have an impact on breast cancer development. Indeed, breast
cancer is a hormone-related disease and hormonal factors, most importantly estrogens,
are known to have an impact on breast cancer development.

The most common hereditary forms of breast cancer are related to mutations in
the breast cancer genes \emph{BRCA1} and \emph{BRCA2}, which explain about 25\% familial
breast cancer in many populations. These, however are rare on the population
level and familial clustering of beast cancer is multifactorial and caused by
moderate risk and low risk genetic variations, which are much more common
\citep{Melchor2013}.




\subsubsection{Breast cancer classification}\label{breast-cancer-classification}

Breast cancers are heterogeneous in their nature and classified by morphology
(the microscopic structure of cancer tissue).
The morphological classification of breast cancer is based on the WHO
classification from 2003 and includes altogether 19 histological subtypes of
invasive breast cancer \citep{Tavassoli2003,Weigelt2009}. Of these, invasive
ductal carcinoma is by far the most common. Additionally, the WHO
classification of breast cancer includes the TNM classification (consisting of
the size of the primary tumor (T), whether the tumor has spread to
lymph nodes (N) or metastasized (M) into other tissues), and grading
into well, moderately or poorly differentiated tumors based on
microscopical examination,
% based on three features; tubule formation as an expression of glandular
% differentiation, nuclear pleomorphism and mitotic counts
with less differentiated tumors having worse prognosis \citep{Tavassoli2003}.

In the clinic several other tumor characteristics, such as patient age, and the
expression of estrogen (ER) and progesterone receptors (PR) and Her2, are used
to group patients into prognostic categories. Treatment regimens can be chosen
using for example the St Gallen criteria \citep{Goldhirsch2007}, which
classifies tumors into highly endocrine responsive, incompletely endocrine
responsive, and endocrine non-responsive based on the expression of estrogen
and progesterone receptors in tumor cells and, consequently, the response
to endocrine treatment.

One of the problems with morphological classification is that over 50\% of
tumors do not show any particular features, although tumors in this large
group have highly variable outcomes. Additionally, endocrine-repsonsive tumors
can become non-responsive as tumor cells accumulate genomic mutations \citep{Oesterreich2013}.
Therefore, personalized molecular diagnostics are required to tailor targeted treatments.

More recently, expression profiling has led to a suggestion of new
classification of breast cancers. Studying the expression profiles of breast
tumors, Perou et al \citep{Perou2000} distinguished four subgroups based on
gene expression, namely ER+/luminal-like, basal-like, Erb-B2+ and normal
breast, The subgroups were shown to correlate with prognosis and a 50-gene
test (PAM50) was devised to classify tumors into the subtypes
\citep{Parker2009}.

Other prognostic tests based on expression signatures, including Oncotype DX
(a 21-gene recurrence score) and MammaPrint (a 70-gene test), have also been
proposed to help to determine the need of adjuvant chemotherapy. These tests
have been suggested to be valid and promising, but their utility in clinical
decision making remains unclear \citep{Azim2013}.




\subsection{MicroRNAs and cancer}

The dysregulation of microRNAs is associated with many human diseases, such as
neurological disorders, diabetes and cancer \citep{Jiang2009}. A
disease-promoting role for miRNAs has been implicated in many different
cancers \citep{Melo2011}, including breast cancer, and research has shown microRNAs
to have important roles in tumor initiation,
progression and metastasis \citep{Lin2015}.

MicroRNA expression signatures
correlate with numerous cancer features, such as tissue of origin, 
progression, prognosis and treatment response, and all studied cancers
have had miRNA expression profiles differing from healthy tissue
\citep{Calin2006}. In fact, microRNAs appear to be generally underexpressed in
cancers \citep{Lu2005}. Therefore, it seems clear that microRNAs participate
in many of the pathways resulting in the hallmarks of cancer, and
examples of miRNAs influencing each of the hallmarks have been found.
Similarly to protein-coding genes, microRNAs can function as tumor
suppressors or oncogenes \citep{Lin2015}. For instance, in a
meta-analysis of dysregulation of miRNAs in breast cancer, van Schoonveld
et al \citep{vanSchooneveld2015} found five major oncogenic and nine major tumor suppressive microRNAs.

% breast cancer to include miR-10b, miR-21, miR-155, miR-373, and miR-520c
% \citep{vanSchooneveld2015}. They also list nine miRNAs as major tumor suppressors for breast cancer,
% namely miR-125b, miR-205, miR-17-92, miR-206, miR-200, miR-146b, miR-126,
% miR-335, and miR-31.

The genetic mechanisms for microRNA involvement in cancer are varied,
including mutations in miRNA or target mRNA sequence, chromosomal
rearrangements of the miRNA-encoding DNA regions and epigenetic changes in DNA
methylation or histones, leading to aberrant miRNA expression
\citep{Calin2006,Melo2011}. For example, a single-nucleotide polymorphism
in the microRNA miR-196a2 has been found to be associated with breast cancer
risk \citep{Gao2011}. A mutation in the sequence of estrogen receptor alpha,
in the target site of miR-453, has been suggested to be associated with a
lower breast cancer risk \citep{Tchatchou2009}, an example of a mutation in
a target transcript affecting miRNA function. MicroRNA function can also be
altered by abnormalities in the miRNA-processing machinery. For instance, a
mutation in the Dicer gene causes a tumor predisposition syndrome known as
DICER1 syndrome \citep{Slade2011}. Another example of this is apparent
dysregulation of Dicer and Drosha in breast cancer \citep{Yan2012}.

The different subtypes of breast cancer, explained above, reflect the genetic
background of the tumor and, accordingly, the subtypes differ in their gene
expression profiles. This also applies for miRNA expression, the different
intrinsic subtypes have different miRNA expression profiles, suggesting their
importance in breast cancer evolution \citep{Blenkiron2007}.
de Rinaldis et al \citep{deRinaldis2013}
identified a 46-miRNA signature that could be used in differentiating the
intrinsic subtypes from each other. In addition to
tumor development, many miRNAs have been found to modulate the response to
breast cancer therapies. These include chemotherapy, antiendocrine therapy,
radiotherapy and targeted therapies.

Accordingly, miRNAs have been studied as biomarkers for diagnosing cancer and
cancer prognosis. Emmadi et al \citep{Emmadi2015} recently found let-7 expression to be negatively
correlated with the Oncotype DX recurrence score in breast cancer.
This corroborated with the earlier finding of let-7 being
downregulated in breast cancer stem cells (tumor cells possessing the ability
of self-renewal) \citep{Yu2007} and later research suggesting let-7 to act as
a tumor suppressor. Several miRNAs have also been associated with breast cancer
metastasis \citep{Chen2016}.

MicroRNAs also show promise as a novel therapeutic tool. Several studies have
tested miRNA-based cancer treatments in animal models \citep{VanRooij2014}.
However, more research this area is needed before microRNA treatments are
ready for the clinical setting.

% Cell-free biomarker:
% Lawrie C.H., Gal S., Dunlop H.M., Pushkaran B., Liggins A.P., Pulford K., Banham A.H., Pezzella F., Boultwood J., Wainscoat J.S., et al. Detection of elevated levels of tumour-associated microRNAs in serum of patients with diffuse large B-cell lymphoma. Br. J. Haematol. 2008;141:672-675.

% Treatments:
% Garzon R., Marcucci G., Croce C.M. Targeting microRNAs in cancer: rationale, strategies and challenges. Nat. Rev. Drug Discov. 2010;9:775-789.
