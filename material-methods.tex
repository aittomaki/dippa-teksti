%!TEX root = dippa.tex
%%% This file contains the materials and methods section of my master's thesis.
%%% Author: Viljami Aittomäki


\section{Materials and methods}\label{materials-and-methods}

\subsection{Research material}

The data analysed in this thesis consists of 283 tumor samples collected from
280 breast cancer patients treated in two Norwegian hospitals. Protein, mRNA
and microRNA expression were measured from each sample. The data were
published by Aure et al. \citep{Aure2015} and are publicly available. Analyses
performed in this thesis used the publicly available preprocessed data.

The data are part of the Oslo2 cohort, which consists of breast cancer
patients with primarily operable disease -- that is stage cT1--cT2 -- treated
in several Norwegian hospitals. Collection of the cohort started in 2006 and
is still ongoing. Therefore, no survival data were available for analysis.
Clinical data for included patients were kindly provided by Aure and
associates and a compiled summary is presented in Table
\ref{clinical-data}. Notably, the vast majority of tumors in the data
were ductal carcinomas, which is in general the most common histological type
of breast cancer. Patient ages ranged from X to Y with a median of 45 years.
No matched control samples of healthy breast tissue were available.
It is worth noting that the data represent a very heterogeneous sampling
of different kinds of breast tumors.

../dippa-analyysi/execute/clinicalLatex/document/document.tex

The mRNA and microRNA expression were measured using Agilent Technologies
SurePrint G3 Human GE 8x60K and Human miRNA Microarray Kit (V2) microarrays,
respectively. These microarrays measure 27958 genes and 887 miRNAs, according
to manufacturer annotation. Protein expression was measured using a reverse
phospatase protein array (RPPA) containing a set of 105 proteins. Most of the
proteins are found on the PI3K/AKT intracellular pathway, which is important
for cell-cycle regulation and, thus, cancer. \textbf{(ref to G.Mills?)}.

% The mRNA and microRNA expression data are publicly available in preprocessed
% format in the Gene Expression Omnibus (GEO) database \citep{GEO} under
% accession IDs xx and xx respectively. For the purpose of this thesis, the raw
% Agilent expression data were kindly provided and used for the analyses instead
% of the preprocessed data. The protein expression data is available in
% Additional file 4 of \citet{norjis} also in preprocessed format.

% Clinical data concerning each patient and cancer were also provided. A summary
% of the clinical parameters is presented in table \ref{clinical-data}. The
% predominant tumor type in the data was ductal carcinoma, which is in general
% the most common histological type of breast cancer.

% Use danish data for validation?





\subsection{Methods}

A Bayesian regression model of protein, mRNA and miRNA expression data was
constructed, and projection predictive variable selection was used to predict
microRNA targets in breast cancer data. Details of the methods used are
presented in this section. All computational analyses were performed in R
\citep{R} and workflow management was handled with Anduril \citep{Ovaska2010}.
Monte-Carlo simulations for the Bayesian regression models were performed with
\emph{RStan} \citep{RStan} using the No-U-Turn variant of a Hamiltonian Monte Carlo
algorithm for sampling posterior distributions. Simulations were run using
computer resources within the Aalto University School of Science "Science-IT"
project.


\subsubsection{Preprocessing and quality control}

MicroRNA and mRNA expression data were downloaded from the Gene Expression
Omnibus (accessions GSE8210 and GSE8212, respectively \citep{Edgar2002}) using
the GEOquery R package \citep{GEOquery}. Protein expression data were
downloaded as a Microsoft Excel sheet from the supplementary data of Aure et
al \citep{Aure2015}. All of the data are in log scale, as is usual for
expression data analyses. For regression analyses, all variables (miRNA, mRNA
and protein) were further scaled to have zero mean and unit variance,
a commonly used transformation in regression.

Notably, the protein data have only a single set of measurements for two sets
of related genes, namely AKT1, AKT2, AKT3 and GSK3A, GSK3B (labeled
respectively as "AKT1/2/3" and "GSK3A/GSK3B" in the original data). Presumably
the protein array used cannot differentiate the proteins of these genes, but
no mention of this is made by Aure et al. For the analyses presented here,
each of these genes was considered separately, and the same expression values
from the single measurement set were used for all of the related genes
respectively.

Out of the 421 microRNAs present in the public dataset, eleven miRNAs (hsa-
miR-1274a, hsa-miR-1274b, hsa-miR-1280, hsa-miR-1308, hsa- miR-1826, hsa-
miR-1974, hsa-miR-1975, hsa-miR-1977, hsa-miR-1979, hsa-miR-720, hsa-
miR-886-3p) were reported as missing from miRBase by miRBase Tracker. Reviewed
on miRBase, these miRNAs are reported as being fragments of other RNA species
(e.g. tRNA or rRNA) and, thus, removed from the database. The eleven miRNAs
were therefore removed from subsequent analyses, leaving 410 miRNAs.

The publicly available expression data are in preprocessed form, and as such,
no further preprocessing of the actual measurements was done. For details on
the preprocessing, the reader is referred to the supplementary data of Aure et
al \citep{Aure2015}. The mRNA expression data are available as probe-level
measurements, these were summarized to gene-level using manufacturer probe
annotations by taking the median of all probes targeting the same gene.
Only the genes present in the protein data were used in analyses.

For assessing the quality of the expression data, distributions of each
microarray and each variable (miRNA, mRNA and protein) were plotted. The data
have been collected from two different hospitals and to assess possible bias
introduced by the separate sites, a principal component analysis (PCA) and
hierarchical clustering of samples were performed separately for each data
type.



\subsubsection{Correlation analysis}


\subsubsection{Regression models}

For predicting protein expression from mRNA and miRNA expression, a similar
regression model to Aure et al was used:
\begin{equation}
	\label{eq:reg-model}
	y = \beta_0 + z \beta_g + X \beta + \epsilon,
\end{equation}
where $y$ denotes the protein expression vector for the protein, which
is produced from translation of the mRNA $z$, $w_g$
is the regression coefficient for the mRNA, $X$ is the matrix of miRNA
expression vectors, and $w_0$ is the intercept term
(for a justification of this equation, see Aure et al \citep{Aure2015}).
A separate model was fitted for each gene.
A model with only the mRNA expression covariate (called the \emph{gene-only model}),
defined as $y = \beta_0 + z \beta_g + \epsilon$, was used as a baseline.
A normally distributed error term with equal errors and no correlation
between observations was assumed for all models.

The likelihood for Bayesian regression was therefore defined as
\begin{equation}
	y | \mathbf{\beta}, \sigma, z, X \sim N(\beta_0 + z \beta_g + X \beta, \sigma^2I),
\end{equation}
where $\mathbf{\beta} = [\beta_0, \beta_g, \beta]$ for convenience.
The intercept and mRNA coefficient were given diffuse Gaussian priors
and $\sigma$ a uniform prior:
\begin{subequations}
  \begin{align}
    \beta_0 & \sim N(0, 5^2) \\
    \beta_g & \sim N(0, 5^2) \\
    \sigma  & \propto 1.
  \end{align}
\end{subequations}
These were also used in the gene-only model.
A hierarchical shrinkage prior was applied to the miRNA coefficients $\beta$,
as defined in Equation \eqref{eq:hs-prior}. The degrees of freedom for the
$\lambda_j$ priors was set at $\nu=3$ (similar to Piironen and Vehtari
\citep{Piironen2015}). The prior for $\tau$ was defined as:
\begin{equation}
    \tau ~ \textup{half-Cauchy}(0, \frac{p_n}/{n}*\sqrt{\textup{log}(n/p_n)}),
\end{equation}
combining the previous suggestions of half-Cauchy and fixed $tau$.
The assumed number of relevant miRNAs, $p_n$, was estimated as follows.
Ensembl gene ID's were downloaded for all protein-coding genes in the human
genome using biomaRt \citep{biomaRt}. From these, a sample of 1000 genes was
taken, and known validated microRNA interaction partners for each sampled gene
were downloaded from miRWalk \citep{Dweep2015}. Genes for which there were no
validated miRNA interactors were assumed to have zero. The mean number of
miRNA interactors per gene was used as an estimate, giving
$\hat{p_n} = 13.75$.



\subsubsection{Variable selection}

Projection predictive variable selection (as described in Section
\ref{sec:bayes-variable- selection}) was used to obtain the relevant set
of microRNAs for each gene. A full model was fitted by drawing 2000 samples
from the posterior using RStan (4 chains, 1000 samples each and the first half
discarded as burn-in). A random sample of $S=1000$ simulation samples from the
full posterior was used to increase projection speed. Then a series of
submodels was obtained by projection from the full model and using a forward
search strategy. That is, the search started from a model including only the
intercept, the mRNA expression $z$ was always added as the first covariate,
and then at each subsequent step, the miRNA covariate $x_j$ giving the largest
decrease in KL-divergence between the full and projected models was chosen.
The forward search was continued up to 200 variables.

For choosing the model size, 10-fold cross validation was used, as proposed by
Piironen and Vehtari \citep{Piironen2016}. That is, the above model selection
process was performed $K=10$ times, each leaving $n/K$ observations out for
evaluation. To judge the appropriote model size, estimation of submodel
predictive performance was used as explained below.

\paragraph{Predictive performance}
Given submodel $M_\perp$ with the posterior predictive distribution
$p(\tilde{y}| \tilde{z}, \tilde{X}_\perp, \mathbf{\theta}_\perp, D_\perp)$,
where $D_\perp$ is the observed data in the current submodel and
$\mathbf{\theta}_\perp$ the projected parameters, the predictive performance
of each submodel was evaluated with the logarithm of the predictive density
(LPD) at each of the left-out observations $(y_*, z_*, X_*)$ The LPD was
estimated by averaging over the simulated posterior samples:
\[
	textup{LPD}_*(M_\perp) \approx \textup{log}\frac{1}{S} \sum_s^S p(y_*| z_*, X_*, \mathbf{\beta}_\perp, \sigma_\perp).
\]
The LPD's from each fold were pooled and a mean over the full set of data
(MLPD) was used as a summary. To compare the predictive performance of a
sbumodel to the full model the difference in MLPD ($\Delta \textup{MLPD}$) was
used. Bayesian bootstrap (with 5000 samples) was used to estimate a
distribution for $\Delta \textup{MLPD}$ by:
\[
	\Delta \textup{MLPD}^{s} = \sum_i^n w_i^{(s)} \left [ LPD_i(M_\perp) - LPD_i(M_{\textup{full}}) \right ],
\]
where $w_i^{(s)}, i = 1, \dotsc,n$, are the bootstrap weights
for the $s$th bootstrap sample (subject to $\sum_i w_i^{(s)} = 1$).
The \emph{bayesboot} R package was used for computing the bootstrap.

\paragraph{Choosing model size}
The threshold for choosing the model size was defined as the smallest
model satisfying
\begin{equation}
	\label{eq:size-condition}
	\textup{Pr}(\Delta \textup{LPD} < U) \geq \alpha,
\end{equation}
where $U = \gamma \textup{E}(LPD_{\textup{full}}-LPD_{0})$, and
$LPD_0$ refers to the intercept-only model. This means that model
size was chosen such that the probability of the difference in utility
between projected and full model being smaller than a constant times
the difference between intercept-only and full model was at least $\alpha$.
Several different values for $\alpha$ and $\gamma$ were experimented with.

\paragraph{Final model selection}
The final projected model was obtained by redoing the projection search up to
the chosen number of variables, using all data for each gene. The miRNAs for
which the 95\% posterior interval did not include the origo (called
\emph{significant miRNAs}) were considered putative interactors with the gene
in question. For some genes the condition in \eqref{eq:size-condition} was not
met after including 200 covariates. In these cases it was concluded that the
miRNA covariates provided no additional information on the protein expression
and, thus, none of them were deemed as targeting the gene. In some cases the
condition was met already by the model with only the mRNA covariate, and the
same conclusion was made.

\paragraph{Lasso regression}
A lasso regression model was also fitted for each gene using the
\emph{glmnet} R package. In this case the mRNA variable was treated equal
to the miRNA variables and subjected to the lasso regularization.
For choosing the regularization parameter $\lambda$, a 10-fold cross
validation was done, and the largest $\lambda$ that gave
a mean square prediction error (MSE) at most one standard deviation
apart from the lowest MSE was used. The covariates included
in the model using the chosen $\lambda$ were considered putative
target interactors. For same genes, this criterion was met by
the intercept-only model, and again, in these cases
none of the miRNAs were deemed as targeting the gene.



\subsubsection{Measuring model fit}

The coefficient of determination $R^2$ of a predictive model is defined as
\begin{equation}
	R^2 = \frac{SS_{\textup{residuals}}}{SS_{\textup{total}}} = \frac{\sum_{i=1}^{n}(\hat{y}_i-y_i)^2}{\sum_{i=1}^{n}(y_i-\bar{y})^2},
\end{equation}
where $\hat{y}_i$ are the predictions made by the model and $\bar{y}$ is the mean of the outcome variable.
$R^2$ indicates the proportion of variance of the outcome variable that is explained
by a statistical model. It provides a measure of how well the model
replicates the observed values and can be used as a measure of model fit.
$R^2$ has the property of being invariant to variable scaling, which
makes it suitable for use with expression data, as expression data
do not have a well defined scale, $R^2$. A caveat of
$R^2$ is that in linear regression it often increases monotonically by adding more explanatory
variables. The adjusted $R^2$, defined as $\bar{R}^2 = 1-(1-R^2)*(n-1)/(n-p)$
(where $n$ is the number of observations and $p$ the number of explanatory
variables) adjusts for the number of regressors relative to the number of
observations, thus penalizing inclusion of additional variables. $\bar{R}^2$
was used for comparing the projected model with the gene only model.



\subsubsection{Verification of predicted targets}

